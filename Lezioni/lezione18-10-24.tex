\section{Lezione del 24/10/2018 [Marmi]}

\begin{definition}[numero algebrico]
	Un numero $ \alpha $ in $ \R $ o $ \C $ si dice algebrico se è radice di un polinomio a coefficienti interi, cioè se esiste $ P \in \Z[x] $ tale che $ P(\alpha) = 0 $. \\
	Si definisce \emph{grado} $ d $ di $ \alpha $ se $ \alpha $ è radice di un polinomio di grado $ d $ e di nessun polinomio di grado minore.  
\end{definition}

\begin{thm}[Liouville]
	Se $ \alpha $ è algebrico di grado $ d \geq 2 $ allora $ \forall \epsilon > 0 $ la disequazione
	\[
		\abs{\alpha - \frac{p}{q}} < \frac{1}{q^{d + \epsilon}}
	\]
	ha solo un numero finito di soluzioni. 
\end{thm}
%
\begin{proof}
	Sia $ P \in \Z[x] $ di grado $ d $ indivisibile tale che $ P(\alpha) = 0 $. Osserviamo che per ogni $ p/q \in \Q $ diverso da $ \alpha $ vale che $ P(p/q) = N/q^d $ per un qualche $ N \in \Z \setminus \{0\} $. \\
	Lo sviluppo in serie di Taylor di $ P $ centrato in $ \alpha $ ha espressione finita e pari a 
	\[
		P(x) = \sum_{k = 1}^{d} \frac{P^{(k)}(\alpha)}{k!} (x - \alpha)^k
	\]
	dove abbiamo omesso il termine di grado zero essendo $ P^{(0)}(\alpha) = P(\alpha) = 0 $. Così
	\[
		\abs{P\left(\frac{p}{q}\right)} \leq \sum_{k = 1}^{d} \frac{\abs{P^{(k)}(\alpha)}}{k!} \abs{\frac{p}{q} - \alpha}^k = \abs{\frac{p}{q} - \alpha} \sum_{k = 1}^{d} \frac{\abs{P^{(k)}(\alpha)}}{k!} \abs{\frac{p}{q} - \alpha}^{k - 1}.
	\]
	Se $ \abs{p/q - \alpha} \leq 1 $, posto $ A(\alpha) \coloneqq d \sup_{1 \leq k \leq d} \frac{1}{k!} \abs{P^{(k)}(\alpha)} $ otteniamo 
	\[
		\frac{\abs{N}}{q^d} \leq \abs{\frac{p}{q} - \alpha} A(\alpha).
	\]
\end{proof}

\subsection{Back to dinamica topologica}
\emph{Setting}: $ (X, d) $ è uno spazio metrico compatto, $ f \colon X \to X $ è una funzione continua e la dinamica è quella data dall'iterazione di $ f $. 

\begin{definition}[funzione topologicamente transitiva]
	$ f $ si dice topologicamente transitiva se esiste un punto la cui orbita è densa cioè $ \exists x \in X : \overline{\mathcal{O}_f(x)} = X $.
\end{definition}

\begin{definition}[funzione minimale]
	$ f $ si dice minimale se $ \forall x \in X, \ \overline{\mathcal{O}_f(x)} = X $. 
\end{definition}

\begin{definition}[insieme minimale]
	Un insieme chiuso, $ f $-invariante e non vuoto $ A \subseteq X $ si dice minimale se $ f\lvert_A $ è minimale. 
\end{definition}

Osserviamo che la transitività topologica così come la minimalità sono invarianti per coniugazione o semi-coniugazione topologica.

\begin{thm}
	Sia $ f \colon X \to X $ un omeomorfismo. I seguenti fatti sono equivalenti.
	\begin{enumerate}[label=(\roman*)]
		\item $ f $ è topologicamente transitiva.
		\item \emph{Idemcomponibilità topologica}: se $ U \subseteq X $ è aperto $ f $-totalmente invariante, i.e. $ U = f(U) = f^{-1}(U) $, allora $ U = \emptyset $ o $ \overline{U} = X $.
		\item Per ogni coppia di aperti non vuoti $ U, V \subseteq X $ esiste un $ n_0 \in \Z $ tale che $ f^{n_0}(U) \cap V \neq \emptyset $. 
		\item $ \{x \in X : \overline{\mathcal{O}_f(x)} = X\} $ è un $ G_\delta $-denso\footnote{%
			Un sottoinsieme $ A $ di uno spazio topologico $ (X, \tau) $ si dice $ G_\delta $-denso se è intersezione numerabile di aperti densi. 
		}.
	\end{enumerate}
\end{thm}
%
\begin{proof}
	Mostriamo le varie implicazioni. 
	\begin{enumerate}
		\item[$ (i) \Rightarrow (ii) $] Per ipotesi $ \exists x \in X : \overline{\mathcal{O}_f(x)} = X $. Allora se $ U \neq \emptyset, \ \exists n \in \Z : f^{n}(U) \in U $ da cui essendo $ f $-totalmente invariante, $ \forall n \in Z, \ f^n(x) \in U $ ovvero $ \overline{U} = \overline{\mathcal{O}_f(x)} = X $. 
		\item[$ (ii) \Rightarrow (iii) $] Dato $ U $ aperto e non vuoto costruiamo un aperto non vuoto e $ f $-totalmente invariante ponendo $ U' \coloneqq \bigcup_{n \in \Z} f^n(U) $. Per ipotesi $ \overline{U'} = X $ da cui $ U' \cap V \neq \emptyset $. Allora per definizione $ \exists n_0 \in \Z : f^{n_0}(U) \cap V \neq \emptyset $.
		\item[$ (iii) \Rightarrow (iv) $] Ricordiamo che gli spazi metrici compatti sono \emph{spazi polacchi} cioè ammettono una base numerabile di aperti $ \{U_i\}_{i \in \N} $. Osserviamo che il fatto che $ \overline{\mathcal{O}_f(x)} = X $ è equivalente a chiedere che $ \forall n \in \N, \ \exists j \in \Z : f^j(x) \in U_n $ e cioè che $ x \in \bigcap_{n \in \N} \bigcup_{m \in \Z} f^{m}(U_n) $. Ora per ipotesi $ \bigcup_{m \in \Z} f^m(x) $ ha intersezione non vuota con ogni aperto non vuoto, cioè è denso in $ X $. Per definizione questo implica che $ \{x \in X : \overline{\mathcal{O}_f(x)} = X\} $ è un $ G_\delta $-denso. 
		\item[$ (iv) \Rightarrow (i) $] Essendo $ \{x \in X : \overline{\mathcal{O}_f(x)} = X\} $ un $ G_\delta $-denso per il Teorema di Ba\^{i}re è anche un denso e pertanto non vuoto. Dunque $ \exists x \in X : \overline{\mathcal{O}_f(x)} = X $ cioè $ f $ è topologicamente transitiva. \qedhere
	\end{enumerate}
\end{proof}

Osserviamo che se $ f \colon X \to X $ è solo continua allora $ (i) $ e $ (ii) $ sono comunque equivalenti a patto che $ X $ non abbia punti isolati. \\

\begin{definition}[integrale primo]
	Un integrale primo è una funzione $ \varphi \colon X \to \R $ tale che $ \varphi \circ f = \varphi $, cioè che è costante sulle orbite. 
\end{definition}

\begin{proposition}
	Se $ f $ è topologicamente transitiva allora gli unici integrali primi continui sono funzioni costanti.
\end{proposition}
%
\begin{proof}
	Se $ \varphi $ è un integrale primo allora $ \varphi(\mathcal{O}_f(\bar{x})) = c $ con $ c \in \R $ dove $ \bar{x} \in X $ è il punto le cui orbite sono dense in $ X $. Allora se $ x \in \R $ esiste una successione $ (x_k)_{k \in \N} \subseteq \mathcal{O}_f(\bar{x}) $ tale che $ x_k \to x $. Essendo $ \varphi $ continua allora $ \varphi(x) = \lim_{k} \varphi(x_k) = \lim_{k} c = c $ da cui segue la tesi. 
\end{proof}

\begin{definition}[topologicamente mescolante]
	Una dinamica definita da $ f $ si dice topologicamente mescolante se $ \forall U, V $ aperti non vuoti $ \exists n_0 \in \N : \forall n \geq n_0, \ f^n(U) \cap V \neq \emptyset $. 
\end{definition}

\begin{example}[dilatazione sul toro e shift]
	Fissato $ m \geq 2 $ intero sia $ E_m \colon \T^1 \to \T^1 $ data da $ E_m(x) \coloneqq mx \pmod{1} $. Su $ \T^1 $ poniamo la distanza euclidea modulo 1, cioè $ d(x, y) = \inf_{p \in \Z} \abs{x - y - p} $ rispetto alla quale è uno spazio metrico compatto. Osserviamo i seguenti fatti
	\begin{itemize}
		\item $ E_m $ è una mappa \emph{espansiva}, cioè se $ x, y \in \T^1 $ e $ d(x, y) < 1/2^n $ allora $ d(E_m(x), E_m(y)) = m d(x, y) $ per un qualche $ n $. (??)
		\item $ E_m $ conserva la misura di Lebesgue 
		\item $ E_m $ è un fattore dello \emph{shift su $ m $ simboli}. \\
		Posto $ \Sigma_m \coloneqq \{0, 1, \ldots, m-1\}^{\N} $ definiamo la mappa $ \sigma \colon \Sigma_m \to \Sigma_m $ di shift su $ m $ simboli che dato un elemento $ (\epsilon_i)_{i \in \N} \in \Sigma_m $ agisce sulla successione avanzando di uno gli indici $ \sigma\left((\epsilon_i)_{i \in \N}\right) \coloneqq (\epsilon_{i + 1})_{i \in \N} $. \\
		Sull'insieme $ \Sigma_m $ è possibile definire un concetto di \emph{profondità} dalla quale si separano due successioni come 
		\[
			a\left((\epsilon_i), (\delta_i)\right) \coloneqq \inf_{i \in \N} \{\epsilon_i \neq \delta_i\}
		\]
		a partire dalla quale possiamo definire una distanza tra successioni 
		\[
			d\left((\epsilon_i), (\delta_i)\right) \coloneqq m^{-a\left((\epsilon_i), (\delta_i)\right)}. 
		\]
		Rispetto a tale distanza $ \Sigma_m $ è uno spazio metrico compatto e $ \sigma $ è una funzione continua ma non è un omeomorfismo. Ciò è invece vero se al posto di $ \Sigma_m $ si considera l'insieme delle successioni bi-infinite $ \tilde{\Sigma}_m \coloneqq \{0, 1, \ldots, m-1\}^{\Z} $. \\
		Prima di tutto mostriamo che $ d $ è una distanza. La proprietà simmetrica è ovvia per costruzione così come il fatto che $ d $ sia positiva. Se $ d((x_i), (y_i)) = 0 $ allora $ a((x_i), (y_i) = +\infty $ ovvero $ x = y $. Per quanto riguarda la disuguaglianza triangolare dobbiamo mostrare che $ \forall (x_i), (y_i), (z_i) \in \Sigma_m $ si ha $ m^{-a((x_i),(z_i))} \leq m^{-a((x_i), (y_i))} + m^{-a((y_i), (z_i))} $. Siano $ \alpha \coloneqq a((x_i), (z_i)) $ e $ \beta \coloneqq a((x_i)) $. Se $ \alpha = \beta $ allora la disuguaglianza è verificata. Se $ \alpha < \beta $ allora $ \forall i \leq \beta-1, \ x_i = y_i $ ed essendo $ x_\alpha \neq z_\alpha $ otteniamo che $ z_\alpha \neq y_\alpha $ e $ \forall i \leq \alpha-1, \ z_i = y_i $, cioè $ a((y_i), (z_i)) = \alpha $. Viceversa se $ \alpha > \beta $ allora $ \forall i \leq \beta-1, x_i=y_i=z_i $ e inoltre $ x_\beta=z_\beta \neq y_\beta $ quindi $ a((y_i), (z_i)) = \beta $: osservando che $ m, \alpha, \beta \in \N $ e che $ m \geq 2 $ abbiamo che $ \alpha > \beta \Rightarrow \alpha \geq \beta - 1 \geq \beta - \log_m 2 $ per ogni $ m $, da cui otteniamo che $ m^{-\alpha} \leq 2m^{-\beta} $ che è la disuguaglianza voluta. \\
        Per mostrare la compattezza verifichiamo che $ \Sigma_m $ è totalmente limitato e completo.
        \begin{itemize}
            \item Le palle di raggio $ R $ e di centro $ \bar{x} \coloneqq (\bar{x}_i)_{i \in \N} $ di questo spazio metrico sono 
            \[
            B_R(\bar{x}) = \{(x_i) \in \Sigma_m : \forall i \in \N : 0 \leq i \leq -\log_m{R}, \ x_i = \bar{x}_i\}.
            \] 
            Osserviamo quindi che, posto $ p \coloneqq \left \lfloor -\log_m{R} \right \rfloor $, ogni successione della forma \linebreak $ (x_1, \ldots, x_p, x_{p+1}, \ldots) $ stà in $ B_R{\left((x_1, \ldots, x_p, 0, \ldots)\right)} $ e pertanto considerando l'insieme delle possibili $ m^p $ successioni distinte che sono costantemente nulle dal $ p+1 $-esimo termine in poi, le palle di raggio $ R $ centrate in tali elementi sono un ricoprimento finito di $ \Sigma_m $. 
            \item Sia ora $ (x^j)_{j \in \N} \coloneqq ((x_i^j)_{i \in \N})_{j \in \N} $ una successione di Cauchy in $ \Sigma_m $, cioè tale che $ \forall \epsilon > 0, \ \exists \bar{j} : \forall j_1, j_2 \geq \bar{j}, \ x_i^{j_1} = x_i^{j_2} \ \forall i \leq -\log_m{\epsilon} $. Essendo la successione $ (x_i^j)_{j \in \N} $ definitivamente costante (a $ i $ fissato, scelto un $ \epsilon $ positivo tale che $ -\log_m{\epsilon} > i $, per la proprietà di Cauchy abbiamo che $ \exists \bar{j} : \forall j \geq \bar{j}, \ x_i^j = \text{cost} $) possiamo porre $ \bar{x}_i \coloneqq \lim_{j \to +\infty} x_i^j $. La successione $ \bar{x} \coloneqq (\bar{x}_i)_{i \in \N} $ è il candidato limite. Dobbiamo mostrare che $ \lim_{j \to +\infty} d(x^j, \bar{x}) = 0 $, cioè che $ \forall I, \ \exists \bar{j} : \forall j \geq \bar{j}, \ \inf{\{i : x_i^j \neq \bar{x}_i\}} \geq I $. Tale proprietà è verificata per costruzione: fissato $ I \in \N $, per ogni $ i \in \{0, \ldots, I\} $ sappiamo che esiste $ j_i \in \N : \forall j \geq j_i, \ x_i^j = \bar{x}_i $; prendendo allora $ \bar{j} \coloneqq \max \{j_1, \ldots, j_I\} $ otteniamo che per ogni $ 0 \leq i \leq I $ e $ \forall j \geq \bar{j} $ si ha $ x_i^j = \bar{x}_i $, cioè $ \inf{\{i : x_i^j \neq \bar{x}_i\}} \geq I $. 
        \end{itemize} 
	\end{itemize}
\end{example}


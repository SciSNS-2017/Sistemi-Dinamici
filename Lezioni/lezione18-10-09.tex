\section{Lezione del 09/10/2018 [Marmi]}
\begin{definition}[gruppo]
	Un gruppo è una coppia $ \mathcal{G} \coloneqq (G, \star) $ dove $ G $ è un insieme e $ {\star \colon G \times G \to G} $ è un'operazione binaria che gode delle seguenti proprietà
	\begin{enumerate}[label=(\roman*)]
		\item \emph{associativa}: $ \forall g_1, g_2, g_3 \in G, \ g_1 \star (g_2 \star g_3) = (g_1 \star g_2) \star g_3 $;
		\item \emph{elemento neutro sinistro}: $  \exists e \in G : \forall g \in G, \ g \star e = g $;
		\item \emph{inverso sinistro}: $ \forall g \in G, \exists g^{-1} \in G : g \star g^{-1} = e $.
	\end{enumerate}
	A partire da queste si mostra facilmente che l'elemento neutro destro è anche elemento neutro sinistro, l'inverso destro è anche inverso sinistro, che l'elemento neutro e l'inverso sono unici. \\
	Se non ci sono ambiguità circa l'operazione definita su $ G $ indicheremo più semplicemente il gruppo $ \mathcal{G} $ facendo riferimento al solo insieme $ G $. 
\end{definition}

\begin{definition}[sistema dinamico]
	Un sistema dinamico è una terna $ (\mathcal{G}, \mathcal{X}, \Phi) $ dove $ {\mathcal{G} \coloneqq (G, \star)} $ è un (semi-)gruppo\footnote{Per \emph{semigruppo} si intende una coppia $ (G,\star) $ dove $ \star $ è associativa.}, $ \mathcal{X} $ è uno spazio, cioè un insieme $ X $ dotato di una qualche struttura (per esempio una topologia), e 
	\begin{align*}
		\Phi \colon G \times X & \to X \\
		(g, x) & \mapsto \Phi(g, x) = \Phi_g(x)
	\end{align*}
	è un'applicazione tale che
	\begin{enumerate}[label=(\roman*)]
		\item $ \forall x \in X, \ \Phi_e(x) = x $ dove $ e $ è l'elemento neutro di $ G $, cioè $ \Phi_e = \Id_X $;
		\item $ \forall g_1, g_2 \in G, \forall x \in X, \ \Phi_{(g_1 \star g_2)}(x) = (\Phi_{g1} \circ \Phi_{g_2})(x) $ cioè $ \Phi_{(g_1 \star g_2)} = \Phi_{g1} \circ \Phi_{g_2} $.
	\end{enumerate}
	Più brevemente diciamo che un sistema dinamico è l'\emph{azione} di un gruppo $ G $ su uno spazio $ X $ definita da una mappa $ \Phi $. 
\end{definition}

Nella maggior parte dei casi useremo come gruppo insiemi numerici $ \N $, $ \Z $ e $ \R $ con le usuali operazioni. Nei primi due casi parleremo di sistemi a \emph{tempo discreto} mentre nell'ultimo di sistemi a \emph{tempo continuo}. Come spazio $ \mathcal{X} $ useremo spesso uno \emph{spazio metrico compatto} (e.g. la sfera $ \S^d $, il toro $ \T^d $ o un intervallo chiuso $ [a, b] $), uno \emph{spazio di misura} o gli insiemi $ \R^d $ e $ \C $ con le usuali strutture. \\

Per quanto riguarda la mappa $ \Phi $ osserviamo che per definizione $ \Phi_g \in \End{(X)} $ ovvero è un \emph{endomorfismo} su $ X $. Tuttavia spesso penseremo a $ \Phi_g \in \Aut{(X)} $ ovvero un \emph{automorfismo} cioè un endomorfismo invertibile. \\

\begin{example}
Sia $ f \in \End{(X)} $. Dato $ n \in \N $ poniamo $ f^n \coloneqq f \circ \cdots \circ f $ ($ f $ composta $ n $ volte) con la convenzione che $ f^1 = f $ e $ f^0 = \Id_X $. Se consideriamo $ \N $ con l'operazione di addizione, l'applicazione $ \Phi^f $ data da $ \Phi_n^f(x) \coloneqq f^n(x) $ definisce un sistema dinamico. \\
Se prendiamo $ f \in \Aut{(X)} $ possiamo considerare la stessa costruzione usando come gruppo $ \Z $ e definendo $ f^{-n} $ come l'inversa di $ f^n $. 
\end{example}

\begin{example}
	Partendo dalla costruzione appena data possiamo prendere $ X = [0, 1] $ e per $ \alpha \in \R $ la funzione $ f(x) \coloneqq x + \alpha \pmod{1} $. Osserviamo che essendo $ f $ invertibile possiamo definire come sopra l'applicazione $ \Phi $ su $ \Z $. Il sistema così definito è un prototipo di \emph{sistema periodico} se $ \alpha \in \Q $ e di \emph{sistema quasi-periodico} se $ \alpha \notin \Q $. \\
	\definecolor{uuuuuu}{rgb}{0.26666666666666666,0.26666666666666666,0.26666666666666666}
\definecolor{zzttqq}{rgb}{0.6,0.2,0}
\definecolor{ttqqqq}{rgb}{0.2,0,0}
\definecolor{ududff}{rgb}{0.30196078431372547,0.30196078431372547,1}
\begin{tikzpicture}[scale=3, line cap=round,line join=round,>=triangle 45,x=1cm,y=1cm]
\clip(-0.4797180045826627,-0.4928702782458507) rectangle (1.1851515676356013,1.2462596633807077);
\fill[line width=2pt,color=zzttqq,fill=zzttqq,fill opacity=0.10000000149011612] (0,0) -- (0,1) -- (1,1) -- (1,0) -- cycle;
\draw[line width=2pt,color=ttqqqq] (0.3333343699667031,0) -- (0.3333343699667031,0);
\draw[line width=2pt,color=ttqqqq] (0.3333343699667031,0) -- (0.3350010333661191,0.0016677000327857683);
\draw[line width=2pt,color=ttqqqq] (0.3350010333661191,0.0016677000327857683) -- (0.33666769676553504,0.0033343634322017257);
\draw[line width=2pt,color=ttqqqq] (0.33666769676553504,0.0033343634322017257) -- (0.338334360164951,0.005001026831617683);
\draw[line width=2pt,color=ttqqqq] (0.338334360164951,0.005001026831617683) -- (0.34000102356436696,0.0066676902310336406);
\draw[line width=2pt,color=ttqqqq] (0.34000102356436696,0.0066676902310336406) -- (0.3416676869637829,0.008334353630449598);
\draw[line width=2pt,color=ttqqqq] (0.3416676869637829,0.008334353630449598) -- (0.34333435036319887,0.010001017029865555);
\draw[line width=2pt,color=ttqqqq] (0.34333435036319887,0.010001017029865555) -- (0.3450010137626148,0.011667680429281513);
\draw[line width=2pt,color=ttqqqq] (0.3450010137626148,0.011667680429281513) -- (0.3466676771620308,0.01333434382869747);
\draw[line width=2pt,color=ttqqqq] (0.3466676771620308,0.01333434382869747) -- (0.34833434056144674,0.015001007228113428);
\draw[line width=2pt,color=ttqqqq] (0.34833434056144674,0.015001007228113428) -- (0.3500010039608627,0.016667670627529385);
\draw[line width=2pt,color=ttqqqq] (0.3500010039608627,0.016667670627529385) -- (0.35166766736027866,0.018334334026945343);
\draw[line width=2pt,color=ttqqqq] (0.35166766736027866,0.018334334026945343) -- (0.3533343307596946,0.0200009974263613);
\draw[line width=2pt,color=ttqqqq] (0.3533343307596946,0.0200009974263613) -- (0.35500099415911057,0.021667660825777257);
\draw[line width=2pt,color=ttqqqq] (0.35500099415911057,0.021667660825777257) -- (0.35666765755852653,0.023334324225193215);
\draw[line width=2pt,color=ttqqqq] (0.35666765755852653,0.023334324225193215) -- (0.3583343209579425,0.025000987624609172);
\draw[line width=2pt,color=ttqqqq] (0.3583343209579425,0.025000987624609172) -- (0.36000098435735844,0.02666765102402513);
\draw[line width=2pt,color=ttqqqq] (0.36000098435735844,0.02666765102402513) -- (0.3616676477567744,0.028334314423441087);
\draw[line width=2pt,color=ttqqqq] (0.3616676477567744,0.028334314423441087) -- (0.36333431115619036,0.030000977822857045);
\draw[line width=2pt,color=ttqqqq] (0.36333431115619036,0.030000977822857045) -- (0.3650009745556063,0.031667641222273);
\draw[line width=2pt,color=ttqqqq] (0.3650009745556063,0.031667641222273) -- (0.3666676379550223,0.03333430462168896);
\draw[line width=2pt,color=ttqqqq] (0.3666676379550223,0.03333430462168896) -- (0.36833430135443823,0.03500096802110492);
\draw[line width=2pt,color=ttqqqq] (0.36833430135443823,0.03500096802110492) -- (0.3700009647538542,0.036667631420520874);
\draw[line width=2pt,color=ttqqqq] (0.3700009647538542,0.036667631420520874) -- (0.37166762815327015,0.03833429481993683);
\draw[line width=2pt,color=ttqqqq] (0.37166762815327015,0.03833429481993683) -- (0.3733342915526861,0.04000095821935279);
\draw[line width=2pt,color=ttqqqq] (0.3733342915526861,0.04000095821935279) -- (0.37500095495210206,0.041667621618768746);
\draw[line width=2pt,color=ttqqqq] (0.37500095495210206,0.041667621618768746) -- (0.376667618351518,0.043334285018184704);
\draw[line width=2pt,color=ttqqqq] (0.376667618351518,0.043334285018184704) -- (0.378334281750934,0.04500094841760066);
\draw[line width=2pt,color=ttqqqq] (0.378334281750934,0.04500094841760066) -- (0.38000094515034993,0.04666761181701662);
\draw[line width=2pt,color=ttqqqq] (0.38000094515034993,0.04666761181701662) -- (0.3816676085497659,0.048334275216432576);
\draw[line width=2pt,color=ttqqqq] (0.3816676085497659,0.048334275216432576) -- (0.38333427194918185,0.050000938615848534);
\draw[line width=2pt,color=ttqqqq] (0.38333427194918185,0.050000938615848534) -- (0.3850009353485978,0.05166760201526449);
\draw[line width=2pt,color=ttqqqq] (0.3850009353485978,0.05166760201526449) -- (0.38666759874801376,0.05333426541468045);
\draw[line width=2pt,color=ttqqqq] (0.38666759874801376,0.05333426541468045) -- (0.3883342621474297,0.055000928814096406);
\draw[line width=2pt,color=ttqqqq] (0.3883342621474297,0.055000928814096406) -- (0.3900009255468457,0.05666759221351236);
\draw[line width=2pt,color=ttqqqq] (0.3900009255468457,0.05666759221351236) -- (0.39166758894626164,0.05833425561292832);
\draw[line width=2pt,color=ttqqqq] (0.39166758894626164,0.05833425561292832) -- (0.3933342523456776,0.06000091901234428);
\draw[line width=2pt,color=ttqqqq] (0.3933342523456776,0.06000091901234428) -- (0.39500091574509355,0.061667582411760236);
\draw[line width=2pt,color=ttqqqq] (0.39500091574509355,0.061667582411760236) -- (0.3966675791445095,0.06333424581117619);
\draw[line width=2pt,color=ttqqqq] (0.3966675791445095,0.06333424581117619) -- (0.39833424254392547,0.06500090921059215);
\draw[line width=2pt,color=ttqqqq] (0.39833424254392547,0.06500090921059215) -- (0.4000009059433414,0.06666757261000811);
\draw[line width=2pt,color=ttqqqq] (0.4000009059433414,0.06666757261000811) -- (0.4016675693427574,0.06833423600942407);
\draw[line width=2pt,color=ttqqqq] (0.4016675693427574,0.06833423600942407) -- (0.40333423274217334,0.07000089940884002);
\draw[line width=2pt,color=ttqqqq] (0.40333423274217334,0.07000089940884002) -- (0.4050008961415893,0.07166756280825598);
\draw[line width=2pt,color=ttqqqq] (0.4050008961415893,0.07166756280825598) -- (0.40666755954100525,0.07333422620767194);
\draw[line width=2pt,color=ttqqqq] (0.40666755954100525,0.07333422620767194) -- (0.4083342229404212,0.0750008896070879);
\draw[line width=2pt,color=ttqqqq] (0.4083342229404212,0.0750008896070879) -- (0.41000088633983717,0.07666755300650385);
\draw[line width=2pt,color=ttqqqq] (0.41000088633983717,0.07666755300650385) -- (0.4116675497392531,0.07833421640591981);
\draw[line width=2pt,color=ttqqqq] (0.4116675497392531,0.07833421640591981) -- (0.4133342131386691,0.08000087980533577);
\draw[line width=2pt,color=ttqqqq] (0.4133342131386691,0.08000087980533577) -- (0.41500087653808504,0.08166754320475172);
\draw[line width=2pt,color=ttqqqq] (0.41500087653808504,0.08166754320475172) -- (0.416667539937501,0.08333420660416768);
\draw[line width=2pt,color=ttqqqq] (0.416667539937501,0.08333420660416768) -- (0.41833420333691695,0.08500087000358364);
\draw[line width=2pt,color=ttqqqq] (0.41833420333691695,0.08500087000358364) -- (0.4200008667363329,0.0866675334029996);
\draw[line width=2pt,color=ttqqqq] (0.4200008667363329,0.0866675334029996) -- (0.42166753013574887,0.08833419680241555);
\draw[line width=2pt,color=ttqqqq] (0.42166753013574887,0.08833419680241555) -- (0.4233341935351648,0.09000086020183151);
\draw[line width=2pt,color=ttqqqq] (0.4233341935351648,0.09000086020183151) -- (0.4250008569345808,0.09166752360124747);
\draw[line width=2pt,color=ttqqqq] (0.4250008569345808,0.09166752360124747) -- (0.42666752033399674,0.09333418700066343);
\draw[line width=2pt,color=ttqqqq] (0.42666752033399674,0.09333418700066343) -- (0.4283341837334127,0.09500085040007938);
\draw[line width=2pt,color=ttqqqq] (0.4283341837334127,0.09500085040007938) -- (0.43000084713282866,0.09666751379949534);
\draw[line width=2pt,color=ttqqqq] (0.43000084713282866,0.09666751379949534) -- (0.4316675105322446,0.0983341771989113);
\draw[line width=2pt,color=ttqqqq] (0.4316675105322446,0.0983341771989113) -- (0.43333417393166057,0.10000084059832726);
\draw[line width=2pt,color=ttqqqq] (0.43333417393166057,0.10000084059832726) -- (0.43500083733107653,0.10166750399774321);
\draw[line width=2pt,color=ttqqqq] (0.43500083733107653,0.10166750399774321) -- (0.4366675007304925,0.10333416739715917);
\draw[line width=2pt,color=ttqqqq] (0.4366675007304925,0.10333416739715917) -- (0.43833416412990844,0.10500083079657513);
\draw[line width=2pt,color=ttqqqq] (0.43833416412990844,0.10500083079657513) -- (0.4400008275293244,0.10666749419599109);
\draw[line width=2pt,color=ttqqqq] (0.4400008275293244,0.10666749419599109) -- (0.44166749092874036,0.10833415759540704);
\draw[line width=2pt,color=ttqqqq] (0.44166749092874036,0.10833415759540704) -- (0.4433341543281563,0.110000820994823);
\draw[line width=2pt,color=ttqqqq] (0.4433341543281563,0.110000820994823) -- (0.4450008177275723,0.11166748439423896);
\draw[line width=2pt,color=ttqqqq] (0.4450008177275723,0.11166748439423896) -- (0.44666748112698823,0.11333414779365492);
\draw[line width=2pt,color=ttqqqq] (0.44666748112698823,0.11333414779365492) -- (0.4483341445264042,0.11500081119307087);
\draw[line width=2pt,color=ttqqqq] (0.4483341445264042,0.11500081119307087) -- (0.45000080792582015,0.11666747459248683);
\draw[line width=2pt,color=ttqqqq] (0.45000080792582015,0.11666747459248683) -- (0.4516674713252361,0.11833413799190279);
\draw[line width=2pt,color=ttqqqq] (0.4516674713252361,0.11833413799190279) -- (0.45333413472465206,0.12000080139131875);
\draw[line width=2pt,color=ttqqqq] (0.45333413472465206,0.12000080139131875) -- (0.455000798124068,0.1216674647907347);
\draw[line width=2pt,color=ttqqqq] (0.455000798124068,0.1216674647907347) -- (0.456667461523484,0.12333412819015066);
\draw[line width=2pt,color=ttqqqq] (0.456667461523484,0.12333412819015066) -- (0.45833412492289993,0.12500079158956662);
\draw[line width=2pt,color=ttqqqq] (0.45833412492289993,0.12500079158956662) -- (0.4600007883223159,0.12666745498898258);
\draw[line width=2pt,color=ttqqqq] (0.4600007883223159,0.12666745498898258) -- (0.46166745172173185,0.12833411838839853);
\draw[line width=2pt,color=ttqqqq] (0.46166745172173185,0.12833411838839853) -- (0.4633341151211478,0.1300007817878145);
\draw[line width=2pt,color=ttqqqq] (0.4633341151211478,0.1300007817878145) -- (0.46500077852056376,0.13166744518723045);
\draw[line width=2pt,color=ttqqqq] (0.46500077852056376,0.13166744518723045) -- (0.4666674419199797,0.1333341085866464);
\draw[line width=2pt,color=ttqqqq] (0.4666674419199797,0.1333341085866464) -- (0.4683341053193957,0.13500077198606236);
\draw[line width=2pt,color=ttqqqq] (0.4683341053193957,0.13500077198606236) -- (0.47000076871881163,0.13666743538547832);
\draw[line width=2pt,color=ttqqqq] (0.47000076871881163,0.13666743538547832) -- (0.4716674321182276,0.13833409878489428);
\draw[line width=2pt,color=ttqqqq] (0.4716674321182276,0.13833409878489428) -- (0.47333409551764355,0.14000076218431023);
\draw[line width=2pt,color=ttqqqq] (0.47333409551764355,0.14000076218431023) -- (0.4750007589170595,0.1416674255837262);
\draw[line width=2pt,color=ttqqqq] (0.4750007589170595,0.1416674255837262) -- (0.47666742231647546,0.14333408898314215);
\draw[line width=2pt,color=ttqqqq] (0.47666742231647546,0.14333408898314215) -- (0.4783340857158914,0.1450007523825581);
\draw[line width=2pt,color=ttqqqq] (0.4783340857158914,0.1450007523825581) -- (0.4800007491153074,0.14666741578197406);
\draw[line width=2pt,color=ttqqqq] (0.4800007491153074,0.14666741578197406) -- (0.48166741251472334,0.14833407918139002);
\draw[line width=2pt,color=ttqqqq] (0.48166741251472334,0.14833407918139002) -- (0.4833340759141393,0.15000074258080598);
\draw[line width=2pt,color=ttqqqq] (0.4833340759141393,0.15000074258080598) -- (0.48500073931355525,0.15166740598022194);
\draw[line width=2pt,color=ttqqqq] (0.48500073931355525,0.15166740598022194) -- (0.4866674027129712,0.1533340693796379);
\draw[line width=2pt,color=ttqqqq] (0.4866674027129712,0.1533340693796379) -- (0.48833406611238717,0.15500073277905385);
\draw[line width=2pt,color=ttqqqq] (0.48833406611238717,0.15500073277905385) -- (0.4900007295118031,0.1566673961784698);
\draw[line width=2pt,color=ttqqqq] (0.4900007295118031,0.1566673961784698) -- (0.4916673929112191,0.15833405957788577);
\draw[line width=2pt,color=ttqqqq] (0.4916673929112191,0.15833405957788577) -- (0.49333405631063504,0.16000072297730172);
\draw[line width=2pt,color=ttqqqq] (0.49333405631063504,0.16000072297730172) -- (0.495000719710051,0.16166738637671768);
\draw[line width=2pt,color=ttqqqq] (0.495000719710051,0.16166738637671768) -- (0.49666738310946695,0.16333404977613364);
\draw[line width=2pt,color=ttqqqq] (0.49666738310946695,0.16333404977613364) -- (0.4983340465088829,0.1650007131755496);
\draw[line width=2pt,color=ttqqqq] (0.4983340465088829,0.1650007131755496) -- (0.5000007099082989,0.1666673765749656);
\draw[line width=2pt,color=ttqqqq] (0.5000007099082989,0.1666673765749656) -- (0.5016673733077149,0.16833403997438162);
\draw[line width=2pt,color=ttqqqq] (0.5016673733077149,0.16833403997438162) -- (0.503334036707131,0.17000070337379763);
\draw[line width=2pt,color=ttqqqq] (0.503334036707131,0.17000070337379763) -- (0.505000700106547,0.17166736677321365);
\draw[line width=2pt,color=ttqqqq] (0.505000700106547,0.17166736677321365) -- (0.506667363505963,0.17333403017262966);
\draw[line width=2pt,color=ttqqqq] (0.506667363505963,0.17333403017262966) -- (0.508334026905379,0.17500069357204567);
\draw[line width=2pt,color=ttqqqq] (0.508334026905379,0.17500069357204567) -- (0.510000690304795,0.1766673569714617);
\draw[line width=2pt,color=ttqqqq] (0.510000690304795,0.1766673569714617) -- (0.511667353704211,0.1783340203708777);
\draw[line width=2pt,color=ttqqqq] (0.511667353704211,0.1783340203708777) -- (0.513334017103627,0.1800006837702937);
\draw[line width=2pt,color=ttqqqq] (0.513334017103627,0.1800006837702937) -- (0.515000680503043,0.18166734716970973);
\draw[line width=2pt,color=ttqqqq] (0.515000680503043,0.18166734716970973) -- (0.516667343902459,0.18333401056912574);
\draw[line width=2pt,color=ttqqqq] (0.516667343902459,0.18333401056912574) -- (0.5183340073018751,0.18500067396854175);
\draw[line width=2pt,color=ttqqqq] (0.5183340073018751,0.18500067396854175) -- (0.5200006707012911,0.18666733736795776);
\draw[line width=2pt,color=ttqqqq] (0.5200006707012911,0.18666733736795776) -- (0.5216673341007071,0.18833400076737378);
\draw[line width=2pt,color=ttqqqq] (0.5216673341007071,0.18833400076737378) -- (0.5233339975001231,0.1900006641667898);
\draw[line width=2pt,color=ttqqqq] (0.5233339975001231,0.1900006641667898) -- (0.5250006608995391,0.1916673275662058);
\draw[line width=2pt,color=ttqqqq] (0.5250006608995391,0.1916673275662058) -- (0.5266673242989551,0.19333399096562182);
\draw[line width=2pt,color=ttqqqq] (0.5266673242989551,0.19333399096562182) -- (0.5283339876983711,0.19500065436503783);
\draw[line width=2pt,color=ttqqqq] (0.5283339876983711,0.19500065436503783) -- (0.5300006510977872,0.19666731776445384);
\draw[line width=2pt,color=ttqqqq] (0.5300006510977872,0.19666731776445384) -- (0.5316673144972032,0.19833398116386985);
\draw[line width=2pt,color=ttqqqq] (0.5316673144972032,0.19833398116386985) -- (0.5333339778966192,0.20000064456328587);
\draw[line width=2pt,color=ttqqqq] (0.5333339778966192,0.20000064456328587) -- (0.5350006412960352,0.20166730796270188);
\draw[line width=2pt,color=ttqqqq] (0.5350006412960352,0.20166730796270188) -- (0.5366673046954512,0.2033339713621179);
\draw[line width=2pt,color=ttqqqq] (0.5366673046954512,0.2033339713621179) -- (0.5383339680948672,0.2050006347615339);
\draw[line width=2pt,color=ttqqqq] (0.5383339680948672,0.2050006347615339) -- (0.5400006314942832,0.20666729816094992);
\draw[line width=2pt,color=ttqqqq] (0.5400006314942832,0.20666729816094992) -- (0.5416672948936992,0.20833396156036593);
\draw[line width=2pt,color=ttqqqq] (0.5416672948936992,0.20833396156036593) -- (0.5433339582931153,0.21000062495978195);
\draw[line width=2pt,color=ttqqqq] (0.5433339582931153,0.21000062495978195) -- (0.5450006216925313,0.21166728835919796);
\draw[line width=2pt,color=ttqqqq] (0.5450006216925313,0.21166728835919796) -- (0.5466672850919473,0.21333395175861397);
\draw[line width=2pt,color=ttqqqq] (0.5466672850919473,0.21333395175861397) -- (0.5483339484913633,0.21500061515802998);
\draw[line width=2pt,color=ttqqqq] (0.5483339484913633,0.21500061515802998) -- (0.5500006118907793,0.216667278557446);
\draw[line width=2pt,color=ttqqqq] (0.5500006118907793,0.216667278557446) -- (0.5516672752901953,0.218333941956862);
\draw[line width=2pt,color=ttqqqq] (0.5516672752901953,0.218333941956862) -- (0.5533339386896113,0.22000060535627802);
\draw[line width=2pt,color=ttqqqq] (0.5533339386896113,0.22000060535627802) -- (0.5550006020890274,0.22166726875569404);
\draw[line width=2pt,color=ttqqqq] (0.5550006020890274,0.22166726875569404) -- (0.5566672654884434,0.22333393215511005);
\draw[line width=2pt,color=ttqqqq] (0.5566672654884434,0.22333393215511005) -- (0.5583339288878594,0.22500059555452606);
\draw[line width=2pt,color=ttqqqq] (0.5583339288878594,0.22500059555452606) -- (0.5600005922872754,0.22666725895394207);
\draw[line width=2pt,color=ttqqqq] (0.5600005922872754,0.22666725895394207) -- (0.5616672556866914,0.2283339223533581);
\draw[line width=2pt,color=ttqqqq] (0.5616672556866914,0.2283339223533581) -- (0.5633339190861074,0.2300005857527741);
\draw[line width=2pt,color=ttqqqq] (0.5633339190861074,0.2300005857527741) -- (0.5650005824855234,0.2316672491521901);
\draw[line width=2pt,color=ttqqqq] (0.5650005824855234,0.2316672491521901) -- (0.5666672458849394,0.23333391255160613);
\draw[line width=2pt,color=ttqqqq] (0.5666672458849394,0.23333391255160613) -- (0.5683339092843555,0.23500057595102214);
\draw[line width=2pt,color=ttqqqq] (0.5683339092843555,0.23500057595102214) -- (0.5700005726837715,0.23666723935043815);
\draw[line width=2pt,color=ttqqqq] (0.5700005726837715,0.23666723935043815) -- (0.5716672360831875,0.23833390274985417);
\draw[line width=2pt,color=ttqqqq] (0.5716672360831875,0.23833390274985417) -- (0.5733338994826035,0.24000056614927018);
\draw[line width=2pt,color=ttqqqq] (0.5733338994826035,0.24000056614927018) -- (0.5750005628820195,0.2416672295486862);
\draw[line width=2pt,color=ttqqqq] (0.5750005628820195,0.2416672295486862) -- (0.5766672262814355,0.2433338929481022);
\draw[line width=2pt,color=ttqqqq] (0.5766672262814355,0.2433338929481022) -- (0.5783338896808515,0.24500055634751822);
\draw[line width=2pt,color=ttqqqq] (0.5783338896808515,0.24500055634751822) -- (0.5800005530802675,0.24666721974693423);
\draw[line width=2pt,color=ttqqqq] (0.5800005530802675,0.24666721974693423) -- (0.5816672164796836,0.24833388314635024);
\draw[line width=2pt,color=ttqqqq] (0.5816672164796836,0.24833388314635024) -- (0.5833338798790996,0.25000054654576626);
\draw[line width=2pt,color=ttqqqq] (0.5833338798790996,0.25000054654576626) -- (0.5850005432785156,0.25166720994518227);
\draw[line width=2pt,color=ttqqqq] (0.5850005432785156,0.25166720994518227) -- (0.5866672066779316,0.2533338733445983);
\draw[line width=2pt,color=ttqqqq] (0.5866672066779316,0.2533338733445983) -- (0.5883338700773476,0.2550005367440143);
\draw[line width=2pt,color=ttqqqq] (0.5883338700773476,0.2550005367440143) -- (0.5900005334767636,0.2566672001434303);
\draw[line width=2pt,color=ttqqqq] (0.5900005334767636,0.2566672001434303) -- (0.5916671968761796,0.2583338635428463);
\draw[line width=2pt,color=ttqqqq] (0.5916671968761796,0.2583338635428463) -- (0.5933338602755956,0.26000052694226233);
\draw[line width=2pt,color=ttqqqq] (0.5933338602755956,0.26000052694226233) -- (0.5950005236750117,0.26166719034167835);
\draw[line width=2pt,color=ttqqqq] (0.5950005236750117,0.26166719034167835) -- (0.5966671870744277,0.26333385374109436);
\draw[line width=2pt,color=ttqqqq] (0.5966671870744277,0.26333385374109436) -- (0.5983338504738437,0.26500051714051037);
\draw[line width=2pt,color=ttqqqq] (0.5983338504738437,0.26500051714051037) -- (0.6000005138732597,0.2666671805399264);
\draw[line width=2pt,color=ttqqqq] (0.6000005138732597,0.2666671805399264) -- (0.6016671772726757,0.2683338439393424);
\draw[line width=2pt,color=ttqqqq] (0.6016671772726757,0.2683338439393424) -- (0.6033338406720917,0.2700005073387584);
\draw[line width=2pt,color=ttqqqq] (0.6033338406720917,0.2700005073387584) -- (0.6050005040715077,0.2716671707381744);
\draw[line width=2pt,color=ttqqqq] (0.6050005040715077,0.2716671707381744) -- (0.6066671674709238,0.27333383413759044);
\draw[line width=2pt,color=ttqqqq] (0.6066671674709238,0.27333383413759044) -- (0.6083338308703398,0.27500049753700645);
\draw[line width=2pt,color=ttqqqq] (0.6083338308703398,0.27500049753700645) -- (0.6100004942697558,0.27666716093642246);
\draw[line width=2pt,color=ttqqqq] (0.6100004942697558,0.27666716093642246) -- (0.6116671576691718,0.2783338243358385);
\draw[line width=2pt,color=ttqqqq] (0.6116671576691718,0.2783338243358385) -- (0.6133338210685878,0.2800004877352545);
\draw[line width=2pt,color=ttqqqq] (0.6133338210685878,0.2800004877352545) -- (0.6150004844680038,0.2816671511346705);
\draw[line width=2pt,color=ttqqqq] (0.6150004844680038,0.2816671511346705) -- (0.6166671478674198,0.2833338145340865);
\draw[line width=2pt,color=ttqqqq] (0.6166671478674198,0.2833338145340865) -- (0.6183338112668358,0.2850004779335025);
\draw[line width=2pt,color=ttqqqq] (0.6183338112668358,0.2850004779335025) -- (0.6200004746662519,0.28666714133291854);
\draw[line width=2pt,color=ttqqqq] (0.6200004746662519,0.28666714133291854) -- (0.6216671380656679,0.28833380473233455);
\draw[line width=2pt,color=ttqqqq] (0.6216671380656679,0.28833380473233455) -- (0.6233338014650839,0.29000046813175057);
\draw[line width=2pt,color=ttqqqq] (0.6233338014650839,0.29000046813175057) -- (0.6250004648644999,0.2916671315311666);
\draw[line width=2pt,color=ttqqqq] (0.6250004648644999,0.2916671315311666) -- (0.6266671282639159,0.2933337949305826);
\draw[line width=2pt,color=ttqqqq] (0.6266671282639159,0.2933337949305826) -- (0.6283337916633319,0.2950004583299986);
\draw[line width=2pt,color=ttqqqq] (0.6283337916633319,0.2950004583299986) -- (0.6300004550627479,0.2966671217294146);
\draw[line width=2pt,color=ttqqqq] (0.6300004550627479,0.2966671217294146) -- (0.631667118462164,0.29833378512883063);
\draw[line width=2pt,color=ttqqqq] (0.631667118462164,0.29833378512883063) -- (0.63333378186158,0.30000044852824664);
\draw[line width=2pt,color=ttqqqq] (0.63333378186158,0.30000044852824664) -- (0.635000445260996,0.30166711192766266);
\draw[line width=2pt,color=ttqqqq] (0.635000445260996,0.30166711192766266) -- (0.636667108660412,0.30333377532707867);
\draw[line width=2pt,color=ttqqqq] (0.636667108660412,0.30333377532707867) -- (0.638333772059828,0.3050004387264947);
\draw[line width=2pt,color=ttqqqq] (0.638333772059828,0.3050004387264947) -- (0.640000435459244,0.3066671021259107);
\draw[line width=2pt,color=ttqqqq] (0.640000435459244,0.3066671021259107) -- (0.64166709885866,0.3083337655253267);
\draw[line width=2pt,color=ttqqqq] (0.64166709885866,0.3083337655253267) -- (0.643333762258076,0.3100004289247427);
\draw[line width=2pt,color=ttqqqq] (0.643333762258076,0.3100004289247427) -- (0.645000425657492,0.31166709232415873);
\draw[line width=2pt,color=ttqqqq] (0.645000425657492,0.31166709232415873) -- (0.6466670890569081,0.31333375572357475);
\draw[line width=2pt,color=ttqqqq] (0.6466670890569081,0.31333375572357475) -- (0.6483337524563241,0.31500041912299076);
\draw[line width=2pt,color=ttqqqq] (0.6483337524563241,0.31500041912299076) -- (0.6500004158557401,0.3166670825224068);
\draw[line width=2pt,color=ttqqqq] (0.6500004158557401,0.3166670825224068) -- (0.6516670792551561,0.3183337459218228);
\draw[line width=2pt,color=ttqqqq] (0.6516670792551561,0.3183337459218228) -- (0.6533337426545721,0.3200004093212388);
\draw[line width=2pt,color=ttqqqq] (0.6533337426545721,0.3200004093212388) -- (0.6550004060539881,0.3216670727206548);
\draw[line width=2pt,color=ttqqqq] (0.6550004060539881,0.3216670727206548) -- (0.6566670694534041,0.3233337361200708);
\draw[line width=2pt,color=ttqqqq] (0.6566670694534041,0.3233337361200708) -- (0.6583337328528202,0.32500039951948684);
\draw[line width=2pt,color=ttqqqq] (0.6583337328528202,0.32500039951948684) -- (0.6600003962522362,0.32666706291890285);
\draw[line width=2pt,color=ttqqqq] (0.6600003962522362,0.32666706291890285) -- (0.6616670596516522,0.32833372631831886);
\draw[line width=2pt,color=ttqqqq] (0.6616670596516522,0.32833372631831886) -- (0.6633337230510682,0.3300003897177349);
\draw[line width=2pt,color=ttqqqq] (0.6633337230510682,0.3300003897177349) -- (0.6650003864504842,0.3316670531171509);
\draw[line width=2pt,color=ttqqqq] (0.6650003864504842,0.3316670531171509) -- (0.6666670498499002,0.3333337165165669);
\draw[line width=2pt,color=ttqqqq] (0.6666670498499002,0.3333337165165669) -- (0.6683337132493162,0.3350003799159829);
\draw[line width=2pt,color=ttqqqq] (0.6683337132493162,0.3350003799159829) -- (0.6700003766487322,0.33666704331539893);
\draw[line width=2pt,color=ttqqqq] (0.6700003766487322,0.33666704331539893) -- (0.6716670400481483,0.33833370671481494);
\draw[line width=2pt,color=ttqqqq] (0.6716670400481483,0.33833370671481494) -- (0.6733337034475643,0.34000037011423095);
\draw[line width=2pt,color=ttqqqq] (0.6733337034475643,0.34000037011423095) -- (0.6750003668469803,0.34166703351364697);
\draw[line width=2pt,color=ttqqqq] (0.6750003668469803,0.34166703351364697) -- (0.6766670302463963,0.343333696913063);
\draw[line width=2pt,color=ttqqqq] (0.6766670302463963,0.343333696913063) -- (0.6783336936458123,0.345000360312479);
\draw[line width=2pt,color=ttqqqq] (0.6783336936458123,0.345000360312479) -- (0.6800003570452283,0.346667023711895);
\draw[line width=2pt,color=ttqqqq] (0.6800003570452283,0.346667023711895) -- (0.6816670204446443,0.348333687111311);
\draw[line width=2pt,color=ttqqqq] (0.6816670204446443,0.348333687111311) -- (0.6833336838440603,0.35000035051072703);
\draw[line width=2pt,color=ttqqqq] (0.6833336838440603,0.35000035051072703) -- (0.6850003472434764,0.35166701391014304);
\draw[line width=2pt,color=ttqqqq] (0.6850003472434764,0.35166701391014304) -- (0.6866670106428924,0.35333367730955906);
\draw[line width=2pt,color=ttqqqq] (0.6866670106428924,0.35333367730955906) -- (0.6883336740423084,0.35500034070897507);
\draw[line width=2pt,color=ttqqqq] (0.6883336740423084,0.35500034070897507) -- (0.6900003374417244,0.3566670041083911);
\draw[line width=2pt,color=ttqqqq] (0.6900003374417244,0.3566670041083911) -- (0.6916670008411404,0.3583336675078071);
\draw[line width=2pt,color=ttqqqq] (0.6916670008411404,0.3583336675078071) -- (0.6933336642405564,0.3600003309072231);
\draw[line width=2pt,color=ttqqqq] (0.6933336642405564,0.3600003309072231) -- (0.6950003276399724,0.3616669943066391);
\draw[line width=2pt,color=ttqqqq] (0.6950003276399724,0.3616669943066391) -- (0.6966669910393885,0.36333365770605514);
\draw[line width=2pt,color=ttqqqq] (0.6966669910393885,0.36333365770605514) -- (0.6983336544388045,0.36500032110547115);
\draw[line width=2pt,color=ttqqqq] (0.6983336544388045,0.36500032110547115) -- (0.7000003178382205,0.36666698450488716);
\draw[line width=2pt,color=ttqqqq] (0.7000003178382205,0.36666698450488716) -- (0.7016669812376365,0.3683336479043032);
\draw[line width=2pt,color=ttqqqq] (0.7016669812376365,0.3683336479043032) -- (0.7033336446370525,0.3700003113037192);
\draw[line width=2pt,color=ttqqqq] (0.7033336446370525,0.3700003113037192) -- (0.7050003080364685,0.3716669747031352);
\draw[line width=2pt,color=ttqqqq] (0.7050003080364685,0.3716669747031352) -- (0.7066669714358845,0.3733336381025512);
\draw[line width=2pt,color=ttqqqq] (0.7066669714358845,0.3733336381025512) -- (0.7083336348353005,0.3750003015019672);
\draw[line width=2pt,color=ttqqqq] (0.7083336348353005,0.3750003015019672) -- (0.7100002982347166,0.37666696490138324);
\draw[line width=2pt,color=ttqqqq] (0.7100002982347166,0.37666696490138324) -- (0.7116669616341326,0.37833362830079925);
\draw[line width=2pt,color=ttqqqq] (0.7116669616341326,0.37833362830079925) -- (0.7133336250335486,0.38000029170021526);
\draw[line width=2pt,color=ttqqqq] (0.7133336250335486,0.38000029170021526) -- (0.7150002884329646,0.3816669550996313);
\draw[line width=2pt,color=ttqqqq] (0.7150002884329646,0.3816669550996313) -- (0.7166669518323806,0.3833336184990473);
\draw[line width=2pt,color=ttqqqq] (0.7166669518323806,0.3833336184990473) -- (0.7183336152317966,0.3850002818984633);
\draw[line width=2pt,color=ttqqqq] (0.7183336152317966,0.3850002818984633) -- (0.7200002786312126,0.3866669452978793);
\draw[line width=2pt,color=ttqqqq] (0.7200002786312126,0.3866669452978793) -- (0.7216669420306286,0.38833360869729533);
\draw[line width=2pt,color=ttqqqq] (0.7216669420306286,0.38833360869729533) -- (0.7233336054300447,0.39000027209671134);
\draw[line width=2pt,color=ttqqqq] (0.7233336054300447,0.39000027209671134) -- (0.7250002688294607,0.39166693549612736);
\draw[line width=2pt,color=ttqqqq] (0.7250002688294607,0.39166693549612736) -- (0.7266669322288767,0.39333359889554337);
\draw[line width=2pt,color=ttqqqq] (0.7266669322288767,0.39333359889554337) -- (0.7283335956282927,0.3950002622949594);
\draw[line width=2pt,color=ttqqqq] (0.7283335956282927,0.3950002622949594) -- (0.7300002590277087,0.3966669256943754);
\draw[line width=2pt,color=ttqqqq] (0.7300002590277087,0.3966669256943754) -- (0.7316669224271247,0.3983335890937914);
\draw[line width=2pt,color=ttqqqq] (0.7316669224271247,0.3983335890937914) -- (0.7333335858265407,0.4000002524932074);
\draw[line width=2pt,color=ttqqqq] (0.7333335858265407,0.4000002524932074) -- (0.7350002492259567,0.40166691589262343);
\draw[line width=2pt,color=ttqqqq] (0.7350002492259567,0.40166691589262343) -- (0.7366669126253728,0.40333357929203945);
\draw[line width=2pt,color=ttqqqq] (0.7366669126253728,0.40333357929203945) -- (0.7383335760247888,0.40500024269145546);
\draw[line width=2pt,color=ttqqqq] (0.7383335760247888,0.40500024269145546) -- (0.7400002394242048,0.40666690609087147);
\draw[line width=2pt,color=ttqqqq] (0.7400002394242048,0.40666690609087147) -- (0.7416669028236208,0.4083335694902875);
\draw[line width=2pt,color=ttqqqq] (0.7416669028236208,0.4083335694902875) -- (0.7433335662230368,0.4100002328897035);
\draw[line width=2pt,color=ttqqqq] (0.7433335662230368,0.4100002328897035) -- (0.7450002296224528,0.4116668962891195);
\draw[line width=2pt,color=ttqqqq] (0.7450002296224528,0.4116668962891195) -- (0.7466668930218688,0.4133335596885355);
\draw[line width=2pt,color=ttqqqq] (0.7466668930218688,0.4133335596885355) -- (0.7483335564212849,0.41500022308795154);
\draw[line width=2pt,color=ttqqqq] (0.7483335564212849,0.41500022308795154) -- (0.7500002198207009,0.41666688648736755);
\draw[line width=2pt,color=ttqqqq] (0.7500002198207009,0.41666688648736755) -- (0.7516668832201169,0.41833354988678356);
\draw[line width=2pt,color=ttqqqq] (0.7516668832201169,0.41833354988678356) -- (0.7533335466195329,0.4200002132861996);
\draw[line width=2pt,color=ttqqqq] (0.7533335466195329,0.4200002132861996) -- (0.7550002100189489,0.4216668766856156);
\draw[line width=2pt,color=ttqqqq] (0.7550002100189489,0.4216668766856156) -- (0.7566668734183649,0.4233335400850316);
\draw[line width=2pt,color=ttqqqq] (0.7566668734183649,0.4233335400850316) -- (0.7583335368177809,0.4250002034844476);
\draw[line width=2pt,color=ttqqqq] (0.7583335368177809,0.4250002034844476) -- (0.7600002002171969,0.4266668668838636);
\draw[line width=2pt,color=ttqqqq] (0.7600002002171969,0.4266668668838636) -- (0.761666863616613,0.42833353028327964);
\draw[line width=2pt,color=ttqqqq] (0.761666863616613,0.42833353028327964) -- (0.763333527016029,0.43000019368269565);
\draw[line width=2pt,color=ttqqqq] (0.763333527016029,0.43000019368269565) -- (0.765000190415445,0.43166685708211167);
\draw[line width=2pt,color=ttqqqq] (0.765000190415445,0.43166685708211167) -- (0.766666853814861,0.4333335204815277);
\draw[line width=2pt,color=ttqqqq] (0.766666853814861,0.4333335204815277) -- (0.768333517214277,0.4350001838809437);
\draw[line width=2pt,color=ttqqqq] (0.768333517214277,0.4350001838809437) -- (0.770000180613693,0.4366668472803597);
\draw[line width=2pt,color=ttqqqq] (0.770000180613693,0.4366668472803597) -- (0.771666844013109,0.4383335106797757);
\draw[line width=2pt,color=ttqqqq] (0.771666844013109,0.4383335106797757) -- (0.773333507412525,0.44000017407919173);
\draw[line width=2pt,color=ttqqqq] (0.773333507412525,0.44000017407919173) -- (0.7750001708119411,0.44166683747860774);
\draw[line width=2pt,color=ttqqqq] (0.7750001708119411,0.44166683747860774) -- (0.7766668342113571,0.44333350087802376);
\draw[line width=2pt,color=ttqqqq] (0.7766668342113571,0.44333350087802376) -- (0.7783334976107731,0.44500016427743977);
\draw[line width=2pt,color=ttqqqq] (0.7783334976107731,0.44500016427743977) -- (0.7800001610101891,0.4466668276768558);
\draw[line width=2pt,color=ttqqqq] (0.7800001610101891,0.4466668276768558) -- (0.7816668244096051,0.4483334910762718);
\draw[line width=2pt,color=ttqqqq] (0.7816668244096051,0.4483334910762718) -- (0.7833334878090211,0.4500001544756878);
\draw[line width=2pt,color=ttqqqq] (0.7833334878090211,0.4500001544756878) -- (0.7850001512084371,0.4516668178751038);
\draw[line width=2pt,color=ttqqqq] (0.7850001512084371,0.4516668178751038) -- (0.7866668146078531,0.45333348127451983);
\draw[line width=2pt,color=ttqqqq] (0.7866668146078531,0.45333348127451983) -- (0.7883334780072692,0.45500014467393585);
\draw[line width=2pt,color=ttqqqq] (0.7883334780072692,0.45500014467393585) -- (0.7900001414066852,0.45666680807335186);
\draw[line width=2pt,color=ttqqqq] (0.7900001414066852,0.45666680807335186) -- (0.7916668048061012,0.4583334714727679);
\draw[line width=2pt,color=ttqqqq] (0.7916668048061012,0.4583334714727679) -- (0.7933334682055172,0.4600001348721839);
\draw[line width=2pt,color=ttqqqq] (0.7933334682055172,0.4600001348721839) -- (0.7950001316049332,0.4616667982715999);
\draw[line width=2pt,color=ttqqqq] (0.7950001316049332,0.4616667982715999) -- (0.7966667950043492,0.4633334616710159);
\draw[line width=2pt,color=ttqqqq] (0.7966667950043492,0.4633334616710159) -- (0.7983334584037652,0.4650001250704319);
\draw[line width=2pt,color=ttqqqq] (0.7983334584037652,0.4650001250704319) -- (0.8000001218031813,0.46666678846984794);
\draw[line width=2pt,color=ttqqqq] (0.8000001218031813,0.46666678846984794) -- (0.8016667852025973,0.46833345186926395);
\draw[line width=2pt,color=ttqqqq] (0.8016667852025973,0.46833345186926395) -- (0.8033334486020133,0.47000011526867996);
\draw[line width=2pt,color=ttqqqq] (0.8033334486020133,0.47000011526867996) -- (0.8050001120014293,0.471666778668096);
\draw[line width=2pt,color=ttqqqq] (0.8050001120014293,0.471666778668096) -- (0.8066667754008453,0.473333442067512);
\draw[line width=2pt,color=ttqqqq] (0.8066667754008453,0.473333442067512) -- (0.8083334388002613,0.475000105466928);
\draw[line width=2pt,color=ttqqqq] (0.8083334388002613,0.475000105466928) -- (0.8100001021996773,0.476666768866344);
\draw[line width=2pt,color=ttqqqq] (0.8100001021996773,0.476666768866344) -- (0.8116667655990933,0.47833343226576003);
\draw[line width=2pt,color=ttqqqq] (0.8116667655990933,0.47833343226576003) -- (0.8133334289985094,0.48000009566517604);
\draw[line width=2pt,color=ttqqqq] (0.8133334289985094,0.48000009566517604) -- (0.8150000923979254,0.48166675906459205);
\draw[line width=2pt,color=ttqqqq] (0.8150000923979254,0.48166675906459205) -- (0.8166667557973414,0.48333342246400807);
\draw[line width=2pt,color=ttqqqq] (0.8166667557973414,0.48333342246400807) -- (0.8183334191967574,0.4850000858634241);
\draw[line width=2pt,color=ttqqqq] (0.8183334191967574,0.4850000858634241) -- (0.8200000825961734,0.4866667492628401);
\draw[line width=2pt,color=ttqqqq] (0.8200000825961734,0.4866667492628401) -- (0.8216667459955894,0.4883334126622561);
\draw[line width=2pt,color=ttqqqq] (0.8216667459955894,0.4883334126622561) -- (0.8233334093950054,0.4900000760616721);
\draw[line width=2pt,color=ttqqqq] (0.8233334093950054,0.4900000760616721) -- (0.8250000727944214,0.49166673946108813);
\draw[line width=2pt,color=ttqqqq] (0.8250000727944214,0.49166673946108813) -- (0.8266667361938375,0.49333340286050414);
\draw[line width=2pt,color=ttqqqq] (0.8266667361938375,0.49333340286050414) -- (0.8283333995932535,0.49500006625992016);
\draw[line width=2pt,color=ttqqqq] (0.8283333995932535,0.49500006625992016) -- (0.8300000629926695,0.49666672965933617);
\draw[line width=2pt,color=ttqqqq] (0.8300000629926695,0.49666672965933617) -- (0.8316667263920855,0.4983333930587522);
\draw[line width=2pt,color=ttqqqq] (0.8316667263920855,0.4983333930587522) -- (0.8333333897915015,0.5000000564581681);
\draw[line width=2pt,color=ttqqqq] (0.8333333897915015,0.5000000564581681) -- (0.8350000531909175,0.5016667198575842);
\draw[line width=2pt,color=ttqqqq] (0.8350000531909175,0.5016667198575842) -- (0.8366667165903335,0.5033333832570002);
\draw[line width=2pt,color=ttqqqq] (0.8366667165903335,0.5033333832570002) -- (0.8383333799897495,0.5050000466564162);
\draw[line width=2pt,color=ttqqqq] (0.8383333799897495,0.5050000466564162) -- (0.8400000433891656,0.5066667100558322);
\draw[line width=2pt,color=ttqqqq] (0.8400000433891656,0.5066667100558322) -- (0.8416667067885816,0.5083333734552482);
\draw[line width=2pt,color=ttqqqq] (0.8416667067885816,0.5083333734552482) -- (0.8433333701879976,0.5100000368546642);
\draw[line width=2pt,color=ttqqqq] (0.8433333701879976,0.5100000368546642) -- (0.8450000335874136,0.5116667002540802);
\draw[line width=2pt,color=ttqqqq] (0.8450000335874136,0.5116667002540802) -- (0.8466666969868296,0.5133333636534962);
\draw[line width=2pt,color=ttqqqq] (0.8466666969868296,0.5133333636534962) -- (0.8483333603862456,0.5150000270529123);
\draw[line width=2pt,color=ttqqqq] (0.8483333603862456,0.5150000270529123) -- (0.8500000237856616,0.5166666904523283);
\draw[line width=2pt,color=ttqqqq] (0.8500000237856616,0.5166666904523283) -- (0.8516666871850777,0.5183333538517443);
\draw[line width=2pt,color=ttqqqq] (0.8516666871850777,0.5183333538517443) -- (0.8533333505844937,0.5200000172511603);
\draw[line width=2pt,color=ttqqqq] (0.8533333505844937,0.5200000172511603) -- (0.8550000139839097,0.5216666806505763);
\draw[line width=2pt,color=ttqqqq] (0.8550000139839097,0.5216666806505763) -- (0.8566666773833257,0.5233333440499923);
\draw[line width=2pt,color=ttqqqq] (0.8566666773833257,0.5233333440499923) -- (0.8583333407827417,0.5250000074494083);
\draw[line width=2pt,color=ttqqqq] (0.8583333407827417,0.5250000074494083) -- (0.8600000041821577,0.5266666708488243);
\draw[line width=2pt,color=ttqqqq] (0.8600000041821577,0.5266666708488243) -- (0.8616666675815737,0.5283333342482404);
\draw[line width=2pt,color=ttqqqq] (0.8616666675815737,0.5283333342482404) -- (0.8633333309809897,0.5299999976476564);
\draw[line width=2pt,color=ttqqqq] (0.8633333309809897,0.5299999976476564) -- (0.8649999943804058,0.5316666610470724);
\draw[line width=2pt,color=ttqqqq] (0.8649999943804058,0.5316666610470724) -- (0.8666666577798218,0.5333333244464884);
\draw[line width=2pt,color=ttqqqq] (0.8666666577798218,0.5333333244464884) -- (0.8683333211792378,0.5349999878459044);
\draw[line width=2pt,color=ttqqqq] (0.8683333211792378,0.5349999878459044) -- (0.8699999845786538,0.5366666512453204);
\draw[line width=2pt,color=ttqqqq] (0.8699999845786538,0.5366666512453204) -- (0.8716666479780698,0.5383333146447364);
\draw[line width=2pt,color=ttqqqq] (0.8716666479780698,0.5383333146447364) -- (0.8733333113774858,0.5399999780441525);
\draw[line width=2pt,color=ttqqqq] (0.8733333113774858,0.5399999780441525) -- (0.8749999747769018,0.5416666414435685);
\draw[line width=2pt,color=ttqqqq] (0.8749999747769018,0.5416666414435685) -- (0.8766666381763178,0.5433333048429845);
\draw[line width=2pt,color=ttqqqq] (0.8766666381763178,0.5433333048429845) -- (0.8783333015757339,0.5449999682424005);
\draw[line width=2pt,color=ttqqqq] (0.8783333015757339,0.5449999682424005) -- (0.8799999649751499,0.5466666316418165);
\draw[line width=2pt,color=ttqqqq] (0.8799999649751499,0.5466666316418165) -- (0.8816666283745659,0.5483332950412325);
\draw[line width=2pt,color=ttqqqq] (0.8816666283745659,0.5483332950412325) -- (0.8833332917739819,0.5499999584406485);
\draw[line width=2pt,color=ttqqqq] (0.8833332917739819,0.5499999584406485) -- (0.8849999551733979,0.5516666218400645);
\draw[line width=2pt,color=ttqqqq] (0.8849999551733979,0.5516666218400645) -- (0.8866666185728139,0.5533332852394806);
\draw[line width=2pt,color=ttqqqq] (0.8866666185728139,0.5533332852394806) -- (0.8883332819722299,0.5549999486388966);
\draw[line width=2pt,color=ttqqqq] (0.8883332819722299,0.5549999486388966) -- (0.889999945371646,0.5566666120383126);
\draw[line width=2pt,color=ttqqqq] (0.889999945371646,0.5566666120383126) -- (0.891666608771062,0.5583332754377286);
\draw[line width=2pt,color=ttqqqq] (0.891666608771062,0.5583332754377286) -- (0.893333272170478,0.5599999388371446);
\draw[line width=2pt,color=ttqqqq] (0.893333272170478,0.5599999388371446) -- (0.894999935569894,0.5616666022365606);
\draw[line width=2pt,color=ttqqqq] (0.894999935569894,0.5616666022365606) -- (0.89666659896931,0.5633332656359766);
\draw[line width=2pt,color=ttqqqq] (0.89666659896931,0.5633332656359766) -- (0.898333262368726,0.5649999290353926);
\draw[line width=2pt,color=ttqqqq] (0.898333262368726,0.5649999290353926) -- (0.899999925768142,0.5666665924348087);
\draw[line width=2pt,color=ttqqqq] (0.899999925768142,0.5666665924348087) -- (0.901666589167558,0.5683332558342247);
\draw[line width=2pt,color=ttqqqq] (0.901666589167558,0.5683332558342247) -- (0.903333252566974,0.5699999192336407);
\draw[line width=2pt,color=ttqqqq] (0.903333252566974,0.5699999192336407) -- (0.9049999159663901,0.5716665826330567);
\draw[line width=2pt,color=ttqqqq] (0.9049999159663901,0.5716665826330567) -- (0.9066665793658061,0.5733332460324727);
\draw[line width=2pt,color=ttqqqq] (0.9066665793658061,0.5733332460324727) -- (0.9083332427652221,0.5749999094318887);
\draw[line width=2pt,color=ttqqqq] (0.9083332427652221,0.5749999094318887) -- (0.9099999061646381,0.5766665728313047);
\draw[line width=2pt,color=ttqqqq] (0.9099999061646381,0.5766665728313047) -- (0.9116665695640541,0.5783332362307207);
\draw[line width=2pt,color=ttqqqq] (0.9116665695640541,0.5783332362307207) -- (0.9133332329634701,0.5799998996301368);
\draw[line width=2pt,color=ttqqqq] (0.9133332329634701,0.5799998996301368) -- (0.9149998963628861,0.5816665630295528);
\draw[line width=2pt,color=ttqqqq] (0.9149998963628861,0.5816665630295528) -- (0.9166665597623022,0.5833332264289688);
\draw[line width=2pt,color=ttqqqq] (0.9166665597623022,0.5833332264289688) -- (0.9183332231617182,0.5849998898283848);
\draw[line width=2pt,color=ttqqqq] (0.9183332231617182,0.5849998898283848) -- (0.9199998865611342,0.5866665532278008);
\draw[line width=2pt,color=ttqqqq] (0.9199998865611342,0.5866665532278008) -- (0.9216665499605502,0.5883332166272168);
\draw[line width=2pt,color=ttqqqq] (0.9216665499605502,0.5883332166272168) -- (0.9233332133599662,0.5899998800266328);
\draw[line width=2pt,color=ttqqqq] (0.9233332133599662,0.5899998800266328) -- (0.9249998767593822,0.5916665434260489);
\draw[line width=2pt,color=ttqqqq] (0.9249998767593822,0.5916665434260489) -- (0.9266665401587982,0.5933332068254649);
\draw[line width=2pt,color=ttqqqq] (0.9266665401587982,0.5933332068254649) -- (0.9283332035582142,0.5949998702248809);
\draw[line width=2pt,color=ttqqqq] (0.9283332035582142,0.5949998702248809) -- (0.9299998669576303,0.5966665336242969);
\draw[line width=2pt,color=ttqqqq] (0.9299998669576303,0.5966665336242969) -- (0.9316665303570463,0.5983331970237129);
\draw[line width=2pt,color=ttqqqq] (0.9316665303570463,0.5983331970237129) -- (0.9333331937564623,0.5999998604231289);
\draw[line width=2pt,color=ttqqqq] (0.9333331937564623,0.5999998604231289) -- (0.9349998571558783,0.6016665238225449);
\draw[line width=2pt,color=ttqqqq] (0.9349998571558783,0.6016665238225449) -- (0.9366665205552943,0.6033331872219609);
\draw[line width=2pt,color=ttqqqq] (0.9366665205552943,0.6033331872219609) -- (0.9383331839547103,0.604999850621377);
\draw[line width=2pt,color=ttqqqq] (0.9383331839547103,0.604999850621377) -- (0.9399998473541263,0.606666514020793);
\draw[line width=2pt,color=ttqqqq] (0.9399998473541263,0.606666514020793) -- (0.9416665107535424,0.608333177420209);
\draw[line width=2pt,color=ttqqqq] (0.9416665107535424,0.608333177420209) -- (0.9433331741529584,0.609999840819625);
\draw[line width=2pt,color=ttqqqq] (0.9433331741529584,0.609999840819625) -- (0.9449998375523744,0.611666504219041);
\draw[line width=2pt,color=ttqqqq] (0.9449998375523744,0.611666504219041) -- (0.9466665009517904,0.613333167618457);
\draw[line width=2pt,color=ttqqqq] (0.9466665009517904,0.613333167618457) -- (0.9483331643512064,0.614999831017873);
\draw[line width=2pt,color=ttqqqq] (0.9483331643512064,0.614999831017873) -- (0.9499998277506224,0.616666494417289);
\draw[line width=2pt,color=ttqqqq] (0.9499998277506224,0.616666494417289) -- (0.9516664911500384,0.6183331578167051);
\draw[line width=2pt,color=ttqqqq] (0.9516664911500384,0.6183331578167051) -- (0.9533331545494544,0.6199998212161211);
\draw[line width=2pt,color=ttqqqq] (0.9533331545494544,0.6199998212161211) -- (0.9549998179488705,0.6216664846155371);
\draw[line width=2pt,color=ttqqqq] (0.9549998179488705,0.6216664846155371) -- (0.9566664813482865,0.6233331480149531);
\draw[line width=2pt,color=ttqqqq] (0.9566664813482865,0.6233331480149531) -- (0.9583331447477025,0.6249998114143691);
\draw[line width=2pt,color=ttqqqq] (0.9583331447477025,0.6249998114143691) -- (0.9599998081471185,0.6266664748137851);
\draw[line width=2pt,color=ttqqqq] (0.9599998081471185,0.6266664748137851) -- (0.9616664715465345,0.6283331382132011);
\draw[line width=2pt,color=ttqqqq] (0.9616664715465345,0.6283331382132011) -- (0.9633331349459505,0.6299998016126171);
\draw[line width=2pt,color=ttqqqq] (0.9633331349459505,0.6299998016126171) -- (0.9649997983453665,0.6316664650120332);
\draw[line width=2pt,color=ttqqqq] (0.9649997983453665,0.6316664650120332) -- (0.9666664617447825,0.6333331284114492);
\draw[line width=2pt,color=ttqqqq] (0.9666664617447825,0.6333331284114492) -- (0.9683331251441986,0.6349997918108652);
\draw[line width=2pt,color=ttqqqq] (0.9683331251441986,0.6349997918108652) -- (0.9699997885436146,0.6366664552102812);
\draw[line width=2pt,color=ttqqqq] (0.9699997885436146,0.6366664552102812) -- (0.9716664519430306,0.6383331186096972);
\draw[line width=2pt,color=ttqqqq] (0.9716664519430306,0.6383331186096972) -- (0.9733331153424466,0.6399997820091132);
\draw[line width=2pt,color=ttqqqq] (0.9733331153424466,0.6399997820091132) -- (0.9749997787418626,0.6416664454085292);
\draw[line width=2pt,color=ttqqqq] (0.9749997787418626,0.6416664454085292) -- (0.9766664421412786,0.6433331088079453);
\draw[line width=2pt,color=ttqqqq] (0.9766664421412786,0.6433331088079453) -- (0.9783331055406946,0.6449997722073613);
\draw[line width=2pt,color=ttqqqq] (0.9783331055406946,0.6449997722073613) -- (0.9799997689401106,0.6466664356067773);
\draw[line width=2pt,color=ttqqqq] (0.9799997689401106,0.6466664356067773) -- (0.9816664323395267,0.6483330990061933);
\draw[line width=2pt,color=ttqqqq] (0.9816664323395267,0.6483330990061933) -- (0.9833330957389427,0.6499997624056093);
\draw[line width=2pt,color=ttqqqq] (0.9833330957389427,0.6499997624056093) -- (0.9849997591383587,0.6516664258050253);
\draw[line width=2pt,color=ttqqqq] (0.9849997591383587,0.6516664258050253) -- (0.9866664225377747,0.6533330892044413);
\draw[line width=2pt,color=ttqqqq] (0.9866664225377747,0.6533330892044413) -- (0.9883330859371907,0.6549997526038573);
\draw[line width=2pt,color=ttqqqq] (0.9883330859371907,0.6549997526038573) -- (0.9899997493366067,0.6566664160032734);
\draw[line width=2pt,color=ttqqqq] (0.9899997493366067,0.6566664160032734) -- (0.9916664127360227,0.6583330794026894);
\draw[line width=2pt,color=ttqqqq] (0.9916664127360227,0.6583330794026894) -- (0.9933330761354388,0.6599997428021054);
\draw[line width=2pt,color=ttqqqq] (0.9933330761354388,0.6599997428021054) -- (0.9949997395348548,0.6616664062015214);
\draw[line width=2pt,color=ttqqqq] (0.9949997395348548,0.6616664062015214) -- (0.9966664029342708,0.6633330696009374);
\draw[line width=2pt,color=ttqqqq] (0.9966664029342708,0.6633330696009374) -- (0.9983330663336868,0.6649997330003534);
\draw[line width=2pt,dotted,color=ttqqqq] (8.43695452262867e-7,0.6666666666666667) -- (0,0.6666666666666667);
\draw[line width=2pt,dotted,color=ttqqqq] (0,0.6666666666666667) -- (0.0024999954544778747,0.6666666666666667);
\draw[line width=2pt,dotted,color=ttqqqq] (0.0024999954544778747,0.6666666666666667) -- (0.004999990908955749,0.6666666666666667);
\draw[line width=2pt,dotted,color=ttqqqq] (0.004999990908955749,0.6666666666666667) -- (0.007499986363433624,0.6666666666666667);
\draw[line width=2pt,dotted,color=ttqqqq] (0.007499986363433624,0.6666666666666667) -- (0.009999981817911499,0.6666666666666667);
\draw[line width=2pt,dotted,color=ttqqqq] (0.009999981817911499,0.6666666666666667) -- (0.012499977272389374,0.6666666666666667);
\draw[line width=2pt,dotted,color=ttqqqq] (0.012499977272389374,0.6666666666666667) -- (0.014999972726867249,0.6666666666666667);
\draw[line width=2pt,dotted,color=ttqqqq] (0.014999972726867249,0.6666666666666667) -- (0.017499968181345124,0.6666666666666667);
\draw[line width=2pt,dotted,color=ttqqqq] (0.017499968181345124,0.6666666666666667) -- (0.019999963635822997,0.6666666666666667);
\draw[line width=2pt,dotted,color=ttqqqq] (0.019999963635822997,0.6666666666666667) -- (0.02249995909030087,0.6666666666666667);
\draw[line width=2pt,dotted,color=ttqqqq] (0.02249995909030087,0.6666666666666667) -- (0.024999954544778744,0.6666666666666667);
\draw[line width=2pt,dotted,color=ttqqqq] (0.024999954544778744,0.6666666666666667) -- (0.027499949999256618,0.6666666666666667);
\draw[line width=2pt,dotted,color=ttqqqq] (0.027499949999256618,0.6666666666666667) -- (0.02999994545373449,0.6666666666666667);
\draw[line width=2pt,dotted,color=ttqqqq] (0.02999994545373449,0.6666666666666667) -- (0.032499940908212364,0.6666666666666667);
\draw[line width=2pt,dotted,color=ttqqqq] (0.032499940908212364,0.6666666666666667) -- (0.03499993636269024,0.6666666666666667);
\draw[line width=2pt,dotted,color=ttqqqq] (0.03499993636269024,0.6666666666666667) -- (0.03749993181716812,0.6666666666666667);
\draw[line width=2pt,dotted,color=ttqqqq] (0.03749993181716812,0.6666666666666667) -- (0.039999927271645995,0.6666666666666667);
\draw[line width=2pt,dotted,color=ttqqqq] (0.039999927271645995,0.6666666666666667) -- (0.04249992272612387,0.6666666666666667);
\draw[line width=2pt,dotted,color=ttqqqq] (0.04249992272612387,0.6666666666666667) -- (0.04499991818060175,0.6666666666666667);
\draw[line width=2pt,dotted,color=ttqqqq] (0.04499991818060175,0.6666666666666667) -- (0.047499913635079626,0.6666666666666667);
\draw[line width=2pt,dotted,color=ttqqqq] (0.047499913635079626,0.6666666666666667) -- (0.0499999090895575,0.6666666666666667);
\draw[line width=2pt,dotted,color=ttqqqq] (0.0499999090895575,0.6666666666666667) -- (0.05249990454403538,0.6666666666666667);
\draw[line width=2pt,dotted,color=ttqqqq] (0.05249990454403538,0.6666666666666667) -- (0.054999899998513256,0.6666666666666667);
\draw[line width=2pt,dotted,color=ttqqqq] (0.054999899998513256,0.6666666666666667) -- (0.05749989545299113,0.6666666666666667);
\draw[line width=2pt,dotted,color=ttqqqq] (0.05749989545299113,0.6666666666666667) -- (0.05999989090746901,0.6666666666666667);
\draw[line width=2pt,dotted,color=ttqqqq] (0.05999989090746901,0.6666666666666667) -- (0.06249988636194689,0.6666666666666667);
\draw[line width=2pt,dotted,color=ttqqqq] (0.06249988636194689,0.6666666666666667) -- (0.06499988181642476,0.6666666666666667);
\draw[line width=2pt,dotted,color=ttqqqq] (0.06499988181642476,0.6666666666666667) -- (0.06749987727090263,0.6666666666666667);
\draw[line width=2pt,dotted,color=ttqqqq] (0.06749987727090263,0.6666666666666667) -- (0.06999987272538051,0.6666666666666667);
\draw[line width=2pt,dotted,color=ttqqqq] (0.06999987272538051,0.6666666666666667) -- (0.07249986817985839,0.6666666666666667);
\draw[line width=2pt,dotted,color=ttqqqq] (0.07249986817985839,0.6666666666666667) -- (0.07499986363433626,0.6666666666666667);
\draw[line width=2pt,dotted,color=ttqqqq] (0.07499986363433626,0.6666666666666667) -- (0.07749985908881414,0.6666666666666667);
\draw[line width=2pt,dotted,color=ttqqqq] (0.07749985908881414,0.6666666666666667) -- (0.07999985454329202,0.6666666666666667);
\draw[line width=2pt,dotted,color=ttqqqq] (0.07999985454329202,0.6666666666666667) -- (0.0824998499977699,0.6666666666666667);
\draw[line width=2pt,dotted,color=ttqqqq] (0.0824998499977699,0.6666666666666667) -- (0.08499984545224777,0.6666666666666667);
\draw[line width=2pt,dotted,color=ttqqqq] (0.08499984545224777,0.6666666666666667) -- (0.08749984090672565,0.6666666666666667);
\draw[line width=2pt,dotted,color=ttqqqq] (0.08749984090672565,0.6666666666666667) -- (0.08999983636120353,0.6666666666666667);
\draw[line width=2pt,dotted,color=ttqqqq] (0.08999983636120353,0.6666666666666667) -- (0.0924998318156814,0.6666666666666667);
\draw[line width=2pt,dotted,color=ttqqqq] (0.0924998318156814,0.6666666666666667) -- (0.09499982727015928,0.6666666666666667);
\draw[line width=2pt,dotted,color=ttqqqq] (0.09499982727015928,0.6666666666666667) -- (0.09749982272463716,0.6666666666666667);
\draw[line width=2pt,dotted,color=ttqqqq] (0.09749982272463716,0.6666666666666667) -- (0.09999981817911503,0.6666666666666667);
\draw[line width=2pt,dotted,color=ttqqqq] (0.09999981817911503,0.6666666666666667) -- (0.10249981363359291,0.6666666666666667);
\draw[line width=2pt,dotted,color=ttqqqq] (0.10249981363359291,0.6666666666666667) -- (0.10499980908807079,0.6666666666666667);
\draw[line width=2pt,dotted,color=ttqqqq] (0.10499980908807079,0.6666666666666667) -- (0.10749980454254866,0.6666666666666667);
\draw[line width=2pt,dotted,color=ttqqqq] (0.10749980454254866,0.6666666666666667) -- (0.10999979999702654,0.6666666666666667);
\draw[line width=2pt,dotted,color=ttqqqq] (0.10999979999702654,0.6666666666666667) -- (0.11249979545150442,0.6666666666666667);
\draw[line width=2pt,dotted,color=ttqqqq] (0.11249979545150442,0.6666666666666667) -- (0.1149997909059823,0.6666666666666667);
\draw[line width=2pt,dotted,color=ttqqqq] (0.1149997909059823,0.6666666666666667) -- (0.11749978636046017,0.6666666666666667);
\draw[line width=2pt,dotted,color=ttqqqq] (0.11749978636046017,0.6666666666666667) -- (0.11999978181493805,0.6666666666666667);
\draw[line width=2pt,dotted,color=ttqqqq] (0.11999978181493805,0.6666666666666667) -- (0.12249977726941592,0.6666666666666667);
\draw[line width=2pt,dotted,color=ttqqqq] (0.12249977726941592,0.6666666666666667) -- (0.1249997727238938,0.6666666666666667);
\draw[line width=2pt,dotted,color=ttqqqq] (0.1249997727238938,0.6666666666666667) -- (0.12749976817837166,0.6666666666666667);
\draw[line width=2pt,dotted,color=ttqqqq] (0.12749976817837166,0.6666666666666667) -- (0.12999976363284954,0.6666666666666667);
\draw[line width=2pt,dotted,color=ttqqqq] (0.12999976363284954,0.6666666666666667) -- (0.13249975908732742,0.6666666666666667);
\draw[line width=2pt,dotted,color=ttqqqq] (0.13249975908732742,0.6666666666666667) -- (0.1349997545418053,0.6666666666666667);
\draw[line width=2pt,dotted,color=ttqqqq] (0.1349997545418053,0.6666666666666667) -- (0.13749974999628317,0.6666666666666667);
\draw[line width=2pt,dotted,color=ttqqqq] (0.13749974999628317,0.6666666666666667) -- (0.13999974545076105,0.6666666666666667);
\draw[line width=2pt,dotted,color=ttqqqq] (0.13999974545076105,0.6666666666666667) -- (0.14249974090523893,0.6666666666666667);
\draw[line width=2pt,dotted,color=ttqqqq] (0.14249974090523893,0.6666666666666667) -- (0.1449997363597168,0.6666666666666667);
\draw[line width=2pt,dotted,color=ttqqqq] (0.1449997363597168,0.6666666666666667) -- (0.14749973181419468,0.6666666666666667);
\draw[line width=2pt,dotted,color=ttqqqq] (0.14749973181419468,0.6666666666666667) -- (0.14999972726867256,0.6666666666666667);
\draw[line width=2pt,dotted,color=ttqqqq] (0.14999972726867256,0.6666666666666667) -- (0.15249972272315043,0.6666666666666667);
\draw[line width=2pt,dotted,color=ttqqqq] (0.15249972272315043,0.6666666666666667) -- (0.1549997181776283,0.6666666666666667);
\draw[line width=2pt,dotted,color=ttqqqq] (0.1549997181776283,0.6666666666666667) -- (0.1574997136321062,0.6666666666666667);
\draw[line width=2pt,dotted,color=ttqqqq] (0.1574997136321062,0.6666666666666667) -- (0.15999970908658406,0.6666666666666667);
\draw[line width=2pt,dotted,color=ttqqqq] (0.15999970908658406,0.6666666666666667) -- (0.16249970454106194,0.6666666666666667);
\draw[line width=2pt,dotted,color=ttqqqq] (0.16249970454106194,0.6666666666666667) -- (0.16499969999553982,0.6666666666666667);
\draw[line width=2pt,dotted,color=ttqqqq] (0.16499969999553982,0.6666666666666667) -- (0.1674996954500177,0.6666666666666667);
\draw[line width=2pt,dotted,color=ttqqqq] (0.1674996954500177,0.6666666666666667) -- (0.16999969090449557,0.6666666666666667);
\draw[line width=2pt,dotted,color=ttqqqq] (0.16999969090449557,0.6666666666666667) -- (0.17249968635897345,0.6666666666666667);
\draw[line width=2pt,dotted,color=ttqqqq] (0.17249968635897345,0.6666666666666667) -- (0.17499968181345132,0.6666666666666667);
\draw[line width=2pt,dotted,color=ttqqqq] (0.17499968181345132,0.6666666666666667) -- (0.1774996772679292,0.6666666666666667);
\draw[line width=2pt,dotted,color=ttqqqq] (0.1774996772679292,0.6666666666666667) -- (0.17999967272240708,0.6666666666666667);
\draw[line width=2pt,dotted,color=ttqqqq] (0.17999967272240708,0.6666666666666667) -- (0.18249966817688495,0.6666666666666667);
\draw[line width=2pt,dotted,color=ttqqqq] (0.18249966817688495,0.6666666666666667) -- (0.18499966363136283,0.6666666666666667);
\draw[line width=2pt,dotted,color=ttqqqq] (0.18499966363136283,0.6666666666666667) -- (0.1874996590858407,0.6666666666666667);
\draw[line width=2pt,dotted,color=ttqqqq] (0.1874996590858407,0.6666666666666667) -- (0.18999965454031859,0.6666666666666667);
\draw[line width=2pt,dotted,color=ttqqqq] (0.18999965454031859,0.6666666666666667) -- (0.19249964999479646,0.6666666666666667);
\draw[line width=2pt,dotted,color=ttqqqq] (0.19249964999479646,0.6666666666666667) -- (0.19499964544927434,0.6666666666666667);
\draw[line width=2pt,dotted,color=ttqqqq] (0.19499964544927434,0.6666666666666667) -- (0.19749964090375222,0.6666666666666667);
\draw[line width=2pt,dotted,color=ttqqqq] (0.19749964090375222,0.6666666666666667) -- (0.1999996363582301,0.6666666666666667);
\draw[line width=2pt,dotted,color=ttqqqq] (0.1999996363582301,0.6666666666666667) -- (0.20249963181270797,0.6666666666666667);
\draw[line width=2pt,dotted,color=ttqqqq] (0.20249963181270797,0.6666666666666667) -- (0.20499962726718585,0.6666666666666667);
\draw[line width=2pt,dotted,color=ttqqqq] (0.20499962726718585,0.6666666666666667) -- (0.20749962272166372,0.6666666666666667);
\draw[line width=2pt,dotted,color=ttqqqq] (0.20749962272166372,0.6666666666666667) -- (0.2099996181761416,0.6666666666666667);
\draw[line width=2pt,dotted,color=ttqqqq] (0.2099996181761416,0.6666666666666667) -- (0.21249961363061948,0.6666666666666667);
\draw[line width=2pt,dotted,color=ttqqqq] (0.21249961363061948,0.6666666666666667) -- (0.21499960908509735,0.6666666666666667);
\draw[line width=2pt,dotted,color=ttqqqq] (0.21499960908509735,0.6666666666666667) -- (0.21749960453957523,0.6666666666666667);
\draw[line width=2pt,dotted,color=ttqqqq] (0.21749960453957523,0.6666666666666667) -- (0.2199995999940531,0.6666666666666667);
\draw[line width=2pt,dotted,color=ttqqqq] (0.2199995999940531,0.6666666666666667) -- (0.22249959544853098,0.6666666666666667);
\draw[line width=2pt,dotted,color=ttqqqq] (0.22249959544853098,0.6666666666666667) -- (0.22499959090300886,0.6666666666666667);
\draw[line width=2pt,dotted,color=ttqqqq] (0.22499959090300886,0.6666666666666667) -- (0.22749958635748674,0.6666666666666667);
\draw[line width=2pt,dotted,color=ttqqqq] (0.22749958635748674,0.6666666666666667) -- (0.22999958181196462,0.6666666666666667);
\draw[line width=2pt,dotted,color=ttqqqq] (0.22999958181196462,0.6666666666666667) -- (0.2324995772664425,0.6666666666666667);
\draw[line width=2pt,dotted,color=ttqqqq] (0.2324995772664425,0.6666666666666667) -- (0.23499957272092037,0.6666666666666667);
\draw[line width=2pt,dotted,color=ttqqqq] (0.23499957272092037,0.6666666666666667) -- (0.23749956817539825,0.6666666666666667);
\draw[line width=2pt,dotted,color=ttqqqq] (0.23749956817539825,0.6666666666666667) -- (0.23999956362987612,0.6666666666666667);
\draw[line width=2pt,dotted,color=ttqqqq] (0.23999956362987612,0.6666666666666667) -- (0.242499559084354,0.6666666666666667);
\draw[line width=2pt,dotted,color=ttqqqq] (0.242499559084354,0.6666666666666667) -- (0.24499955453883188,0.6666666666666667);
\draw[line width=2pt,dotted,color=ttqqqq] (0.24499955453883188,0.6666666666666667) -- (0.24749954999330975,0.6666666666666667);
\draw[line width=2pt,dotted,color=ttqqqq] (0.24749954999330975,0.6666666666666667) -- (0.24999954544778763,0.6666666666666667);
\draw[line width=2pt,dotted,color=ttqqqq] (0.24999954544778763,0.6666666666666667) -- (0.2524995409022655,0.6666666666666667);
\draw[line width=2pt,dotted,color=ttqqqq] (0.2524995409022655,0.6666666666666667) -- (0.2549995363567434,0.6666666666666667);
\draw[line width=2pt,dotted,color=ttqqqq] (0.2549995363567434,0.6666666666666667) -- (0.25749953181122126,0.6666666666666667);
\draw[line width=2pt,dotted,color=ttqqqq] (0.25749953181122126,0.6666666666666667) -- (0.25999952726569914,0.6666666666666667);
\draw[line width=2pt,dotted,color=ttqqqq] (0.25999952726569914,0.6666666666666667) -- (0.262499522720177,0.6666666666666667);
\draw[line width=2pt,dotted,color=ttqqqq] (0.262499522720177,0.6666666666666667) -- (0.2649995181746549,0.6666666666666667);
\draw[line width=2pt,dotted,color=ttqqqq] (0.2649995181746549,0.6666666666666667) -- (0.26749951362913277,0.6666666666666667);
\draw[line width=2pt,dotted,color=ttqqqq] (0.26749951362913277,0.6666666666666667) -- (0.26999950908361064,0.6666666666666667);
\draw[line width=2pt,dotted,color=ttqqqq] (0.26999950908361064,0.6666666666666667) -- (0.2724995045380885,0.6666666666666667);
\draw[line width=2pt,dotted,color=ttqqqq] (0.2724995045380885,0.6666666666666667) -- (0.2749994999925664,0.6666666666666667);
\draw[line width=2pt,dotted,color=ttqqqq] (0.2749994999925664,0.6666666666666667) -- (0.2774994954470443,0.6666666666666667);
\draw[line width=2pt,dotted,color=ttqqqq] (0.2774994954470443,0.6666666666666667) -- (0.27999949090152215,0.6666666666666667);
\draw[line width=2pt,dotted,color=ttqqqq] (0.27999949090152215,0.6666666666666667) -- (0.28249948635600003,0.6666666666666667);
\draw[line width=2pt,dotted,color=ttqqqq] (0.28249948635600003,0.6666666666666667) -- (0.2849994818104779,0.6666666666666667);
\draw[line width=2pt,dotted,color=ttqqqq] (0.2849994818104779,0.6666666666666667) -- (0.2874994772649558,0.6666666666666667);
\draw[line width=2pt,dotted,color=ttqqqq] (0.2874994772649558,0.6666666666666667) -- (0.28999947271943366,0.6666666666666667);
\draw[line width=2pt,dotted,color=ttqqqq] (0.28999947271943366,0.6666666666666667) -- (0.29249946817391154,0.6666666666666667);
\draw[line width=2pt,dotted,color=ttqqqq] (0.29249946817391154,0.6666666666666667) -- (0.2949994636283894,0.6666666666666667);
\draw[line width=2pt,dotted,color=ttqqqq] (0.2949994636283894,0.6666666666666667) -- (0.2974994590828673,0.6666666666666667);
\draw[line width=2pt,dotted,color=ttqqqq] (0.2974994590828673,0.6666666666666667) -- (0.29999945453734517,0.6666666666666667);
\draw[line width=2pt,dotted,color=ttqqqq] (0.29999945453734517,0.6666666666666667) -- (0.30249944999182304,0.6666666666666667);
\draw[line width=2pt,dotted,color=ttqqqq] (0.30249944999182304,0.6666666666666667) -- (0.3049994454463009,0.6666666666666667);
\draw[line width=2pt,dotted,color=ttqqqq] (0.3049994454463009,0.6666666666666667) -- (0.3074994409007788,0.6666666666666667);
\draw[line width=2pt,dotted,color=ttqqqq] (0.3074994409007788,0.6666666666666667) -- (0.3099994363552567,0.6666666666666667);
\draw[line width=2pt,dotted,color=ttqqqq] (0.3099994363552567,0.6666666666666667) -- (0.31249943180973455,0.6666666666666667);
\draw[line width=2pt,dotted,color=ttqqqq] (0.31249943180973455,0.6666666666666667) -- (0.31499942726421243,0.6666666666666667);
\draw[line width=2pt,dotted,color=ttqqqq] (0.31499942726421243,0.6666666666666667) -- (0.3174994227186903,0.6666666666666667);
\draw[line width=2pt,dotted,color=ttqqqq] (0.3174994227186903,0.6666666666666667) -- (0.3199994181731682,0.6666666666666667);
\draw[line width=2pt,dotted,color=ttqqqq] (0.3199994181731682,0.6666666666666667) -- (0.32249941362764606,0.6666666666666667);
\draw[line width=2pt,dotted,color=ttqqqq] (0.32249941362764606,0.6666666666666667) -- (0.32499940908212394,0.6666666666666667);
\draw[line width=2pt,dotted,color=ttqqqq] (0.32499940908212394,0.6666666666666667) -- (0.3274994045366018,0.6666666666666667);
\draw[line width=2pt,dotted,color=ttqqqq] (0.3274994045366018,0.6666666666666667) -- (0.3299993999910797,0.6666666666666667);
\draw[line width=2pt,dotted,color=ttqqqq] (0.3299993999910797,0.6666666666666667) -- (0.33249939544555757,0.6666666666666667);
\draw[line width=2pt,dotted,color=ttqqqq] (0.33249939544555757,0.6666666666666667) -- (0.33499939090003544,0.6666666666666667);
\draw[line width=2pt,dotted,color=ttqqqq] (0.33499939090003544,0.6666666666666667) -- (0.3374993863545133,0.6666666666666667);
\draw[line width=2pt,dotted,color=ttqqqq] (0.3374993863545133,0.6666666666666667) -- (0.3399993818089912,0.6666666666666667);
\draw[line width=2pt,dotted,color=ttqqqq] (0.3399993818089912,0.6666666666666667) -- (0.3424993772634691,0.6666666666666667);
\draw[line width=2pt,dotted,color=ttqqqq] (0.3424993772634691,0.6666666666666667) -- (0.34499937271794695,0.6666666666666667);
\draw[line width=2pt,dotted,color=ttqqqq] (0.34499937271794695,0.6666666666666667) -- (0.3474993681724248,0.6666666666666667);
\draw[line width=2pt,dotted,color=ttqqqq] (0.3474993681724248,0.6666666666666667) -- (0.3499993636269027,0.6666666666666667);
\draw[line width=2pt,dotted,color=ttqqqq] (0.3499993636269027,0.6666666666666667) -- (0.3524993590813806,0.6666666666666667);
\draw[line width=2pt,dotted,color=ttqqqq] (0.3524993590813806,0.6666666666666667) -- (0.35499935453585846,0.6666666666666667);
\draw[line width=2pt,dotted,color=ttqqqq] (0.35499935453585846,0.6666666666666667) -- (0.35749934999033633,0.6666666666666667);
\draw[line width=2pt,dotted,color=ttqqqq] (0.35749934999033633,0.6666666666666667) -- (0.3599993454448142,0.6666666666666667);
\draw[line width=2pt,dotted,color=ttqqqq] (0.3599993454448142,0.6666666666666667) -- (0.3624993408992921,0.6666666666666667);
\draw[line width=2pt,dotted,color=ttqqqq] (0.3624993408992921,0.6666666666666667) -- (0.36499933635376997,0.6666666666666667);
\draw[line width=2pt,dotted,color=ttqqqq] (0.36499933635376997,0.6666666666666667) -- (0.36749933180824784,0.6666666666666667);
\draw[line width=2pt,dotted,color=ttqqqq] (0.36749933180824784,0.6666666666666667) -- (0.3699993272627257,0.6666666666666667);
\draw[line width=2pt,dotted,color=ttqqqq] (0.3699993272627257,0.6666666666666667) -- (0.3724993227172036,0.6666666666666667);
\draw[line width=2pt,dotted,color=ttqqqq] (0.3724993227172036,0.6666666666666667) -- (0.3749993181716815,0.6666666666666667);
\draw[line width=2pt,dotted,color=ttqqqq] (0.3749993181716815,0.6666666666666667) -- (0.37749931362615935,0.6666666666666667);
\draw[line width=2pt,dotted,color=ttqqqq] (0.37749931362615935,0.6666666666666667) -- (0.3799993090806372,0.6666666666666667);
\draw[line width=2pt,dotted,color=ttqqqq] (0.3799993090806372,0.6666666666666667) -- (0.3824993045351151,0.6666666666666667);
\draw[line width=2pt,dotted,color=ttqqqq] (0.3824993045351151,0.6666666666666667) -- (0.384999299989593,0.6666666666666667);
\draw[line width=2pt,dotted,color=ttqqqq] (0.384999299989593,0.6666666666666667) -- (0.38749929544407086,0.6666666666666667);
\draw[line width=2pt,dotted,color=ttqqqq] (0.38749929544407086,0.6666666666666667) -- (0.38999929089854873,0.6666666666666667);
\draw[line width=2pt,dotted,color=ttqqqq] (0.38999929089854873,0.6666666666666667) -- (0.3924992863530266,0.6666666666666667);
\draw[line width=2pt,dotted,color=ttqqqq] (0.3924992863530266,0.6666666666666667) -- (0.3949992818075045,0.6666666666666667);
\draw[line width=2pt,dotted,color=ttqqqq] (0.3949992818075045,0.6666666666666667) -- (0.39749927726198236,0.6666666666666667);
\draw[line width=2pt,dotted,color=ttqqqq] (0.39749927726198236,0.6666666666666667) -- (0.39999927271646024,0.6666666666666667);
\draw[line width=2pt,dotted,color=ttqqqq] (0.39999927271646024,0.6666666666666667) -- (0.4024992681709381,0.6666666666666667);
\draw[line width=2pt,dotted,color=ttqqqq] (0.4024992681709381,0.6666666666666667) -- (0.404999263625416,0.6666666666666667);
\draw[line width=2pt,dotted,color=ttqqqq] (0.404999263625416,0.6666666666666667) -- (0.40749925907989387,0.6666666666666667);
\draw[line width=2pt,dotted,color=ttqqqq] (0.40749925907989387,0.6666666666666667) -- (0.40999925453437175,0.6666666666666667);
\draw[line width=2pt,dotted,color=ttqqqq] (0.40999925453437175,0.6666666666666667) -- (0.4124992499888496,0.6666666666666667);
\draw[line width=2pt,dotted,color=ttqqqq] (0.4124992499888496,0.6666666666666667) -- (0.4149992454433275,0.6666666666666667);
\draw[line width=2pt,dotted,color=ttqqqq] (0.4149992454433275,0.6666666666666667) -- (0.4174992408978054,0.6666666666666667);
\draw[line width=2pt,dotted,color=ttqqqq] (0.4174992408978054,0.6666666666666667) -- (0.41999923635228326,0.6666666666666667);
\draw[line width=2pt,dotted,color=ttqqqq] (0.41999923635228326,0.6666666666666667) -- (0.42249923180676113,0.6666666666666667);
\draw[line width=2pt,dotted,color=ttqqqq] (0.42249923180676113,0.6666666666666667) -- (0.424999227261239,0.6666666666666667);
\draw[line width=2pt,dotted,color=ttqqqq] (0.424999227261239,0.6666666666666667) -- (0.4274992227157169,0.6666666666666667);
\draw[line width=2pt,dotted,color=ttqqqq] (0.4274992227157169,0.6666666666666667) -- (0.42999921817019476,0.6666666666666667);
\draw[line width=2pt,dotted,color=ttqqqq] (0.42999921817019476,0.6666666666666667) -- (0.43249921362467264,0.6666666666666667);
\draw[line width=2pt,dotted,color=ttqqqq] (0.43249921362467264,0.6666666666666667) -- (0.4349992090791505,0.6666666666666667);
\draw[line width=2pt,dotted,color=ttqqqq] (0.4349992090791505,0.6666666666666667) -- (0.4374992045336284,0.6666666666666667);
\draw[line width=2pt,dotted,color=ttqqqq] (0.4374992045336284,0.6666666666666667) -- (0.43999919998810627,0.6666666666666667);
\draw[line width=2pt,dotted,color=ttqqqq] (0.43999919998810627,0.6666666666666667) -- (0.44249919544258415,0.6666666666666667);
\draw[line width=2pt,dotted,color=ttqqqq] (0.44249919544258415,0.6666666666666667) -- (0.444999190897062,0.6666666666666667);
\draw[line width=2pt,dotted,color=ttqqqq] (0.444999190897062,0.6666666666666667) -- (0.4474991863515399,0.6666666666666667);
\draw[line width=2pt,dotted,color=ttqqqq] (0.4474991863515399,0.6666666666666667) -- (0.4499991818060178,0.6666666666666667);
\draw[line width=2pt,dotted,color=ttqqqq] (0.4499991818060178,0.6666666666666667) -- (0.45249917726049566,0.6666666666666667);
\draw[line width=2pt,dotted,color=ttqqqq] (0.45249917726049566,0.6666666666666667) -- (0.45499917271497353,0.6666666666666667);
\draw[line width=2pt,dotted,color=ttqqqq] (0.45499917271497353,0.6666666666666667) -- (0.4574991681694514,0.6666666666666667);
\draw[line width=2pt,dotted,color=ttqqqq] (0.4574991681694514,0.6666666666666667) -- (0.4599991636239293,0.6666666666666667);
\draw[line width=2pt,dotted,color=ttqqqq] (0.4599991636239293,0.6666666666666667) -- (0.46249915907840716,0.6666666666666667);
\draw[line width=2pt,dotted,color=ttqqqq] (0.46249915907840716,0.6666666666666667) -- (0.46499915453288504,0.6666666666666667);
\draw[line width=2pt,dotted,color=ttqqqq] (0.46499915453288504,0.6666666666666667) -- (0.4674991499873629,0.6666666666666667);
\draw[line width=2pt,dotted,color=ttqqqq] (0.4674991499873629,0.6666666666666667) -- (0.4699991454418408,0.6666666666666667);
\draw[line width=2pt,dotted,color=ttqqqq] (0.4699991454418408,0.6666666666666667) -- (0.47249914089631867,0.6666666666666667);
\draw[line width=2pt,dotted,color=ttqqqq] (0.47249914089631867,0.6666666666666667) -- (0.47499913635079655,0.6666666666666667);
\draw[line width=2pt,dotted,color=ttqqqq] (0.47499913635079655,0.6666666666666667) -- (0.4774991318052744,0.6666666666666667);
\draw[line width=2pt,dotted,color=ttqqqq] (0.4774991318052744,0.6666666666666667) -- (0.4799991272597523,0.6666666666666667);
\draw[line width=2pt,dotted,color=ttqqqq] (0.4799991272597523,0.6666666666666667) -- (0.4824991227142302,0.6666666666666667);
\draw[line width=2pt,dotted,color=ttqqqq] (0.4824991227142302,0.6666666666666667) -- (0.48499911816870805,0.6666666666666667);
\draw[line width=2pt,dotted,color=ttqqqq] (0.48499911816870805,0.6666666666666667) -- (0.48749911362318593,0.6666666666666667);
\draw[line width=2pt,dotted,color=ttqqqq] (0.48749911362318593,0.6666666666666667) -- (0.4899991090776638,0.6666666666666667);
\draw[line width=2pt,dotted,color=ttqqqq] (0.4899991090776638,0.6666666666666667) -- (0.4924991045321417,0.6666666666666667);
\draw[line width=2pt,dotted,color=ttqqqq] (0.4924991045321417,0.6666666666666667) -- (0.49499909998661956,0.6666666666666667);
\draw[line width=2pt,dotted,color=ttqqqq] (0.49499909998661956,0.6666666666666667) -- (0.49749909544109744,0.6666666666666667);
\draw[line width=2pt,dotted,color=ttqqqq] (0.49749909544109744,0.6666666666666667) -- (0.4999990908955753,0.6666666666666667);
\draw[line width=2pt,dotted,color=ttqqqq] (0.4999990908955753,0.6666666666666667) -- (0.5024990863500531,0.6666666666666667);
\draw[line width=2pt,dotted,color=ttqqqq] (0.5024990863500531,0.6666666666666667) -- (0.504999081804531,0.6666666666666667);
\draw[line width=2pt,dotted,color=ttqqqq] (0.504999081804531,0.6666666666666667) -- (0.5074990772590089,0.6666666666666667);
\draw[line width=2pt,dotted,color=ttqqqq] (0.5074990772590089,0.6666666666666667) -- (0.5099990727134868,0.6666666666666667);
\draw[line width=2pt,dotted,color=ttqqqq] (0.5099990727134868,0.6666666666666667) -- (0.5124990681679646,0.6666666666666667);
\draw[line width=2pt,dotted,color=ttqqqq] (0.5124990681679646,0.6666666666666667) -- (0.5149990636224425,0.6666666666666667);
\draw[line width=2pt,dotted,color=ttqqqq] (0.5149990636224425,0.6666666666666667) -- (0.5174990590769204,0.6666666666666667);
\draw[line width=2pt,dotted,color=ttqqqq] (0.5174990590769204,0.6666666666666667) -- (0.5199990545313983,0.6666666666666667);
\draw[line width=2pt,dotted,color=ttqqqq] (0.5199990545313983,0.6666666666666667) -- (0.5224990499858762,0.6666666666666667);
\draw[line width=2pt,dotted,color=ttqqqq] (0.5224990499858762,0.6666666666666667) -- (0.524999045440354,0.6666666666666667);
\draw[line width=2pt,dotted,color=ttqqqq] (0.524999045440354,0.6666666666666667) -- (0.5274990408948319,0.6666666666666667);
\draw[line width=2pt,dotted,color=ttqqqq] (0.5274990408948319,0.6666666666666667) -- (0.5299990363493098,0.6666666666666667);
\draw[line width=2pt,dotted,color=ttqqqq] (0.5299990363493098,0.6666666666666667) -- (0.5324990318037877,0.6666666666666667);
\draw[line width=2pt,dotted,color=ttqqqq] (0.5324990318037877,0.6666666666666667) -- (0.5349990272582655,0.6666666666666667);
\draw[line width=2pt,dotted,color=ttqqqq] (0.5349990272582655,0.6666666666666667) -- (0.5374990227127434,0.6666666666666667);
\draw[line width=2pt,dotted,color=ttqqqq] (0.5374990227127434,0.6666666666666667) -- (0.5399990181672213,0.6666666666666667);
\draw[line width=2pt,dotted,color=ttqqqq] (0.5399990181672213,0.6666666666666667) -- (0.5424990136216992,0.6666666666666667);
\draw[line width=2pt,dotted,color=ttqqqq] (0.5424990136216992,0.6666666666666667) -- (0.544999009076177,0.6666666666666667);
\draw[line width=2pt,dotted,color=ttqqqq] (0.544999009076177,0.6666666666666667) -- (0.5474990045306549,0.6666666666666667);
\draw[line width=2pt,dotted,color=ttqqqq] (0.5474990045306549,0.6666666666666667) -- (0.5499989999851328,0.6666666666666667);
\draw[line width=2pt,dotted,color=ttqqqq] (0.5499989999851328,0.6666666666666667) -- (0.5524989954396107,0.6666666666666667);
\draw[line width=2pt,dotted,color=ttqqqq] (0.5524989954396107,0.6666666666666667) -- (0.5549989908940886,0.6666666666666667);
\draw[line width=2pt,dotted,color=ttqqqq] (0.5549989908940886,0.6666666666666667) -- (0.5574989863485664,0.6666666666666667);
\draw[line width=2pt,dotted,color=ttqqqq] (0.5574989863485664,0.6666666666666667) -- (0.5599989818030443,0.6666666666666667);
\draw[line width=2pt,dotted,color=ttqqqq] (0.5599989818030443,0.6666666666666667) -- (0.5624989772575222,0.6666666666666667);
\draw[line width=2pt,dotted,color=ttqqqq] (0.5624989772575222,0.6666666666666667) -- (0.5649989727120001,0.6666666666666667);
\draw[line width=2pt,dotted,color=ttqqqq] (0.5649989727120001,0.6666666666666667) -- (0.5674989681664779,0.6666666666666667);
\draw[line width=2pt,dotted,color=ttqqqq] (0.5674989681664779,0.6666666666666667) -- (0.5699989636209558,0.6666666666666667);
\draw[line width=2pt,dotted,color=ttqqqq] (0.5699989636209558,0.6666666666666667) -- (0.5724989590754337,0.6666666666666667);
\draw[line width=2pt,dotted,color=ttqqqq] (0.5724989590754337,0.6666666666666667) -- (0.5749989545299116,0.6666666666666667);
\draw[line width=2pt,dotted,color=ttqqqq] (0.5749989545299116,0.6666666666666667) -- (0.5774989499843894,0.6666666666666667);
\draw[line width=2pt,dotted,color=ttqqqq] (0.5774989499843894,0.6666666666666667) -- (0.5799989454388673,0.6666666666666667);
\draw[line width=2pt,dotted,color=ttqqqq] (0.5799989454388673,0.6666666666666667) -- (0.5824989408933452,0.6666666666666667);
\draw[line width=2pt,dotted,color=ttqqqq] (0.5824989408933452,0.6666666666666667) -- (0.5849989363478231,0.6666666666666667);
\draw[line width=2pt,dotted,color=ttqqqq] (0.5849989363478231,0.6666666666666667) -- (0.587498931802301,0.6666666666666667);
\draw[line width=2pt,dotted,color=ttqqqq] (0.587498931802301,0.6666666666666667) -- (0.5899989272567788,0.6666666666666667);
\draw[line width=2pt,dotted,color=ttqqqq] (0.5899989272567788,0.6666666666666667) -- (0.5924989227112567,0.6666666666666667);
\draw[line width=2pt,dotted,color=ttqqqq] (0.5924989227112567,0.6666666666666667) -- (0.5949989181657346,0.6666666666666667);
\draw[line width=2pt,dotted,color=ttqqqq] (0.5949989181657346,0.6666666666666667) -- (0.5974989136202125,0.6666666666666667);
\draw[line width=2pt,dotted,color=ttqqqq] (0.5974989136202125,0.6666666666666667) -- (0.5999989090746903,0.6666666666666667);
\draw[line width=2pt,dotted,color=ttqqqq] (0.5999989090746903,0.6666666666666667) -- (0.6024989045291682,0.6666666666666667);
\draw[line width=2pt,dotted,color=ttqqqq] (0.6024989045291682,0.6666666666666667) -- (0.6049988999836461,0.6666666666666667);
\draw[line width=2pt,dotted,color=ttqqqq] (0.6049988999836461,0.6666666666666667) -- (0.607498895438124,0.6666666666666667);
\draw[line width=2pt,dotted,color=ttqqqq] (0.607498895438124,0.6666666666666667) -- (0.6099988908926018,0.6666666666666667);
\draw[line width=2pt,dotted,color=ttqqqq] (0.6099988908926018,0.6666666666666667) -- (0.6124988863470797,0.6666666666666667);
\draw[line width=2pt,dotted,color=ttqqqq] (0.6124988863470797,0.6666666666666667) -- (0.6149988818015576,0.6666666666666667);
\draw[line width=2pt,dotted,color=ttqqqq] (0.6149988818015576,0.6666666666666667) -- (0.6174988772560355,0.6666666666666667);
\draw[line width=2pt,dotted,color=ttqqqq] (0.6174988772560355,0.6666666666666667) -- (0.6199988727105133,0.6666666666666667);
\draw[line width=2pt,dotted,color=ttqqqq] (0.6199988727105133,0.6666666666666667) -- (0.6224988681649912,0.6666666666666667);
\draw[line width=2pt,dotted,color=ttqqqq] (0.6224988681649912,0.6666666666666667) -- (0.6249988636194691,0.6666666666666667);
\draw[line width=2pt,dotted,color=ttqqqq] (0.6249988636194691,0.6666666666666667) -- (0.627498859073947,0.6666666666666667);
\draw[line width=2pt,dotted,color=ttqqqq] (0.627498859073947,0.6666666666666667) -- (0.6299988545284249,0.6666666666666667);
\draw[line width=2pt,dotted,color=ttqqqq] (0.6299988545284249,0.6666666666666667) -- (0.6324988499829027,0.6666666666666667);
\draw[line width=2pt,dotted,color=ttqqqq] (0.6324988499829027,0.6666666666666667) -- (0.6349988454373806,0.6666666666666667);
\draw[line width=2pt,dotted,color=ttqqqq] (0.6349988454373806,0.6666666666666667) -- (0.6374988408918585,0.6666666666666667);
\draw[line width=2pt,dotted,color=ttqqqq] (0.6374988408918585,0.6666666666666667) -- (0.6399988363463364,0.6666666666666667);
\draw[line width=2pt,dotted,color=ttqqqq] (0.6399988363463364,0.6666666666666667) -- (0.6424988318008142,0.6666666666666667);
\draw[line width=2pt,dotted,color=ttqqqq] (0.6424988318008142,0.6666666666666667) -- (0.6449988272552921,0.6666666666666667);
\draw[line width=2pt,dotted,color=ttqqqq] (0.6449988272552921,0.6666666666666667) -- (0.64749882270977,0.6666666666666667);
\draw[line width=2pt,dotted,color=ttqqqq] (0.64749882270977,0.6666666666666667) -- (0.6499988181642479,0.6666666666666667);
\draw[line width=2pt,dotted,color=ttqqqq] (0.6499988181642479,0.6666666666666667) -- (0.6524988136187257,0.6666666666666667);
\draw[line width=2pt,dotted,color=ttqqqq] (0.6524988136187257,0.6666666666666667) -- (0.6549988090732036,0.6666666666666667);
\draw[line width=2pt,dotted,color=ttqqqq] (0.6549988090732036,0.6666666666666667) -- (0.6574988045276815,0.6666666666666667);
\draw[line width=2pt,dotted,color=ttqqqq] (0.6574988045276815,0.6666666666666667) -- (0.6599987999821594,0.6666666666666667);
\draw[line width=2pt,dotted,color=ttqqqq] (0.6599987999821594,0.6666666666666667) -- (0.6624987954366373,0.6666666666666667);
\draw[line width=2pt,dotted,color=ttqqqq] (0.6624987954366373,0.6666666666666667) -- (0.6649987908911151,0.6666666666666667);
\draw[line width=2pt,dotted,color=ttqqqq] (0.6649987908911151,0.6666666666666667) -- (0.667498786345593,0.6666666666666667);
\draw[line width=2pt,dotted,color=ttqqqq] (0.667498786345593,0.6666666666666667) -- (0.6699987818000709,0.6666666666666667);
\draw[line width=2pt,dotted,color=ttqqqq] (0.6699987818000709,0.6666666666666667) -- (0.6724987772545488,0.6666666666666667);
\draw[line width=2pt,dotted,color=ttqqqq] (0.6724987772545488,0.6666666666666667) -- (0.6749987727090266,0.6666666666666667);
\draw[line width=2pt,dotted,color=ttqqqq] (0.6749987727090266,0.6666666666666667) -- (0.6774987681635045,0.6666666666666667);
\draw[line width=2pt,dotted,color=ttqqqq] (0.6774987681635045,0.6666666666666667) -- (0.6799987636179824,0.6666666666666667);
\draw[line width=2pt,dotted,color=ttqqqq] (0.6799987636179824,0.6666666666666667) -- (0.6824987590724603,0.6666666666666667);
\draw[line width=2pt,dotted,color=ttqqqq] (0.6824987590724603,0.6666666666666667) -- (0.6849987545269381,0.6666666666666667);
\draw[line width=2pt,dotted,color=ttqqqq] (0.6849987545269381,0.6666666666666667) -- (0.687498749981416,0.6666666666666667);
\draw[line width=2pt,dotted,color=ttqqqq] (0.687498749981416,0.6666666666666667) -- (0.6899987454358939,0.6666666666666667);
\draw[line width=2pt,dotted,color=ttqqqq] (0.6899987454358939,0.6666666666666667) -- (0.6924987408903718,0.6666666666666667);
\draw[line width=2pt,dotted,color=ttqqqq] (0.6924987408903718,0.6666666666666667) -- (0.6949987363448497,0.6666666666666667);
\draw[line width=2pt,dotted,color=ttqqqq] (0.6949987363448497,0.6666666666666667) -- (0.6974987317993275,0.6666666666666667);
\draw[line width=2pt,dotted,color=ttqqqq] (0.6974987317993275,0.6666666666666667) -- (0.6999987272538054,0.6666666666666667);
\draw[line width=2pt,dotted,color=ttqqqq] (0.6999987272538054,0.6666666666666667) -- (0.7024987227082833,0.6666666666666667);
\draw[line width=2pt,dotted,color=ttqqqq] (0.7024987227082833,0.6666666666666667) -- (0.7049987181627612,0.6666666666666667);
\draw[line width=2pt,dotted,color=ttqqqq] (0.7049987181627612,0.6666666666666667) -- (0.707498713617239,0.6666666666666667);
\draw[line width=2pt,dotted,color=ttqqqq] (0.707498713617239,0.6666666666666667) -- (0.7099987090717169,0.6666666666666667);
\draw[line width=2pt,dotted,color=ttqqqq] (0.7099987090717169,0.6666666666666667) -- (0.7124987045261948,0.6666666666666667);
\draw[line width=2pt,dotted,color=ttqqqq] (0.7124987045261948,0.6666666666666667) -- (0.7149986999806727,0.6666666666666667);
\draw[line width=2pt,dotted,color=ttqqqq] (0.7149986999806727,0.6666666666666667) -- (0.7174986954351505,0.6666666666666667);
\draw[line width=2pt,dotted,color=ttqqqq] (0.7174986954351505,0.6666666666666667) -- (0.7199986908896284,0.6666666666666667);
\draw[line width=2pt,dotted,color=ttqqqq] (0.7199986908896284,0.6666666666666667) -- (0.7224986863441063,0.6666666666666667);
\draw[line width=2pt,dotted,color=ttqqqq] (0.7224986863441063,0.6666666666666667) -- (0.7249986817985842,0.6666666666666667);
\draw[line width=2pt,dotted,color=ttqqqq] (0.7249986817985842,0.6666666666666667) -- (0.727498677253062,0.6666666666666667);
\draw[line width=2pt,dotted,color=ttqqqq] (0.727498677253062,0.6666666666666667) -- (0.7299986727075399,0.6666666666666667);
\draw[line width=2pt,dotted,color=ttqqqq] (0.7299986727075399,0.6666666666666667) -- (0.7324986681620178,0.6666666666666667);
\draw[line width=2pt,dotted,color=ttqqqq] (0.7324986681620178,0.6666666666666667) -- (0.7349986636164957,0.6666666666666667);
\draw[line width=2pt,dotted,color=ttqqqq] (0.7349986636164957,0.6666666666666667) -- (0.7374986590709736,0.6666666666666667);
\draw[line width=2pt,dotted,color=ttqqqq] (0.7374986590709736,0.6666666666666667) -- (0.7399986545254514,0.6666666666666667);
\draw[line width=2pt,dotted,color=ttqqqq] (0.7399986545254514,0.6666666666666667) -- (0.7424986499799293,0.6666666666666667);
\draw[line width=2pt,dotted,color=ttqqqq] (0.7424986499799293,0.6666666666666667) -- (0.7449986454344072,0.6666666666666667);
\draw[line width=2pt,dotted,color=ttqqqq] (0.7449986454344072,0.6666666666666667) -- (0.7474986408888851,0.6666666666666667);
\draw[line width=2pt,dotted,color=ttqqqq] (0.7474986408888851,0.6666666666666667) -- (0.749998636343363,0.6666666666666667);
\draw[line width=2pt,dotted,color=ttqqqq] (0.749998636343363,0.6666666666666667) -- (0.7524986317978408,0.6666666666666667);
\draw[line width=2pt,dotted,color=ttqqqq] (0.7524986317978408,0.6666666666666667) -- (0.7549986272523187,0.6666666666666667);
\draw[line width=2pt,dotted,color=ttqqqq] (0.7549986272523187,0.6666666666666667) -- (0.7574986227067966,0.6666666666666667);
\draw[line width=2pt,dotted,color=ttqqqq] (0.7574986227067966,0.6666666666666667) -- (0.7599986181612745,0.6666666666666667);
\draw[line width=2pt,dotted,color=ttqqqq] (0.7599986181612745,0.6666666666666667) -- (0.7624986136157523,0.6666666666666667);
\draw[line width=2pt,dotted,color=ttqqqq] (0.7624986136157523,0.6666666666666667) -- (0.7649986090702302,0.6666666666666667);
\draw[line width=2pt,dotted,color=ttqqqq] (0.7649986090702302,0.6666666666666667) -- (0.7674986045247081,0.6666666666666667);
\draw[line width=2pt,dotted,color=ttqqqq] (0.7674986045247081,0.6666666666666667) -- (0.769998599979186,0.6666666666666667);
\draw[line width=2pt,dotted,color=ttqqqq] (0.769998599979186,0.6666666666666667) -- (0.7724985954336638,0.6666666666666667);
\draw[line width=2pt,dotted,color=ttqqqq] (0.7724985954336638,0.6666666666666667) -- (0.7749985908881417,0.6666666666666667);
\draw[line width=2pt,dotted,color=ttqqqq] (0.7749985908881417,0.6666666666666667) -- (0.7774985863426196,0.6666666666666667);
\draw[line width=2pt,dotted,color=ttqqqq] (0.7774985863426196,0.6666666666666667) -- (0.7799985817970975,0.6666666666666667);
\draw[line width=2pt,dotted,color=ttqqqq] (0.7799985817970975,0.6666666666666667) -- (0.7824985772515753,0.6666666666666667);
\draw[line width=2pt,dotted,color=ttqqqq] (0.7824985772515753,0.6666666666666667) -- (0.7849985727060532,0.6666666666666667);
\draw[line width=2pt,dotted,color=ttqqqq] (0.7849985727060532,0.6666666666666667) -- (0.7874985681605311,0.6666666666666667);
\draw[line width=2pt,dotted,color=ttqqqq] (0.7874985681605311,0.6666666666666667) -- (0.789998563615009,0.6666666666666667);
\draw[line width=2pt,dotted,color=ttqqqq] (0.789998563615009,0.6666666666666667) -- (0.7924985590694869,0.6666666666666667);
\draw[line width=2pt,dotted,color=ttqqqq] (0.7924985590694869,0.6666666666666667) -- (0.7949985545239647,0.6666666666666667);
\draw[line width=2pt,dotted,color=ttqqqq] (0.7949985545239647,0.6666666666666667) -- (0.7974985499784426,0.6666666666666667);
\draw[line width=2pt,dotted,color=ttqqqq] (0.7974985499784426,0.6666666666666667) -- (0.7999985454329205,0.6666666666666667);
\draw[line width=2pt,dotted,color=ttqqqq] (0.7999985454329205,0.6666666666666667) -- (0.8024985408873984,0.6666666666666667);
\draw[line width=2pt,dotted,color=ttqqqq] (0.8024985408873984,0.6666666666666667) -- (0.8049985363418762,0.6666666666666667);
\draw[line width=2pt,dotted,color=ttqqqq] (0.8049985363418762,0.6666666666666667) -- (0.8074985317963541,0.6666666666666667);
\draw[line width=2pt,dotted,color=ttqqqq] (0.8074985317963541,0.6666666666666667) -- (0.809998527250832,0.6666666666666667);
\draw[line width=2pt,dotted,color=ttqqqq] (0.809998527250832,0.6666666666666667) -- (0.8124985227053099,0.6666666666666667);
\draw[line width=2pt,dotted,color=ttqqqq] (0.8124985227053099,0.6666666666666667) -- (0.8149985181597877,0.6666666666666667);
\draw[line width=2pt,dotted,color=ttqqqq] (0.8149985181597877,0.6666666666666667) -- (0.8174985136142656,0.6666666666666667);
\draw[line width=2pt,dotted,color=ttqqqq] (0.8174985136142656,0.6666666666666667) -- (0.8199985090687435,0.6666666666666667);
\draw[line width=2pt,dotted,color=ttqqqq] (0.8199985090687435,0.6666666666666667) -- (0.8224985045232214,0.6666666666666667);
\draw[line width=2pt,dotted,color=ttqqqq] (0.8224985045232214,0.6666666666666667) -- (0.8249984999776993,0.6666666666666667);
\draw[line width=2pt,dotted,color=ttqqqq] (0.8249984999776993,0.6666666666666667) -- (0.8274984954321771,0.6666666666666667);
\draw[line width=2pt,dotted,color=ttqqqq] (0.8274984954321771,0.6666666666666667) -- (0.829998490886655,0.6666666666666667);
\draw[line width=2pt,dotted,color=ttqqqq] (0.829998490886655,0.6666666666666667) -- (0.8324984863411329,0.6666666666666667);
\draw[line width=2pt,dotted,color=ttqqqq] (0.8324984863411329,0.6666666666666667) -- (0.8349984817956108,0.6666666666666667);
\draw[line width=2pt,dotted,color=ttqqqq] (0.8349984817956108,0.6666666666666667) -- (0.8374984772500886,0.6666666666666667);
\draw[line width=2pt,dotted,color=ttqqqq] (0.8374984772500886,0.6666666666666667) -- (0.8399984727045665,0.6666666666666667);
\draw[line width=2pt,dotted,color=ttqqqq] (0.8399984727045665,0.6666666666666667) -- (0.8424984681590444,0.6666666666666667);
\draw[line width=2pt,dotted,color=ttqqqq] (0.8424984681590444,0.6666666666666667) -- (0.8449984636135223,0.6666666666666667);
\draw[line width=2pt,dotted,color=ttqqqq] (0.8449984636135223,0.6666666666666667) -- (0.8474984590680001,0.6666666666666667);
\draw[line width=2pt,dotted,color=ttqqqq] (0.8474984590680001,0.6666666666666667) -- (0.849998454522478,0.6666666666666667);
\draw[line width=2pt,dotted,color=ttqqqq] (0.849998454522478,0.6666666666666667) -- (0.8524984499769559,0.6666666666666667);
\draw[line width=2pt,dotted,color=ttqqqq] (0.8524984499769559,0.6666666666666667) -- (0.8549984454314338,0.6666666666666667);
\draw[line width=2pt,dotted,color=ttqqqq] (0.8549984454314338,0.6666666666666667) -- (0.8574984408859117,0.6666666666666667);
\draw[line width=2pt,dotted,color=ttqqqq] (0.8574984408859117,0.6666666666666667) -- (0.8599984363403895,0.6666666666666667);
\draw[line width=2pt,dotted,color=ttqqqq] (0.8599984363403895,0.6666666666666667) -- (0.8624984317948674,0.6666666666666667);
\draw[line width=2pt,dotted,color=ttqqqq] (0.8624984317948674,0.6666666666666667) -- (0.8649984272493453,0.6666666666666667);
\draw[line width=2pt,dotted,color=ttqqqq] (0.8649984272493453,0.6666666666666667) -- (0.8674984227038232,0.6666666666666667);
\draw[line width=2pt,dotted,color=ttqqqq] (0.8674984227038232,0.6666666666666667) -- (0.869998418158301,0.6666666666666667);
\draw[line width=2pt,dotted,color=ttqqqq] (0.869998418158301,0.6666666666666667) -- (0.8724984136127789,0.6666666666666667);
\draw[line width=2pt,dotted,color=ttqqqq] (0.8724984136127789,0.6666666666666667) -- (0.8749984090672568,0.6666666666666667);
\draw[line width=2pt,dotted,color=ttqqqq] (0.8749984090672568,0.6666666666666667) -- (0.8774984045217347,0.6666666666666667);
\draw[line width=2pt,dotted,color=ttqqqq] (0.8774984045217347,0.6666666666666667) -- (0.8799983999762125,0.6666666666666667);
\draw[line width=2pt,dotted,color=ttqqqq] (0.8799983999762125,0.6666666666666667) -- (0.8824983954306904,0.6666666666666667);
\draw[line width=2pt,dotted,color=ttqqqq] (0.8824983954306904,0.6666666666666667) -- (0.8849983908851683,0.6666666666666667);
\draw[line width=2pt,dotted,color=ttqqqq] (0.8849983908851683,0.6666666666666667) -- (0.8874983863396462,0.6666666666666667);
\draw[line width=2pt,dotted,color=ttqqqq] (0.8874983863396462,0.6666666666666667) -- (0.889998381794124,0.6666666666666667);
\draw[line width=2pt,dotted,color=ttqqqq] (0.889998381794124,0.6666666666666667) -- (0.8924983772486019,0.6666666666666667);
\draw[line width=2pt,dotted,color=ttqqqq] (0.8924983772486019,0.6666666666666667) -- (0.8949983727030798,0.6666666666666667);
\draw[line width=2pt,dotted,color=ttqqqq] (0.8949983727030798,0.6666666666666667) -- (0.8974983681575577,0.6666666666666667);
\draw[line width=2pt,dotted,color=ttqqqq] (0.8974983681575577,0.6666666666666667) -- (0.8999983636120356,0.6666666666666667);
\draw[line width=2pt,dotted,color=ttqqqq] (0.8999983636120356,0.6666666666666667) -- (0.9024983590665134,0.6666666666666667);
\draw[line width=2pt,dotted,color=ttqqqq] (0.9024983590665134,0.6666666666666667) -- (0.9049983545209913,0.6666666666666667);
\draw[line width=2pt,dotted,color=ttqqqq] (0.9049983545209913,0.6666666666666667) -- (0.9074983499754692,0.6666666666666667);
\draw[line width=2pt,dotted,color=ttqqqq] (0.9074983499754692,0.6666666666666667) -- (0.9099983454299471,0.6666666666666667);
\draw[line width=2pt,dotted,color=ttqqqq] (0.9099983454299471,0.6666666666666667) -- (0.9124983408844249,0.6666666666666667);
\draw[line width=2pt,dotted,color=ttqqqq] (0.9124983408844249,0.6666666666666667) -- (0.9149983363389028,0.6666666666666667);
\draw[line width=2pt,dotted,color=ttqqqq] (0.9149983363389028,0.6666666666666667) -- (0.9174983317933807,0.6666666666666667);
\draw[line width=2pt,dotted,color=ttqqqq] (0.9174983317933807,0.6666666666666667) -- (0.9199983272478586,0.6666666666666667);
\draw[line width=2pt,dotted,color=ttqqqq] (0.9199983272478586,0.6666666666666667) -- (0.9224983227023364,0.6666666666666667);
\draw[line width=2pt,dotted,color=ttqqqq] (0.9224983227023364,0.6666666666666667) -- (0.9249983181568143,0.6666666666666667);
\draw[line width=2pt,dotted,color=ttqqqq] (0.9249983181568143,0.6666666666666667) -- (0.9274983136112922,0.6666666666666667);
\draw[line width=2pt,dotted,color=ttqqqq] (0.9274983136112922,0.6666666666666667) -- (0.9299983090657701,0.6666666666666667);
\draw[line width=2pt,dotted,color=ttqqqq] (0.9299983090657701,0.6666666666666667) -- (0.932498304520248,0.6666666666666667);
\draw[line width=2pt,dotted,color=ttqqqq] (0.932498304520248,0.6666666666666667) -- (0.9349982999747258,0.6666666666666667);
\draw[line width=2pt,dotted,color=ttqqqq] (0.9349982999747258,0.6666666666666667) -- (0.9374982954292037,0.6666666666666667);
\draw[line width=2pt,dotted,color=ttqqqq] (0.9374982954292037,0.6666666666666667) -- (0.9399982908836816,0.6666666666666667);
\draw[line width=2pt,dotted,color=ttqqqq] (0.9399982908836816,0.6666666666666667) -- (0.9424982863381595,0.6666666666666667);
\draw[line width=2pt,dotted,color=ttqqqq] (0.9424982863381595,0.6666666666666667) -- (0.9449982817926373,0.6666666666666667);
\draw[line width=2pt,dotted,color=ttqqqq] (0.9449982817926373,0.6666666666666667) -- (0.9474982772471152,0.6666666666666667);
\draw[line width=2pt,dotted,color=ttqqqq] (0.9474982772471152,0.6666666666666667) -- (0.9499982727015931,0.6666666666666667);
\draw[line width=2pt,dotted,color=ttqqqq] (0.9499982727015931,0.6666666666666667) -- (0.952498268156071,0.6666666666666667);
\draw[line width=2pt,dotted,color=ttqqqq] (0.952498268156071,0.6666666666666667) -- (0.9549982636105488,0.6666666666666667);
\draw[line width=2pt,dotted,color=ttqqqq] (0.9549982636105488,0.6666666666666667) -- (0.9574982590650267,0.6666666666666667);
\draw[line width=2pt,dotted,color=ttqqqq] (0.9574982590650267,0.6666666666666667) -- (0.9599982545195046,0.6666666666666667);
\draw[line width=2pt,dotted,color=ttqqqq] (0.9599982545195046,0.6666666666666667) -- (0.9624982499739825,0.6666666666666667);
\draw[line width=2pt,dotted,color=ttqqqq] (0.9624982499739825,0.6666666666666667) -- (0.9649982454284604,0.6666666666666667);
\draw[line width=2pt,dotted,color=ttqqqq] (0.9649982454284604,0.6666666666666667) -- (0.9674982408829382,0.6666666666666667);
\draw[line width=2pt,dotted,color=ttqqqq] (0.9674982408829382,0.6666666666666667) -- (0.9699982363374161,0.6666666666666667);
\draw[line width=2pt,dotted,color=ttqqqq] (0.9699982363374161,0.6666666666666667) -- (0.972498231791894,0.6666666666666667);
\draw[line width=2pt,dotted,color=ttqqqq] (0.972498231791894,0.6666666666666667) -- (0.9749982272463719,0.6666666666666667);
\draw[line width=2pt,dotted,color=ttqqqq] (0.9749982272463719,0.6666666666666667) -- (0.9774982227008497,0.6666666666666667);
\draw[line width=2pt,dotted,color=ttqqqq] (0.9774982227008497,0.6666666666666667) -- (0.9799982181553276,0.6666666666666667);
\draw[line width=2pt,dotted,color=ttqqqq] (0.9799982181553276,0.6666666666666667) -- (0.9824982136098055,0.6666666666666667);
\draw[line width=2pt,dotted,color=ttqqqq] (0.9824982136098055,0.6666666666666667) -- (0.9849982090642834,0.6666666666666667);
\draw[line width=2pt,dotted,color=ttqqqq] (0.9849982090642834,0.6666666666666667) -- (0.9874982045187612,0.6666666666666667);
\draw[line width=2pt,dotted,color=ttqqqq] (0.9874982045187612,0.6666666666666667) -- (0.9899981999732391,0.6666666666666667);
\draw[line width=2pt,dotted,color=ttqqqq] (0.9899981999732391,0.6666666666666667) -- (0.992498195427717,0.6666666666666667);
\draw[line width=2pt,dotted,color=ttqqqq] (0.992498195427717,0.6666666666666667) -- (0.9949981908821949,0.6666666666666667);
\draw[line width=2pt,dotted,color=ttqqqq] (0.9949981908821949,0.6666666666666667) -- (0.9974981863366728,0.6666666666666667);
\draw[line width=2pt,dotted,color=ttqqqq] (0.9974981863366728,0.6666666666666667) -- (0.9999981817911506,0.6666666666666667);
\draw [line width=2pt,color=zzttqq] (0,0)-- (0,1);
\draw [line width=2pt,color=zzttqq] (0,1)-- (1,1);
\draw [line width=2pt,color=zzttqq] (1,1)-- (1,0);
\draw [line width=2pt,color=zzttqq] (1,0)-- (0,0);
\draw [line width=2pt,dotted,color=ttqqqq] (0.3333333333333333,0)-- (0.3333333333333333,1);
\begin{scriptsize}
\draw [fill=ududff] (0,0) circle (2.5pt);
\draw[color=ududff] (-0.038946779707611484,-0.02934184339083823) node {$A$};
\draw [fill=ududff] (0,1) circle (2.5pt);
\draw[color=ududff] (-0.053319754431797935,1.0462357651357799) node {$B$};
\draw [fill=ududff] (1,0) circle (2.5pt);
\draw[color=ududff] (1.019862358307457,-0.046110313902389066) node {$C$};
\draw [fill=ududff] (1,1) circle (2.5pt);
\draw[color=ududff] (1.024653349882186,1.0390492777736866) node {$D$};
\draw [fill=uuuuuu] (0.3333333333333333,0) circle (2pt);
\draw[color=uuuuuu] (0.3323550673338718,-0.02934184339083823) node {$E$};
\end{scriptsize}
\end{tikzpicture} \texttt{Completare questo schifo.}
\end{example}

Prendiamo come gruppo $ \R $ o $ [0, +\infty) $. In tale caso data l'applicazione $ \Phi_t(x) = \Phi(t, x) $ prende il nome di \emph{flusso} o \emph{semi-flusso} rispettivamente. \\

Un esempio di sistema dinamico a tempo continuo è dato da un'equazione differenziale ordinaria (ODE) del primo ordine\footnote{Di seguito considereremo quasi solo ODE del primo ordine in quanto equazioni differenziali di ordine superiore possono essere ricondotte a questa con il solito cambio di variabile a sistemi di ODE del primo ordine.} autonoma
\begin{equation}
	\begin{cases}
		\dot{x} = v(x) \\
		x(0) = x_0
	\end{cases}
\end{equation} 
dove $ x, x_0 \in \R^n $ e $ v \colon \R^n \to \R^n $ è un campo vettoriale. Se supponiamo che $ v $ sia di classe $ \mathcal{C}^1 $ allora abbiamo esistenza e unicità della soluzione \texttt{(e dipendenza continua dai parametri iniziali ??)}, cioè esiste $ \tau > 0 $ e un'unica funzione $ \phi \colon [0, \tau) \to \R^n $ tale che $ \phi(0) = x_0 $ e $ \phi'(t) = v(\phi(0)) $ per ogni $ t \in [0, \tau) $. \\
Se supponiamo per esempio che $ v $ sia un'applicazione lineare $ v(x) \coloneqq A x $ con $ A \in \mathrm{Mat}_{n \times n}(\R) $ allora abbiamo che la soluzione è prolungabile a tutto l'asse reale e introducendo la nozione di esponenziale di una matrice\footnote{%
	Data $ A \in \mathrm{Mat}_{n \times n}(\R) $ si pone 
	\[ \exp(A) \coloneqq \sum_{k = 0}^{+\infty} \frac{A^k}{k!}. \] 
}
si può scrivere nella forma 
\[ \phi(t) = \exp{\left(t \, A\right)} \, x_0. \]

\begin{definition}[orbita e spazio delle orbite]
	Data $ f \in \Aut{(X)} $ e $ \Phi^f \colon \Z \times X \to X $ definiamo orbita di $ x \in X $ come 
	\[
	\mathcal{O}^f(x) \coloneqq \{f^n(x) : n \in \Z\}.
	\]
	Le orbite definiscono una naturale relazione di equivalenza $ x \sim y \iff \exists n \in \Z : y = f^n(x) \iff y \in \mathcal{O}^f(x) \iff x \in \mathcal{O}^f(y) $. Chiamiamo lo spazio quoziente $ \faktor{X}{\sim} $ spazio delle orbite. \texttt{Topologia quoziente?}
\end{definition}

Pendolo semplice...

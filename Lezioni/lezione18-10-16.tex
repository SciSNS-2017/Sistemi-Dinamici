\section{Lezione del 16/10/2018 [Marmi]}
\begin{definition}[orbita pre-periodica e periodica] Sia $f\colon X \to X$ un sistema dinamico. Un'orbita $\mathcal{O}^f(x)$ si dice pre-periodica se contiene un numero finito di elementi. Se inoltre $ f $ è invertibile l'orbita si dice periodica e la sua cardinalità si dice periodo.
    
Infine, se $f$ non è invertibile, possono esistere punti $ x $ (che costituiscono il pre-periodo) tali che $ \forall n > 0\ f^n (x) \neq x $.
\end{definition}

\begin{example}[Congettura di Collatz]
Si consideri il sistema dinamico $f \colon \N \to \N$:
\[
    f(n) \coloneqq
    \begin{cases}
    	n/2 & \text{se $ n $ è pari}\\
    	3n+1 		& \text{se $ n $ è dispari} \\	
    \end{cases}
\]
La congettura\footnote{È attualmente un problema aperto.} di Collatz asserisce che tutti gli $n \in \N$ sono preperiodici e che l'unico ciclo è $ 1 \to 4 \to 2 \to 1 $.
\end{example}

\subsection{Coniugazione e misure invarianti}
\begin{definition}[Sistemi dinamici coniugati]
Siano $f\colon X \to X$ e $g\colon Y \to Y$ due sistemi dinamici. Questi si dicono coniugati se esiste $h \colon X \to Y$ invertibile tale che $h \circ f = g \circ h$, cioè tale da far commutare il seguente diagramma:
\begin{center}
	\begin{tikzcd}
	X \arrow[r, "f"] \arrow[d, "h"]	& X \arrow[d, "h"] \\
	Y \arrow[r, "g"] & Y 
	\end{tikzcd}
\end{center}
Se $h$ è solamente surgettiva si dice che $g$ è un \emph{fattore} di $f$ oppure che $f$ è un'\emph{estensione} di $g$. Se invece $h$ è solo iniettiva allora si dice che $f$ è un \emph{sottosistema} di $g$.
\end{definition}

\begin{example} \label{ex:Ulam_Mandelbrot}
Si considerino i seguenti sistemi dinamici $ \C \to \C$:
\[ Q_\lambda(z) \coloneqq \lambda z (1-z) \]
\[ P_c(z) \coloneqq z^2 + c \qquad \text{con } c = - \frac{\lambda^2}{4} +  \frac{\lambda}{2}. \]
Le funzioni $ Q_\lambda $ sono dette \emph{trasformazioni di Ulam-Von Neumann}, mentre $P_c$ è la funzione che genera l'\emph{insieme di Mandelbrot}.
I due sistemi risultano coniugati attraverso la funzione
\[ h_\lambda(z) = -\lambda z + \frac{\lambda}{2}\;. \]
Infatti si verifica che $ h\circ Q_\lambda = P_c \circ h $.
\end{example}

\begin{example}
    Sia $ Q_4 \colon [0,1] \to [0,1] $ come definita nell'esempio \ref{ex:Ulam_Mandelbrot} e sia $ T\colon [0,1] \to [0,1] $ la mappa a tenda:
    \[
        T(x) \coloneqq
        \begin{cases}
            2x   & \text{se } 0 \leq x \leq 1/2 \\
            2-2x & \text{se } 1/2 \leq x \leq 1
        \end{cases}.
    \]
    Allora i due sistemi sono coniugati tramite $ h(x) = \sin^2\left(\frac{\pi x}{2}\right) $, cioè si ha $ Q_4\circ h = h\circ T $.
    Inoltre, poiché $ T $ conserva la misura di Lebesgue, usando la \eqref{eq:pushforward-misure} si ottiene che $ Q_4 $ conserva la misura:
    \[ \dif{h_\sharp \lambda}(x) = \frac{\dif x}{\pi\sqrt{x(1-x)}}\; . \]
\end{example}

\begin{example}
    Sia $ Q_4\colon (0,1)\to(0,1) $, $ S\colon \R \to \R $ definita come
    \[ S(y) \coloneqq \log\left(\frac{4 e^y}{(1-e^y)^2}\right) \]
    e $ h\colon (0,1)\to\R $:
    \[ h(x) \coloneqq \operatorname{logit}(x) \coloneqq \log\left(\frac{x}{1-x}\right) \]
    Allora $ h\circ Q_4 = S \circ h $ e S conserva la misura:
    \[ \dif\mu(y) = \frac{\dif y}{\pi\left( e^{y/2} - e^{-y/2} \right) } \; . \]
\end{example}

\subsection{Dinamica topologica}
\emph{Setting}: $ X $ spazio metrico compatto e $ f\colon X\to X $ automorfismo.
\begin{definition}[Sottoinsieme invariante]
    Un sottoinsieme $ \Lambda \subset X $ si dice $ f $-invariante se $ \forall x\in\Lambda $ e $ \forall m \in \Z $, $ f^m(x) \in \Lambda $.
    Se tale proprietà vale solo per $ m \geq 0 $ o $ m \leq 0 $, $ \Lambda $ si dirà rispettivamente positivamente o negativamente $ f $-invariante.
\end{definition}

\begin{oss}
    Orbite periodiche sono insiemi invarianti, così come lo sono unioni di orbite.
\end{oss}

\begin{definition}[Punto errante]
    Un punto $ x\in X $ si dice errante se $ \exists\; U\ni x $ intorno tale che $ \left( \bigcup_{\; \abs{m}>0} f^m(U) \right) \cap U = \emptyset $.
    
    Se $ f $ è solo un endomorfismo prenderemo solo gli $ m > 0 $. Nel caso di un sistema dinamico a tempo continuo, invece, si applica la stessa definizione dopo aver discretizzato il tempo ($ t = \tau\Z $).
\end{definition}
\begin{example}
    Sia $ X = \R $ e $ T_\alpha (x) \coloneqq x + \alpha $ la traslazione di $ \alpha $. In questo caso tutti gli $ x\in\R $ sono punti erranti.
\end{example}
\begin{exercise}
    Sia $ \Omega = \left\{ x\in X : x \text{ non è errante} \right\} $. Mostrare che $ \Omega $ è chiuso e invariante e che $ X $ compatto $ \Rightarrow \Omega \neq \emptyset $.
\end{exercise}
\begin{solution}
    \textcolor{red}{content}
\end{solution}

\begin{definition}[$ \alpha $ e $ \omega $-limite]
    Dato $ x\in X $ si definiscono gli insiemi:
    \begin{align*}
        \alpha(x) \coloneqq \left\{ y\in X : \exists\; n_j \nearrow -\infty, f^{n_j}(x) \to y \right\} \\
        \omega(x) \coloneqq \left\{ y\in X : \exists\; n_j \searrow +\infty, f^{n_j}(x) \to y \right\}
    \end{align*}
    ossia l'insieme dei punti di accumulazione dell'orbita di x nel passato e nel futuro rispettivamente.
\end{definition}

\begin{definition}[Punto ricorrente]
    Un punto $ x\in X $ si dice ricorrente se $ x\in \alpha(x) \cap \omega(x) $. Se vale solo $ x\in \alpha $ (risp. $ \omega $) il punto si dirà negativamente (risp. positivamente) ricorrente.
\end{definition}
\begin{exercise}
    Dato $ x\in X $, $ \alpha(x) $ e $ \omega(x) $ sono chiusi e invarianti.
\end{exercise}

\begin{example}
    \textcolor{red}{spirale}
\end{example}

\begin{example}
    \textcolor{red}{Schifezza a forma di 8.}
\end{example}

\begin{example}
    \textcolor{red}{Cosa sul toro}
\end{example}

\begin{exercise}[Teorema di Dirichlet]
    
\end{exercise}

\begin{exercise}
    $ \sum_{n=0}^{+\infty} 10^{-n!} $ è di Liouville.
\end{exercise}
\begin{exercise}
    Gli irrazionali algebrici sono diofantei.
\end{exercise}
\begin{exercise}
    Quasi ogni reale è diofanteo.\\
    Hint: stimare la misura di Lebesgue di $ ( (0,1) \setminus \mathrm{CD}(\gamma,\tau) ),\; \tau > 0 $.
\end{exercise}
\section{Lezione del 16/10/2018 [Marmi]}
\begin{definition}[orbita pre-periodica e periodica] Sia $f\colon X \to X$ un sistema dinamico. Un'orbita $\mathcal{O}^f(x)$ si dice pre-periodica se contiene un numero finito di elementi. Se inoltre $ f $ è invertibile l'orbita si dice periodica e la sua cardinalità si dice periodo.

Infine, se $f$ non è invertibile, possono esistere punti $ x $ (che costituiscono il pre-periodo) tali che $ \forall n > 0\ f^n (x) \neq x $.
\end{definition}

\begin{example}[Congettura di Collatz]
Si consideri il sistema dinamico $f \colon \N \to \N$:
\[
    f(n) \coloneqq
    \begin{cases}
        n/2  & \text{se $ n $ è pari}    \\
        3n+1 & \text{se $ n $ è dispari}
    \end{cases}
\]
La congettura\footnote{È attualmente un problema aperto.} di Collatz asserisce che tutti gli $n \in \N$ sono preperiodici e che l'unico ciclo è $ 1 \to 4 \to 2 \to 1 $.
\end{example}

\subsection{Coniugazione e misure invarianti}
\begin{definition}[Sistemi dinamici coniugati]
Siano $f\colon X \to X$ e $g\colon Y \to Y$ due sistemi dinamici. Questi si dicono coniugati se esiste $h \colon X \to Y$ invertibile tale che $h \circ f = g \circ h$, cioè tale da far commutare il seguente diagramma:
\begin{center}
    \begin{tikzcd}
        X \arrow[r, "f"] \arrow[d, "h"]  & X \arrow[d, "h"] \\
        Y \arrow[r, "g"] & Y
    \end{tikzcd}
\end{center}
Se $h$ è solamente surgettiva si dice che $g$ è un \emph{fattore} di $f$ oppure che $f$ è un'\emph{estensione} di $g$. Se invece $h$ è solo iniettiva allora si dice che $f$ è un \emph{sottosistema} di $g$.
\end{definition}

\begin{example} \label{ex:Ulam_Mandelbrot}
Si considerino i seguenti sistemi dinamici $ \C \to \C$:
\[ Q_\lambda(z) \coloneqq \lambda z (1-z) \]
\[ P_c(z) \coloneqq z^2 + c \qquad \text{con } c = - \frac{\lambda^2}{4} +  \frac{\lambda}{2}. \]
Le funzioni $ Q_\lambda $ sono dette \emph{trasformazioni di Ulam-Von Neumann}, mentre $P_c$ è la funzione che genera l'\emph{insieme di Mandelbrot}.
I due sistemi risultano coniugati attraverso la funzione
\[ h_\lambda(z) = -\lambda z + \frac{\lambda}{2}\;. \]
Infatti si verifica che $ h\circ Q_\lambda = P_c \circ h $.
\end{example}

\begin{example} \label{ex:Ulam_tenda}
    Sia $ Q_4 \colon [0,1] \to [0,1] $ come definita nell'esempio \ref{ex:Ulam_Mandelbrot} e sia $ T\colon [0,1] \to [0,1] $ la mappa a tenda:
    \[
        T(x) \coloneqq
        \begin{cases}
            2x   & \text{se } 0 \leq x \leq 1/2 \\
            2-2x & \text{se } 1/2 \leq x \leq 1
        \end{cases}.
    \]
    Allora i due sistemi sono coniugati tramite $ h(x) = \sin^2\left(\frac{\pi x}{2}\right) $, cioè si ha $ Q_4\circ h = h\circ T $.
    Inoltre, poiché $ T $ conserva la misura di Lebesgue, usando la \eqref{eq:pushforward-misure} si ottiene che $ Q_4 $ conserva la misura:
    \[ \dif{h_\sharp \lambda}(x) = \frac{\dif x}{\pi\sqrt{x(1-x)}}\; . \]

    \iffigureon
    \begin{figure}
        \begin{center}
            \subfloat[Mappa a tenda]
            { \definecolor{qqqqcc}{rgb}{0.,0.,0.8}
\begin{tikzpicture}[line cap=round,line join=round,>=triangle 45,x=5.0cm,y=5.0cm]
\draw[->,color=black] (-0.1,0.) -- (1.1,0.);
\foreach \x in {,0.2,0.4,0.6,0.8,1.}
\draw[shift={(\x,0)},color=black] (0pt,2pt) -- (0pt,-2pt) node[below] {\footnotesize $\x$};
\draw[->,color=black] (0.,-0.1) -- (0.,1.1);
\foreach \y in {,0.2,0.4,0.6,0.8,1.}
\draw[shift={(0,\y)},color=black] (2pt,0pt) -- (-2pt,0pt) node[left] {\footnotesize $\y$};
\draw[color=black] (0pt,-10pt) node[right] {\footnotesize $0$};
\clip(-0.1,-0.1) rectangle (1.1,1.1);
\draw[line width=1pt,color=qqqqcc] (8.000000000003847E-7,0.0) -- (0.0,0.0);
\draw[line width=1pt,color=qqqqcc] (0.0,0.0) -- (0.0024999956818200016,0.004999991363640003);
\draw[line width=1pt,color=qqqqcc] (0.0024999956818200016,0.004999991363640003) -- (0.004999991363640003,0.009999982727280006);
\draw[line width=1pt,color=qqqqcc] (0.004999991363640003,0.009999982727280006) -- (0.007499987045460005,0.01499997409092001);
\draw[line width=1pt,color=qqqqcc] (0.007499987045460005,0.01499997409092001) -- (0.009999982727280006,0.019999965454560013);
\draw[line width=1pt,color=qqqqcc] (0.009999982727280006,0.019999965454560013) -- (0.012499978409100007,0.024999956818200015);
\draw[line width=1pt,color=qqqqcc] (0.012499978409100007,0.024999956818200015) -- (0.014999974090920009,0.029999948181840017);
\draw[line width=1pt,color=qqqqcc] (0.014999974090920009,0.029999948181840017) -- (0.01749996977274001,0.03499993954548002);
\draw[line width=1pt,color=qqqqcc] (0.01749996977274001,0.03499993954548002) -- (0.019999965454560013,0.039999930909120025);
\draw[line width=1pt,color=qqqqcc] (0.019999965454560013,0.039999930909120025) -- (0.022499961136380014,0.04499992227276003);
\draw[line width=1pt,color=qqqqcc] (0.022499961136380014,0.04499992227276003) -- (0.024999956818200015,0.04999991363640003);
\draw[line width=1pt,color=qqqqcc] (0.024999956818200015,0.04999991363640003) -- (0.027499952500020016,0.05499990500004003);
\draw[line width=1pt,color=qqqqcc] (0.027499952500020016,0.05499990500004003) -- (0.029999948181840017,0.059999896363680034);
\draw[line width=1pt,color=qqqqcc] (0.029999948181840017,0.059999896363680034) -- (0.03249994386366002,0.06499988772732004);
\draw[line width=1pt,color=qqqqcc] (0.03249994386366002,0.06499988772732004) -- (0.03499993954548002,0.06999987909096005);
\draw[line width=1pt,color=qqqqcc] (0.03499993954548002,0.06999987909096005) -- (0.037499935227300024,0.07499987045460005);
\draw[line width=1pt,color=qqqqcc] (0.037499935227300024,0.07499987045460005) -- (0.039999930909120025,0.07999986181824005);
\draw[line width=1pt,color=qqqqcc] (0.039999930909120025,0.07999986181824005) -- (0.042499926590940026,0.08499985318188005);
\draw[line width=1pt,color=qqqqcc] (0.042499926590940026,0.08499985318188005) -- (0.04499992227276003,0.08999984454552006);
\draw[line width=1pt,color=qqqqcc] (0.04499992227276003,0.08999984454552006) -- (0.04749991795458003,0.09499983590916006);
\draw[line width=1pt,color=qqqqcc] (0.04749991795458003,0.09499983590916006) -- (0.04999991363640003,0.09999982727280006);
\draw[line width=1pt,color=qqqqcc] (0.04999991363640003,0.09999982727280006) -- (0.05249990931822003,0.10499981863644006);
\draw[line width=1pt,color=qqqqcc] (0.05249990931822003,0.10499981863644006) -- (0.05499990500004003,0.10999981000008006);
\draw[line width=1pt,color=qqqqcc] (0.05499990500004003,0.10999981000008006) -- (0.05749990068186003,0.11499980136372007);
\draw[line width=1pt,color=qqqqcc] (0.05749990068186003,0.11499980136372007) -- (0.059999896363680034,0.11999979272736007);
\draw[line width=1pt,color=qqqqcc] (0.059999896363680034,0.11999979272736007) -- (0.062499892045500036,0.12499978409100007);
\draw[line width=1pt,color=qqqqcc] (0.062499892045500036,0.12499978409100007) -- (0.06499988772732004,0.1299997754546401);
\draw[line width=1pt,color=qqqqcc] (0.06499988772732004,0.1299997754546401) -- (0.06749988340914005,0.1349997668182801);
\draw[line width=1pt,color=qqqqcc] (0.06749988340914005,0.1349997668182801) -- (0.06999987909096006,0.13999975818192012);
\draw[line width=1pt,color=qqqqcc] (0.06999987909096006,0.13999975818192012) -- (0.07249987477278007,0.14499974954556014);
\draw[line width=1pt,color=qqqqcc] (0.07249987477278007,0.14499974954556014) -- (0.07499987045460008,0.14999974090920015);
\draw[line width=1pt,color=qqqqcc] (0.07499987045460008,0.14999974090920015) -- (0.07749986613642008,0.15499973227284017);
\draw[line width=1pt,color=qqqqcc] (0.07749986613642008,0.15499973227284017) -- (0.07999986181824009,0.15999972363648018);
\draw[line width=1pt,color=qqqqcc] (0.07999986181824009,0.15999972363648018) -- (0.0824998575000601,0.1649997150001202);
\draw[line width=1pt,color=qqqqcc] (0.0824998575000601,0.1649997150001202) -- (0.08499985318188011,0.16999970636376022);
\draw[line width=1pt,color=qqqqcc] (0.08499985318188011,0.16999970636376022) -- (0.08749984886370012,0.17499969772740023);
\draw[line width=1pt,color=qqqqcc] (0.08749984886370012,0.17499969772740023) -- (0.08999984454552012,0.17999968909104025);
\draw[line width=1pt,color=qqqqcc] (0.08999984454552012,0.17999968909104025) -- (0.09249984022734013,0.18499968045468027);
\draw[line width=1pt,color=qqqqcc] (0.09249984022734013,0.18499968045468027) -- (0.09499983590916014,0.18999967181832028);
\draw[line width=1pt,color=qqqqcc] (0.09499983590916014,0.18999967181832028) -- (0.09749983159098015,0.1949996631819603);
\draw[line width=1pt,color=qqqqcc] (0.09749983159098015,0.1949996631819603) -- (0.09999982727280016,0.1999996545456003);
\draw[line width=1pt,color=qqqqcc] (0.09999982727280016,0.1999996545456003) -- (0.10249982295462017,0.20499964590924033);
\draw[line width=1pt,color=qqqqcc] (0.10249982295462017,0.20499964590924033) -- (0.10499981863644017,0.20999963727288035);
\draw[line width=1pt,color=qqqqcc] (0.10499981863644017,0.20999963727288035) -- (0.10749981431826018,0.21499962863652036);
\draw[line width=1pt,color=qqqqcc] (0.10749981431826018,0.21499962863652036) -- (0.10999981000008019,0.21999962000016038);
\draw[line width=1pt,color=qqqqcc] (0.10999981000008019,0.21999962000016038) -- (0.1124998056819002,0.2249996113638004);
\draw[line width=1pt,color=qqqqcc] (0.1124998056819002,0.2249996113638004) -- (0.1149998013637202,0.2299996027274404);
\draw[line width=1pt,color=qqqqcc] (0.1149998013637202,0.2299996027274404) -- (0.11749979704554021,0.23499959409108043);
\draw[line width=1pt,color=qqqqcc] (0.11749979704554021,0.23499959409108043) -- (0.11999979272736022,0.23999958545472044);
\draw[line width=1pt,color=qqqqcc] (0.11999979272736022,0.23999958545472044) -- (0.12249978840918023,0.24499957681836046);
\draw[line width=1pt,color=qqqqcc] (0.12249978840918023,0.24499957681836046) -- (0.12499978409100024,0.24999956818200048);
\draw[line width=1pt,color=qqqqcc] (0.12499978409100024,0.24999956818200048) -- (0.12749977977282023,0.25499955954564046);
\draw[line width=1pt,color=qqqqcc] (0.12749977977282023,0.25499955954564046) -- (0.12999977545464023,0.25999955090928045);
\draw[line width=1pt,color=qqqqcc] (0.12999977545464023,0.25999955090928045) -- (0.13249977113646022,0.26499954227292044);
\draw[line width=1pt,color=qqqqcc] (0.13249977113646022,0.26499954227292044) -- (0.13499976681828021,0.26999953363656043);
\draw[line width=1pt,color=qqqqcc] (0.13499976681828021,0.26999953363656043) -- (0.1374997625001002,0.2749995250002004);
\draw[line width=1pt,color=qqqqcc] (0.1374997625001002,0.2749995250002004) -- (0.1399997581819202,0.2799995163638404);
\draw[line width=1pt,color=qqqqcc] (0.1399997581819202,0.2799995163638404) -- (0.1424997538637402,0.2849995077274804);
\draw[line width=1pt,color=qqqqcc] (0.1424997538637402,0.2849995077274804) -- (0.1449997495455602,0.2899994990911204);
\draw[line width=1pt,color=qqqqcc] (0.1449997495455602,0.2899994990911204) -- (0.14749974522738019,0.29499949045476037);
\draw[line width=1pt,color=qqqqcc] (0.14749974522738019,0.29499949045476037) -- (0.14999974090920018,0.29999948181840036);
\draw[line width=1pt,color=qqqqcc] (0.14999974090920018,0.29999948181840036) -- (0.15249973659102017,0.30499947318204035);
\draw[line width=1pt,color=qqqqcc] (0.15249973659102017,0.30499947318204035) -- (0.15499973227284017,0.30999946454568034);
\draw[line width=1pt,color=qqqqcc] (0.15499973227284017,0.30999946454568034) -- (0.15749972795466016,0.3149994559093203);
\draw[line width=1pt,color=qqqqcc] (0.15749972795466016,0.3149994559093203) -- (0.15999972363648016,0.3199994472729603);
\draw[line width=1pt,color=qqqqcc] (0.15999972363648016,0.3199994472729603) -- (0.16249971931830015,0.3249994386366003);
\draw[line width=1pt,color=qqqqcc] (0.16249971931830015,0.3249994386366003) -- (0.16499971500012015,0.3299994300002403);
\draw[line width=1pt,color=qqqqcc] (0.16499971500012015,0.3299994300002403) -- (0.16749971068194014,0.3349994213638803);
\draw[line width=1pt,color=qqqqcc] (0.16749971068194014,0.3349994213638803) -- (0.16999970636376013,0.33999941272752027);
\draw[line width=1pt,color=qqqqcc] (0.16999970636376013,0.33999941272752027) -- (0.17249970204558013,0.34499940409116026);
\draw[line width=1pt,color=qqqqcc] (0.17249970204558013,0.34499940409116026) -- (0.17499969772740012,0.34999939545480024);
\draw[line width=1pt,color=qqqqcc] (0.17499969772740012,0.34999939545480024) -- (0.17749969340922012,0.35499938681844023);
\draw[line width=1pt,color=qqqqcc] (0.17749969340922012,0.35499938681844023) -- (0.1799996890910401,0.3599993781820802);
\draw[line width=1pt,color=qqqqcc] (0.1799996890910401,0.3599993781820802) -- (0.1824996847728601,0.3649993695457202);
\draw[line width=1pt,color=qqqqcc] (0.1824996847728601,0.3649993695457202) -- (0.1849996804546801,0.3699993609093602);
\draw[line width=1pt,color=qqqqcc] (0.1849996804546801,0.3699993609093602) -- (0.1874996761365001,0.3749993522730002);
\draw[line width=1pt,color=qqqqcc] (0.1874996761365001,0.3749993522730002) -- (0.1899996718183201,0.3799993436366402);
\draw[line width=1pt,color=qqqqcc] (0.1899996718183201,0.3799993436366402) -- (0.19249966750014008,0.38499933500028016);
\draw[line width=1pt,color=qqqqcc] (0.19249966750014008,0.38499933500028016) -- (0.19499966318196008,0.38999932636392015);
\draw[line width=1pt,color=qqqqcc] (0.19499966318196008,0.38999932636392015) -- (0.19749965886378007,0.39499931772756014);
\draw[line width=1pt,color=qqqqcc] (0.19749965886378007,0.39499931772756014) -- (0.19999965454560006,0.39999930909120013);
\draw[line width=1pt,color=qqqqcc] (0.19999965454560006,0.39999930909120013) -- (0.20249965022742006,0.4049993004548401);
\draw[line width=1pt,color=qqqqcc] (0.20249965022742006,0.4049993004548401) -- (0.20499964590924005,0.4099992918184801);
\draw[line width=1pt,color=qqqqcc] (0.20499964590924005,0.4099992918184801) -- (0.20749964159106005,0.4149992831821201);
\draw[line width=1pt,color=qqqqcc] (0.20749964159106005,0.4149992831821201) -- (0.20999963727288004,0.4199992745457601);
\draw[line width=1pt,color=qqqqcc] (0.20999963727288004,0.4199992745457601) -- (0.21249963295470004,0.42499926590940007);
\draw[line width=1pt,color=qqqqcc] (0.21249963295470004,0.42499926590940007) -- (0.21499962863652003,0.42999925727304006);
\draw[line width=1pt,color=qqqqcc] (0.21499962863652003,0.42999925727304006) -- (0.21749962431834002,0.43499924863668005);
\draw[line width=1pt,color=qqqqcc] (0.21749962431834002,0.43499924863668005) -- (0.21999962000016002,0.43999924000032004);
\draw[line width=1pt,color=qqqqcc] (0.21999962000016002,0.43999924000032004) -- (0.22249961568198,0.44499923136396);
\draw[line width=1pt,color=qqqqcc] (0.22249961568198,0.44499923136396) -- (0.2249996113638,0.4499992227276);
\draw[line width=1pt,color=qqqqcc] (0.2249996113638,0.4499992227276) -- (0.22749960704562,0.45499921409124);
\draw[line width=1pt,color=qqqqcc] (0.22749960704562,0.45499921409124) -- (0.22999960272744,0.45999920545488);
\draw[line width=1pt,color=qqqqcc] (0.22999960272744,0.45999920545488) -- (0.23249959840926,0.46499919681852);
\draw[line width=1pt,color=qqqqcc] (0.23249959840926,0.46499919681852) -- (0.23499959409107998,0.46999918818215997);
\draw[line width=1pt,color=qqqqcc] (0.23499959409107998,0.46999918818215997) -- (0.23749958977289998,0.47499917954579995);
\draw[line width=1pt,color=qqqqcc] (0.23749958977289998,0.47499917954579995) -- (0.23999958545471997,0.47999917090943994);
\draw[line width=1pt,color=qqqqcc] (0.23999958545471997,0.47999917090943994) -- (0.24249958113653997,0.48499916227307993);
\draw[line width=1pt,color=qqqqcc] (0.24249958113653997,0.48499916227307993) -- (0.24499957681835996,0.4899991536367199);
\draw[line width=1pt,color=qqqqcc] (0.24499957681835996,0.4899991536367199) -- (0.24749957250017995,0.4949991450003599);
\draw[line width=1pt,color=qqqqcc] (0.24749957250017995,0.4949991450003599) -- (0.24999956818199995,0.4999991363639999);
\draw[line width=1pt,color=qqqqcc] (0.24999956818199995,0.4999991363639999) -- (0.25249956386381994,0.5049991277276399);
\draw[line width=1pt,color=qqqqcc] (0.25249956386381994,0.5049991277276399) -- (0.25499955954563996,0.5099991190912799);
\draw[line width=1pt,color=qqqqcc] (0.25499955954563996,0.5099991190912799) -- (0.25749955522746,0.51499911045492);
\draw[line width=1pt,color=qqqqcc] (0.25749955522746,0.51499911045492) -- (0.25999955090928,0.51999910181856);
\draw[line width=1pt,color=qqqqcc] (0.25999955090928,0.51999910181856) -- (0.26249954659110003,0.5249990931822001);
\draw[line width=1pt,color=qqqqcc] (0.26249954659110003,0.5249990931822001) -- (0.26499954227292005,0.5299990845458401);
\draw[line width=1pt,color=qqqqcc] (0.26499954227292005,0.5299990845458401) -- (0.2674995379547401,0.5349990759094801);
\draw[line width=1pt,color=qqqqcc] (0.2674995379547401,0.5349990759094801) -- (0.2699995336365601,0.5399990672731202);
\draw[line width=1pt,color=qqqqcc] (0.2699995336365601,0.5399990672731202) -- (0.2724995293183801,0.5449990586367602);
\draw[line width=1pt,color=qqqqcc] (0.2724995293183801,0.5449990586367602) -- (0.27499952500020014,0.5499990500004003);
\draw[line width=1pt,color=qqqqcc] (0.27499952500020014,0.5499990500004003) -- (0.27749952068202016,0.5549990413640403);
\draw[line width=1pt,color=qqqqcc] (0.27749952068202016,0.5549990413640403) -- (0.2799995163638402,0.5599990327276804);
\draw[line width=1pt,color=qqqqcc] (0.2799995163638402,0.5599990327276804) -- (0.2824995120456602,0.5649990240913204);
\draw[line width=1pt,color=qqqqcc] (0.2824995120456602,0.5649990240913204) -- (0.28499950772748023,0.5699990154549605);
\draw[line width=1pt,color=qqqqcc] (0.28499950772748023,0.5699990154549605) -- (0.28749950340930025,0.5749990068186005);
\draw[line width=1pt,color=qqqqcc] (0.28749950340930025,0.5749990068186005) -- (0.28999949909112027,0.5799989981822405);
\draw[line width=1pt,color=qqqqcc] (0.28999949909112027,0.5799989981822405) -- (0.2924994947729403,0.5849989895458806);
\draw[line width=1pt,color=qqqqcc] (0.2924994947729403,0.5849989895458806) -- (0.2949994904547603,0.5899989809095206);
\draw[line width=1pt,color=qqqqcc] (0.2949994904547603,0.5899989809095206) -- (0.29749948613658034,0.5949989722731607);
\draw[line width=1pt,color=qqqqcc] (0.29749948613658034,0.5949989722731607) -- (0.29999948181840036,0.5999989636368007);
\draw[line width=1pt,color=qqqqcc] (0.29999948181840036,0.5999989636368007) -- (0.3024994775002204,0.6049989550004408);
\draw[line width=1pt,color=qqqqcc] (0.3024994775002204,0.6049989550004408) -- (0.3049994731820404,0.6099989463640808);
\draw[line width=1pt,color=qqqqcc] (0.3049994731820404,0.6099989463640808) -- (0.3074994688638604,0.6149989377277209);
\draw[line width=1pt,color=qqqqcc] (0.3074994688638604,0.6149989377277209) -- (0.30999946454568045,0.6199989290913609);
\draw[line width=1pt,color=qqqqcc] (0.30999946454568045,0.6199989290913609) -- (0.31249946022750047,0.6249989204550009);
\draw[line width=1pt,color=qqqqcc] (0.31249946022750047,0.6249989204550009) -- (0.3149994559093205,0.629998911818641);
\draw[line width=1pt,color=qqqqcc] (0.3149994559093205,0.629998911818641) -- (0.3174994515911405,0.634998903182281);
\draw[line width=1pt,color=qqqqcc] (0.3174994515911405,0.634998903182281) -- (0.31999944727296054,0.6399988945459211);
\draw[line width=1pt,color=qqqqcc] (0.31999944727296054,0.6399988945459211) -- (0.32249944295478056,0.6449988859095611);
\draw[line width=1pt,color=qqqqcc] (0.32249944295478056,0.6449988859095611) -- (0.3249994386366006,0.6499988772732012);
\draw[line width=1pt,color=qqqqcc] (0.3249994386366006,0.6499988772732012) -- (0.3274994343184206,0.6549988686368412);
\draw[line width=1pt,color=qqqqcc] (0.3274994343184206,0.6549988686368412) -- (0.3299994300002406,0.6599988600004812);
\draw[line width=1pt,color=qqqqcc] (0.3299994300002406,0.6599988600004812) -- (0.33249942568206065,0.6649988513641213);
\draw[line width=1pt,color=qqqqcc] (0.33249942568206065,0.6649988513641213) -- (0.33499942136388067,0.6699988427277613);
\draw[line width=1pt,color=qqqqcc] (0.33499942136388067,0.6699988427277613) -- (0.3374994170457007,0.6749988340914014);
\draw[line width=1pt,color=qqqqcc] (0.3374994170457007,0.6749988340914014) -- (0.3399994127275207,0.6799988254550414);
\draw[line width=1pt,color=qqqqcc] (0.3399994127275207,0.6799988254550414) -- (0.34249940840934073,0.6849988168186815);
\draw[line width=1pt,color=qqqqcc] (0.34249940840934073,0.6849988168186815) -- (0.34499940409116076,0.6899988081823215);
\draw[line width=1pt,color=qqqqcc] (0.34499940409116076,0.6899988081823215) -- (0.3474993997729808,0.6949987995459616);
\draw[line width=1pt,color=qqqqcc] (0.3474993997729808,0.6949987995459616) -- (0.3499993954548008,0.6999987909096016);
\draw[line width=1pt,color=qqqqcc] (0.3499993954548008,0.6999987909096016) -- (0.3524993911366208,0.7049987822732416);
\draw[line width=1pt,color=qqqqcc] (0.3524993911366208,0.7049987822732416) -- (0.35499938681844084,0.7099987736368817);
\draw[line width=1pt,color=qqqqcc] (0.35499938681844084,0.7099987736368817) -- (0.35749938250026086,0.7149987650005217);
\draw[line width=1pt,color=qqqqcc] (0.35749938250026086,0.7149987650005217) -- (0.3599993781820809,0.7199987563641618);
\draw[line width=1pt,color=qqqqcc] (0.3599993781820809,0.7199987563641618) -- (0.3624993738639009,0.7249987477278018);
\draw[line width=1pt,color=qqqqcc] (0.3624993738639009,0.7249987477278018) -- (0.36499936954572093,0.7299987390914419);
\draw[line width=1pt,color=qqqqcc] (0.36499936954572093,0.7299987390914419) -- (0.36749936522754095,0.7349987304550819);
\draw[line width=1pt,color=qqqqcc] (0.36749936522754095,0.7349987304550819) -- (0.369999360909361,0.739998721818722);
\draw[line width=1pt,color=qqqqcc] (0.369999360909361,0.739998721818722) -- (0.372499356591181,0.744998713182362);
\draw[line width=1pt,color=qqqqcc] (0.372499356591181,0.744998713182362) -- (0.374999352273001,0.749998704546002);
\draw[line width=1pt,color=qqqqcc] (0.374999352273001,0.749998704546002) -- (0.37749934795482104,0.7549986959096421);
\draw[line width=1pt,color=qqqqcc] (0.37749934795482104,0.7549986959096421) -- (0.37999934363664106,0.7599986872732821);
\draw[line width=1pt,color=qqqqcc] (0.37999934363664106,0.7599986872732821) -- (0.3824993393184611,0.7649986786369222);
\draw[line width=1pt,color=qqqqcc] (0.3824993393184611,0.7649986786369222) -- (0.3849993350002811,0.7699986700005622);
\draw[line width=1pt,color=qqqqcc] (0.3849993350002811,0.7699986700005622) -- (0.38749933068210113,0.7749986613642023);
\draw[line width=1pt,color=qqqqcc] (0.38749933068210113,0.7749986613642023) -- (0.38999932636392115,0.7799986527278423);
\draw[line width=1pt,color=qqqqcc] (0.38999932636392115,0.7799986527278423) -- (0.3924993220457412,0.7849986440914823);
\draw[line width=1pt,color=qqqqcc] (0.3924993220457412,0.7849986440914823) -- (0.3949993177275612,0.7899986354551224);
\draw[line width=1pt,color=qqqqcc] (0.3949993177275612,0.7899986354551224) -- (0.3974993134093812,0.7949986268187624);
\draw[line width=1pt,color=qqqqcc] (0.3974993134093812,0.7949986268187624) -- (0.39999930909120124,0.7999986181824025);
\draw[line width=1pt,color=qqqqcc] (0.39999930909120124,0.7999986181824025) -- (0.40249930477302126,0.8049986095460425);
\draw[line width=1pt,color=qqqqcc] (0.40249930477302126,0.8049986095460425) -- (0.4049993004548413,0.8099986009096826);
\draw[line width=1pt,color=qqqqcc] (0.4049993004548413,0.8099986009096826) -- (0.4074992961366613,0.8149985922733226);
\draw[line width=1pt,color=qqqqcc] (0.4074992961366613,0.8149985922733226) -- (0.4099992918184813,0.8199985836369627);
\draw[line width=1pt,color=qqqqcc] (0.4099992918184813,0.8199985836369627) -- (0.41249928750030135,0.8249985750006027);
\draw[line width=1pt,color=qqqqcc] (0.41249928750030135,0.8249985750006027) -- (0.41499928318212137,0.8299985663642427);
\draw[line width=1pt,color=qqqqcc] (0.41499928318212137,0.8299985663642427) -- (0.4174992788639414,0.8349985577278828);
\draw[line width=1pt,color=qqqqcc] (0.4174992788639414,0.8349985577278828) -- (0.4199992745457614,0.8399985490915228);
\draw[line width=1pt,color=qqqqcc] (0.4199992745457614,0.8399985490915228) -- (0.42249927022758144,0.8449985404551629);
\draw[line width=1pt,color=qqqqcc] (0.42249927022758144,0.8449985404551629) -- (0.42499926590940146,0.8499985318188029);
\draw[line width=1pt,color=qqqqcc] (0.42499926590940146,0.8499985318188029) -- (0.4274992615912215,0.854998523182443);
\draw[line width=1pt,color=qqqqcc] (0.4274992615912215,0.854998523182443) -- (0.4299992572730415,0.859998514546083);
\draw[line width=1pt,color=qqqqcc] (0.4299992572730415,0.859998514546083) -- (0.4324992529548615,0.864998505909723);
\draw[line width=1pt,color=qqqqcc] (0.4324992529548615,0.864998505909723) -- (0.43499924863668155,0.8699984972733631);
\draw[line width=1pt,color=qqqqcc] (0.43499924863668155,0.8699984972733631) -- (0.43749924431850157,0.8749984886370031);
\draw[line width=1pt,color=qqqqcc] (0.43749924431850157,0.8749984886370031) -- (0.4399992400003216,0.8799984800006432);
\draw[line width=1pt,color=qqqqcc] (0.4399992400003216,0.8799984800006432) -- (0.4424992356821416,0.8849984713642832);
\draw[line width=1pt,color=qqqqcc] (0.4424992356821416,0.8849984713642832) -- (0.44499923136396163,0.8899984627279233);
\draw[line width=1pt,color=qqqqcc] (0.44499923136396163,0.8899984627279233) -- (0.44749922704578166,0.8949984540915633);
\draw[line width=1pt,color=qqqqcc] (0.44749922704578166,0.8949984540915633) -- (0.4499992227276017,0.8999984454552034);
\draw[line width=1pt,color=qqqqcc] (0.4499992227276017,0.8999984454552034) -- (0.4524992184094217,0.9049984368188434);
\draw[line width=1pt,color=qqqqcc] (0.4524992184094217,0.9049984368188434) -- (0.4549992140912417,0.9099984281824834);
\draw[line width=1pt,color=qqqqcc] (0.4549992140912417,0.9099984281824834) -- (0.45749920977306174,0.9149984195461235);
\draw[line width=1pt,color=qqqqcc] (0.45749920977306174,0.9149984195461235) -- (0.45999920545488177,0.9199984109097635);
\draw[line width=1pt,color=qqqqcc] (0.45999920545488177,0.9199984109097635) -- (0.4624992011367018,0.9249984022734036);
\draw[line width=1pt,color=qqqqcc] (0.4624992011367018,0.9249984022734036) -- (0.4649991968185218,0.9299983936370436);
\draw[line width=1pt,color=qqqqcc] (0.4649991968185218,0.9299983936370436) -- (0.46749919250034183,0.9349983850006837);
\draw[line width=1pt,color=qqqqcc] (0.46749919250034183,0.9349983850006837) -- (0.46999918818216185,0.9399983763643237);
\draw[line width=1pt,color=qqqqcc] (0.46999918818216185,0.9399983763643237) -- (0.4724991838639819,0.9449983677279638);
\draw[line width=1pt,color=qqqqcc] (0.4724991838639819,0.9449983677279638) -- (0.4749991795458019,0.9499983590916038);
\draw[line width=1pt,color=qqqqcc] (0.4749991795458019,0.9499983590916038) -- (0.4774991752276219,0.9549983504552438);
\draw[line width=1pt,color=qqqqcc] (0.4774991752276219,0.9549983504552438) -- (0.47999917090944194,0.9599983418188839);
\draw[line width=1pt,color=qqqqcc] (0.47999917090944194,0.9599983418188839) -- (0.48249916659126196,0.9649983331825239);
\draw[line width=1pt,color=qqqqcc] (0.48249916659126196,0.9649983331825239) -- (0.484999162273082,0.969998324546164);
\draw[line width=1pt,color=qqqqcc] (0.484999162273082,0.969998324546164) -- (0.487499157954902,0.974998315909804);
\draw[line width=1pt,color=qqqqcc] (0.487499157954902,0.974998315909804) -- (0.48999915363672203,0.9799983072734441);
\draw[line width=1pt,color=qqqqcc] (0.48999915363672203,0.9799983072734441) -- (0.49249914931854205,0.9849982986370841);
\draw[line width=1pt,color=qqqqcc] (0.49249914931854205,0.9849982986370841) -- (0.4949991450003621,0.9899982900007241);
\draw[line width=1pt,color=qqqqcc] (0.4949991450003621,0.9899982900007241) -- (0.4974991406821821,0.9949982813643642);
\draw[line width=1pt,color=qqqqcc] (0.4974991406821821,0.9949982813643642) -- (0.4999991363640021,0.9999982727280042);
\draw[line width=1pt,color=qqqqcc] (0.4999991363640021,0.9999982727280042) -- (0.5024991320458221,0.9950017359083558);
\draw[line width=1pt,color=qqqqcc] (0.5024991320458221,0.9950017359083558) -- (0.5049991277276421,0.9900017445447158);
\draw[line width=1pt,color=qqqqcc] (0.5049991277276421,0.9900017445447158) -- (0.5074991234094621,0.9850017531810757);
\draw[line width=1pt,color=qqqqcc] (0.5074991234094621,0.9850017531810757) -- (0.5099991190912821,0.9800017618174357);
\draw[line width=1pt,color=qqqqcc] (0.5099991190912821,0.9800017618174357) -- (0.5124991147731022,0.9750017704537957);
\draw[line width=1pt,color=qqqqcc] (0.5124991147731022,0.9750017704537957) -- (0.5149991104549222,0.9700017790901556);
\draw[line width=1pt,color=qqqqcc] (0.5149991104549222,0.9700017790901556) -- (0.5174991061367422,0.9650017877265156);
\draw[line width=1pt,color=qqqqcc] (0.5174991061367422,0.9650017877265156) -- (0.5199991018185622,0.9600017963628755);
\draw[line width=1pt,color=qqqqcc] (0.5199991018185622,0.9600017963628755) -- (0.5224990975003823,0.9550018049992355);
\draw[line width=1pt,color=qqqqcc] (0.5224990975003823,0.9550018049992355) -- (0.5249990931822023,0.9500018136355954);
\draw[line width=1pt,color=qqqqcc] (0.5249990931822023,0.9500018136355954) -- (0.5274990888640223,0.9450018222719554);
\draw[line width=1pt,color=qqqqcc] (0.5274990888640223,0.9450018222719554) -- (0.5299990845458423,0.9400018309083154);
\draw[line width=1pt,color=qqqqcc] (0.5299990845458423,0.9400018309083154) -- (0.5324990802276623,0.9350018395446753);
\draw[line width=1pt,color=qqqqcc] (0.5324990802276623,0.9350018395446753) -- (0.5349990759094824,0.9300018481810353);
\draw[line width=1pt,color=qqqqcc] (0.5349990759094824,0.9300018481810353) -- (0.5374990715913024,0.9250018568173952);
\draw[line width=1pt,color=qqqqcc] (0.5374990715913024,0.9250018568173952) -- (0.5399990672731224,0.9200018654537552);
\draw[line width=1pt,color=qqqqcc] (0.5399990672731224,0.9200018654537552) -- (0.5424990629549424,0.9150018740901151);
\draw[line width=1pt,color=qqqqcc] (0.5424990629549424,0.9150018740901151) -- (0.5449990586367625,0.9100018827264751);
\draw[line width=1pt,color=qqqqcc] (0.5449990586367625,0.9100018827264751) -- (0.5474990543185825,0.905001891362835);
\draw[line width=1pt,color=qqqqcc] (0.5474990543185825,0.905001891362835) -- (0.5499990500004025,0.900001899999195);
\draw[line width=1pt,color=qqqqcc] (0.5499990500004025,0.900001899999195) -- (0.5524990456822225,0.895001908635555);
\draw[line width=1pt,color=qqqqcc] (0.5524990456822225,0.895001908635555) -- (0.5549990413640425,0.8900019172719149);
\draw[line width=1pt,color=qqqqcc] (0.5549990413640425,0.8900019172719149) -- (0.5574990370458626,0.8850019259082749);
\draw[line width=1pt,color=qqqqcc] (0.5574990370458626,0.8850019259082749) -- (0.5599990327276826,0.8800019345446348);
\draw[line width=1pt,color=qqqqcc] (0.5599990327276826,0.8800019345446348) -- (0.5624990284095026,0.8750019431809948);
\draw[line width=1pt,color=qqqqcc] (0.5624990284095026,0.8750019431809948) -- (0.5649990240913226,0.8700019518173547);
\draw[line width=1pt,color=qqqqcc] (0.5649990240913226,0.8700019518173547) -- (0.5674990197731427,0.8650019604537147);
\draw[line width=1pt,color=qqqqcc] (0.5674990197731427,0.8650019604537147) -- (0.5699990154549627,0.8600019690900746);
\draw[line width=1pt,color=qqqqcc] (0.5699990154549627,0.8600019690900746) -- (0.5724990111367827,0.8550019777264346);
\draw[line width=1pt,color=qqqqcc] (0.5724990111367827,0.8550019777264346) -- (0.5749990068186027,0.8500019863627946);
\draw[line width=1pt,color=qqqqcc] (0.5749990068186027,0.8500019863627946) -- (0.5774990025004227,0.8450019949991545);
\draw[line width=1pt,color=qqqqcc] (0.5774990025004227,0.8450019949991545) -- (0.5799989981822428,0.8400020036355145);
\draw[line width=1pt,color=qqqqcc] (0.5799989981822428,0.8400020036355145) -- (0.5824989938640628,0.8350020122718744);
\draw[line width=1pt,color=qqqqcc] (0.5824989938640628,0.8350020122718744) -- (0.5849989895458828,0.8300020209082344);
\draw[line width=1pt,color=qqqqcc] (0.5849989895458828,0.8300020209082344) -- (0.5874989852277028,0.8250020295445943);
\draw[line width=1pt,color=qqqqcc] (0.5874989852277028,0.8250020295445943) -- (0.5899989809095229,0.8200020381809543);
\draw[line width=1pt,color=qqqqcc] (0.5899989809095229,0.8200020381809543) -- (0.5924989765913429,0.8150020468173143);
\draw[line width=1pt,color=qqqqcc] (0.5924989765913429,0.8150020468173143) -- (0.5949989722731629,0.8100020554536742);
\draw[line width=1pt,color=qqqqcc] (0.5949989722731629,0.8100020554536742) -- (0.5974989679549829,0.8050020640900342);
\draw[line width=1pt,color=qqqqcc] (0.5974989679549829,0.8050020640900342) -- (0.5999989636368029,0.8000020727263941);
\draw[line width=1pt,color=qqqqcc] (0.5999989636368029,0.8000020727263941) -- (0.602498959318623,0.7950020813627541);
\draw[line width=1pt,color=qqqqcc] (0.602498959318623,0.7950020813627541) -- (0.604998955000443,0.790002089999114);
\draw[line width=1pt,color=qqqqcc] (0.604998955000443,0.790002089999114) -- (0.607498950682263,0.785002098635474);
\draw[line width=1pt,color=qqqqcc] (0.607498950682263,0.785002098635474) -- (0.609998946364083,0.7800021072718339);
\draw[line width=1pt,color=qqqqcc] (0.609998946364083,0.7800021072718339) -- (0.612498942045903,0.7750021159081939);
\draw[line width=1pt,color=qqqqcc] (0.612498942045903,0.7750021159081939) -- (0.6149989377277231,0.7700021245445539);
\draw[line width=1pt,color=qqqqcc] (0.6149989377277231,0.7700021245445539) -- (0.6174989334095431,0.7650021331809138);
\draw[line width=1pt,color=qqqqcc] (0.6174989334095431,0.7650021331809138) -- (0.6199989290913631,0.7600021418172738);
\draw[line width=1pt,color=qqqqcc] (0.6199989290913631,0.7600021418172738) -- (0.6224989247731831,0.7550021504536337);
\draw[line width=1pt,color=qqqqcc] (0.6224989247731831,0.7550021504536337) -- (0.6249989204550032,0.7500021590899937);
\draw[line width=1pt,color=qqqqcc] (0.6249989204550032,0.7500021590899937) -- (0.6274989161368232,0.7450021677263536);
\draw[line width=1pt,color=qqqqcc] (0.6274989161368232,0.7450021677263536) -- (0.6299989118186432,0.7400021763627136);
\draw[line width=1pt,color=qqqqcc] (0.6299989118186432,0.7400021763627136) -- (0.6324989075004632,0.7350021849990735);
\draw[line width=1pt,color=qqqqcc] (0.6324989075004632,0.7350021849990735) -- (0.6349989031822832,0.7300021936354335);
\draw[line width=1pt,color=qqqqcc] (0.6349989031822832,0.7300021936354335) -- (0.6374988988641033,0.7250022022717935);
\draw[line width=1pt,color=qqqqcc] (0.6374988988641033,0.7250022022717935) -- (0.6399988945459233,0.7200022109081534);
\draw[line width=1pt,color=qqqqcc] (0.6399988945459233,0.7200022109081534) -- (0.6424988902277433,0.7150022195445134);
\draw[line width=1pt,color=qqqqcc] (0.6424988902277433,0.7150022195445134) -- (0.6449988859095633,0.7100022281808733);
\draw[line width=1pt,color=qqqqcc] (0.6449988859095633,0.7100022281808733) -- (0.6474988815913834,0.7050022368172333);
\draw[line width=1pt,color=qqqqcc] (0.6474988815913834,0.7050022368172333) -- (0.6499988772732034,0.7000022454535932);
\draw[line width=1pt,color=qqqqcc] (0.6499988772732034,0.7000022454535932) -- (0.6524988729550234,0.6950022540899532);
\draw[line width=1pt,color=qqqqcc] (0.6524988729550234,0.6950022540899532) -- (0.6549988686368434,0.6900022627263132);
\draw[line width=1pt,color=qqqqcc] (0.6549988686368434,0.6900022627263132) -- (0.6574988643186634,0.6850022713626731);
\draw[line width=1pt,color=qqqqcc] (0.6574988643186634,0.6850022713626731) -- (0.6599988600004835,0.6800022799990331);
\draw[line width=1pt,color=qqqqcc] (0.6599988600004835,0.6800022799990331) -- (0.6624988556823035,0.675002288635393);
\draw[line width=1pt,color=qqqqcc] (0.6624988556823035,0.675002288635393) -- (0.6649988513641235,0.670002297271753);
\draw[line width=1pt,color=qqqqcc] (0.6649988513641235,0.670002297271753) -- (0.6674988470459435,0.6650023059081129);
\draw[line width=1pt,color=qqqqcc] (0.6674988470459435,0.6650023059081129) -- (0.6699988427277636,0.6600023145444729);
\draw[line width=1pt,color=qqqqcc] (0.6699988427277636,0.6600023145444729) -- (0.6724988384095836,0.6550023231808328);
\draw[line width=1pt,color=qqqqcc] (0.6724988384095836,0.6550023231808328) -- (0.6749988340914036,0.6500023318171928);
\draw[line width=1pt,color=qqqqcc] (0.6749988340914036,0.6500023318171928) -- (0.6774988297732236,0.6450023404535528);
\draw[line width=1pt,color=qqqqcc] (0.6774988297732236,0.6450023404535528) -- (0.6799988254550436,0.6400023490899127);
\draw[line width=1pt,color=qqqqcc] (0.6799988254550436,0.6400023490899127) -- (0.6824988211368637,0.6350023577262727);
\draw[line width=1pt,color=qqqqcc] (0.6824988211368637,0.6350023577262727) -- (0.6849988168186837,0.6300023663626326);
\draw[line width=1pt,color=qqqqcc] (0.6849988168186837,0.6300023663626326) -- (0.6874988125005037,0.6250023749989926);
\draw[line width=1pt,color=qqqqcc] (0.6874988125005037,0.6250023749989926) -- (0.6899988081823237,0.6200023836353525);
\draw[line width=1pt,color=qqqqcc] (0.6899988081823237,0.6200023836353525) -- (0.6924988038641438,0.6150023922717125);
\draw[line width=1pt,color=qqqqcc] (0.6924988038641438,0.6150023922717125) -- (0.6949987995459638,0.6100024009080725);
\draw[line width=1pt,color=qqqqcc] (0.6949987995459638,0.6100024009080725) -- (0.6974987952277838,0.6050024095444324);
\draw[line width=1pt,color=qqqqcc] (0.6974987952277838,0.6050024095444324) -- (0.6999987909096038,0.6000024181807924);
\draw[line width=1pt,color=qqqqcc] (0.6999987909096038,0.6000024181807924) -- (0.7024987865914238,0.5950024268171523);
\draw[line width=1pt,color=qqqqcc] (0.7024987865914238,0.5950024268171523) -- (0.7049987822732439,0.5900024354535123);
\draw[line width=1pt,color=qqqqcc] (0.7049987822732439,0.5900024354535123) -- (0.7074987779550639,0.5850024440898722);
\draw[line width=1pt,color=qqqqcc] (0.7074987779550639,0.5850024440898722) -- (0.7099987736368839,0.5800024527262322);
\draw[line width=1pt,color=qqqqcc] (0.7099987736368839,0.5800024527262322) -- (0.7124987693187039,0.5750024613625921);
\draw[line width=1pt,color=qqqqcc] (0.7124987693187039,0.5750024613625921) -- (0.714998765000524,0.5700024699989521);
\draw[line width=1pt,color=qqqqcc] (0.714998765000524,0.5700024699989521) -- (0.717498760682344,0.565002478635312);
\draw[line width=1pt,color=qqqqcc] (0.717498760682344,0.565002478635312) -- (0.719998756364164,0.560002487271672);
\draw[line width=1pt,color=qqqqcc] (0.719998756364164,0.560002487271672) -- (0.722498752045984,0.555002495908032);
\draw[line width=1pt,color=qqqqcc] (0.722498752045984,0.555002495908032) -- (0.724998747727804,0.5500025045443919);
\draw[line width=1pt,color=qqqqcc] (0.724998747727804,0.5500025045443919) -- (0.7274987434096241,0.5450025131807519);
\draw[line width=1pt,color=qqqqcc] (0.7274987434096241,0.5450025131807519) -- (0.7299987390914441,0.5400025218171118);
\draw[line width=1pt,color=qqqqcc] (0.7299987390914441,0.5400025218171118) -- (0.7324987347732641,0.5350025304534718);
\draw[line width=1pt,color=qqqqcc] (0.7324987347732641,0.5350025304534718) -- (0.7349987304550841,0.5300025390898317);
\draw[line width=1pt,color=qqqqcc] (0.7349987304550841,0.5300025390898317) -- (0.7374987261369041,0.5250025477261917);
\draw[line width=1pt,color=qqqqcc] (0.7374987261369041,0.5250025477261917) -- (0.7399987218187242,0.5200025563625517);
\draw[line width=1pt,color=qqqqcc] (0.7399987218187242,0.5200025563625517) -- (0.7424987175005442,0.5150025649989116);
\draw[line width=1pt,color=qqqqcc] (0.7424987175005442,0.5150025649989116) -- (0.7449987131823642,0.5100025736352716);
\draw[line width=1pt,color=qqqqcc] (0.7449987131823642,0.5100025736352716) -- (0.7474987088641842,0.5050025822716315);
\draw[line width=1pt,color=qqqqcc] (0.7474987088641842,0.5050025822716315) -- (0.7499987045460043,0.5000025909079915);
\draw[line width=1pt,color=qqqqcc] (0.7499987045460043,0.5000025909079915) -- (0.7524987002278243,0.49500259954435144);
\draw[line width=1pt,color=qqqqcc] (0.7524987002278243,0.49500259954435144) -- (0.7549986959096443,0.4900026081807114);
\draw[line width=1pt,color=qqqqcc] (0.7549986959096443,0.4900026081807114) -- (0.7574986915914643,0.48500261681707135);
\draw[line width=1pt,color=qqqqcc] (0.7574986915914643,0.48500261681707135) -- (0.7599986872732843,0.4800026254534313);
\draw[line width=1pt,color=qqqqcc] (0.7599986872732843,0.4800026254534313) -- (0.7624986829551044,0.47500263408979126);
\draw[line width=1pt,color=qqqqcc] (0.7624986829551044,0.47500263408979126) -- (0.7649986786369244,0.4700026427261512);
\draw[line width=1pt,color=qqqqcc] (0.7649986786369244,0.4700026427261512) -- (0.7674986743187444,0.4650026513625112);
\draw[line width=1pt,color=qqqqcc] (0.7674986743187444,0.4650026513625112) -- (0.7699986700005644,0.46000265999887113);
\draw[line width=1pt,color=qqqqcc] (0.7699986700005644,0.46000265999887113) -- (0.7724986656823845,0.4550026686352311);
\draw[line width=1pt,color=qqqqcc] (0.7724986656823845,0.4550026686352311) -- (0.7749986613642045,0.45000267727159105);
\draw[line width=1pt,color=qqqqcc] (0.7749986613642045,0.45000267727159105) -- (0.7774986570460245,0.445002685907951);
\draw[line width=1pt,color=qqqqcc] (0.7774986570460245,0.445002685907951) -- (0.7799986527278445,0.44000269454431096);
\draw[line width=1pt,color=qqqqcc] (0.7799986527278445,0.44000269454431096) -- (0.7824986484096645,0.4350027031806709);
\draw[line width=1pt,color=qqqqcc] (0.7824986484096645,0.4350027031806709) -- (0.7849986440914846,0.43000271181703087);
\draw[line width=1pt,color=qqqqcc] (0.7849986440914846,0.43000271181703087) -- (0.7874986397733046,0.4250027204533908);
\draw[line width=1pt,color=qqqqcc] (0.7874986397733046,0.4250027204533908) -- (0.7899986354551246,0.4200027290897508);
\draw[line width=1pt,color=qqqqcc] (0.7899986354551246,0.4200027290897508) -- (0.7924986311369446,0.41500273772611074);
\draw[line width=1pt,color=qqqqcc] (0.7924986311369446,0.41500273772611074) -- (0.7949986268187647,0.4100027463624707);
\draw[line width=1pt,color=qqqqcc] (0.7949986268187647,0.4100027463624707) -- (0.7974986225005847,0.40500275499883065);
\draw[line width=1pt,color=qqqqcc] (0.7974986225005847,0.40500275499883065) -- (0.7999986181824047,0.4000027636351906);
\draw[line width=1pt,color=qqqqcc] (0.7999986181824047,0.4000027636351906) -- (0.8024986138642247,0.39500277227155056);
\draw[line width=1pt,color=qqqqcc] (0.8024986138642247,0.39500277227155056) -- (0.8049986095460447,0.3900027809079105);
\draw[line width=1pt,color=qqqqcc] (0.8049986095460447,0.3900027809079105) -- (0.8074986052278648,0.3850027895442705);
\draw[line width=1pt,color=qqqqcc] (0.8074986052278648,0.3850027895442705) -- (0.8099986009096848,0.38000279818063043);
\draw[line width=1pt,color=qqqqcc] (0.8099986009096848,0.38000279818063043) -- (0.8124985965915048,0.3750028068169904);
\draw[line width=1pt,color=qqqqcc] (0.8124985965915048,0.3750028068169904) -- (0.8149985922733248,0.37000281545335034);
\draw[line width=1pt,color=qqqqcc] (0.8149985922733248,0.37000281545335034) -- (0.8174985879551449,0.3650028240897103);
\draw[line width=1pt,color=qqqqcc] (0.8174985879551449,0.3650028240897103) -- (0.8199985836369649,0.36000283272607025);
\draw[line width=1pt,color=qqqqcc] (0.8199985836369649,0.36000283272607025) -- (0.8224985793187849,0.3550028413624302);
\draw[line width=1pt,color=qqqqcc] (0.8224985793187849,0.3550028413624302) -- (0.8249985750006049,0.35000284999879017);
\draw[line width=1pt,color=qqqqcc] (0.8249985750006049,0.35000284999879017) -- (0.8274985706824249,0.3450028586351501);
\draw[line width=1pt,color=qqqqcc] (0.8274985706824249,0.3450028586351501) -- (0.829998566364245,0.3400028672715101);
\draw[line width=1pt,color=qqqqcc] (0.829998566364245,0.3400028672715101) -- (0.832498562046065,0.33500287590787003);
\draw[line width=1pt,color=qqqqcc] (0.832498562046065,0.33500287590787003) -- (0.834998557727885,0.33000288454423);
\draw[line width=1pt,color=qqqqcc] (0.834998557727885,0.33000288454423) -- (0.837498553409705,0.32500289318058995);
\draw[line width=1pt,color=qqqqcc] (0.837498553409705,0.32500289318058995) -- (0.839998549091525,0.3200029018169499);
\draw[line width=1pt,color=qqqqcc] (0.839998549091525,0.3200029018169499) -- (0.8424985447733451,0.31500291045330986);
\draw[line width=1pt,color=qqqqcc] (0.8424985447733451,0.31500291045330986) -- (0.8449985404551651,0.3100029190896698);
\draw[line width=1pt,color=qqqqcc] (0.8449985404551651,0.3100029190896698) -- (0.8474985361369851,0.30500292772602977);
\draw[line width=1pt,color=qqqqcc] (0.8474985361369851,0.30500292772602977) -- (0.8499985318188051,0.3000029363623897);
\draw[line width=1pt,color=qqqqcc] (0.8499985318188051,0.3000029363623897) -- (0.8524985275006252,0.2950029449987497);
\draw[line width=1pt,color=qqqqcc] (0.8524985275006252,0.2950029449987497) -- (0.8549985231824452,0.29000295363510964);
\draw[line width=1pt,color=qqqqcc] (0.8549985231824452,0.29000295363510964) -- (0.8574985188642652,0.2850029622714696);
\draw[line width=1pt,color=qqqqcc] (0.8574985188642652,0.2850029622714696) -- (0.8599985145460852,0.28000297090782955);
\draw[line width=1pt,color=qqqqcc] (0.8599985145460852,0.28000297090782955) -- (0.8624985102279052,0.2750029795441895);
\draw[line width=1pt,color=qqqqcc] (0.8624985102279052,0.2750029795441895) -- (0.8649985059097253,0.27000298818054946);
\draw[line width=1pt,color=qqqqcc] (0.8649985059097253,0.27000298818054946) -- (0.8674985015915453,0.2650029968169094);
\draw[line width=1pt,color=qqqqcc] (0.8674985015915453,0.2650029968169094) -- (0.8699984972733653,0.2600030054532694);
\draw[line width=1pt,color=qqqqcc] (0.8699984972733653,0.2600030054532694) -- (0.8724984929551853,0.25500301408962933);
\draw[line width=1pt,color=qqqqcc] (0.8724984929551853,0.25500301408962933) -- (0.8749984886370054,0.2500030227259893);
\draw[line width=1pt,color=qqqqcc] (0.8749984886370054,0.2500030227259893) -- (0.8774984843188254,0.24500303136234924);
\draw[line width=1pt,color=qqqqcc] (0.8774984843188254,0.24500303136234924) -- (0.8799984800006454,0.2400030399987092);
\draw[line width=1pt,color=qqqqcc] (0.8799984800006454,0.2400030399987092) -- (0.8824984756824654,0.23500304863506916);
\draw[line width=1pt,color=qqqqcc] (0.8824984756824654,0.23500304863506916) -- (0.8849984713642854,0.2300030572714291);
\draw[line width=1pt,color=qqqqcc] (0.8849984713642854,0.2300030572714291) -- (0.8874984670461055,0.22500306590778907);
\draw[line width=1pt,color=qqqqcc] (0.8874984670461055,0.22500306590778907) -- (0.8899984627279255,0.22000307454414902);
\draw[line width=1pt,color=qqqqcc] (0.8899984627279255,0.22000307454414902) -- (0.8924984584097455,0.21500308318050898);
\draw[line width=1pt,color=qqqqcc] (0.8924984584097455,0.21500308318050898) -- (0.8949984540915655,0.21000309181686894);
\draw[line width=1pt,color=qqqqcc] (0.8949984540915655,0.21000309181686894) -- (0.8974984497733856,0.2050031004532289);
\draw[line width=1pt,color=qqqqcc] (0.8974984497733856,0.2050031004532289) -- (0.8999984454552056,0.20000310908958885);
\draw[line width=1pt,color=qqqqcc] (0.8999984454552056,0.20000310908958885) -- (0.9024984411370256,0.1950031177259488);
\draw[line width=1pt,color=qqqqcc] (0.9024984411370256,0.1950031177259488) -- (0.9049984368188456,0.19000312636230876);
\draw[line width=1pt,color=qqqqcc] (0.9049984368188456,0.19000312636230876) -- (0.9074984325006656,0.18500313499866872);
\draw[line width=1pt,color=qqqqcc] (0.9074984325006656,0.18500313499866872) -- (0.9099984281824857,0.18000314363502867);
\draw[line width=1pt,color=qqqqcc] (0.9099984281824857,0.18000314363502867) -- (0.9124984238643057,0.17500315227138863);
\draw[line width=1pt,color=qqqqcc] (0.9124984238643057,0.17500315227138863) -- (0.9149984195461257,0.17000316090774859);
\draw[line width=1pt,color=qqqqcc] (0.9149984195461257,0.17000316090774859) -- (0.9174984152279457,0.16500316954410854);
\draw[line width=1pt,color=qqqqcc] (0.9174984152279457,0.16500316954410854) -- (0.9199984109097658,0.1600031781804685);
\draw[line width=1pt,color=qqqqcc] (0.9199984109097658,0.1600031781804685) -- (0.9224984065915858,0.15500318681682845);
\draw[line width=1pt,color=qqqqcc] (0.9224984065915858,0.15500318681682845) -- (0.9249984022734058,0.1500031954531884);
\draw[line width=1pt,color=qqqqcc] (0.9249984022734058,0.1500031954531884) -- (0.9274983979552258,0.14500320408954837);
\draw[line width=1pt,color=qqqqcc] (0.9274983979552258,0.14500320408954837) -- (0.9299983936370458,0.14000321272590832);
\draw[line width=1pt,color=qqqqcc] (0.9299983936370458,0.14000321272590832) -- (0.9324983893188659,0.13500322136226828);
\draw[line width=1pt,color=qqqqcc] (0.9324983893188659,0.13500322136226828) -- (0.9349983850006859,0.13000322999862823);
\draw[line width=1pt,color=qqqqcc] (0.9349983850006859,0.13000322999862823) -- (0.9374983806825059,0.1250032386349882);
\draw[line width=1pt,color=qqqqcc] (0.9374983806825059,0.1250032386349882) -- (0.9399983763643259,0.12000324727134815);
\draw[line width=1pt,color=qqqqcc] (0.9399983763643259,0.12000324727134815) -- (0.942498372046146,0.1150032559077081);
\draw[line width=1pt,color=qqqqcc] (0.942498372046146,0.1150032559077081) -- (0.944998367727966,0.11000326454406806);
\draw[line width=1pt,color=qqqqcc] (0.944998367727966,0.11000326454406806) -- (0.947498363409786,0.10500327318042801);
\draw[line width=1pt,color=qqqqcc] (0.947498363409786,0.10500327318042801) -- (0.949998359091606,0.10000328181678797);
\draw[line width=1pt,color=qqqqcc] (0.949998359091606,0.10000328181678797) -- (0.952498354773426,0.09500329045314793);
\draw[line width=1pt,color=qqqqcc] (0.952498354773426,0.09500329045314793) -- (0.9549983504552461,0.09000329908950788);
\draw[line width=1pt,color=qqqqcc] (0.9549983504552461,0.09000329908950788) -- (0.9574983461370661,0.08500330772586784);
\draw[line width=1pt,color=qqqqcc] (0.9574983461370661,0.08500330772586784) -- (0.9599983418188861,0.0800033163622278);
\draw[line width=1pt,color=qqqqcc] (0.9599983418188861,0.0800033163622278) -- (0.9624983375007061,0.07500332499858775);
\draw[line width=1pt,color=qqqqcc] (0.9624983375007061,0.07500332499858775) -- (0.9649983331825261,0.0700033336349477);
\draw[line width=1pt,color=qqqqcc] (0.9649983331825261,0.0700033336349477) -- (0.9674983288643462,0.06500334227130766);
\draw[line width=1pt,color=qqqqcc] (0.9674983288643462,0.06500334227130766) -- (0.9699983245461662,0.06000335090766762);
\draw[line width=1pt,color=qqqqcc] (0.9699983245461662,0.06000335090766762) -- (0.9724983202279862,0.055003359544027575);
\draw[line width=1pt,color=qqqqcc] (0.9724983202279862,0.055003359544027575) -- (0.9749983159098062,0.05000336818038753);
\draw[line width=1pt,color=qqqqcc] (0.9749983159098062,0.05000336818038753) -- (0.9774983115916263,0.04500337681674749);
\draw[line width=1pt,color=qqqqcc] (0.9774983115916263,0.04500337681674749) -- (0.9799983072734463,0.04000338545310744);
\draw[line width=1pt,color=qqqqcc] (0.9799983072734463,0.04000338545310744) -- (0.9824983029552663,0.0350033940894674);
\draw[line width=1pt,color=qqqqcc] (0.9824983029552663,0.0350033940894674) -- (0.9849982986370863,0.030003402725827355);
\draw[line width=1pt,color=qqqqcc] (0.9849982986370863,0.030003402725827355) -- (0.9874982943189063,0.02500341136218731);
\draw[line width=1pt,color=qqqqcc] (0.9874982943189063,0.02500341136218731) -- (0.9899982900007264,0.020003419998547267);
\draw[line width=1pt,color=qqqqcc] (0.9899982900007264,0.020003419998547267) -- (0.9924982856825464,0.015003428634907223);
\draw[line width=1pt,color=qqqqcc] (0.9924982856825464,0.015003428634907223) -- (0.9949982813643664,0.01000343727126718);
\draw[line width=1pt,color=qqqqcc] (0.9949982813643664,0.01000343727126718) -- (0.9974982770461864,0.005003445907627135);
\draw[line width=1pt,color=qqqqcc] (0.9974982770461864,0.005003445907627135) -- (0.9999982727280065,0.0);
\draw [line width=0.8pt] (0.,1.)-- (1.,1.);
\draw [line width=0.8pt] (1.,1.)-- (1.,0.);
\draw [line width=0.8pt] (0.,1.)-- (0.,0.);
\draw [line width=0.8pt] (0.,0.)-- (1.,0.);
\end{tikzpicture}
 }
            \subfloat[Trasformazione $ Q_4 $]
            { \definecolor{ttzzqq}{rgb}{0.2,0.6,0.}
\begin{tikzpicture}[line cap=round,line join=round,>=triangle 45,x=5.0cm,y=5.0cm]
\draw[->,color=black] (-0.1,0.) -- (1.1,0.);
\foreach \x in {,0.2,0.4,0.6,0.8,1.}
\draw[shift={(\x,0)},color=black] (0pt,2pt) -- (0pt,-2pt) node[below] {\footnotesize $\x$};
\draw[->,color=black] (0.,-0.1) -- (0.,1.1);
\foreach \y in {,0.2,0.4,0.6,0.8,1.}
\draw[shift={(0,\y)},color=black] (2pt,0pt) -- (-2pt,0pt) node[left] {\footnotesize $\y$};
\draw[color=black] (0pt,-10pt) node[right] {\footnotesize $0$};
\clip(-0.1,-0.1) rectangle (1.1,1.1);
\draw [line width=0.8pt] (0.,1.)-- (1.,1.);
\draw [line width=0.8pt] (1.,1.)-- (1.,0.);
\draw[line width=1pt,color=ttzzqq] (8.000000000003847E-7,0.0) -- (0.0,0.0);
\draw[line width=1pt,color=ttzzqq] (0.0,0.0) -- (0.0024999956818200016,0.009974982813643531);
\draw[line width=1pt,color=ttzzqq] (0.0024999956818200016,0.009974982813643531) -- (0.004999991363640003,0.019899965800014113);
\draw[line width=1pt,color=ttzzqq] (0.004999991363640003,0.019899965800014113) -- (0.007499987045460005,0.02977494895911175);
\draw[line width=1pt,color=ttzzqq] (0.007499987045460005,0.02977494895911175) -- (0.009999982727280006,0.03959993229093643);
\draw[line width=1pt,color=ttzzqq] (0.009999982727280006,0.03959993229093643) -- (0.012499978409100007,0.04937491579548817);
\draw[line width=1pt,color=ttzzqq] (0.012499978409100007,0.04937491579548817) -- (0.014999974090920009,0.05909989947276695);
\draw[line width=1pt,color=ttzzqq] (0.014999974090920009,0.05909989947276695) -- (0.01749996977274001,0.06877488332277279);
\draw[line width=1pt,color=ttzzqq] (0.01749996977274001,0.06877488332277279) -- (0.019999965454560013,0.07839986734550568);
\draw[line width=1pt,color=ttzzqq] (0.019999965454560013,0.07839986734550568) -- (0.022499961136380014,0.08797485154096561);
\draw[line width=1pt,color=ttzzqq] (0.022499961136380014,0.08797485154096561) -- (0.024999956818200015,0.0974998359091526);
\draw[line width=1pt,color=ttzzqq] (0.024999956818200015,0.0974998359091526) -- (0.027499952500020016,0.10697482045006663);
\draw[line width=1pt,color=ttzzqq] (0.027499952500020016,0.10697482045006663) -- (0.029999948181840017,0.11639980516370772);
\draw[line width=1pt,color=ttzzqq] (0.029999948181840017,0.11639980516370772) -- (0.03249994386366002,0.12577479005007586);
\draw[line width=1pt,color=ttzzqq] (0.03249994386366002,0.12577479005007586) -- (0.03499993954548002,0.13509977510917107);
\draw[line width=1pt,color=ttzzqq] (0.03499993954548002,0.13509977510917107) -- (0.037499935227300024,0.1443747603409933);
\draw[line width=1pt,color=ttzzqq] (0.037499935227300024,0.1443747603409933) -- (0.039999930909120025,0.1535997457455426);
\draw[line width=1pt,color=ttzzqq] (0.039999930909120025,0.1535997457455426) -- (0.042499926590940026,0.16277473132281894);
\draw[line width=1pt,color=ttzzqq] (0.042499926590940026,0.16277473132281894) -- (0.04499992227276003,0.17189971707282234);
\draw[line width=1pt,color=ttzzqq] (0.04499992227276003,0.17189971707282234) -- (0.04749991795458003,0.18097470299555277);
\draw[line width=1pt,color=ttzzqq] (0.04749991795458003,0.18097470299555277) -- (0.04999991363640003,0.18999968909101028);
\draw[line width=1pt,color=ttzzqq] (0.04999991363640003,0.18999968909101028) -- (0.05249990931822003,0.19897467535919483);
\draw[line width=1pt,color=ttzzqq] (0.05249990931822003,0.19897467535919483) -- (0.05499990500004003,0.2078996618001064);
\draw[line width=1pt,color=ttzzqq] (0.05499990500004003,0.2078996618001064) -- (0.05749990068186003,0.21677464841374505);
\draw[line width=1pt,color=ttzzqq] (0.05749990068186003,0.21677464841374505) -- (0.059999896363680034,0.22559963520011075);
\draw[line width=1pt,color=ttzzqq] (0.059999896363680034,0.22559963520011075) -- (0.062499892045500036,0.23437462215920352);
\draw[line width=1pt,color=ttzzqq] (0.062499892045500036,0.23437462215920352) -- (0.06499988772732004,0.24309960929102334);
\draw[line width=1pt,color=ttzzqq] (0.06499988772732004,0.24309960929102334) -- (0.06749988340914005,0.2517745965955702);
\draw[line width=1pt,color=ttzzqq] (0.06749988340914005,0.2517745965955702) -- (0.06999987909096006,0.26039958407284414);
\draw[line width=1pt,color=ttzzqq] (0.06999987909096006,0.26039958407284414) -- (0.07249987477278007,0.2689745717228451);
\draw[line width=1pt,color=ttzzqq] (0.07249987477278007,0.2689745717228451) -- (0.07499987045460008,0.27749955954557315);
\draw[line width=1pt,color=ttzzqq] (0.07499987045460008,0.27749955954557315) -- (0.07749986613642008,0.2859745475410282);
\draw[line width=1pt,color=ttzzqq] (0.07749986613642008,0.2859745475410282) -- (0.07999986181824009,0.29439953570921035);
\draw[line width=1pt,color=ttzzqq] (0.07999986181824009,0.29439953570921035) -- (0.0824998575000601,0.3027745240501195);
\draw[line width=1pt,color=ttzzqq] (0.0824998575000601,0.3027745240501195) -- (0.08499985318188011,0.31109951256375573);
\draw[line width=1pt,color=ttzzqq] (0.08499985318188011,0.31109951256375573) -- (0.08749984886370012,0.319374501250119);
\draw[line width=1pt,color=ttzzqq] (0.08749984886370012,0.319374501250119) -- (0.08999984454552012,0.32759949010920936);
\draw[line width=1pt,color=ttzzqq] (0.08999984454552012,0.32759949010920936) -- (0.09249984022734013,0.3357744791410267);
\draw[line width=1pt,color=ttzzqq] (0.09249984022734013,0.3357744791410267) -- (0.09499983590916014,0.34389946834557117);
\draw[line width=1pt,color=ttzzqq] (0.09499983590916014,0.34389946834557117) -- (0.09749983159098015,0.35197445772284264);
\draw[line width=1pt,color=ttzzqq] (0.09749983159098015,0.35197445772284264) -- (0.09999982727280016,0.35999944727284117);
\draw[line width=1pt,color=ttzzqq] (0.09999982727280016,0.35999944727284117) -- (0.10249982295462017,0.36797443699556676);
\draw[line width=1pt,color=ttzzqq] (0.10249982295462017,0.36797443699556676) -- (0.10499981863644017,0.3758994268910194);
\draw[line width=1pt,color=ttzzqq] (0.10499981863644017,0.3758994268910194) -- (0.10749981431826018,0.38377441695919906);
\draw[line width=1pt,color=ttzzqq] (0.10749981431826018,0.38377441695919906) -- (0.10999981000008019,0.3915994072001058);
\draw[line width=1pt,color=ttzzqq] (0.10999981000008019,0.3915994072001058) -- (0.1124998056819002,0.39937439761373955);
\draw[line width=1pt,color=ttzzqq] (0.1124998056819002,0.39937439761373955) -- (0.1149998013637202,0.4070993882001004);
\draw[line width=1pt,color=ttzzqq] (0.1149998013637202,0.4070993882001004) -- (0.11749979704554021,0.4147743789591883);
\draw[line width=1pt,color=ttzzqq] (0.11749979704554021,0.4147743789591883) -- (0.11999979272736022,0.42239936989100324);
\draw[line width=1pt,color=ttzzqq] (0.11999979272736022,0.42239936989100324) -- (0.12249978840918023,0.42997436099554526);
\draw[line width=1pt,color=ttzzqq] (0.12249978840918023,0.42997436099554526) -- (0.12499978409100024,0.4374993522728143);
\draw[line width=1pt,color=ttzzqq] (0.12499978409100024,0.4374993522728143) -- (0.12749977977282023,0.4449743437228103);
\draw[line width=1pt,color=ttzzqq] (0.12749977977282023,0.4449743437228103) -- (0.12999977545464023,0.4523993353455334);
\draw[line width=1pt,color=ttzzqq] (0.12999977545464023,0.4523993353455334) -- (0.13249977113646022,0.45977432714098354);
\draw[line width=1pt,color=ttzzqq] (0.13249977113646022,0.45977432714098354) -- (0.13499976681828021,0.46709931910916075);
\draw[line width=1pt,color=ttzzqq] (0.13499976681828021,0.46709931910916075) -- (0.1374997625001002,0.47437431125006496);
\draw[line width=1pt,color=ttzzqq] (0.1374997625001002,0.47437431125006496) -- (0.1399997581819202,0.4815993035636963);
\draw[line width=1pt,color=ttzzqq] (0.1399997581819202,0.4815993035636963) -- (0.1424997538637402,0.4887742960500546);
\draw[line width=1pt,color=ttzzqq] (0.1424997538637402,0.4887742960500546) -- (0.1449997495455602,0.4958992887091401);
\draw[line width=1pt,color=ttzzqq] (0.1449997495455602,0.4958992887091401) -- (0.14749974522738019,0.5029742815409525);
\draw[line width=1pt,color=ttzzqq] (0.14749974522738019,0.5029742815409525) -- (0.14999974090920018,0.509999274545492);
\draw[line width=1pt,color=ttzzqq] (0.14999974090920018,0.509999274545492) -- (0.15249973659102017,0.5169742677227586);
\draw[line width=1pt,color=ttzzqq] (0.15249973659102017,0.5169742677227586) -- (0.15499973227284017,0.5238992610727522);
\draw[line width=1pt,color=ttzzqq] (0.15499973227284017,0.5238992610727522) -- (0.15749972795466016,0.5307742545954728);
\draw[line width=1pt,color=ttzzqq] (0.15749972795466016,0.5307742545954728) -- (0.15999972363648016,0.5375992482909205);
\draw[line width=1pt,color=ttzzqq] (0.15999972363648016,0.5375992482909205) -- (0.16249971931830015,0.5443742421590952);
\draw[line width=1pt,color=ttzzqq] (0.16249971931830015,0.5443742421590952) -- (0.16499971500012015,0.5510992361999971);
\draw[line width=1pt,color=ttzzqq] (0.16499971500012015,0.5510992361999971) -- (0.16749971068194014,0.557774230413626);
\draw[line width=1pt,color=ttzzqq] (0.16749971068194014,0.557774230413626) -- (0.16999970636376013,0.5643992247999818);
\draw[line width=1pt,color=ttzzqq] (0.16999970636376013,0.5643992247999818) -- (0.17249970204558013,0.5709742193590649);
\draw[line width=1pt,color=ttzzqq] (0.17249970204558013,0.5709742193590649) -- (0.17499969772740012,0.5774992140908749);
\draw[line width=1pt,color=ttzzqq] (0.17499969772740012,0.5774992140908749) -- (0.17749969340922012,0.5839742089954119);
\draw[line width=1pt,color=ttzzqq] (0.17749969340922012,0.5839742089954119) -- (0.1799996890910401,0.590399204072676);
\draw[line width=1pt,color=ttzzqq] (0.1799996890910401,0.590399204072676) -- (0.1824996847728601,0.5967741993226672);
\draw[line width=1pt,color=ttzzqq] (0.1824996847728601,0.5967741993226672) -- (0.1849996804546801,0.6030991947453854);
\draw[line width=1pt,color=ttzzqq] (0.1849996804546801,0.6030991947453854) -- (0.1874996761365001,0.6093741903408307);
\draw[line width=1pt,color=ttzzqq] (0.1874996761365001,0.6093741903408307) -- (0.1899996718183201,0.615599186109003);
\draw[line width=1pt,color=ttzzqq] (0.1899996718183201,0.615599186109003) -- (0.19249966750014008,0.6217741820499024);
\draw[line width=1pt,color=ttzzqq] (0.19249966750014008,0.6217741820499024) -- (0.19499966318196008,0.6278991781635288);
\draw[line width=1pt,color=ttzzqq] (0.19499966318196008,0.6278991781635288) -- (0.19749965886378007,0.6339741744498822);
\draw[line width=1pt,color=ttzzqq] (0.19749965886378007,0.6339741744498822) -- (0.19999965454560006,0.6399991709089627);
\draw[line width=1pt,color=ttzzqq] (0.19999965454560006,0.6399991709089627) -- (0.20249965022742006,0.6459741675407703);
\draw[line width=1pt,color=ttzzqq] (0.20249965022742006,0.6459741675407703) -- (0.20499964590924005,0.651899164345305);
\draw[line width=1pt,color=ttzzqq] (0.20499964590924005,0.651899164345305) -- (0.20749964159106005,0.6577741613225667);
\draw[line width=1pt,color=ttzzqq] (0.20749964159106005,0.6577741613225667) -- (0.20999963727288004,0.6635991584725554);
\draw[line width=1pt,color=ttzzqq] (0.20999963727288004,0.6635991584725554) -- (0.21249963295470004,0.6693741557952713);
\draw[line width=1pt,color=ttzzqq] (0.21249963295470004,0.6693741557952713) -- (0.21499962863652003,0.675099153290714);
\draw[line width=1pt,color=ttzzqq] (0.21499962863652003,0.675099153290714) -- (0.21749962431834002,0.6807741509588839);
\draw[line width=1pt,color=ttzzqq] (0.21749962431834002,0.6807741509588839) -- (0.21999962000016002,0.6863991487997808);
\draw[line width=1pt,color=ttzzqq] (0.21999962000016002,0.6863991487997808) -- (0.22249961568198,0.6919741468134047);
\draw[line width=1pt,color=ttzzqq] (0.22249961568198,0.6919741468134047) -- (0.2249996113638,0.6974991449997558);
\draw[line width=1pt,color=ttzzqq] (0.2249996113638,0.6974991449997558) -- (0.22749960704562,0.7029741433588339);
\draw[line width=1pt,color=ttzzqq] (0.22749960704562,0.7029741433588339) -- (0.22999960272744,0.708399141890639);
\draw[line width=1pt,color=ttzzqq] (0.22999960272744,0.708399141890639) -- (0.23249959840926,0.7137741405951713);
\draw[line width=1pt,color=ttzzqq] (0.23249959840926,0.7137741405951713) -- (0.23499959409107998,0.7190991394724305);
\draw[line width=1pt,color=ttzzqq] (0.23499959409107998,0.7190991394724305) -- (0.23749958977289998,0.7243741385224168);
\draw[line width=1pt,color=ttzzqq] (0.23749958977289998,0.7243741385224168) -- (0.23999958545471997,0.7295991377451301);
\draw[line width=1pt,color=ttzzqq] (0.23999958545471997,0.7295991377451301) -- (0.24249958113653997,0.7347741371405705);
\draw[line width=1pt,color=ttzzqq] (0.24249958113653997,0.7347741371405705) -- (0.24499957681835996,0.739899136708738);
\draw[line width=1pt,color=ttzzqq] (0.24499957681835996,0.739899136708738) -- (0.24749957250017995,0.7449741364496325);
\draw[line width=1pt,color=ttzzqq] (0.24749957250017995,0.7449741364496325) -- (0.24999956818199995,0.749999136363254);
\draw[line width=1pt,color=ttzzqq] (0.24999956818199995,0.749999136363254) -- (0.25249956386381994,0.7549741364496027);
\draw[line width=1pt,color=ttzzqq] (0.25249956386381994,0.7549741364496027) -- (0.25499955954563996,0.7598991367086783);
\draw[line width=1pt,color=ttzzqq] (0.25499955954563996,0.7598991367086783) -- (0.25749955522746,0.7647741371404811);
\draw[line width=1pt,color=ttzzqq] (0.25749955522746,0.7647741371404811) -- (0.25999955090928,0.7695991377450109);
\draw[line width=1pt,color=ttzzqq] (0.25999955090928,0.7695991377450109) -- (0.26249954659110003,0.7743741385222678);
\draw[line width=1pt,color=ttzzqq] (0.26249954659110003,0.7743741385222678) -- (0.26499954227292005,0.7790991394722516);
\draw[line width=1pt,color=ttzzqq] (0.26499954227292005,0.7790991394722516) -- (0.2674995379547401,0.7837741405949626);
\draw[line width=1pt,color=ttzzqq] (0.2674995379547401,0.7837741405949626) -- (0.2699995336365601,0.7883991418904005);
\draw[line width=1pt,color=ttzzqq] (0.2699995336365601,0.7883991418904005) -- (0.2724995293183801,0.7929741433585658);
\draw[line width=1pt,color=ttzzqq] (0.2724995293183801,0.7929741433585658) -- (0.27499952500020014,0.7974991449994577);
\draw[line width=1pt,color=ttzzqq] (0.27499952500020014,0.7974991449994577) -- (0.27749952068202016,0.801974146813077);
\draw[line width=1pt,color=ttzzqq] (0.27749952068202016,0.801974146813077) -- (0.2799995163638402,0.8063991487994231);
\draw[line width=1pt,color=ttzzqq] (0.2799995163638402,0.8063991487994231) -- (0.2824995120456602,0.8107741509584965);
\draw[line width=1pt,color=ttzzqq] (0.2824995120456602,0.8107741509584965) -- (0.28499950772748023,0.8150991532902966);
\draw[line width=1pt,color=ttzzqq] (0.28499950772748023,0.8150991532902966) -- (0.28749950340930025,0.8193741557948241);
\draw[line width=1pt,color=ttzzqq] (0.28749950340930025,0.8193741557948241) -- (0.28999949909112027,0.8235991584720783);
\draw[line width=1pt,color=ttzzqq] (0.28999949909112027,0.8235991584720783) -- (0.2924994947729403,0.8277741613220599);
\draw[line width=1pt,color=ttzzqq] (0.2924994947729403,0.8277741613220599) -- (0.2949994904547603,0.8318991643447683);
\draw[line width=1pt,color=ttzzqq] (0.2949994904547603,0.8318991643447683) -- (0.29749948613658034,0.835974167540204);
\draw[line width=1pt,color=ttzzqq] (0.29749948613658034,0.835974167540204) -- (0.29999948181840036,0.8399991709083665);
\draw[line width=1pt,color=ttzzqq] (0.29999948181840036,0.8399991709083665) -- (0.3024994775002204,0.8439741744492563);
\draw[line width=1pt,color=ttzzqq] (0.3024994775002204,0.8439741744492563) -- (0.3049994731820404,0.8478991781628729);
\draw[line width=1pt,color=ttzzqq] (0.3049994731820404,0.8478991781628729) -- (0.3074994688638604,0.8517741820492167);
\draw[line width=1pt,color=ttzzqq] (0.3074994688638604,0.8517741820492167) -- (0.30999946454568045,0.8555991861082873);
\draw[line width=1pt,color=ttzzqq] (0.30999946454568045,0.8555991861082873) -- (0.31249946022750047,0.8593741903400853);
\draw[line width=1pt,color=ttzzqq] (0.31249946022750047,0.8593741903400853) -- (0.3149994559093205,0.8630991947446102);
\draw[line width=1pt,color=ttzzqq] (0.3149994559093205,0.8630991947446102) -- (0.3174994515911405,0.8667741993218622);
\draw[line width=1pt,color=ttzzqq] (0.3174994515911405,0.8667741993218622) -- (0.31999944727296054,0.8703992040718411);
\draw[line width=1pt,color=ttzzqq] (0.31999944727296054,0.8703992040718411) -- (0.32249944295478056,0.8739742089945473);
\draw[line width=1pt,color=ttzzqq] (0.32249944295478056,0.8739742089945473) -- (0.3249994386366006,0.8774992140899802);
\draw[line width=1pt,color=ttzzqq] (0.3249994386366006,0.8774992140899802) -- (0.3274994343184206,0.8809742193581405);
\draw[line width=1pt,color=ttzzqq] (0.3274994343184206,0.8809742193581405) -- (0.3299994300002406,0.8843992247990276);
\draw[line width=1pt,color=ttzzqq] (0.3299994300002406,0.8843992247990276) -- (0.33249942568206065,0.887774230412642);
\draw[line width=1pt,color=ttzzqq] (0.33249942568206065,0.887774230412642) -- (0.33499942136388067,0.8910992361989831);
\draw[line width=1pt,color=ttzzqq] (0.33499942136388067,0.8910992361989831) -- (0.3374994170457007,0.8943742421580516);
\draw[line width=1pt,color=ttzzqq] (0.3374994170457007,0.8943742421580516) -- (0.3399994127275207,0.8975992482898468);
\draw[line width=1pt,color=ttzzqq] (0.3399994127275207,0.8975992482898468) -- (0.34249940840934073,0.9007742545943694);
\draw[line width=1pt,color=ttzzqq] (0.34249940840934073,0.9007742545943694) -- (0.34499940409116076,0.9038992610716188);
\draw[line width=1pt,color=ttzzqq] (0.34499940409116076,0.9038992610716188) -- (0.3474993997729808,0.9069742677215955);
\draw[line width=1pt,color=ttzzqq] (0.3474993997729808,0.9069742677215955) -- (0.3499993954548008,0.909999274544299);
\draw[line width=1pt,color=ttzzqq] (0.3499993954548008,0.909999274544299) -- (0.3524993911366208,0.9129742815397298);
\draw[line width=1pt,color=ttzzqq] (0.3524993911366208,0.9129742815397298) -- (0.35499938681844084,0.9158992887078873);
\draw[line width=1pt,color=ttzzqq] (0.35499938681844084,0.9158992887078873) -- (0.35749938250026086,0.9187742960487723);
\draw[line width=1pt,color=ttzzqq] (0.35749938250026086,0.9187742960487723) -- (0.3599993781820809,0.9215993035623838);
\draw[line width=1pt,color=ttzzqq] (0.3599993781820809,0.9215993035623838) -- (0.3624993738639009,0.9243743112487229);
\draw[line width=1pt,color=ttzzqq] (0.3624993738639009,0.9243743112487229) -- (0.36499936954572093,0.9270993191077886);
\draw[line width=1pt,color=ttzzqq] (0.36499936954572093,0.9270993191077886) -- (0.36749936522754095,0.9297743271395817);
\draw[line width=1pt,color=ttzzqq] (0.36749936522754095,0.9297743271395817) -- (0.369999360909361,0.9323993353441016);
\draw[line width=1pt,color=ttzzqq] (0.369999360909361,0.9323993353441016) -- (0.372499356591181,0.9349743437213488);
\draw[line width=1pt,color=ttzzqq] (0.372499356591181,0.9349743437213488) -- (0.374999352273001,0.9374993522713228);
\draw[line width=1pt,color=ttzzqq] (0.374999352273001,0.9374993522713228) -- (0.37749934795482104,0.9399743609940241);
\draw[line width=1pt,color=ttzzqq] (0.37749934795482104,0.9399743609940241) -- (0.37999934363664106,0.942399369889452);
\draw[line width=1pt,color=ttzzqq] (0.37999934363664106,0.942399369889452) -- (0.3824993393184611,0.9447743789576075);
\draw[line width=1pt,color=ttzzqq] (0.3824993393184611,0.9447743789576075) -- (0.3849993350002811,0.9470993881984896);
\draw[line width=1pt,color=ttzzqq] (0.3849993350002811,0.9470993881984896) -- (0.38749933068210113,0.9493743976120992);
\draw[line width=1pt,color=ttzzqq] (0.38749933068210113,0.9493743976120992) -- (0.38999932636392115,0.9515994071984354);
\draw[line width=1pt,color=ttzzqq] (0.38999932636392115,0.9515994071984354) -- (0.3924993220457412,0.953774416957499);
\draw[line width=1pt,color=ttzzqq] (0.3924993220457412,0.953774416957499) -- (0.3949993177275612,0.9558994268892893);
\draw[line width=1pt,color=ttzzqq] (0.3949993177275612,0.9558994268892893) -- (0.3974993134093812,0.957974436993807);
\draw[line width=1pt,color=ttzzqq] (0.3974993134093812,0.957974436993807) -- (0.39999930909120124,0.9599994472710515);
\draw[line width=1pt,color=ttzzqq] (0.39999930909120124,0.9599994472710515) -- (0.40249930477302126,0.9619744577210233);
\draw[line width=1pt,color=ttzzqq] (0.40249930477302126,0.9619744577210233) -- (0.4049993004548413,0.9638994683437219);
\draw[line width=1pt,color=ttzzqq] (0.4049993004548413,0.9638994683437219) -- (0.4074992961366613,0.9657744791391477);
\draw[line width=1pt,color=ttzzqq] (0.4074992961366613,0.9657744791391477) -- (0.4099992918184813,0.9675994901073004);
\draw[line width=1pt,color=ttzzqq] (0.4099992918184813,0.9675994901073004) -- (0.41249928750030135,0.9693745012481804);
\draw[line width=1pt,color=ttzzqq] (0.41249928750030135,0.9693745012481804) -- (0.41499928318212137,0.9710995125617872);
\draw[line width=1pt,color=ttzzqq] (0.41499928318212137,0.9710995125617872) -- (0.4174992788639414,0.9727745240481213);
\draw[line width=1pt,color=ttzzqq] (0.4174992788639414,0.9727745240481213) -- (0.4199992745457614,0.9743995357071821);
\draw[line width=1pt,color=ttzzqq] (0.4199992745457614,0.9743995357071821) -- (0.42249927022758144,0.9759745475389703);
\draw[line width=1pt,color=ttzzqq] (0.42249927022758144,0.9759745475389703) -- (0.42499926590940146,0.9774995595434852);
\draw[line width=1pt,color=ttzzqq] (0.42499926590940146,0.9774995595434852) -- (0.4274992615912215,0.9789745717207275);
\draw[line width=1pt,color=ttzzqq] (0.4274992615912215,0.9789745717207275) -- (0.4299992572730415,0.9803995840706966);
\draw[line width=1pt,color=ttzzqq] (0.4299992572730415,0.9803995840706966) -- (0.4324992529548615,0.9817745965933931);
\draw[line width=1pt,color=ttzzqq] (0.4324992529548615,0.9817745965933931) -- (0.43499924863668155,0.9830996092888161);
\draw[line width=1pt,color=ttzzqq] (0.43499924863668155,0.9830996092888161) -- (0.43749924431850157,0.9843746221569667);
\draw[line width=1pt,color=ttzzqq] (0.43749924431850157,0.9843746221569667) -- (0.4399992400003216,0.9855996351978439);
\draw[line width=1pt,color=ttzzqq] (0.4399992400003216,0.9855996351978439) -- (0.4424992356821416,0.9867746484114485);
\draw[line width=1pt,color=ttzzqq] (0.4424992356821416,0.9867746484114485) -- (0.44499923136396163,0.9878996617977798);
\draw[line width=1pt,color=ttzzqq] (0.44499923136396163,0.9878996617977798) -- (0.44749922704578166,0.9889746753568386);
\draw[line width=1pt,color=ttzzqq] (0.44749922704578166,0.9889746753568386) -- (0.4499992227276017,0.989999689088624);
\draw[line width=1pt,color=ttzzqq] (0.4499992227276017,0.989999689088624) -- (0.4524992184094217,0.9909747029931368);
\draw[line width=1pt,color=ttzzqq] (0.4524992184094217,0.9909747029931368) -- (0.4549992140912417,0.9918997170703763);
\draw[line width=1pt,color=ttzzqq] (0.4549992140912417,0.9918997170703763) -- (0.45749920977306174,0.9927747313203432);
\draw[line width=1pt,color=ttzzqq] (0.45749920977306174,0.9927747313203432) -- (0.45999920545488177,0.9935997457430369);
\draw[line width=1pt,color=ttzzqq] (0.45999920545488177,0.9935997457430369) -- (0.4624992011367018,0.994374760338458);
\draw[line width=1pt,color=ttzzqq] (0.4624992011367018,0.994374760338458) -- (0.4649991968185218,0.9950997751066056);
\draw[line width=1pt,color=ttzzqq] (0.4649991968185218,0.9950997751066056) -- (0.46749919250034183,0.9957747900474807);
\draw[line width=1pt,color=ttzzqq] (0.46749919250034183,0.9957747900474807) -- (0.46999918818216185,0.9963998051610825);
\draw[line width=1pt,color=ttzzqq] (0.46999918818216185,0.9963998051610825) -- (0.4724991838639819,0.9969748204474118);
\draw[line width=1pt,color=ttzzqq] (0.4724991838639819,0.9969748204474118) -- (0.4749991795458019,0.9974998359064677);
\draw[line width=1pt,color=ttzzqq] (0.4749991795458019,0.9974998359064677) -- (0.4774991752276219,0.997974851538251);
\draw[line width=1pt,color=ttzzqq] (0.4774991752276219,0.997974851538251) -- (0.47999917090944194,0.9983998673427611);
\draw[line width=1pt,color=ttzzqq] (0.47999917090944194,0.9983998673427611) -- (0.48249916659126196,0.9987748833199985);
\draw[line width=1pt,color=ttzzqq] (0.48249916659126196,0.9987748833199985) -- (0.484999162273082,0.9990998994699626);
\draw[line width=1pt,color=ttzzqq] (0.484999162273082,0.9990998994699626) -- (0.487499157954902,0.9993749157926541);
\draw[line width=1pt,color=ttzzqq] (0.487499157954902,0.9993749157926541) -- (0.48999915363672203,0.9995999322880723);
\draw[line width=1pt,color=ttzzqq] (0.48999915363672203,0.9995999322880723) -- (0.49249914931854205,0.999774948956218);
\draw[line width=1pt,color=ttzzqq] (0.49249914931854205,0.999774948956218) -- (0.4949991450003621,0.9998999657970903);
\draw[line width=1pt,color=ttzzqq] (0.4949991450003621,0.9998999657970903) -- (0.4974991406821821,0.9999749828106901);
\draw[line width=1pt,color=ttzzqq] (0.4974991406821821,0.9999749828106901) -- (0.4999991363640021,0.9999999999970164);
\draw[line width=1pt,color=ttzzqq] (0.4999991363640021,0.9999999999970164) -- (0.5024991320458221,0.9999750173560702);
\draw[line width=1pt,color=ttzzqq] (0.5024991320458221,0.9999750173560702) -- (0.5049991277276421,0.9999000348878508);
\draw[line width=1pt,color=ttzzqq] (0.5049991277276421,0.9999000348878508) -- (0.5074991234094621,0.9997750525923587);
\draw[line width=1pt,color=ttzzqq] (0.5074991234094621,0.9997750525923587) -- (0.5099991190912821,0.9996000704695934);
\draw[line width=1pt,color=ttzzqq] (0.5099991190912821,0.9996000704695934) -- (0.5124991147731022,0.9993750885195553);
\draw[line width=1pt,color=ttzzqq] (0.5124991147731022,0.9993750885195553) -- (0.5149991104549222,0.9991001067422441);
\draw[line width=1pt,color=ttzzqq] (0.5149991104549222,0.9991001067422441) -- (0.5174991061367422,0.9987751251376601);
\draw[line width=1pt,color=ttzzqq] (0.5174991061367422,0.9987751251376601) -- (0.5199991018185622,0.9984001437058031);
\draw[line width=1pt,color=ttzzqq] (0.5199991018185622,0.9984001437058031) -- (0.5224990975003823,0.9979751624466732);
\draw[line width=1pt,color=ttzzqq] (0.5224990975003823,0.9979751624466732) -- (0.5249990931822023,0.9975001813602703);
\draw[line width=1pt,color=ttzzqq] (0.5249990931822023,0.9975001813602703) -- (0.5274990888640223,0.9969752004465944);
\draw[line width=1pt,color=ttzzqq] (0.5274990888640223,0.9969752004465944) -- (0.5299990845458423,0.9964002197056456);
\draw[line width=1pt,color=ttzzqq] (0.5299990845458423,0.9964002197056456) -- (0.5324990802276623,0.9957752391374238);
\draw[line width=1pt,color=ttzzqq] (0.5324990802276623,0.9957752391374238) -- (0.5349990759094824,0.9951002587419292);
\draw[line width=1pt,color=ttzzqq] (0.5349990759094824,0.9951002587419292) -- (0.5374990715913024,0.9943752785191615);
\draw[line width=1pt,color=ttzzqq] (0.5374990715913024,0.9943752785191615) -- (0.5399990672731224,0.9936002984691209);
\draw[line width=1pt,color=ttzzqq] (0.5399990672731224,0.9936002984691209) -- (0.5424990629549424,0.9927753185918073);
\draw[line width=1pt,color=ttzzqq] (0.5424990629549424,0.9927753185918073) -- (0.5449990586367625,0.9919003388872208);
\draw[line width=1pt,color=ttzzqq] (0.5449990586367625,0.9919003388872208) -- (0.5474990543185825,0.9909753593553614);
\draw[line width=1pt,color=ttzzqq] (0.5474990543185825,0.9909753593553614) -- (0.5499990500004025,0.990000379996229);
\draw[line width=1pt,color=ttzzqq] (0.5499990500004025,0.990000379996229) -- (0.5524990456822225,0.9889754008098236);
\draw[line width=1pt,color=ttzzqq] (0.5524990456822225,0.9889754008098236) -- (0.5549990413640425,0.9879004217961453);
\draw[line width=1pt,color=ttzzqq] (0.5549990413640425,0.9879004217961453) -- (0.5574990370458626,0.9867754429551941);
\draw[line width=1pt,color=ttzzqq] (0.5574990370458626,0.9867754429551941) -- (0.5599990327276826,0.9856004642869699);
\draw[line width=1pt,color=ttzzqq] (0.5599990327276826,0.9856004642869699) -- (0.5624990284095026,0.9843754857914727);
\draw[line width=1pt,color=ttzzqq] (0.5624990284095026,0.9843754857914727) -- (0.5649990240913226,0.9831005074687026);
\draw[line width=1pt,color=ttzzqq] (0.5649990240913226,0.9831005074687026) -- (0.5674990197731427,0.9817755293186596);
\draw[line width=1pt,color=ttzzqq] (0.5674990197731427,0.9817755293186596) -- (0.5699990154549627,0.9804005513413436);
\draw[line width=1pt,color=ttzzqq] (0.5699990154549627,0.9804005513413436) -- (0.5724990111367827,0.9789755735367547);
\draw[line width=1pt,color=ttzzqq] (0.5724990111367827,0.9789755735367547) -- (0.5749990068186027,0.9775005959048927);
\draw[line width=1pt,color=ttzzqq] (0.5749990068186027,0.9775005959048927) -- (0.5774990025004227,0.9759756184457579);
\draw[line width=1pt,color=ttzzqq] (0.5774990025004227,0.9759756184457579) -- (0.5799989981822428,0.9744006411593501);
\draw[line width=1pt,color=ttzzqq] (0.5799989981822428,0.9744006411593501) -- (0.5824989938640628,0.9727756640456693);
\draw[line width=1pt,color=ttzzqq] (0.5824989938640628,0.9727756640456693) -- (0.5849989895458828,0.9711006871047156);
\draw[line width=1pt,color=ttzzqq] (0.5849989895458828,0.9711006871047156) -- (0.5874989852277028,0.969375710336489);
\draw[line width=1pt,color=ttzzqq] (0.5874989852277028,0.969375710336489) -- (0.5899989809095229,0.9676007337409893);
\draw[line width=1pt,color=ttzzqq] (0.5899989809095229,0.9676007337409893) -- (0.5924989765913429,0.9657757573182169);
\draw[line width=1pt,color=ttzzqq] (0.5924989765913429,0.9657757573182169) -- (0.5949989722731629,0.9639007810681713);
\draw[line width=1pt,color=ttzzqq] (0.5949989722731629,0.9639007810681713) -- (0.5974989679549829,0.9619758049908529);
\draw[line width=1pt,color=ttzzqq] (0.5974989679549829,0.9619758049908529) -- (0.5999989636368029,0.9600008290862615);
\draw[line width=1pt,color=ttzzqq] (0.5999989636368029,0.9600008290862615) -- (0.602498959318623,0.9579758533543971);
\draw[line width=1pt,color=ttzzqq] (0.602498959318623,0.9579758533543971) -- (0.604998955000443,0.9559008777952598);
\draw[line width=1pt,color=ttzzqq] (0.604998955000443,0.9559008777952598) -- (0.607498950682263,0.9537759024088496);
\draw[line width=1pt,color=ttzzqq] (0.607498950682263,0.9537759024088496) -- (0.609998946364083,0.9516009271951663);
\draw[line width=1pt,color=ttzzqq] (0.609998946364083,0.9516009271951663) -- (0.612498942045903,0.9493759521542102);
\draw[line width=1pt,color=ttzzqq] (0.612498942045903,0.9493759521542102) -- (0.6149989377277231,0.9471009772859811);
\draw[line width=1pt,color=ttzzqq] (0.6149989377277231,0.9471009772859811) -- (0.6174989334095431,0.944776002590479);
\draw[line width=1pt,color=ttzzqq] (0.6174989334095431,0.944776002590479) -- (0.6199989290913631,0.942401028067704);
\draw[line width=1pt,color=ttzzqq] (0.6199989290913631,0.942401028067704) -- (0.6224989247731831,0.939976053717656);
\draw[line width=1pt,color=ttzzqq] (0.6224989247731831,0.939976053717656) -- (0.6249989204550032,0.9375010795403351);
\draw[line width=1pt,color=ttzzqq] (0.6249989204550032,0.9375010795403351) -- (0.6274989161368232,0.9349761055357413);
\draw[line width=1pt,color=ttzzqq] (0.6274989161368232,0.9349761055357413) -- (0.6299989118186432,0.9324011317038745);
\draw[line width=1pt,color=ttzzqq] (0.6299989118186432,0.9324011317038745) -- (0.6324989075004632,0.9297761580447348);
\draw[line width=1pt,color=ttzzqq] (0.6324989075004632,0.9297761580447348) -- (0.6349989031822832,0.927101184558322);
\draw[line width=1pt,color=ttzzqq] (0.6349989031822832,0.927101184558322) -- (0.6374988988641033,0.9243762112446364);
\draw[line width=1pt,color=ttzzqq] (0.6374988988641033,0.9243762112446364) -- (0.6399988945459233,0.9216012381036778);
\draw[line width=1pt,color=ttzzqq] (0.6399988945459233,0.9216012381036778) -- (0.6424988902277433,0.9187762651354462);
\draw[line width=1pt,color=ttzzqq] (0.6424988902277433,0.9187762651354462) -- (0.6449988859095633,0.9159012923399418);
\draw[line width=1pt,color=ttzzqq] (0.6449988859095633,0.9159012923399418) -- (0.6474988815913834,0.9129763197171643);
\draw[line width=1pt,color=ttzzqq] (0.6474988815913834,0.9129763197171643) -- (0.6499988772732034,0.9100013472671139);
\draw[line width=1pt,color=ttzzqq] (0.6499988772732034,0.9100013472671139) -- (0.6524988729550234,0.9069763749897906);
\draw[line width=1pt,color=ttzzqq] (0.6524988729550234,0.9069763749897906) -- (0.6549988686368434,0.9039014028851943);
\draw[line width=1pt,color=ttzzqq] (0.6549988686368434,0.9039014028851943) -- (0.6574988643186634,0.9007764309533249);
\draw[line width=1pt,color=ttzzqq] (0.6574988643186634,0.9007764309533249) -- (0.6599988600004835,0.8976014591941828);
\draw[line width=1pt,color=ttzzqq] (0.6599988600004835,0.8976014591941828) -- (0.6624988556823035,0.8943764876077677);
\draw[line width=1pt,color=ttzzqq] (0.6624988556823035,0.8943764876077677) -- (0.6649988513641235,0.8911015161940795);
\draw[line width=1pt,color=ttzzqq] (0.6649988513641235,0.8911015161940795) -- (0.6674988470459435,0.8877765449531184);
\draw[line width=1pt,color=ttzzqq] (0.6674988470459435,0.8877765449531184) -- (0.6699988427277636,0.8844015738848845);
\draw[line width=1pt,color=ttzzqq] (0.6699988427277636,0.8844015738848845) -- (0.6724988384095836,0.8809766029893775);
\draw[line width=1pt,color=ttzzqq] (0.6724988384095836,0.8809766029893775) -- (0.6749988340914036,0.8775016322665976);
\draw[line width=1pt,color=ttzzqq] (0.6749988340914036,0.8775016322665976) -- (0.6774988297732236,0.8739766617165448);
\draw[line width=1pt,color=ttzzqq] (0.6774988297732236,0.8739766617165448) -- (0.6799988254550436,0.870401691339219);
\draw[line width=1pt,color=ttzzqq] (0.6799988254550436,0.870401691339219) -- (0.6824988211368637,0.8667767211346202);
\draw[line width=1pt,color=ttzzqq] (0.6824988211368637,0.8667767211346202) -- (0.6849988168186837,0.8631017511027484);
\draw[line width=1pt,color=ttzzqq] (0.6849988168186837,0.8631017511027484) -- (0.6874988125005037,0.8593767812436038);
\draw[line width=1pt,color=ttzzqq] (0.6874988125005037,0.8593767812436038) -- (0.6899988081823237,0.8556018115571862);
\draw[line width=1pt,color=ttzzqq] (0.6899988081823237,0.8556018115571862) -- (0.6924988038641438,0.8517768420434957);
\draw[line width=1pt,color=ttzzqq] (0.6924988038641438,0.8517768420434957) -- (0.6949987995459638,0.8479018727025321);
\draw[line width=1pt,color=ttzzqq] (0.6949987995459638,0.8479018727025321) -- (0.6974987952277838,0.8439769035342957);
\draw[line width=1pt,color=ttzzqq] (0.6974987952277838,0.8439769035342957) -- (0.6999987909096038,0.8400019345387862);
\draw[line width=1pt,color=ttzzqq] (0.6999987909096038,0.8400019345387862) -- (0.7024987865914238,0.8359769657160039);
\draw[line width=1pt,color=ttzzqq] (0.7024987865914238,0.8359769657160039) -- (0.7049987822732439,0.8319019970659486);
\draw[line width=1pt,color=ttzzqq] (0.7049987822732439,0.8319019970659486) -- (0.7074987779550639,0.8277770285886203);
\draw[line width=1pt,color=ttzzqq] (0.7074987779550639,0.8277770285886203) -- (0.7099987736368839,0.8236020602840192);
\draw[line width=1pt,color=ttzzqq] (0.7099987736368839,0.8236020602840192) -- (0.7124987693187039,0.819377092152145);
\draw[line width=1pt,color=ttzzqq] (0.7124987693187039,0.819377092152145) -- (0.714998765000524,0.8151021241929979);
\draw[line width=1pt,color=ttzzqq] (0.714998765000524,0.8151021241929979) -- (0.717498760682344,0.8107771564065779);
\draw[line width=1pt,color=ttzzqq] (0.717498760682344,0.8107771564065779) -- (0.719998756364164,0.8064021887928848);
\draw[line width=1pt,color=ttzzqq] (0.719998756364164,0.8064021887928848) -- (0.722498752045984,0.8019772213519188);
\draw[line width=1pt,color=ttzzqq] (0.722498752045984,0.8019772213519188) -- (0.724998747727804,0.79750225408368);
\draw[line width=1pt,color=ttzzqq] (0.724998747727804,0.79750225408368) -- (0.7274987434096241,0.7929772869881682);
\draw[line width=1pt,color=ttzzqq] (0.7274987434096241,0.7929772869881682) -- (0.7299987390914441,0.7884023200653834);
\draw[line width=1pt,color=ttzzqq] (0.7299987390914441,0.7884023200653834) -- (0.7324987347732641,0.7837773533153256);
\draw[line width=1pt,color=ttzzqq] (0.7324987347732641,0.7837773533153256) -- (0.7349987304550841,0.7791023867379948);
\draw[line width=1pt,color=ttzzqq] (0.7349987304550841,0.7791023867379948) -- (0.7374987261369041,0.7743774203333912);
\draw[line width=1pt,color=ttzzqq] (0.7374987261369041,0.7743774203333912) -- (0.7399987218187242,0.7696024541015146);
\draw[line width=1pt,color=ttzzqq] (0.7399987218187242,0.7696024541015146) -- (0.7424987175005442,0.764777488042365);
\draw[line width=1pt,color=ttzzqq] (0.7424987175005442,0.764777488042365) -- (0.7449987131823642,0.7599025221559426);
\draw[line width=1pt,color=ttzzqq] (0.7449987131823642,0.7599025221559426) -- (0.7474987088641842,0.7549775564422471);
\draw[line width=1pt,color=ttzzqq] (0.7474987088641842,0.7549775564422471) -- (0.7499987045460043,0.7500025909012786);
\draw[line width=1pt,color=ttzzqq] (0.7499987045460043,0.7500025909012786) -- (0.7524987002278243,0.7449776255330374);
\draw[line width=1pt,color=ttzzqq] (0.7524987002278243,0.7449776255330374) -- (0.7549986959096443,0.739902660337523);
\draw[line width=1pt,color=ttzzqq] (0.7549986959096443,0.739902660337523) -- (0.7574986915914643,0.7347776953147358);
\draw[line width=1pt,color=ttzzqq] (0.7574986915914643,0.7347776953147358) -- (0.7599986872732843,0.7296027304646756);
\draw[line width=1pt,color=ttzzqq] (0.7599986872732843,0.7296027304646756) -- (0.7624986829551044,0.7243777657873424);
\draw[line width=1pt,color=ttzzqq] (0.7624986829551044,0.7243777657873424) -- (0.7649986786369244,0.7191028012827363);
\draw[line width=1pt,color=ttzzqq] (0.7649986786369244,0.7191028012827363) -- (0.7674986743187444,0.7137778369508573);
\draw[line width=1pt,color=ttzzqq] (0.7674986743187444,0.7137778369508573) -- (0.7699986700005644,0.7084028727917052);
\draw[line width=1pt,color=ttzzqq] (0.7699986700005644,0.7084028727917052) -- (0.7724986656823845,0.7029779088052803);
\draw[line width=1pt,color=ttzzqq] (0.7724986656823845,0.7029779088052803) -- (0.7749986613642045,0.6975029449915824);
\draw[line width=1pt,color=ttzzqq] (0.7749986613642045,0.6975029449915824) -- (0.7774986570460245,0.6919779813506115);
\draw[line width=1pt,color=ttzzqq] (0.7774986570460245,0.6919779813506115) -- (0.7799986527278445,0.6864030178823677);
\draw[line width=1pt,color=ttzzqq] (0.7799986527278445,0.6864030178823677) -- (0.7824986484096645,0.6807780545868509);
\draw[line width=1pt,color=ttzzqq] (0.7824986484096645,0.6807780545868509) -- (0.7849986440914846,0.6751030914640612);
\draw[line width=1pt,color=ttzzqq] (0.7849986440914846,0.6751030914640612) -- (0.7874986397733046,0.6693781285139986);
\draw[line width=1pt,color=ttzzqq] (0.7874986397733046,0.6693781285139986) -- (0.7899986354551246,0.663603165736663);
\draw[line width=1pt,color=ttzzqq] (0.7899986354551246,0.663603165736663) -- (0.7924986311369446,0.6577782031320544);
\draw[line width=1pt,color=ttzzqq] (0.7924986311369446,0.6577782031320544) -- (0.7949986268187647,0.6519032407001729);
\draw[line width=1pt,color=ttzzqq] (0.7949986268187647,0.6519032407001729) -- (0.7974986225005847,0.6459782784410184);
\draw[line width=1pt,color=ttzzqq] (0.7974986225005847,0.6459782784410184) -- (0.7999986181824047,0.6400033163545911);
\draw[line width=1pt,color=ttzzqq] (0.7999986181824047,0.6400033163545911) -- (0.8024986138642247,0.6339783544408907);
\draw[line width=1pt,color=ttzzqq] (0.8024986138642247,0.6339783544408907) -- (0.8049986095460447,0.6279033926999174);
\draw[line width=1pt,color=ttzzqq] (0.8049986095460447,0.6279033926999174) -- (0.8074986052278648,0.6217784311316711);
\draw[line width=1pt,color=ttzzqq] (0.8074986052278648,0.6217784311316711) -- (0.8099986009096848,0.6156034697361519);
\draw[line width=1pt,color=ttzzqq] (0.8099986009096848,0.6156034697361519) -- (0.8124985965915048,0.6093785085133597);
\draw[line width=1pt,color=ttzzqq] (0.8124985965915048,0.6093785085133597) -- (0.8149985922733248,0.6031035474632946);
\draw[line width=1pt,color=ttzzqq] (0.8149985922733248,0.6031035474632946) -- (0.8174985879551449,0.5967785865859566);
\draw[line width=1pt,color=ttzzqq] (0.8174985879551449,0.5967785865859566) -- (0.8199985836369649,0.5904036258813455);
\draw[line width=1pt,color=ttzzqq] (0.8199985836369649,0.5904036258813455) -- (0.8224985793187849,0.5839786653494616);
\draw[line width=1pt,color=ttzzqq] (0.8224985793187849,0.5839786653494616) -- (0.8249985750006049,0.5775037049903047);
\draw[line width=1pt,color=ttzzqq] (0.8249985750006049,0.5775037049903047) -- (0.8274985706824249,0.5709787448038749);
\draw[line width=1pt,color=ttzzqq] (0.8274985706824249,0.5709787448038749) -- (0.829998566364245,0.5644037847901721);
\draw[line width=1pt,color=ttzzqq] (0.829998566364245,0.5644037847901721) -- (0.832498562046065,0.5577788249491963);
\draw[line width=1pt,color=ttzzqq] (0.832498562046065,0.5577788249491963) -- (0.834998557727885,0.5511038652809476);
\draw[line width=1pt,color=ttzzqq] (0.834998557727885,0.5511038652809476) -- (0.837498553409705,0.5443789057854259);
\draw[line width=1pt,color=ttzzqq] (0.837498553409705,0.5443789057854259) -- (0.839998549091525,0.5376039464626313);
\draw[line width=1pt,color=ttzzqq] (0.839998549091525,0.5376039464626313) -- (0.8424985447733451,0.5307789873125638);
\draw[line width=1pt,color=ttzzqq] (0.8424985447733451,0.5307789873125638) -- (0.8449985404551651,0.5239040283352233);
\draw[line width=1pt,color=ttzzqq] (0.8449985404551651,0.5239040283352233) -- (0.8474985361369851,0.5169790695306098);
\draw[line width=1pt,color=ttzzqq] (0.8474985361369851,0.5169790695306098) -- (0.8499985318188051,0.5100041108987234);
\draw[line width=1pt,color=ttzzqq] (0.8499985318188051,0.5100041108987234) -- (0.8524985275006252,0.502979152439564);
\draw[line width=1pt,color=ttzzqq] (0.8524985275006252,0.502979152439564) -- (0.8549985231824452,0.49590419415313175);
\draw[line width=1pt,color=ttzzqq] (0.8549985231824452,0.49590419415313175) -- (0.8574985188642652,0.48877923603942647);
\draw[line width=1pt,color=ttzzqq] (0.8574985188642652,0.48877923603942647) -- (0.8599985145460852,0.48160427809844825);
\draw[line width=1pt,color=ttzzqq] (0.8599985145460852,0.48160427809844825) -- (0.8624985102279052,0.4743793203301971);
\draw[line width=1pt,color=ttzzqq] (0.8624985102279052,0.4743793203301971) -- (0.8649985059097253,0.467104362734673);
\draw[line width=1pt,color=ttzzqq] (0.8649985059097253,0.467104362734673) -- (0.8674985015915453,0.45977940531187594);
\draw[line width=1pt,color=ttzzqq] (0.8674985015915453,0.45977940531187594) -- (0.8699984972733653,0.4524044480618059);
\draw[line width=1pt,color=ttzzqq] (0.8699984972733653,0.4524044480618059) -- (0.8724984929551853,0.444979490984463);
\draw[line width=1pt,color=ttzzqq] (0.8724984929551853,0.444979490984463) -- (0.8749984886370054,0.4375045340798471);
\draw[line width=1pt,color=ttzzqq] (0.8749984886370054,0.4375045340798471) -- (0.8774984843188254,0.4299795773479582);
\draw[line width=1pt,color=ttzzqq] (0.8774984843188254,0.4299795773479582) -- (0.8799984800006454,0.42240462078879637);
\draw[line width=1pt,color=ttzzqq] (0.8799984800006454,0.42240462078879637) -- (0.8824984756824654,0.41477966440236164);
\draw[line width=1pt,color=ttzzqq] (0.8824984756824654,0.41477966440236164) -- (0.8849984713642854,0.4071047081886539);
\draw[line width=1pt,color=ttzzqq] (0.8849984713642854,0.4071047081886539) -- (0.8874984670461055,0.39937975214767324);
\draw[line width=1pt,color=ttzzqq] (0.8874984670461055,0.39937975214767324) -- (0.8899984627279255,0.39160479627941963);
\draw[line width=1pt,color=ttzzqq] (0.8899984627279255,0.39160479627941963) -- (0.8924984584097455,0.3837798405838931);
\draw[line width=1pt,color=ttzzqq] (0.8924984584097455,0.3837798405838931) -- (0.8949984540915655,0.3759048850610936);
\draw[line width=1pt,color=ttzzqq] (0.8949984540915655,0.3759048850610936) -- (0.8974984497733856,0.3679799297110211);
\draw[line width=1pt,color=ttzzqq] (0.8974984497733856,0.3679799297110211) -- (0.8999984454552056,0.3600049745336757);
\draw[line width=1pt,color=ttzzqq] (0.8999984454552056,0.3600049745336757) -- (0.9024984411370256,0.3519800195290574);
\draw[line width=1pt,color=ttzzqq] (0.9024984411370256,0.3519800195290574) -- (0.9049984368188456,0.34390506469716603);
\draw[line width=1pt,color=ttzzqq] (0.9049984368188456,0.34390506469716603) -- (0.9074984325006656,0.3357801100380018);
\draw[line width=1pt,color=ttzzqq] (0.9074984325006656,0.3357801100380018) -- (0.9099984281824857,0.3276051555515646);
\draw[line width=1pt,color=ttzzqq] (0.9099984281824857,0.3276051555515646) -- (0.9124984238643057,0.31938020123785443);
\draw[line width=1pt,color=ttzzqq] (0.9124984238643057,0.31938020123785443) -- (0.9149984195461257,0.3111052470968713);
\draw[line width=1pt,color=ttzzqq] (0.9149984195461257,0.3111052470968713) -- (0.9174984152279457,0.30278029312861526);
\draw[line width=1pt,color=ttzzqq] (0.9174984152279457,0.30278029312861526) -- (0.9199984109097658,0.29440533933308627);
\draw[line width=1pt,color=ttzzqq] (0.9199984109097658,0.29440533933308627) -- (0.9224984065915858,0.2859803857102843);
\draw[line width=1pt,color=ttzzqq] (0.9224984065915858,0.2859803857102843) -- (0.9249984022734058,0.27750543226020935);
\draw[line width=1pt,color=ttzzqq] (0.9249984022734058,0.27750543226020935) -- (0.9274983979552258,0.26898047898286154);
\draw[line width=1pt,color=ttzzqq] (0.9274983979552258,0.26898047898286154) -- (0.9299983936370458,0.26040552587824073);
\draw[line width=1pt,color=ttzzqq] (0.9299983936370458,0.26040552587824073) -- (0.9324983893188659,0.2517805729463469);
\draw[line width=1pt,color=ttzzqq] (0.9324983893188659,0.2517805729463469) -- (0.9349983850006859,0.24310562018718024);
\draw[line width=1pt,color=ttzzqq] (0.9349983850006859,0.24310562018718024) -- (0.9374983806825059,0.23438066760074058);
\draw[line width=1pt,color=ttzzqq] (0.9374983806825059,0.23438066760074058) -- (0.9399983763643259,0.22560571518702796);
\draw[line width=1pt,color=ttzzqq] (0.9399983763643259,0.22560571518702796) -- (0.942498372046146,0.2167807629460424);
\draw[line width=1pt,color=ttzzqq] (0.942498372046146,0.2167807629460424) -- (0.944998367727966,0.2079058108777839);
\draw[line width=1pt,color=ttzzqq] (0.944998367727966,0.2079058108777839) -- (0.947498363409786,0.19898085898225243);
\draw[line width=1pt,color=ttzzqq] (0.947498363409786,0.19898085898225243) -- (0.949998359091606,0.19000590725944802);
\draw[line width=1pt,color=ttzzqq] (0.949998359091606,0.19000590725944802) -- (0.952498354773426,0.18098095570937067);
\draw[line width=1pt,color=ttzzqq] (0.952498354773426,0.18098095570937067) -- (0.9549983504552461,0.17190600433202036);
\draw[line width=1pt,color=ttzzqq] (0.9549983504552461,0.17190600433202036) -- (0.9574983461370661,0.1627810531273971);
\draw[line width=1pt,color=ttzzqq] (0.9574983461370661,0.1627810531273971) -- (0.9599983418188861,0.15360610209550088);
\draw[line width=1pt,color=ttzzqq] (0.9599983418188861,0.15360610209550088) -- (0.9624983375007061,0.14438115123633172);
\draw[line width=1pt,color=ttzzqq] (0.9624983375007061,0.14438115123633172) -- (0.9649983331825261,0.13510620054988962);
\draw[line width=1pt,color=ttzzqq] (0.9649983331825261,0.13510620054988962) -- (0.9674983288643462,0.12578125003617455);
\draw[line width=1pt,color=ttzzqq] (0.9674983288643462,0.12578125003617455) -- (0.9699983245461662,0.11640629969518654);
\draw[line width=1pt,color=ttzzqq] (0.9699983245461662,0.11640629969518654) -- (0.9724983202279862,0.10698134952692558);
\draw[line width=1pt,color=ttzzqq] (0.9724983202279862,0.10698134952692558) -- (0.9749983159098062,0.09750639953139167);
\draw[line width=1pt,color=ttzzqq] (0.9749983159098062,0.09750639953139167) -- (0.9774983115916263,0.0879814497085848);
\draw[line width=1pt,color=ttzzqq] (0.9774983115916263,0.0879814497085848) -- (0.9799983072734463,0.078406500058505);
\draw[line width=1pt,color=ttzzqq] (0.9799983072734463,0.078406500058505) -- (0.9824983029552663,0.06878155058115223);
\draw[line width=1pt,color=ttzzqq] (0.9824983029552663,0.06878155058115223) -- (0.9849982986370863,0.059106601276526526);
\draw[line width=1pt,color=ttzzqq] (0.9849982986370863,0.059106601276526526) -- (0.9874982943189063,0.04938165214462786);
\draw[line width=1pt,color=ttzzqq] (0.9874982943189063,0.04938165214462786) -- (0.9899982900007264,0.039606703185456255);
\draw[line width=1pt,color=ttzzqq] (0.9899982900007264,0.039606703185456255) -- (0.9924982856825464,0.029781754399011692);
\draw[line width=1pt,color=ttzzqq] (0.9924982856825464,0.029781754399011692) -- (0.9949982813643664,0.01990680578529418);
\draw[line width=1pt,color=ttzzqq] (0.9949982813643664,0.01990680578529418) -- (0.9974982770461864,0.00998185734430372);
\draw[line width=1pt,color=ttzzqq] (0.9974982770461864,0.00998185734430372) -- (0.9999982727280065,0.0);
\draw [line width=0.8pt] (0.,1.)-- (0.,0.);
\draw [line width=0.8pt] (0.,0.)-- (1.,0.);
\end{tikzpicture}
 }
        \end{center}
        \caption{funzione dell'Esempio \ref{ex:Ulam_tenda}.}
    \end{figure}
    \fi
\end{example}

\begin{example}
    Sia $ Q_4\colon (0,1)\to(0,1) $, $ S\colon \R \to \R $ definita come
    \[ S(y) \coloneqq \log\left(\frac{4 e^y}{(1-e^y)^2}\right) \]
    e $ h\colon (0,1)\to\R $:
    \[ h(x) \coloneqq \operatorname{logit}(x) \coloneqq \log\left(\frac{x}{1-x}\right) \]
    Allora $ h\circ Q_4 = S \circ h $ e S conserva la misura:
    \[ \dif\mu(y) = \frac{\dif y}{\pi\left( e^{y/2} - e^{-y/2} \right) } \; . \]
\end{example}

\subsection{Dinamica topologica}
\emph{Setting}: $ X $ spazio metrico compatto e $ f\colon X\to X $ automorfismo.
\begin{definition}[Sottoinsieme invariante]
    Un sottoinsieme $ \Lambda \subset X $ si dice $ f $-invariante se $ \forall x\in\Lambda $ e $ \forall m \in \Z $, $ f^m(x) \in \Lambda $ o, equivalentemente, se $ f(\Lambda) \subseteq \Lambda $.
    Se tale proprietà vale solo per $ m \geq 0 $ o $ m \leq 0 $, $ \Lambda $ si dirà rispettivamente positivamente o negativamente $ f $-invariante.
\end{definition}

\begin{oss}
    Orbite periodiche sono insiemi invarianti, così come lo sono unioni di orbite.
\end{oss}

\begin{definition}[Punto errante]
    Un punto $ x\in X $ si dice errante se $ \exists\; U\ni x $ intorno tale che $ \left( \bigcup_{\; \abs{m}>0} f^m(U) \right) \cap U = \emptyset $.
    Se $ f $ è solo un endomorfismo prenderemo solo gli $ m > 0 $. Nel caso di un sistema dinamico a tempo continuo, invece, si applica la stessa definizione dopo aver discretizzato il tempo ($ t = \tau\Z $).
\end{definition}
\begin{example}
    Sia $ X = \R $ e $ T_\alpha (x) \coloneqq x + \alpha $ la traslazione di $ \alpha $. In questo caso tutti gli $ x\in\R $ sono punti erranti.
\end{example}
\begin{exercise}
    Sia $ \Omega = \left\{ x\in X : x \text{ non è errante} \right\} $. Mostrare che $ \Omega $ è chiuso e $ f $-invariante e che $ X $ compatto $ \Rightarrow \Omega \neq \emptyset $.
\end{exercise}
\begin{solution}
    Sia $ \bar x \in \clo{\Omega} $. Allora $ \forall U\ni \bar x $ intorno aperto $ \exists y\in\Omega\cap U $. Ma $ U $ è intorno anche di $ y\in \Omega $, per cui $ \exists m\neq 0 : f^m(U)\cap U=\emptyset $ e quindi $ \bar x \in \Omega $, da cui $ \Omega $ è chiuso.

    Per dimostrare che $ \Omega $ è invariante basta far vedere che $ f(\Omega) \subseteq \Omega $. Considero dunque $ x\in f(\Omega) $, $ U\ni x $ intorno e $ V \coloneqq f^{-1}(U) $; allora $ \exists \bar x\in\Omega : x = f(\bar x) $. Poiché $ \bar x\in\Omega $, $ \exists m\neq 0 : f^m(V) \cap V \neq \emptyset $. Ora $ f^m(U) \cap U = f^m(f(V)) \cap f(V) = f(f^m(V)) \cap f(V) \supseteq f(f^m(V)\cap V) \neq \emptyset $ e quindi $ x\in\Omega $.

    \textcolor{red}{$ X $ compatto $ \Rightarrow \Omega \neq \emptyset $.}
    %Per dimostrare che se $ X $ è compatto allora $ \Omega \neq \emptyset $ procediamo per assurdo; supponiamo dunque che $ \forall x\in X\; \exists U_x\ni x \text{ intorno aperto}: \forall m\neq 0\; f^m(U)\cap U = 0 $. L'unione $ \bigcup_{x\in X}U_x $ è un ricoprimento aperto di $ X $, dal quale per compattezza si può estrarre un sottoricoprimento finito $ \{U_i\}_{i=0,\ldots,N} $. Considero le immagini $ f^k(U_0) $ al variare di $ k\neq 0 $. Queste non possono intersecare $ U_0 $ e, poiché gli $ \{U_i\} $ sono in numero finito, esiste un $ j\in\{1,\ldots,N\} $ tale che $ f^k(U_0)\cap U_j \neq \emptyset $ per almeno due valori di $ k $, che chiamiamo $ \alpha $ e $ \beta $ con $ \beta > \alpha $.
\end{solution}

\begin{definition}[$ \alpha $ e $ \omega $-limite]
    Dato $ x\in X $ si definiscono gli insiemi:
    \begin{align*}
        \alpha(x) \coloneqq \left\{ y\in X : \exists\; n_j \nearrow -\infty, f^{n_j}(x) \to y \right\} \\
        \omega(x) \coloneqq \left\{ y\in X : \exists\; n_j \searrow +\infty, f^{n_j}(x) \to y \right\}
    \end{align*}
    ossia l'insieme dei punti aderenti all'orbita di $ x $ nel passato e nel futuro rispettivamente.
\end{definition}

\begin{definition}[Punto ricorrente]
    Un punto $ x\in X $ si dice ricorrente se $ x\in \alpha(x) \cap \omega(x) $. Se vale solo $ x\in \alpha $ (risp. $ \omega $) il punto si dirà negativamente (risp. positivamente) ricorrente.
\end{definition}
\begin{exercise}
    Dato $ x\in X $, $ \alpha(x) $ e $ \omega(x) $ sono chiusi e invarianti.
\end{exercise}

\begin{example}
    Si consideri il sistema dinamico definito su $\R^2$ che in coordinate polari segue la legge
    \[
        \begin{cases}
            \dot{\theta}  = 1      \\
            \dot{r}       = (1-r)r
        \end{cases}
    \]
    Gli insiemi limite sono
    \begin{align*}
        \alpha(x) &=
        \begin{cases}
            \{ (0,0) \}                                & \norm{x} < 1 \\
            \left \{x \in \R^2: \norm{x} = 1 \right \} & \norm{x} = 1 \\
            \emptyset                                  & \norm{x} > 1
        \end{cases} \\
        \omega(x) &=
        \begin{cases}
            \{ (0,0) \} & x = (0,0) \\
            \left \{x \in \R^2: \norm{x} = 1 \right \} & x \neq (0,0)
        \end{cases}
    \end{align*}
\end{example}

\begin{example}
    \textcolor{red}{Schifezza a forma di 8, è indispensabile la figura.}
\end{example}

\begin{example}[Rotazioni sul toro]
    Prendiamo come spazio delle fasi $ X = \T^1 $. Consideriamo il sistema dinamico dato dall'iterazione della rotazione $ R_\alpha\colon X\to X $ definita come
    \[ R_\alpha(x) \coloneqq x + \alpha \pmod{1}. \]
    Dotiamo inoltre $ X $ della distanza
    \[ d(x,y) \coloneqq \min_{p\in\Z} \abs{x-y-p}. \]
    Mostriamo innanzi tutto che le orbite sono periodiche $ \iff \alpha\in\Q $:
    \begin{itemize}
        \item[$\Rightarrow$] Si ha $ \forall x\in\T^1\; \exists m\in\Z : x = x + m\alpha \pmod{1} $. Ma allora $ m\alpha = k $ per qualche $ k\in Z $ e quindi dev'essere $ \alpha\in\Q $.
        \item[$\Leftarrow$] Sia $ \alpha = p/q $. $ R_\alpha^m(x) = x + m\frac{p}{q} \pmod{1} $, quindi dopo $ q $ passi $ x $ torna in se stesso.
    \end{itemize}
    Mostriamo ora che se $ \alpha\in\R\setminus\Q $, allora tutte le orbite sono dense nel toro.
    \textcolor{red}{mancante}.
\end{example}

\begin{exercise}
    Dimostrare che per ogni successione finita di cifre decimali esiste una potenza di 2 tale che la sua rappresentazione decimale inizi con tale successione.
\end{exercise}
\begin{solution}
    Sia $ N\in\N $ il numero corrispondente alla data sequenza di cifre decimali e sia $ n\in\N $ tale che $ 2^n \geq N $. Allora la differenza $ d $ tra il numero di cifre di $ 2^n $ e il numero di cifre di $ N $ (espressi in base 10) è
    \[ d = \lfloor \log_{10}2^n\rfloor - \lfloor \log_{10}N \rfloor \]
    La condizione da imporre è dunque che esista un $ n\in\N $ tale che
    \[ 0 \leq \frac{2^n}{10^{d}} - N < 1 \, . \]
    Manipolando questa relazione, otteniamo
    \[ 0 \leq \{ n\log_{10}2 \} - \{ \log_{10}N \} < \log_{10}\left( 1 + \frac{1}{N} \right) \]
    Come nell'esercizio \ref{ex:potenze_di_due_cancro}, abbiamo $ \{ n\log_{10}2 \} = R^n_{\log_{10}(2)}(0) $. L'esistenza di un $ n $ che soddisfa la precedente relazione segue  dalla densità delle orbite delle rotazioni irrazionali nel toro, avendo preso $ \epsilon = \log_{10}\left( 1 + 1/N \right) $.
\end{solution}

\begin{exercise}[Teorema di Dirichlet]
    Sia $ \alpha \in \R\setminus\Q $. Allora l'equazione
    \[ \abs{\alpha - \frac{p}{q}} < \frac{1}{q^2} \]
    ha infinite soluzioni $ p/q \in\Q $ distinte.
\end{exercise}

\begin{definition}[Numeri diofantei]
    Dato $ \gamma > 0 $ e $ \tau \geq 0 $, definiamo l'insieme
    \[ \mathrm{CD(\gamma,\tau)} \coloneqq \left\{ \alpha\in\R\setminus\Q : \abs{\alpha - \frac{p}{q}} \geq \frac{\gamma}{q^{2+\tau}}\quad \forall p/q\in\Q \right\} \]
    Definiamo poi
    \[ \mathrm{CD(\tau)} \coloneqq \bigcup_{\gamma > 0} \mathrm{CD}(\gamma,\tau) \]
    e infine l'insieme dei numeri diofantei
    \[ \mathrm{CD} \coloneqq \bigcup_{\tau \geq 0} \mathrm{CD}(\tau) \]
\end{definition}

\begin{definition}[Numeri di Liouville]
    Diciamo che $ x $ è di Liouville se $ x\in (\R\setminus\Q) \setminus \mathrm{CD} $.
\end{definition}

\begin{exercise}
    $ \sum_{n=0}^{+\infty} 10^{-n!} $ è di Liouville.
\end{exercise}
\begin{exercise}
    Gli irrazionali algebrici sono diofantei.
\end{exercise}
\begin{exercise}
    Quasi ogni reale è diofanteo.\\
    Hint: stimare la misura di Lebesgue di $ ( (0,1) \setminus \mathrm{CD}(\gamma,\tau) ),\; \tau > 0 $.
\end{exercise}

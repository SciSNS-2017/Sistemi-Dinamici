\section{Lezione del 30/10/18 [Bindini]}

\subsection{Convergenza debole}
\emph{Setting}: $ V $ spazio di Hilbert su $ \K = \R \text{ o } \C $.

\begin{definition}[convergenza debole]
    Diciamo che $ (v_n)_{n \in \N} \subseteq V $ converge debolmente a $ v \in V $, e scriviamo $ v_n \todeb v $, se $ \forall w \in V, \ {\langle v_n, w \rangle} \to {\langle v, w \rangle} $ (o analogamente $ {\langle v_n - v, w \rangle} \to 0 $) dove la convergenza del prodotto scalare è intesa rispetto alla distanza euclidea su $ \K $. 
\end{definition}

\begin{oss}
    Su $ \C^n $ o più in generale spazi di dimensione finita la convergenza debole è equivalente a quella rispetto alla norma. Infatti per avere la convergenza in norma è sufficiente guardare la convergenza sugli elementi di una base $ \{e^j\} $ e $ v_n^j = {\langle v_n, e^j\rangle} \to {\langle v, e^j\rangle} = v^j $. Il viceversa è invece ovvio.
\end{oss}

\begin{lemma}
    Sia $ (v_n)_{n \in \N} \subseteq V $. Se $ v_n \to v $ allora $ v_n \todeb v $.
\end{lemma}

\begin{lemma}
    Sia $ (v_n)_{n \in \N} \subseteq V $. Se $ v_n \todeb v $ e $ \norm{v_n} \to \norm{v} $ allora $ v_n \to v $.
\end{lemma}
\begin{proof}
    Infatti $ \norm{v_n - v}^2 = \norm{v_n}^2 + \norm{v}^2 - 2 {\langle v_n, v \rangle} \to 2\norm{v}^2 - 2\norm{v}^2 = 0 $.
\end{proof}

\begin{thm} \label{thm:Banach-Alaoglu-facile}
    La palla $ B_1(0) \subseteq V $ è relativamente compatta rispetto alla convergenza debole, cioè per ogni $ (v_n)_{n \in \N} \subseteq V $ con $ \norm{v_n} \leq 1 $ esiste una sottosuccessione $ (v_{n_k})_{k \in \N} \subseteq V $ e un $ v \in V $ tali che $ v_{n_k} \todeb v $.
\end{thm}
\begin{proof}
    (Idea)
    Per ogni $ v \in V $, consideriamo $ D_v \coloneqq \overline{B_{\norm{v}}(0)} \subseteq \C $ e l'insieme $ K \coloneqq \prod_{v \in V} D_v $. Essendo $ D_v $ compatto per ogni $ v $, $ K $ è compatto per il Teorema di Tychonoff. Sia ora 
    \begin{align*}
        \varphi \colon B_1(0) \subseteq V & \to K \\
        w & \mapsto ({\langle w, v \rangle})_{v \in V}
    \end{align*}
    Tale mappa è ben definita essendo $ \abs{{\langle v_n, v \rangle}} \leq \norm{w} \norm{v} \leq \norm{w} $. Basterà allora mostrare che la ``restrizione di $ \varphi $ nell'immagine'' $ \psi \colon B_1(0) \to \varphi(B_1(0)) $ è continua con inversa continua (prendendo su $ K $ al topologia indotta e su $ B_1(0) $ quella indotta della convergenza debole) e che $ \varphi(B_1(0)) $. Essendo allora $ \varphi(B_1(0)) $ chiusa in un compatto otteniamo che è compatta e dato che $ \psi $ è un omeomorfismo otteniamo che $ B_1(0) $ è compatta.
\end{proof}

\begin{proposition}
    Sia $ \{e_n\}_{n \in \N} $ un insieme ortonormale in $ V $. Le seguenti proprietà sono equivalenti. 
    \begin{enumerate}[label=(\roman*)]
        \item $ \forall v \in V, \ v = \sum_{n = 0}^{+\infty} {\langle v, e_n \rangle} e_n $;
        \item $ \forall v \in V, \ \exists \{a_n\}_{n \in \N} \subseteq \K : v = \sum_{n = 0}^{+\infty} a_n e_n $;
        \item $ \operatorname{Span}{(\{e_n\})} $\footnote{Con $ \operatorname{Span}{(\{w_n\})} $ si intende l'insieme delle combinazioni lineari \emph{finite} di elementi $ w_n $.} è denso in $ V $;
        \item $ {\langle v, e_n \rangle} = 0 \ \forall n \in \N \Rightarrow v = 0 $.
    \end{enumerate}
    Se è verificata una di queste proprietà si dice che $ \{e_n\} $ è un \emph{insieme completo}.
\end{proposition}
\begin{proof}
    Mostriamo le varie implicazioni.
    \begin{itemize}
        \item[$ (i) \Rightarrow (ii) $] Ovvio.
        \item[$ (ii) \Rightarrow (iii) $] Prendo $ v \in V $ che scrivo come $ \sum_{n = 0}^{+\infty} a_n v_n $. Ora se $ w_{N} \coloneqq \sum_{n = 0}^{N} a_ne_n $ ho per ipotesi che $ w_N \to v $. 
        \item[$ (iii) \Rightarrow (iv) $] Sia $ v \in V $ tale che $ \forall n \in \N, \ {\langle v, e_n \rangle} = 0 $. Siano $ (w_k)_{k \in \N} \subseteq \operatorname{Span}{(\{e_n\})} $ tali che $ w_k \to v $. Ma allora $ w_k \todeb v $ quindi $ {\langle v, v \rangle} = \lim_{k} {\langle v, w_k \rangle} = 0 $ essendo $ {\langle v, w_k \rangle} = \sum_{n = 0}^{N} a_n^k {\langle v, e_n \rangle} = 0 $. Così $ \norm{v} = 0 \Rightarrow v = 0 $.
        \item[$ (iv) \Rightarrow (i) $] Fissato $ v \in V $ poniamo $ v_N \coloneqq \sum_{n = 0}^{N} {\langle v, e_n \rangle} e_n $. Essendo $ \{e_n\} $ un insieme ortonormale
        \[
            {\langle v - v_N, v_N \rangle} = {\langle v, v_N \rangle} - {\langle v_N, v_N \rangle} = \sum_{n = 0}^{N} \overline{{\langle v, e_n \rangle}}{\langle v, e_n \rangle} - \norm{v_N}^2 = 0.
        \]
        Pertanto
        \[
            \norm{v}^2 = \norm{v - v_N}^2 + \norm{v_N}^2 \geq \norm{v_N}^2 = \sum_{n = 0}^{N} \abs{{\langle v, e_n \rangle}}^2.
        \]
        Ora $ \norm{v_N} \leq \norm{v} $ quindi per il Teorema \ref{thm:Banach-Alaoglu-facile} esiste $ (v_{N_k}) $ e $ \tilde{v} $ con $ \norm{\tilde{v}} \leq \norm{v} $ tale che $ v_{N_k} \todeb \tilde{v} $. Osservando quindi che
        \[
            {\langle v - \tilde{v}, e_n \rangle} = {\langle v, e_n \rangle} - {\langle \tilde{v}, e_n \rangle} = {\langle v, e_n \rangle} - \lim_{k \to + \infty} {\langle v_{N_k}, e_n \rangle} = {\langle v, e_n \rangle} - {\langle v, e_n \rangle} = 0.
        \]
        Per ipotesi allora $ \tilde{v} = v $. Mostriamo che $ (v_{N_k}) $ tende a $ v $ in norma. Infatti
        \[
            \norm{v - v_{N_k}}^2 = {\langle v - v_{N_k}, v - v_{N_k}\rangle} = {\langle v - v_{N_k}, v \rangle} - {\langle v - v_{N_k}, v_{N_k} \rangle} = {\langle v - v_{N_k}, v \rangle} \to 0.
        \]
        Finora abbiamo dunque estratto una sottosuccessione di $ v_N $ che tende a $ v $. Possiamo applicare lo stesso procedimento su un'arbitraria sottosuccessione di $ v_N $, estraendo da ciascuna una sotto-sottosuccessione convergente a $ v $. Per un noto lemma, quindi, l'intera successione $ v_N $ deve convergere a $ v $. \qedhere
    \end{itemize}
\end{proof}

\subsection{Spazio $ L^2([0, 2\pi]) $}
Consideriamo lo spazio $ L^2([0, 2\pi]) $.
Una sua base ortonormale di è $ e_n(\theta) = \frac{1}{\sqrt{2\pi}}e^{i n \theta} $ con $ n \in \Z $ e $ \theta \in [0, 2\pi] $ ed è detta \emph{base di Fourier}.

\begin{oss}
    Al contrario delle funzioni di $ L^2(\T^1) $, le funzioni di $ L^2([0, 2\pi]) $ possono assumere valori diversi in $ 0 $ e $ 2\pi $. Prima di passare $ [0,2\pi] $ al quoziente, quindi, i due spazi non sono del tutto analoghi.
\end{oss}

\begin{definition}[polinomi trigonometrici]
    Chiamiamo polinomi trigonometrici di grado minore o uguale a $ N $ gli elementi di
    $ T_N \coloneqq \left\{\sum_{n = -N}^{N} a_n e^{i n \theta}, \ a_n \in \C\right\} $
\end{definition}

\begin{thm}
    La base di Fourier è un insieme completo di $ L^2([0,2\pi]) $.
\end{thm}
\begin{proof}
    Road map
    \begin{enumerate}
        \item Se $ f \in L^2([0, 2\pi]) $, $ \hat{f}(n) \coloneqq {\langle f, e_n \rangle} = \frac{1}{2\pi} \int_{0}^{2\pi} f(\theta) e^{-i n\theta} \dif{\theta} $ sono i candidati coefficienti;
        \item Mostro che per ogni $ p \in T_N $ vale $ \norm{f - \sum_{n = -N}^{N} \hat{f}(n) e_n}_2 \leq \norm{f - p}_2 $;
        \item Considero la famiglia di mollificatori $ \{Q_k\} $, con $ Q_k(t) \coloneqq \lambda_k (\frac{1 + \cos{t}}{2})^k $, dove $ \lambda_k $ è il fattore di normalizzazione. Questo è un polinomio trigonometrico di grado $ k $ e che ``tende alla delta di Dirac''. Mostro che
        \[
            (f * Q_k)(x) \coloneqq \int_{-\pi}^{\pi} f(y) Q_k(x - y) \dif{y} \overset{k}{\to} f(x)
        \]
        e uso il punto precedente per concludere. \textcolor{red}{Non chiaro}.
    \end{enumerate}
\end{proof}

\begin{exercise}
    Si calcolino i coefficienti di Fourier per la funzione
    \[
        f(\theta) \coloneqq
        \begin{cases}
            1  & \text{se } 0 \leq \theta \leq \pi \\
            -1 & \text{se } \pi < \theta \leq 2\pi
        \end{cases}
    \]
\end{exercise}

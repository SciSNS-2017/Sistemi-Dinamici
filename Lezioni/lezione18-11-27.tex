\section{Lezione del 27/11/2018 [Marmi]}

\subsection{Mescolamento di un sistema dinamico misurabile}
\emph{Setting}: $ (X, \mathcal{A}, \mu, f) $ sistema dinamico misurabile. \\

Intuitivamente in un sistema "mescolante" un insieme viene spalmato su tutto lo spazio delle fasi sotto l'iterazione di $ f $. Se $ A $ e $ B $ sono insiemi misurabili, consideriamo l'evoluzione di $ A $ sotto $ f $: essendo lo spazio di probabilità $ \mu(f^{-n}(A) \cap B) $ è la "percentuale" di $ f^{-n}(A) $ contenuta in $ B $; se allora $ \mu $ è $ f $-invariante, $ \mu(A) $ viene preservato dall'iterazione di $ f $ e quindi ci aspettiamo che $ \mu(f^{-n}(A) \cap B) $ tenda a $ \mu(A) \mu(B) $. Più formalmente diamo la seguente definizione. 

\begin{definition}[mescolamento forte]
    Un sistema dinamico misurabile si dice (fortemente) mescolante se $ \forall A, B \in \mathcal{A} $
    \[
        \lim_{n \to +\infty} \mu\left(f^{-n}(A) \cap B\right) = \mu(A) \mu(B).
    \]
\end{definition}

\begin{definition}[operatore di Koopman]
    Definiamo l'operatore di Koopman $ \mathcal{U}_f $ associato a $ (X, \mathcal{A}, \mu, f) $ come la mappa che a $ \varphi \in L^2(X, \mathcal{A}, \mu, \R) $ associa la funzione $ \mathcal{U}_f \varphi \coloneqq \varphi \circ f $.  
\end{definition}

\begin{exercise}
    Mostrare che $ \mathcal{U}_f $ è un'isometria di $ L^2(X, \mathcal{A}, \mu, \R) $.
\end{exercise}
\begin{solution}
    Basta mostrare che per ogni $ \varphi, \psi \in L^2(X, \mathcal{A}, \mu, \R) $ si ha $ \langle{\mathcal{U}_f \varphi, \mathcal{U}_f \psi}\rangle = \langle{\varphi, \psi}\rangle $ dove $ \langle{\, , \, }\rangle $ è il prodotto scalare standard di $ L^2 $. Infatti usando l'$ f $-invarianza della misura
   \begin{align*}
       \int_{X} \mathcal{U}_f \varphi \; \mathcal{U}_f \psi \dif{\mu} & =  \int_{X} \varphi(f(x)) \psi(f(x)) \dif{\mu}(x) \\
       & = \int_{f(X)} \varphi(y) \psi(y) \dif{\mu}(f^{-1}(y)) = \int_{f(X)} \varphi(y) \psi(y) \dif{\mu}(y) \\
       & \textcolor{red}={} \int_{X} \varphi(y) \psi(y) \dif{\mu}(y).
   \end{align*}
\end{solution}


\section{Lezione del 05/02/19 [Tantari]}
Nella lezione precedente si è ottenuto che, tranne sulla semiretta $ h=0 $, $ \beta \geq 1 $, vale
\[ \lim_{n \to +\infty}\mean{m_n}_{P_\beta} = \bar{M}(\beta, h) \]
\[ \lim_{n \to +\infty}\mean{m_n^2}_{P_\beta} = \bar{M}^2(\beta,h) \]
e quindi
\[ \lim_{n \to +\infty}\mean{\left(m_n-\mean{m_n}_{P_\beta}\right)^2}_{P_\beta} = \lim_{n \to +\infty} \left( \mean{m_n^2}_{P_\beta} - \mean{m_n}^2_{P_\beta} \right) = 0. \]
In generale, una famiglia di osservabili $ \mathcal{O}_n $ la cui varianza tenda a zero per $ n\to +\infty $ si dice \emph{automediante}; ciò significa che nel limite termodinamico il valore dell'osservabile diventa deterministico.\\

Un altro modo per calcolare la varianza di un'osservabile è il seguente:
supponiamo di poter riscrivere l'hamiltoniana esplicitando l'osservabile di interesse:
\[ \ham(\sigma) = \mathcal{K}(\sigma) + \lambda\mathcal{O}(\sigma). \]
Allora la varianza di $ \mathcal{O} $ si ottiene calcolando
\begin{align*}
    \dpd[2]{}{\lambda} \left( \log Z_n \right) & = \dpd{}{\lambda} \left[ \frac{1}{Z_n} \sum_{\sigma \in \Sigma} \exp\left( K(\sigma) + \lambda \mathcal{O}(\sigma) \right) \mathcal{O}(\sigma) \right] \\
    & = -\frac{1}{Z_n^2} \left( \sum_{\sigma \in \Sigma} \exp\left( K(\sigma) + \lambda \mathcal{O}(\sigma) \right) \right)^2 + \frac{1}{Z_n} \sum_{\sigma \in \Sigma} \exp\left( K(\sigma) + \lambda \mathcal{O}(\sigma) \right) \mathcal{O}^2(\sigma) \\
    & = \mean{\mathcal{O}^2}_{P_\beta} - \mean{\mathcal{O}}^2_{P_\beta} = \mean{ \left(\mathcal{O} - \mean{\mathcal{O}}_{P_\beta} \right)^2 }_{P_\beta}.
\end{align*}
Tornando al caso specifico della magnetizzazione media, prendendo $ \mathcal{O} = \beta n m_n $  e $ \lambda = h $ si ha:
\[ \dpd[2]{}{h}\left( \frac{\log Z_n}{n} \right) = \beta^2 n \mean{\left(m_n-\mean{m_n}_{P_\beta}\right)^2}_{P_\beta} \]
e quindi:
\[ \lim_{n \to +\infty} \mean{\left(m_n-\mean{m_n}_{P_\beta}\right)^2}_{P_\beta} = \frac{1}{\beta^2} \lim_{n \to +\infty} \left[ \frac{1}{n} \dpd[2]{}{h}\left( \frac{\log Z_n}{n} \right) \right] = 0. \]
\\

Da ultimo possiamo calcolare esplicitamente la distribuzione di $ m_n $, cioè $ f(x) = P_\beta \left( m_n^{-1} (\{x\}) \right) $. Detto $ k $ il numero di spin uguali a 1, notiamo che $ m_n $ può assumere solo i valori $ \left(-1 + \frac{2k}{n}\right)_{k=0,\ldots,n} $; se poniamo $ m_n = x $ allora $ k = \frac{1}{2}(1+x) n $ e quindi
\begin{align}\label{eq:distrMagn}
    f(x) & = P_\beta \left( m_n^{-1} (\{x\}) \right) = \sum_{\sigma\in m_n^{-1}(\{x\})} \frac{1}{Z_n} \exp\left( \beta\left( \frac{n}{2}m_n^2(\sigma) + nhm_n(\sigma) -\frac{1}{2}\right) \right) \nonumber \\
    & = \frac{1}{Z_n} \binom{n}{\frac{1}{2}(1+x) n} \exp\left( \beta\left( \frac{n}{2}x^2 + nhx - \frac{1}{2} \right) \right).
\end{align}
Infatti tutti i $ \sigma $ tali che $ m_n(\sigma) = x $ hanno la stessa probabilità, e sono $ \binom{n}{k} $ (il numero di modi in cui si possono scegliere gli spin uguali a 1). In figura \ref{fig:distrMagn} sono riportati alcuni grafici della \eqref{eq:distrMagn} al variare di $ h $ e $ \beta $.

\iffigureon
\begin{figure}[p]
    \centering
    \subfloat[$ h=0, \beta=1 $.]{
%<<<<<<<WARNING>>>>>>>
% PGF/Tikz doesn't support the following mathematical functions:
% cosh, acosh, sinh, asinh, tanh, atanh,
% x^r with r not integer

% Plotting will be done using GNUPLOT
% GNUPLOT must be installed and you must allow Latex to call external
% programs by adding the following option to your compiler
% shell-escape    OR    enable-write18 
% Example: pdflatex --shell-escape file.tex 

\definecolor{wwwwww}{rgb}{0.4,0.4,0.4}
\begin{tikzpicture}[line cap=round,line join=round,>=triangle 45,x=1cm,y=1cm]
\begin{axis}[
x=3.5cm,y=4cm,
axis lines=middle,
xmin=-1.2,
xmax=1.2,
ymin=-0.027640247367904026,
ymax=0.1,
xtick={-1,-0.8,...,1},
ytick={-0.02,0,...,0.1},
y post scale = 12,
ylabel = $f(x)$,
xlabel = $x$]
\clip(-1.3753518986999413,-0.027640247367904026) rectangle (1.4564267138000453,0.13969846222973287);
\draw[line width=1.5pt,color=wwwwww] (-0.9999996436130685,0) -- (-0.9999996436130685,0);
\draw[line width=1.5pt,color=wwwwww] (-0.9999996436130685,0) -- (-0.994999649173213,0);
\draw[line width=1.5pt,color=wwwwww] (-0.994999649173213,0) -- (-0.9899996547333576,0);
\draw[line width=1.5pt,color=wwwwww] (-0.9899996547333576,0) -- (-0.9849996602935022,0);
\draw[line width=1.5pt,color=wwwwww] (-0.9849996602935022,0) -- (-0.9799996658536467,0);
\draw[line width=1.5pt,color=wwwwww] (-0.9799996658536467,0) -- (-0.9749996714137913,0);
\draw[line width=1.5pt,color=wwwwww] (-0.9749996714137913,0) -- (-0.9699996769739359,0);
\draw[line width=1.5pt,color=wwwwww] (-0.9699996769739359,0) -- (-0.9649996825340804,0);
\draw[line width=1.5pt,color=wwwwww] (-0.9649996825340804,0) -- (-0.959999688094225,0);
\draw[line width=1.5pt,color=wwwwww] (-0.959999688094225,0) -- (-0.9549996936543695,0);
\draw[line width=1.5pt,color=wwwwww] (-0.9549996936543695,0) -- (-0.9499996992145141,0);
\draw[line width=1.5pt,color=wwwwww] (-0.9499996992145141,0) -- (-0.9449997047746587,0);
\draw[line width=1.5pt,color=wwwwww] (-0.9449997047746587,0) -- (-0.9399997103348032,0);
\draw[line width=1.5pt,color=wwwwww] (-0.9399997103348032,0) -- (-0.9349997158949478,0);
\draw[line width=1.5pt,color=wwwwww] (-0.9349997158949478,0) -- (-0.9299997214550924,0);
\draw[line width=1.5pt,color=wwwwww] (-0.9299997214550924,0) -- (-0.9249997270152369,0);
\draw[line width=1.5pt,color=wwwwww] (-0.9249997270152369,0) -- (-0.9199997325753815,0);
\draw[line width=1.5pt,color=wwwwww] (-0.9199997325753815,0) -- (-0.914999738135526,0);
\draw[line width=1.5pt,color=wwwwww] (-0.914999738135526,0) -- (-0.9099997436956706,0);
\draw[line width=1.5pt,color=wwwwww] (-0.9099997436956706,0) -- (-0.9049997492558152,0);
\draw[line width=1.5pt,color=wwwwww] (-0.9049997492558152,0) -- (-0.8999997548159597,0);
\draw[line width=1.5pt,color=wwwwww] (-0.8999997548159597,0) -- (-0.8949997603761043,0);
\draw[line width=1.5pt,color=wwwwww] (-0.8949997603761043,0) -- (-0.8899997659362489,0);
\draw[line width=1.5pt,color=wwwwww] (-0.8899997659362489,0) -- (-0.8849997714963934,0);
\draw[line width=1.5pt,color=wwwwww] (-0.8849997714963934,0) -- (-0.879999777056538,0);
\draw[line width=1.5pt,color=wwwwww] (-0.879999777056538,0) -- (-0.8749997826166825,0);
\draw[line width=1.5pt,color=wwwwww] (-0.8749997826166825,0) -- (-0.8699997881768271,0);
\draw[line width=1.5pt,color=wwwwww] (-0.8699997881768271,0) -- (-0.8649997937369717,0);
\draw[line width=1.5pt,color=wwwwww] (-0.8649997937369717,0) -- (-0.8599997992971162,0);
\draw[line width=1.5pt,color=wwwwww] (-0.8599997992971162,0) -- (-0.8549998048572608,0);
\draw[line width=1.5pt,color=wwwwww] (-0.8549998048572608,0) -- (-0.8499998104174054,0);
\draw[line width=1.5pt,color=wwwwww] (-0.8499998104174054,0) -- (-0.8449998159775499,0);
\draw[line width=1.5pt,color=wwwwww] (-0.8449998159775499,0) -- (-0.8399998215376945,0);
\draw[line width=1.5pt,color=wwwwww] (-0.8399998215376945,0) -- (-0.834999827097839,0);
\draw[line width=1.5pt,color=wwwwww] (-0.834999827097839,0) -- (-0.8299998326579836,0);
\draw[line width=1.5pt,color=wwwwww] (-0.8299998326579836,0) -- (-0.8249998382181282,0);
\draw[line width=1.5pt,color=wwwwww] (-0.8249998382181282,0) -- (-0.8199998437782727,0);
\draw[line width=1.5pt,color=wwwwww] (-0.8199998437782727,0) -- (-0.8149998493384173,0);
\draw[line width=1.5pt,color=wwwwww] (-0.8149998493384173,0) -- (-0.8099998548985619,0);
\draw[line width=1.5pt,color=wwwwww] (-0.8099998548985619,0) -- (-0.8049998604587064,0);
\draw[line width=1.5pt,color=wwwwww] (-0.8049998604587064,0) -- (-0.799999866018851,0);
\draw[line width=1.5pt,color=wwwwww] (-0.799999866018851,0) -- (-0.7949998715789955,0);
\draw[line width=1.5pt,color=wwwwww] (-0.7949998715789955,0) -- (-0.7899998771391401,0);
\draw[line width=1.5pt,color=wwwwww] (-0.7899998771391401,0) -- (-0.7849998826992847,0);
\draw[line width=1.5pt,color=wwwwww] (-0.7849998826992847,0) -- (-0.7799998882594292,0);
\draw[line width=1.5pt,color=wwwwww] (-0.7799998882594292,0) -- (-0.7749998938195738,0);
\draw[line width=1.5pt,color=wwwwww] (-0.7749998938195738,0) -- (-0.7699998993797184,0);
\draw[line width=1.5pt,color=wwwwww] (-0.7699998993797184,0) -- (-0.7649999049398629,0);
\draw[line width=1.5pt,color=wwwwww] (-0.7649999049398629,0) -- (-0.7599999105000075,0);
\draw[line width=1.5pt,color=wwwwww] (-0.7599999105000075,0) -- (-0.754999916060152,0);
\draw[line width=1.5pt,color=wwwwww] (-0.754999916060152,0) -- (-0.7499999216202966,0);
\draw[line width=1.5pt,color=wwwwww] (-0.7499999216202966,0) -- (-0.7449999271804412,0);
\draw[line width=1.5pt,color=wwwwww] (-0.7449999271804412,0) -- (-0.7399999327405857,0);
\draw[line width=1.5pt,color=wwwwww] (-0.7399999327405857,0) -- (-0.7349999383007303,0);
\draw[line width=1.5pt,color=wwwwww] (-0.7349999383007303,0) -- (-0.7299999438608749,0);
\draw[line width=1.5pt,color=wwwwww] (-0.7299999438608749,0) -- (-0.7249999494210194,0);
\draw[line width=1.5pt,color=wwwwww] (-0.7249999494210194,0) -- (-0.719999954981164,0);
\draw[line width=1.5pt,color=wwwwww] (-0.719999954981164,0) -- (-0.7149999605413085,0);
\draw[line width=1.5pt,color=wwwwww] (-0.7149999605413085,0) -- (-0.7099999661014531,0);
\draw[line width=1.5pt,color=wwwwww] (-0.7099999661014531,0) -- (-0.7049999716615977,0);
\draw[line width=1.5pt,color=wwwwww] (-0.7049999716615977,0) -- (-0.6999999772217422,0);
\draw[line width=1.5pt,color=wwwwww] (-0.6999999772217422,0) -- (-0.6949999827818868,0);
\draw[line width=1.5pt,color=wwwwww] (-0.6949999827818868,0) -- (-0.6899999883420314,0);
\draw[line width=1.5pt,color=wwwwww] (-0.6899999883420314,0) -- (-0.6849999939021759,0);
\draw[line width=1.5pt,color=wwwwww] (-0.6849999939021759,0) -- (-0.6799999994623205,0);
\draw[line width=1.5pt,color=wwwwww] (-0.6799999994623205,0) -- (-0.675000005022465,0);
\draw[line width=1.5pt,color=wwwwww] (-0.675000005022465,0) -- (-0.6700000105826096,0);
\draw[line width=1.5pt,color=wwwwww] (-0.6700000105826096,0) -- (-0.6650000161427542,0);
\draw[line width=1.5pt,color=wwwwww] (-0.6650000161427542,0) -- (-0.6600000217028987,0);
\draw[line width=1.5pt,color=wwwwww] (-0.6600000217028987,0) -- (-0.6550000272630433,0);
\draw[line width=1.5pt,color=wwwwww] (-0.6550000272630433,0) -- (-0.6500000328231879,0);
\draw[line width=1.5pt,color=wwwwww] (-0.6500000328231879,0) -- (-0.6450000383833324,0);
\draw[line width=1.5pt,color=wwwwww] (-0.6450000383833324,0) -- (-0.640000043943477,0.00100050208681597);
\draw[line width=1.5pt,color=wwwwww] (-0.640000043943477,0.00100050208681597) -- (-0.6350000495036215,0.0010955356995884266);
\draw[line width=1.5pt,color=wwwwww] (-0.6350000495036215,0.0010955356995884266) -- (-0.6300000550637661,0.0011963737513878505);
\draw[line width=1.5pt,color=wwwwww] (-0.6300000550637661,0.0011963737513878505) -- (-0.6250000606239107,0.0013030750994755447);
\draw[line width=1.5pt,color=wwwwww] (-0.6250000606239107,0.0013030750994755447) -- (-0.6200000661840552,0.001415675817939642);
\draw[line width=1.5pt,color=wwwwww] (-0.6200000661840552,0.001415675817939642) -- (-0.6150000717441998,0.001534188522138856);
\draw[line width=1.5pt,color=wwwwww] (-0.6150000717441998,0.001534188522138856) -- (-0.6100000773043444,0.0016586018833918278);
\draw[line width=1.5pt,color=wwwwww] (-0.6100000773043444,0.0016586018833918278) -- (-0.6050000828644889,0.0017888803373215798);
\draw[line width=1.5pt,color=wwwwww] (-0.6050000828644889,0.0017888803373215798) -- (-0.6000000884246335,0.001924963986771934);
\draw[line width=1.5pt,color=wwwwww] (-0.6000000884246335,0.001924963986771934) -- (-0.595000093984778,0.002066768697814782);
\draw[line width=1.5pt,color=wwwwww] (-0.595000093984778,0.002066768697814782) -- (-0.5900000995449226,0.0022141863850583276);
\draw[line width=1.5pt,color=wwwwww] (-0.5900000995449226,0.0022141863850583276) -- (-0.5850001051050672,0.002367085480315293);
\draw[line width=1.5pt,color=wwwwww] (-0.5850001051050672,0.002367085480315293) -- (-0.5800001106652117,0.002525311576703927);
\draw[line width=1.5pt,color=wwwwww] (-0.5800001106652117,0.002525311576703927) -- (-0.5750001162253563,0.002688688238469714);
\draw[line width=1.5pt,color=wwwwww] (-0.5750001162253563,0.002688688238469714) -- (-0.5700001217855009,0.0028570179652275393);
\draw[line width=1.5pt,color=wwwwww] (-0.5700001217855009,0.0028570179652275393) -- (-0.5650001273456454,0.0030300832979897672);
\draw[line width=1.5pt,color=wwwwww] (-0.5650001273456454,0.0030300832979897672) -- (-0.56000013290579,0.0032076480532084806);
\draw[line width=1.5pt,color=wwwwww] (-0.56000013290579,0.0032076480532084806) -- (-0.5550001384659345,0.003389458670180216);
\draw[line width=1.5pt,color=wwwwww] (-0.5550001384659345,0.003389458670180216) -- (-0.5500001440260791,0.003575245656525998);
\draw[line width=1.5pt,color=wwwwww] (-0.5500001440260791,0.003575245656525998) -- (-0.5450001495862237,0.0037647251160247536);
\draw[line width=1.5pt,color=wwwwww] (-0.5450001495862237,0.0037647251160247536) -- (-0.5400001551463682,0.00395760034288987);
\draw[line width=1.5pt,color=wwwwww] (-0.5400001551463682,0.00395760034288987) -- (-0.5350001607065128,0.0041535634665863675);
\draw[line width=1.5pt,color=wwwwww] (-0.5350001607065128,0.0041535634665863675) -- (-0.5300001662666574,0.004352297131502704);
\draw[line width=1.5pt,color=wwwwww] (-0.5300001662666574,0.004352297131502704) -- (-0.5250001718268019,0.004553476196167551);
\draw[line width=1.5pt,color=wwwwww] (-0.5250001718268019,0.004553476196167551) -- (-0.5200001773869465,0.004756769437263037);
\draw[line width=1.5pt,color=wwwwww] (-0.5200001773869465,0.004756769437263037) -- (-0.515000182947091,0.004961841244357736);
\draw[line width=1.5pt,color=wwwwww] (-0.515000182947091,0.004961841244357736) -- (-0.5100001885072356,0.005168353292127936);
\draw[line width=1.5pt,color=wwwwww] (-0.5100001885072356,0.005168353292127936) -- (-0.5050001940673802,0.005375966177721235);
\draw[line width=1.5pt,color=wwwwww] (-0.5050001940673802,0.005375966177721235) -- (-0.5000001996275247,0.00558434101192475);
\draw[line width=1.5pt,color=wwwwww] (-0.5000001996275247,0.00558434101192475) -- (-0.4950002051876693,0.005793140953879345);
\draw[line width=1.5pt,color=wwwwww] (-0.4950002051876693,0.005793140953879345) -- (-0.49000021074781386,0.006002032680159279);
\draw[line width=1.5pt,color=wwwwww] (-0.49000021074781386,0.006002032680159279) -- (-0.4850002163079584,0.006210687780201904);
\draw[line width=1.5pt,color=wwwwww] (-0.4850002163079584,0.006210687780201904) -- (-0.480000221868103,0.006418784071184512);
\draw[line width=1.5pt,color=wwwwww] (-0.480000221868103,0.006418784071184512) -- (-0.47500022742824755,0.006626006826599412);
\draw[line width=1.5pt,color=wwwwww] (-0.47500022742824755,0.006626006826599412) -- (-0.4700002329883921,0.006832049913922415);
\draw[line width=1.5pt,color=wwwwww] (-0.4700002329883921,0.006832049913922415) -- (-0.46500023854853667,0.0070366168377895555);
\draw[line width=1.5pt,color=wwwwww] (-0.46500023854853667,0.0070366168377895555) -- (-0.46000024410868123,0.007239421686232554);
\draw[line width=1.5pt,color=wwwwww] (-0.46000024410868123,0.007239421686232554) -- (-0.4550002496688258,0.007440189978431672);
\draw[line width=1.5pt,color=wwwwww] (-0.4550002496688258,0.007440189978431672) -- (-0.45000025522897036,0.007638659413459769);
\draw[line width=1.5pt,color=wwwwww] (-0.45000025522897036,0.007638659413459769) -- (-0.4450002607891149,0.007834580520297892);
\draw[line width=1.5pt,color=wwwwww] (-0.4450002607891149,0.007834580520297892) -- (-0.4400002663492595,0.00802771721024395);
\draw[line width=1.5pt,color=wwwwww] (-0.4400002663492595,0.00802771721024395) -- (-0.43500027190940405,0.008217847233553736);
\draw[line width=1.5pt,color=wwwwww] (-0.43500027190940405,0.008217847233553736) -- (-0.4300002774695486,0.008404762542806919);
\draw[line width=1.5pt,color=wwwwww] (-0.4300002774695486,0.008404762542806919) -- (-0.42500028302969317,0.008588269566078669);
\draw[line width=1.5pt,color=wwwwww] (-0.42500028302969317,0.008588269566078669) -- (-0.42000028858983773,0.008768189393481894);
\draw[line width=1.5pt,color=wwwwww] (-0.42000028858983773,0.008768189393481894) -- (-0.4150002941499823,0.008944357881096055);
\draw[line width=1.5pt,color=wwwwww] (-0.4150002941499823,0.008944357881096055) -- (-0.41000029971012686,0.009116625676655415);
\draw[line width=1.5pt,color=wwwwww] (-0.41000029971012686,0.009116625676655415) -- (-0.4050003052702714,0.009284858171630607);
\draw[line width=1.5pt,color=wwwwww] (-0.4050003052702714,0.009284858171630607) -- (-0.400000310830416,0.009448935384617002);
\draw[line width=1.5pt,color=wwwwww] (-0.400000310830416,0.009448935384617002) -- (-0.39500031639056055,0.009608751781002432);
\draw[line width=1.5pt,color=wwwwww] (-0.39500031639056055,0.009608751781002432) -- (-0.3900003219507051,0.009764216034134415);
\draw[line width=1.5pt,color=wwwwww] (-0.3900003219507051,0.009764216034134415) -- (-0.38500032751084967,0.009915250733102834);
\draw[line width=1.5pt,color=wwwwww] (-0.38500032751084967,0.009915250733102834) -- (-0.38000033307099423,0.010061792042362747);
\draw[line width=1.5pt,color=wwwwww] (-0.38000033307099423,0.010061792042362747) -- (-0.3750003386311388,0.010203789318308757);
\draw[line width=1.5pt,color=wwwwww] (-0.3750003386311388,0.010203789318308757) -- (-0.37000034419128336,0.010341204687875223);
\draw[line width=1.5pt,color=wwwwww] (-0.37000034419128336,0.010341204687875223) -- (-0.3650003497514279,0.010474012594084274);
\draw[line width=1.5pt,color=wwwwww] (-0.3650003497514279,0.010474012594084274) -- (-0.3600003553115725,0.01060219931329628);
\draw[line width=1.5pt,color=wwwwww] (-0.3600003553115725,0.01060219931329628) -- (-0.35500036087171705,0.010725762448802316);
\draw[line width=1.5pt,color=wwwwww] (-0.35500036087171705,0.010725762448802316) -- (-0.3500003664318616,0.010844710405123123);
\draw[line width=1.5pt,color=wwwwww] (-0.3500003664318616,0.010844710405123123) -- (-0.34500037199200617,0.01095906184720841);
\draw[line width=1.5pt,color=wwwwww] (-0.34500037199200617,0.01095906184720841) -- (-0.34000037755215073,0.011068845148485816);
\draw[line width=1.5pt,color=wwwwww] (-0.34000037755215073,0.011068845148485816) -- (-0.3350003831122953,0.011174097831447847);
\draw[line width=1.5pt,color=wwwwww] (-0.3350003831122953,0.011174097831447847) -- (-0.33000038867243986,0.011274866004239988);
\draw[line width=1.5pt,color=wwwwww] (-0.33000038867243986,0.011274866004239988) -- (-0.3250003942325844,0.01137120379646908);
\draw[line width=1.5pt,color=wwwwww] (-0.3250003942325844,0.01137120379646908) -- (-0.320000399792729,0.011463172797124837);
\draw[line width=1.5pt,color=wwwwww] (-0.320000399792729,0.011463172797124837) -- (-0.31500040535287355,0.01155084149733891);
\draw[line width=1.5pt,color=wwwwww] (-0.31500040535287355,0.01155084149733891) -- (-0.3100004109130181,0.011634284740406531);
\draw[line width=1.5pt,color=wwwwww] (-0.3100004109130181,0.011634284740406531) -- (-0.30500041647316267,0.011713583181259106);
\draw[line width=1.5pt,color=wwwwww] (-0.30500041647316267,0.011713583181259106) -- (-0.30000042203330723,0.011788822757309773);
\draw[line width=1.5pt,color=wwwwww] (-0.30000042203330723,0.011788822757309773) -- (-0.2950004275934518,0.011860094172421665);
\draw[line width=1.5pt,color=wwwwww] (-0.2950004275934518,0.011860094172421665) -- (-0.29000043315359636,0.011927492395453949);
\draw[line width=1.5pt,color=wwwwww] (-0.29000043315359636,0.011927492395453949) -- (-0.2850004387137409,0.011991116174675884);
\draw[line width=1.5pt,color=wwwwww] (-0.2850004387137409,0.011991116174675884) -- (-0.2800004442738855,0.012051067569125532);
\draw[line width=1.5pt,color=wwwwww] (-0.2800004442738855,0.012051067569125532) -- (-0.27500044983403005,0.01210745149776878);
\draw[line width=1.5pt,color=wwwwww] (-0.27500044983403005,0.01210745149776878) -- (-0.2700004553941746,0.012160375307187044);
\draw[line width=1.5pt,color=wwwwww] (-0.2700004553941746,0.012160375307187044) -- (-0.2650004609543192,0.012209948358308416);
\draw[line width=1.5pt,color=wwwwww] (-0.2650004609543192,0.012209948358308416) -- (-0.26000046651446374,0.012256281632564498);
\draw[line width=1.5pt,color=wwwwww] (-0.26000046651446374,0.012256281632564498) -- (-0.2550004720746083,0.012299487357691406);
\draw[line width=1.5pt,color=wwwwww] (-0.2550004720746083,0.012299487357691406) -- (-0.25000047763475286,0.012339678653309271);
\draw[line width=1.5pt,color=wwwwww] (-0.25000047763475286,0.012339678653309271) -- (-0.2450004831948974,0.012376969196227987);
\draw[line width=1.5pt,color=wwwwww] (-0.2450004831948974,0.012376969196227987) -- (-0.24000048875504193,0.012411472905362824);
\draw[line width=1.5pt,color=wwwwww] (-0.24000048875504193,0.012411472905362824) -- (-0.23500049431518646,0.012443303646043898);
\draw[line width=1.5pt,color=wwwwww] (-0.23500049431518646,0.012443303646043898) -- (-0.230000499875331,0.012472574953388507);
\draw[line width=1.5pt,color=wwwwww] (-0.230000499875331,0.012472574953388507) -- (-0.22500050543547553,0.012499399774372066);
\draw[line width=1.5pt,color=wwwwww] (-0.22500050543547553,0.012499399774372066) -- (-0.22000051099562007,0.012523890228121353);
\draw[line width=1.5pt,color=wwwwww] (-0.22000051099562007,0.012523890228121353) -- (-0.2150005165557646,0.012546157383930216);
\draw[line width=1.5pt,color=wwwwww] (-0.2150005165557646,0.012546157383930216) -- (-0.21000052211590914,0.012566311056416306);
\draw[line width=1.5pt,color=wwwwww] (-0.21000052211590914,0.012566311056416306) -- (-0.20500052767605367,0.012584459617249267);
\draw[line width=1.5pt,color=wwwwww] (-0.20500052767605367,0.012584459617249267) -- (-0.2000005332361982,0.012600709822756363);
\draw[line width=1.5pt,color=wwwwww] (-0.2000005332361982,0.012600709822756363) -- (-0.19500053879634274,0.012615166656824895);
\draw[line width=1.5pt,color=wwwwww] (-0.19500053879634274,0.012615166656824895) -- (-0.19000054435648728,0.01262793318830503);
\draw[line width=1.5pt,color=wwwwww] (-0.19000054435648728,0.01262793318830503) -- (-0.1850005499166318,0.012639110442333411);
\draw[line width=1.5pt,color=wwwwww] (-0.1850005499166318,0.012639110442333411) -- (-0.18000055547677635,0.012648797284754069);
\draw[line width=1.5pt,color=wwwwww] (-0.18000055547677635,0.012648797284754069) -- (-0.17500056103692088,0.012657090319019319);
\draw[line width=1.5pt,color=wwwwww] (-0.17500056103692088,0.012657090319019319) -- (-0.17000056659706542,0.012664083794792284);
\draw[line width=1.5pt,color=wwwwww] (-0.17000056659706542,0.012664083794792284) -- (-0.16500057215720995,0.012669869527558898);
\draw[line width=1.5pt,color=wwwwww] (-0.16500057215720995,0.012669869527558898) -- (-0.1600005777173545,0.012674536828566108);
\draw[line width=1.5pt,color=wwwwww] (-0.1600005777173545,0.012674536828566108) -- (-0.15500058327749902,0.012678172444358954);
\draw[line width=1.5pt,color=wwwwww] (-0.15500058327749902,0.012678172444358954) -- (-0.15000058883764356,0.012680860505261805);
\draw[line width=1.5pt,color=wwwwww] (-0.15000058883764356,0.012680860505261805) -- (-0.1450005943977881,0.012682682482132267);
\draw[line width=1.5pt,color=wwwwww] (-0.1450005943977881,0.012682682482132267) -- (-0.14000059995793263,0.012683717150729079);
\draw[line width=1.5pt,color=wwwwww] (-0.14000059995793263,0.012683717150729079) -- (-0.13500060551807716,0.012684040563082345);
\draw[line width=1.5pt,color=wwwwww] (-0.13500060551807716,0.012684040563082345) -- (-0.1300006110782217,0.0126837260252424);
\draw[line width=1.5pt,color=wwwwww] (-0.1300006110782217,0.0126837260252424) -- (-0.12500061663836623,0.012682844080827241);
\draw[line width=1.5pt,color=wwwwww] (-0.12500061663836623,0.012682844080827241) -- (-0.12000062219851076,0.012681462499804277);
\draw[line width=1.5pt,color=wwwwww] (-0.12000062219851076,0.012681462499804277) -- (-0.1150006277586553,0.012679646271959132);
\draw[line width=1.5pt,color=wwwwww] (-0.1150006277586553,0.012679646271959132) -- (-0.11000063331879983,0.012677457604530248);
\draw[line width=1.5pt,color=wwwwww] (-0.11000063331879983,0.012677457604530248) -- (-0.10500063887894437,0.012674955923528164);
\draw[line width=1.5pt,color=wwwwww] (-0.10500063887894437,0.012674955923528164) -- (-0.1000006444390889,0.012672197878244628);
\draw[line width=1.5pt,color=wwwwww] (-0.1000006444390889,0.012672197878244628) -- (-0.09500064999923344,0.012669237348529596);
\draw[line width=1.5pt,color=wwwwww] (-0.09500064999923344,0.012669237348529596) -- (-0.09000065555937797,0.012666125454393351);
\draw[line width=1.5pt,color=wwwwww] (-0.09000065555937797,0.012666125454393351) -- (-0.08500066111952251,0.012662910567535322);
\draw[line width=1.5pt,color=wwwwww] (-0.08500066111952251,0.012662910567535322) -- (-0.08000066667966704,0.012659638324439492);
\draw[line width=1.5pt,color=wwwwww] (-0.08000066667966704,0.012659638324439492) -- (-0.07500067223981158,0.012656351640660327);
\draw[line width=1.5pt,color=wwwwww] (-0.07500067223981158,0.012656351640660327) -- (-0.07000067779995611,0.01265309072598273);
\draw[line width=1.5pt,color=wwwwww] (-0.07000067779995611,0.01265309072598273) -- (-0.06500068336010065,0.01264989310013675);
\draw[line width=1.5pt,color=wwwwww] (-0.06500068336010065,0.01264989310013675) -- (-0.06000068892024518,0.012646793608805346);
\draw[line width=1.5pt,color=wwwwww] (-0.06000068892024518,0.012646793608805346) -- (-0.05500069448038972,0.012643824439610818);
\draw[line width=1.5pt,color=wwwwww] (-0.05500069448038972,0.012643824439610818) -- (-0.05000070004053425,0.01264101513789359);
\draw[line width=1.5pt,color=wwwwww] (-0.05000070004053425,0.01264101513789359) -- (-0.045000705600678786,0.01263839262200564);
\draw[line width=1.5pt,color=wwwwww] (-0.045000705600678786,0.01263839262200564) -- (-0.04000071116082332,0.012635981197965237);
\draw[line width=1.5pt,color=wwwwww] (-0.04000071116082332,0.012635981197965237) -- (-0.035000716720967856,0.012633802573241827);
\draw[line width=1.5pt,color=wwwwww] (-0.035000716720967856,0.012633802573241827) -- (-0.030000722281112394,0.012631875869529248);
\draw[line width=1.5pt,color=wwwwww] (-0.030000722281112394,0.012631875869529248) -- (-0.025000727841256933,0.01263021763436697);
\draw[line width=1.5pt,color=wwwwww] (-0.025000727841256933,0.01263021763436697) -- (-0.02000073340140147,0.012628841851458034);
\draw[line width=1.5pt,color=wwwwww] (-0.02000073340140147,0.012628841851458034) -- (-0.015000738961546009,0.012627759949587916);
\draw[line width=1.5pt,color=wwwwww] (-0.015000738961546009,0.012627759949587916) -- (-0.010000744521690547,0.012626980810062123);
\draw[line width=1.5pt,color=wwwwww] (-0.010000744521690547,0.012626980810062123) -- (-0.005000750081835085,0.012626510772540211);
\draw[line width=1.5pt,color=wwwwww] (-0.005000750081835085,0.012626510772540211) -- (0,0.01262635363927);
\draw[line width=1.5pt,color=wwwwww] (0,0.01262635363927) -- (0.004999994439855463,0.012626510725085208);
\draw[line width=1.5pt,color=wwwwww] (0.004999994439855463,0.012626510725085208) -- (0.009999988879710925,0.012626980715544587);
\draw[line width=1.5pt,color=wwwwww] (0.009999988879710925,0.012626980715544587) -- (0.014999983319566389,0.012627759808772518);
\draw[line width=1.5pt,color=wwwwww] (0.014999983319566389,0.012627759808772518) -- (0.01999997775942185,0.012628841665491058);
\draw[line width=1.5pt,color=wwwwww] (0.01999997775942185,0.012628841665491058) -- (0.024999972199277312,0.01263021740478678);
\draw[line width=1.5pt,color=wwwwww] (0.024999972199277312,0.01263021740478678) -- (0.029999966639132774,0.012631875598254638);
\draw[line width=1.5pt,color=wwwwww] (0.029999966639132774,0.012631875598254638) -- (0.034999961078988236,0.012633802262580564);
\draw[line width=1.5pt,color=wwwwww] (0.034999961078988236,0.012633802262580564) -- (0.0399999555188437,0.012635980850609836);
\draw[line width=1.5pt,color=wwwwww] (0.0399999555188437,0.012635980850609836) -- (0.044999949958699166,0.012638392241041343);
\draw[line width=1.5pt,color=wwwwww] (0.044999949958699166,0.012638392241041343) -- (0.04999994439855463,0.012641014726798624);
\draw[line width=1.5pt,color=wwwwww] (0.04999994439855463,0.012641014726798624) -- (0.054999938838410097,0.012643824002256615);
\draw[line width=1.5pt,color=wwwwww] (0.054999938838410097,0.012643824002256615) -- (0.05999993327826556,0.01264679314945682);
\draw[line width=1.5pt,color=wwwwww] (0.05999993327826556,0.01264679314945682) -- (0.06499992771812102,0.012649892623457618);
\draw[line width=1.5pt,color=wwwwww] (0.06499992771812102,0.012649892623457618) -- (0.06999992215797649,0.01265309023703499);
\draw[line width=1.5pt,color=wwwwww] (0.06999992215797649,0.01265309023703499) -- (0.07499991659783195,0.012656351144917389);
\draw[line width=1.5pt,color=wwwwww] (0.07499991659783195,0.012656351144917389) -- (0.07999991103768742,0.012659637827771136);
\draw[line width=1.5pt,color=wwwwww] (0.07999991103768742,0.012659637827771136) -- (0.08499990547754288,0.01266291007622378);
\draw[line width=1.5pt,color=wwwwww] (0.08499990547754288,0.01266291007622378) -- (0.08999989991739835,0.012666124975129808);
\draw[line width=1.5pt,color=wwwwww] (0.08999989991739835,0.012666124975129808) -- (0.09499989435725381,0.01266923688841727);
\draw[line width=1.5pt,color=wwwwww] (0.09499989435725381,0.01266923688841727) -- (0.09999988879710928,0.012672197444800148);
\draw[line width=1.5pt,color=wwwwww] (0.09999988879710928,0.012672197444800148) -- (0.10499988323696474,0.012674955524686173);
\draw[line width=1.5pt,color=wwwwww] (0.10499988323696474,0.012674955524686173) -- (0.1099998776768202,0.012677457248641883);
\draw[line width=1.5pt,color=wwwwww] (0.1099998776768202,0.012677457248641883) -- (0.11499987211667567,0.01267964596779547);
\draw[line width=1.5pt,color=wwwwww] (0.11499987211667567,0.01267964596779547) -- (0.11999986655653114,0.012681462256556307);
\draw[line width=1.5pt,color=wwwwww] (0.11999986655653114,0.012681462256556307) -- (0.1249998609963866,0.012682843908108568);
\draw[line width=1.5pt,color=wwwwww] (0.1249998609963866,0.012682843908108568) -- (0.12999985543624207,0.01268372593308151);
\draw[line width=1.5pt,color=wwwwww] (0.12999985543624207,0.01268372593308151) -- (0.13499984987609753,0.012684040561933998);
\draw[line width=1.5pt,color=wwwwww] (0.13499984987609753,0.012684040561933998) -- (0.139999844315953,0.012683717251463202);
\draw[line width=1.5pt,color=wwwwww] (0.139999844315953,0.012683717251463202) -- (0.14499983875580846,0.012682682696034274);
\draw[line width=1.5pt,color=wwwwww] (0.14499983875580846,0.012682682696034274) -- (0.14999983319566393,0.012680860844031184);
\draw[line width=1.5pt,color=wwwwww] (0.14999983319566393,0.012680860844031184) -- (0.1549998276355194,0.012678172920105682);
\draw[line width=1.5pt,color=wwwwww] (0.1549998276355194,0.012678172920105682) -- (0.15999982207537486,0.012674537453805555);
\draw[line width=1.5pt,color=wwwwww] (0.15999982207537486,0.012674537453805555) -- (0.16499981651523032,0.012669870315205119);
\draw[line width=1.5pt,color=wwwwww] (0.16499981651523032,0.012669870315205119) -- (0.1699998109550858,0.01266408475814849);
\draw[line width=1.5pt,color=wwwwww] (0.1699998109550858,0.01266408475814849) -- (0.17499980539494125,0.012657091471774814);
\draw[line width=1.5pt,color=wwwwww] (0.17499980539494125,0.012657091471774814) -- (0.17999979983479672,0.012648798640971676);
\draw[line width=1.5pt,color=wwwwww] (0.17999979983479672,0.012648798640971676) -- (0.18499979427465219,0.012639112016434912);
\draw[line width=1.5pt,color=wwwwww] (0.18499979427465219,0.012639112016434912) -- (0.18999978871450765,0.012627934995063397);
\draw[line width=1.5pt,color=wwwwww] (0.18999978871450765,0.012627934995063397) -- (0.19499978315436312,0.012615168711346708);
\draw[line width=1.5pt,color=wwwwww] (0.19499978315436312,0.012615168711346708) -- (0.19999977759421858,0.012600712140471251);
\draw[line width=1.5pt,color=wwwwww] (0.19999977759421858,0.012600712140471251) -- (0.20499977203407405,0.012584462213876737);
\draw[line width=1.5pt,color=wwwwww] (0.20499977203407405,0.012584462213876737) -- (0.2099997664739295,0.01256631394796474);
\draw[line width=1.5pt,color=wwwwww] (0.2099997664739295,0.01256631394796474) -- (0.21499976091378498,0.012546160586656313);
\draw[line width=1.5pt,color=wwwwww] (0.21499976091378498,0.012546160586656313) -- (0.21999975535364044,0.012523893758518923);
\draw[line width=1.5pt,color=wwwwww] (0.21999975535364044,0.012523893758518923) -- (0.2249997497934959,0.012499403649135063);
\draw[line width=1.5pt,color=wwwwww] (0.2249997497934959,0.012499403649135063) -- (0.22999974423335137,0.01247257918938703);
\draw[line width=1.5pt,color=wwwwww] (0.22999974423335137,0.01247257918938703) -- (0.23499973867320684,0.01244330826029396);
\draw[line width=1.5pt,color=wwwwww] (0.23499973867320684,0.01244330826029396) -- (0.2399997331130623,0.012411477914983919);
\draw[line width=1.5pt,color=wwwwww] (0.2399997331130623,0.012411477914983919) -- (0.24499972755291777,0.012376974618406332);
\draw[line width=1.5pt,color=wwwwww] (0.24499972755291777,0.012376974618406332) -- (0.24999972199277323,0.012339684505266431);
\draw[line width=1.5pt,color=wwwwww] (0.24999972199277323,0.012339684505266431) -- (0.2549997164326287,0.012299493656632875);
\draw[line width=1.5pt,color=wwwwww] (0.2549997164326287,0.012299493656632875) -- (0.25999971087248414,0.012256288395638117);
\draw[line width=1.5pt,color=wwwwww] (0.25999971087248414,0.012256288395638117) -- (0.2649997053123396,0.012209955602551642);
\draw[line width=1.5pt,color=wwwwww] (0.2649997053123396,0.012209955602551642) -- (0.269999699752195,0.012160383049474055);
\draw[line width=1.5pt,color=wwwwww] (0.269999699752195,0.012160383049474055) -- (0.27499969419205045,0.012107459754757823);
\draw[line width=1.5pt,color=wwwwww] (0.27499969419205045,0.012107459754757823) -- (0.2799996886319059,0.012051076357204307);
\draw[line width=1.5pt,color=wwwwww] (0.2799996886319059,0.012051076357204307) -- (0.2849996830717613,0.011991125509894125);
\draw[line width=1.5pt,color=wwwwww] (0.2849996830717613,0.011991125509894125) -- (0.28999967751161676,0.011927502293463114);
\draw[line width=1.5pt,color=wwwwww] (0.28999967751161676,0.011927502293463114) -- (0.2949996719514722,0.011860104648406972);
\draw[line width=1.5pt,color=wwwwww] (0.2949996719514722,0.011860104648406972) -- (0.29999966639132764,0.011788833825921599);
\draw[line width=1.5pt,color=wwwwww] (0.29999966639132764,0.011788833825921599) -- (0.3049996608311831,0.011713594856539071);
\draw[line width=1.5pt,color=wwwwww] (0.3049996608311831,0.011713594856539071) -- (0.3099996552710385,0.01163429703571651);
\draw[line width=1.5pt,color=wwwwww] (0.3099996552710385,0.01163429703571651) -- (0.31499964971089395,0.011550854425280633);
\draw[line width=1.5pt,color=wwwwww] (0.31499964971089395,0.011550854425280633) -- (0.3199996441507494,0.011463186369461783);
\draw[line width=1.5pt,color=wwwwww] (0.3199996441507494,0.011463186369461783) -- (0.3249996385906048,0.011371218024045825);
\draw[line width=1.5pt,color=wwwwww] (0.3249996385906048,0.011371218024045825) -- (0.32999963303046026,0.011274880896900947);
\draw[line width=1.5pt,color=wwwwww] (0.32999963303046026,0.011274880896900947) -- (0.3349996274703157,0.011174113397948106);
\draw[line width=1.5pt,color=wwwwww] (0.3349996274703157,0.011174113397948106) -- (0.33999962191017113,0.011068861396420071);
\draw[line width=1.5pt,color=wwwwww] (0.33999962191017113,0.011068861396420071) -- (0.34499961635002657,0.010959078782905652);
\draw[line width=1.5pt,color=wwwwww] (0.34499961635002657,0.010959078782905652) -- (0.349999610789882,0.010844728033576946);
\draw[line width=1.5pt,color=wwwwww] (0.349999610789882,0.010844728033576946) -- (0.35499960522973745,0.010725780773582138);
\draw[line width=1.5pt,color=wwwwww] (0.35499960522973745,0.010725780773582138) -- (0.3599995996695929,0.010602218336463054);
\draw[line width=1.5pt,color=wwwwww] (0.3599995996695929,0.010602218336463054) -- (0.3649995941094483,0.010474032316104583);
\draw[line width=1.5pt,color=wwwwww] (0.3649995941094483,0.010474032316104583) -- (0.36999958854930376,0.010341225107544936);
\draw[line width=1.5pt,color=wwwwww] (0.36999958854930376,0.010341225107544936) -- (0.3749995829891592,0.010203810432668938);
\draw[line width=1.5pt,color=wwwwww] (0.3749995829891592,0.010203810432668938) -- (0.37999957742901463,0.010061813846636129);
\draw[line width=1.5pt,color=wwwwww] (0.37999957742901463,0.010061813846636129) -- (0.38499957186887007,0.009915273220609117);
\draw[line width=1.5pt,color=wwwwww] (0.38499957186887007,0.009915273220609117) -- (0.3899995663087255,0.009764239196235112);
\draw[line width=1.5pt,color=wwwwww] (0.3899995663087255,0.009764239196235112) -- (0.39499956074858095,0.009608775607031885);
\draw[line width=1.5pt,color=wwwwww] (0.39499956074858095,0.009608775607031885) -- (0.3999995551884364,0.009448959861834102);
\draw[line width=1.5pt,color=wwwwww] (0.3999995551884364,0.009448959861834102) -- (0.4049995496282918,0.009284883285168464);
\draw[line width=1.5pt,color=wwwwww] (0.4049995496282918,0.009284883285168464) -- (0.40999954406814726,0.009116651409474423);
\draw[line width=1.5pt,color=wwwwww] (0.40999954406814726,0.009116651409474423) -- (0.4149995385080027,0.008944384213963976);
\draw[line width=1.5pt,color=wwwwww] (0.4149995385080027,0.008944384213963976) -- (0.41999953294785813,0.00876821630494535);
\draw[line width=1.5pt,color=wwwwww] (0.41999953294785813,0.00876821630494535) -- (0.42499952738771357,0.00858829703244824);
\draw[line width=1.5pt,color=wwwwww] (0.42499952738771357,0.00858829703244824) -- (0.429999521827569,0.008404790538156392);
\draw[line width=1.5pt,color=wwwwww] (0.429999521827569,0.008404790538156392) -- (0.43499951626742445,0.008217875729735449);
\draw[line width=1.5pt,color=wwwwww] (0.43499951626742445,0.008217875729735449) -- (0.4399995107072799,0.008027746176904556);
\draw[line width=1.5pt,color=wwwwww] (0.4399995107072799,0.008027746176904556) -- (0.4449995051471353,0.007834609924913595);
\draw[line width=1.5pt,color=wwwwww] (0.4449995051471353,0.007834609924913595) -- (0.44999949958699076,0.007638689221386313);
\draw[line width=1.5pt,color=wwwwww] (0.44999949958699076,0.007638689221386313) -- (0.4549994940268462,0.007440220152964858);
\draw[line width=1.5pt,color=wwwwww] (0.4549994940268462,0.007440220152964858) -- (0.45999948846670163,0.007239452188687007);
\draw[line width=1.5pt,color=wwwwww] (0.45999948846670163,0.007239452188687007) -- (0.46499948290655707,0.007036647627587116);
\draw[line width=1.5pt,color=wwwwww] (0.46499948290655707,0.007036647627587116) -- (0.4699994773464125,0.006832080948695178);
\draw[line width=1.5pt,color=wwwwww] (0.4699994773464125,0.006832080948695178) -- (0.47499947178626795,0.006626038062317157);
\draw[line width=1.5pt,color=wwwwww] (0.47499947178626795,0.006626038062317157) -- (0.4799994662261234,0.006418815462284302);
\draw[line width=1.5pt,color=wwwwww] (0.4799994662261234,0.006418815462284302) -- (0.4849994606659788,0.00621071927974231);
\draw[line width=1.5pt,color=wwwwww] (0.4849994606659788,0.00621071927974231) -- (0.48999945510583426,0.006002064239982985);
\draw[line width=1.5pt,color=wwwwww] (0.48999945510583426,0.006002064239982985) -- (0.4949994495456897,0.005793172524792083);
\draw[line width=1.5pt,color=wwwwww] (0.4949994495456897,0.005793172524792083) -- (0.49999944398554513,0.0055843725438888174);
\draw[line width=1.5pt,color=wwwwww] (0.49999944398554513,0.0055843725438888174) -- (0.5049994384254006,0.005375997620061409);
\draw[line width=1.5pt,color=wwwwww] (0.5049994384254006,0.005375997620061409) -- (0.5099994328652561,0.005168384593744012);
\draw[line width=1.5pt,color=wwwwww] (0.5099994328652561,0.005168384593744012) -- (0.5149994273051115,0.004961872353955767);
\draw[line width=1.5pt,color=wwwwww] (0.5149994273051115,0.004961872353955767) -- (0.5199994217449669,0.004756800303591635);
\draw[line width=1.5pt,color=wwwwww] (0.5199994217449669,0.004756800303591635) -- (0.5249994161848224,0.004553506768258476);
\draw[line width=1.5pt,color=wwwwww] (0.5249994161848224,0.004553506768258476) -- (0.5299994106246778,0.004352327358923281);
\draw[line width=1.5pt,color=wwwwww] (0.5299994106246778,0.004352327358923281) -- (0.5349994050645333,0.004153593299697469);
\draw[line width=1.5pt,color=wwwwww] (0.5349994050645333,0.004153593299697469) -- (0.5399993995043887,0.003957629733098811);
\draw[line width=1.5pt,color=wwwwww] (0.5399993995043887,0.003957629733098811) -- (0.5449993939442441,0.00376475401604314);
\draw[line width=1.5pt,color=wwwwww] (0.5449993939442441,0.00376475401604314) -- (0.5499993883840996,0.0035752740206273937);
\draw[line width=1.5pt,color=wwwwww] (0.5499993883840996,0.0035752740206273937) -- (0.554999382823955,0.0033894864544517776);
\draw[line width=1.5pt,color=wwwwww] (0.554999382823955,0.0033894864544517776) -- (0.5599993772638104,0.0032076752157938924);
\draw[line width=1.5pt,color=wwwwww] (0.5599993772638104,0.0032076752157938924) -- (0.5649993717036659,0.003030109799327577);
\draw[line width=1.5pt,color=wwwwww] (0.5649993717036659,0.003030109799327577) -- (0.5699993661435213,0.00285704376827338);
\draw[line width=1.5pt,color=wwwwww] (0.5699993661435213,0.00285704376827338) -- (0.5749993605833768,0.002688713308913197);
\draw[line width=1.5pt,color=wwwwww] (0.5749993605833768,0.002688713308913197) -- (0.5799993550232322,0.0025253358831595226);
\draw[line width=1.5pt,color=wwwwww] (0.5799993550232322,0.0025253358831595226) -- (0.5849993494630876,0.0023671089944988346);
\draw[line width=1.5pt,color=wwwwww] (0.5849993494630876,0.0023671089944988346) -- (0.5899993439029431,0.0022142090819481926);
\draw[line width=1.5pt,color=wwwwww] (0.5899993439029431,0.0022142090819481926) -- (0.5949993383427985,0.0020667905557826523);
\draw[line width=1.5pt,color=wwwwww] (0.5949993383427985,0.0020667905557826523) -- (0.5999993327826539,0.0019249849876945776);
\draw[line width=1.5pt,color=wwwwww] (0.5999993327826539,0.0019249849876945776) -- (0.6049993272225094,0.0017889004666639958);
\draw[line width=1.5pt,color=wwwwww] (0.6049993272225094,0.0017889004666639958) -- (0.6099993216623648,0.0016586211302651122);
\draw[line width=1.5pt,color=wwwwww] (0.6099993216623648,0.0016586211302651122) -- (0.6149993161022203,0.001534206879329498);
\draw[line width=1.5pt,color=wwwwww] (0.6149993161022203,0.001534206879329498) -- (0.6199993105420757,0.0014156932819090436);
\draw[line width=1.5pt,color=wwwwww] (0.6199993105420757,0.0014156932819090436) -- (0.6249993049819311,0.0013030916703299423);
\draw[line width=1.5pt,color=wwwwww] (0.6249993049819311,0.0013030916703299423) -- (0.6299992994217866,0.0011963894328209663);
\draw[line width=1.5pt,color=wwwwww] (0.6299992994217866,0.0011963894328209663) -- (0.634999293861642,0.001095550498792886);
\draw[line width=1.5pt,color=wwwwww] (0.634999293861642,0.001095550498792886) -- (0.6399992883014974,0.0010005160143672343);
\draw[line width=1.5pt,color=wwwwww] (0.6399992883014974,0.0010005160143672343) -- (0.6449992827413529,0);
\draw[line width=1.5pt,color=wwwwww] (0.6449992827413529,0) -- (0.6499992771812083,0);
\draw[line width=1.5pt,color=wwwwww] (0.6499992771812083,0) -- (0.6549992716210638,0);
\draw[line width=1.5pt,color=wwwwww] (0.6549992716210638,0) -- (0.6599992660609192,0);
\draw[line width=1.5pt,color=wwwwww] (0.6599992660609192,0) -- (0.6649992605007746,0);
\draw[line width=1.5pt,color=wwwwww] (0.6649992605007746,0) -- (0.6699992549406301,0);
\draw[line width=1.5pt,color=wwwwww] (0.6699992549406301,0) -- (0.6749992493804855,0);
\draw[line width=1.5pt,color=wwwwww] (0.6749992493804855,0) -- (0.6799992438203409,0);
\draw[line width=1.5pt,color=wwwwww] (0.6799992438203409,0) -- (0.6849992382601964,0);
\draw[line width=1.5pt,color=wwwwww] (0.6849992382601964,0) -- (0.6899992327000518,0);
\draw[line width=1.5pt,color=wwwwww] (0.6899992327000518,0) -- (0.6949992271399072,0);
\draw[line width=1.5pt,color=wwwwww] (0.6949992271399072,0) -- (0.6999992215797627,0);
\draw[line width=1.5pt,color=wwwwww] (0.6999992215797627,0) -- (0.7049992160196181,0);
\draw[line width=1.5pt,color=wwwwww] (0.7049992160196181,0) -- (0.7099992104594736,0);
\draw[line width=1.5pt,color=wwwwww] (0.7099992104594736,0) -- (0.714999204899329,0);
\draw[line width=1.5pt,color=wwwwww] (0.714999204899329,0) -- (0.7199991993391844,0);
\draw[line width=1.5pt,color=wwwwww] (0.7199991993391844,0) -- (0.7249991937790399,0);
\draw[line width=1.5pt,color=wwwwww] (0.7249991937790399,0) -- (0.7299991882188953,0);
\draw[line width=1.5pt,color=wwwwww] (0.7299991882188953,0) -- (0.7349991826587507,0);
\draw[line width=1.5pt,color=wwwwww] (0.7349991826587507,0) -- (0.7399991770986062,0);
\draw[line width=1.5pt,color=wwwwww] (0.7399991770986062,0) -- (0.7449991715384616,0);
\draw[line width=1.5pt,color=wwwwww] (0.7449991715384616,0) -- (0.7499991659783171,0);
\draw[line width=1.5pt,color=wwwwww] (0.7499991659783171,0) -- (0.7549991604181725,0);
\draw[line width=1.5pt,color=wwwwww] (0.7549991604181725,0) -- (0.7599991548580279,0);
\draw[line width=1.5pt,color=wwwwww] (0.7599991548580279,0) -- (0.7649991492978834,0);
\draw[line width=1.5pt,color=wwwwww] (0.7649991492978834,0) -- (0.7699991437377388,0);
\draw[line width=1.5pt,color=wwwwww] (0.7699991437377388,0) -- (0.7749991381775942,0);
\draw[line width=1.5pt,color=wwwwww] (0.7749991381775942,0) -- (0.7799991326174497,0);
\draw[line width=1.5pt,color=wwwwww] (0.7799991326174497,0) -- (0.7849991270573051,0);
\draw[line width=1.5pt,color=wwwwww] (0.7849991270573051,0) -- (0.7899991214971606,0);
\draw[line width=1.5pt,color=wwwwww] (0.7899991214971606,0) -- (0.794999115937016,0);
\draw[line width=1.5pt,color=wwwwww] (0.794999115937016,0) -- (0.7999991103768714,0);
\draw[line width=1.5pt,color=wwwwww] (0.7999991103768714,0) -- (0.8049991048167269,0);
\draw[line width=1.5pt,color=wwwwww] (0.8049991048167269,0) -- (0.8099990992565823,0);
\draw[line width=1.5pt,color=wwwwww] (0.8099990992565823,0) -- (0.8149990936964377,0);
\draw[line width=1.5pt,color=wwwwww] (0.8149990936964377,0) -- (0.8199990881362932,0);
\draw[line width=1.5pt,color=wwwwww] (0.8199990881362932,0) -- (0.8249990825761486,0);
\draw[line width=1.5pt,color=wwwwww] (0.8249990825761486,0) -- (0.8299990770160041,0);
\draw[line width=1.5pt,color=wwwwww] (0.8299990770160041,0) -- (0.8349990714558595,0);
\draw[line width=1.5pt,color=wwwwww] (0.8349990714558595,0) -- (0.8399990658957149,0);
\draw[line width=1.5pt,color=wwwwww] (0.8399990658957149,0) -- (0.8449990603355704,0);
\draw[line width=1.5pt,color=wwwwww] (0.8449990603355704,0) -- (0.8499990547754258,0);
\draw[line width=1.5pt,color=wwwwww] (0.8499990547754258,0) -- (0.8549990492152812,0);
\draw[line width=1.5pt,color=wwwwww] (0.8549990492152812,0) -- (0.8599990436551367,0);
\draw[line width=1.5pt,color=wwwwww] (0.8599990436551367,0) -- (0.8649990380949921,0);
\draw[line width=1.5pt,color=wwwwww] (0.8649990380949921,0) -- (0.8699990325348476,0);
\draw[line width=1.5pt,color=wwwwww] (0.8699990325348476,0) -- (0.874999026974703,0);
\draw[line width=1.5pt,color=wwwwww] (0.874999026974703,0) -- (0.8799990214145584,0);
\draw[line width=1.5pt,color=wwwwww] (0.8799990214145584,0) -- (0.8849990158544139,0);
\draw[line width=1.5pt,color=wwwwww] (0.8849990158544139,0) -- (0.8899990102942693,0);
\draw[line width=1.5pt,color=wwwwww] (0.8899990102942693,0) -- (0.8949990047341247,0);
\draw[line width=1.5pt,color=wwwwww] (0.8949990047341247,0) -- (0.8999989991739802,0);
\draw[line width=1.5pt,color=wwwwww] (0.8999989991739802,0) -- (0.9049989936138356,0);
\draw[line width=1.5pt,color=wwwwww] (0.9049989936138356,0) -- (0.9099989880536911,0);
\draw[line width=1.5pt,color=wwwwww] (0.9099989880536911,0) -- (0.9149989824935465,0);
\draw[line width=1.5pt,color=wwwwww] (0.9149989824935465,0) -- (0.9199989769334019,0);
\draw[line width=1.5pt,color=wwwwww] (0.9199989769334019,0) -- (0.9249989713732574,0);
\draw[line width=1.5pt,color=wwwwww] (0.9249989713732574,0) -- (0.9299989658131128,0);
\draw[line width=1.5pt,color=wwwwww] (0.9299989658131128,0) -- (0.9349989602529682,0);
\draw[line width=1.5pt,color=wwwwww] (0.9349989602529682,0) -- (0.9399989546928237,0);
\draw[line width=1.5pt,color=wwwwww] (0.9399989546928237,0) -- (0.9449989491326791,0);
\draw[line width=1.5pt,color=wwwwww] (0.9449989491326791,0) -- (0.9499989435725346,0);
\draw[line width=1.5pt,color=wwwwww] (0.9499989435725346,0) -- (0.95499893801239,0);
\draw[line width=1.5pt,color=wwwwww] (0.95499893801239,0) -- (0.9599989324522454,0);
\draw[line width=1.5pt,color=wwwwww] (0.9599989324522454,0) -- (0.9649989268921009,0);
\draw[line width=1.5pt,color=wwwwww] (0.9649989268921009,0) -- (0.9699989213319563,0);
\draw[line width=1.5pt,color=wwwwww] (0.9699989213319563,0) -- (0.9749989157718117,0);
\draw[line width=1.5pt,color=wwwwww] (0.9749989157718117,0) -- (0.9799989102116672,0);
\draw[line width=1.5pt,color=wwwwww] (0.9799989102116672,0) -- (0.9849989046515226,0);
\draw[line width=1.5pt,color=wwwwww] (0.9849989046515226,0) -- (0.9899988990913781,0);
\draw[line width=1.5pt,color=wwwwww] (0.9899988990913781,0) -- (0.9949988935312335,0);
\end{axis}
\end{tikzpicture}}
    \subfloat[$ h=0, \beta<1 $.]{ 
%<<<<<<<WARNING>>>>>>>
% PGF/Tikz doesn't support the following mathematical functions:
% cosh, acosh, sinh, asinh, tanh, atanh,
% x^r with r not integer

% Plotting will be done using GNUPLOT
% GNUPLOT must be installed and you must allow Latex to call external
% programs by adding the following option to your compiler
% shell-escape    OR    enable-write18 
% Example: pdflatex --shell-escape file.tex 

\definecolor{wwwwww}{rgb}{0.4,0.4,0.4}
\begin{tikzpicture}[line cap=round,line join=round,>=triangle 45,x=1cm,y=1cm]
\begin{axis}[
x=3.5cm,y=4cm,
axis lines=middle,
xmin=-1.2,
xmax=1.2,
ymin=-0.027640247367904026,
ymax=0.1,
xtick={-1,-0.8,...,1},
ytick={-0.02,0,...,0.1},
y post scale = 12,
ylabel = $f(x)$,
xlabel = $x$]
\clip(-1.3753518986999413,-0.02671256188785095) rectangle (1.4564267138000453,0.1350097836780805);
\draw[line width=1.5pt,color=wwwwww] (-0.9999996436130685,0) -- (-0.9999996436130685,0);
\draw[line width=1.5pt,color=wwwwww] (-0.9999996436130685,0) -- (-0.994999649173213,0);
\draw[line width=1.5pt,color=wwwwww] (-0.994999649173213,0) -- (-0.9899996547333576,0);
\draw[line width=1.5pt,color=wwwwww] (-0.9899996547333576,0) -- (-0.9849996602935022,0);
\draw[line width=1.5pt,color=wwwwww] (-0.9849996602935022,0) -- (-0.9799996658536467,0);
\draw[line width=1.5pt,color=wwwwww] (-0.9799996658536467,0) -- (-0.9749996714137913,0);
\draw[line width=1.5pt,color=wwwwww] (-0.9749996714137913,0) -- (-0.9699996769739359,0);
\draw[line width=1.5pt,color=wwwwww] (-0.9699996769739359,0) -- (-0.9649996825340804,0);
\draw[line width=1.5pt,color=wwwwww] (-0.9649996825340804,0) -- (-0.959999688094225,0);
\draw[line width=1.5pt,color=wwwwww] (-0.959999688094225,0) -- (-0.9549996936543695,0);
\draw[line width=1.5pt,color=wwwwww] (-0.9549996936543695,0) -- (-0.9499996992145141,0);
\draw[line width=1.5pt,color=wwwwww] (-0.9499996992145141,0) -- (-0.9449997047746587,0);
\draw[line width=1.5pt,color=wwwwww] (-0.9449997047746587,0) -- (-0.9399997103348032,0);
\draw[line width=1.5pt,color=wwwwww] (-0.9399997103348032,0) -- (-0.9349997158949478,0);
\draw[line width=1.5pt,color=wwwwww] (-0.9349997158949478,0) -- (-0.9299997214550924,0);
\draw[line width=1.5pt,color=wwwwww] (-0.9299997214550924,0) -- (-0.9249997270152369,0);
\draw[line width=1.5pt,color=wwwwww] (-0.9249997270152369,0) -- (-0.9199997325753815,0);
\draw[line width=1.5pt,color=wwwwww] (-0.9199997325753815,0) -- (-0.914999738135526,0);
\draw[line width=1.5pt,color=wwwwww] (-0.914999738135526,0) -- (-0.9099997436956706,0);
\draw[line width=1.5pt,color=wwwwww] (-0.9099997436956706,0) -- (-0.9049997492558152,0);
\draw[line width=1.5pt,color=wwwwww] (-0.9049997492558152,0) -- (-0.8999997548159597,0);
\draw[line width=1.5pt,color=wwwwww] (-0.8999997548159597,0) -- (-0.8949997603761043,0);
\draw[line width=1.5pt,color=wwwwww] (-0.8949997603761043,0) -- (-0.8899997659362489,0);
\draw[line width=1.5pt,color=wwwwww] (-0.8899997659362489,0) -- (-0.8849997714963934,0);
\draw[line width=1.5pt,color=wwwwww] (-0.8849997714963934,0) -- (-0.879999777056538,0);
\draw[line width=1.5pt,color=wwwwww] (-0.879999777056538,0) -- (-0.8749997826166825,0);
\draw[line width=1.5pt,color=wwwwww] (-0.8749997826166825,0) -- (-0.8699997881768271,0);
\draw[line width=1.5pt,color=wwwwww] (-0.8699997881768271,0) -- (-0.8649997937369717,0);
\draw[line width=1.5pt,color=wwwwww] (-0.8649997937369717,0) -- (-0.8599997992971162,0);
\draw[line width=1.5pt,color=wwwwww] (-0.8599997992971162,0) -- (-0.8549998048572608,0);
\draw[line width=1.5pt,color=wwwwww] (-0.8549998048572608,0) -- (-0.8499998104174054,0);
\draw[line width=1.5pt,color=wwwwww] (-0.8499998104174054,0) -- (-0.8449998159775499,0);
\draw[line width=1.5pt,color=wwwwww] (-0.8449998159775499,0) -- (-0.8399998215376945,0);
\draw[line width=1.5pt,color=wwwwww] (-0.8399998215376945,0) -- (-0.834999827097839,0);
\draw[line width=1.5pt,color=wwwwww] (-0.834999827097839,0) -- (-0.8299998326579836,0);
\draw[line width=1.5pt,color=wwwwww] (-0.8299998326579836,0) -- (-0.8249998382181282,0);
\draw[line width=1.5pt,color=wwwwww] (-0.8249998382181282,0) -- (-0.8199998437782727,0);
\draw[line width=1.5pt,color=wwwwww] (-0.8199998437782727,0) -- (-0.8149998493384173,0);
\draw[line width=1.5pt,color=wwwwww] (-0.8149998493384173,0) -- (-0.8099998548985619,0);
\draw[line width=1.5pt,color=wwwwww] (-0.8099998548985619,0) -- (-0.8049998604587064,0);
\draw[line width=1.5pt,color=wwwwww] (-0.8049998604587064,0) -- (-0.799999866018851,0);
\draw[line width=1.5pt,color=wwwwww] (-0.799999866018851,0) -- (-0.7949998715789955,0);
\draw[line width=1.5pt,color=wwwwww] (-0.7949998715789955,0) -- (-0.7899998771391401,0);
\draw[line width=1.5pt,color=wwwwww] (-0.7899998771391401,0) -- (-0.7849998826992847,0);
\draw[line width=1.5pt,color=wwwwww] (-0.7849998826992847,0) -- (-0.7799998882594292,0);
\draw[line width=1.5pt,color=wwwwww] (-0.7799998882594292,0) -- (-0.7749998938195738,0);
\draw[line width=1.5pt,color=wwwwww] (-0.7749998938195738,0) -- (-0.7699998993797184,0);
\draw[line width=1.5pt,color=wwwwww] (-0.7699998993797184,0) -- (-0.7649999049398629,0);
\draw[line width=1.5pt,color=wwwwww] (-0.7649999049398629,0) -- (-0.7599999105000075,0);
\draw[line width=1.5pt,color=wwwwww] (-0.7599999105000075,0) -- (-0.754999916060152,0);
\draw[line width=1.5pt,color=wwwwww] (-0.754999916060152,0) -- (-0.7499999216202966,0);
\draw[line width=1.5pt,color=wwwwww] (-0.7499999216202966,0) -- (-0.7449999271804412,0);
\draw[line width=1.5pt,color=wwwwww] (-0.7449999271804412,0) -- (-0.7399999327405857,0);
\draw[line width=1.5pt,color=wwwwww] (-0.7399999327405857,0) -- (-0.7349999383007303,0);
\draw[line width=1.5pt,color=wwwwww] (-0.7349999383007303,0) -- (-0.7299999438608749,0);
\draw[line width=1.5pt,color=wwwwww] (-0.7299999438608749,0) -- (-0.7249999494210194,0);
\draw[line width=1.5pt,color=wwwwww] (-0.7249999494210194,0) -- (-0.719999954981164,0);
\draw[line width=1.5pt,color=wwwwww] (-0.719999954981164,0) -- (-0.7149999605413085,0);
\draw[line width=1.5pt,color=wwwwww] (-0.7149999605413085,0) -- (-0.7099999661014531,0);
\draw[line width=1.5pt,color=wwwwww] (-0.7099999661014531,0) -- (-0.7049999716615977,0);
\draw[line width=1.5pt,color=wwwwww] (-0.7049999716615977,0) -- (-0.6999999772217422,0);
\draw[line width=1.5pt,color=wwwwww] (-0.6999999772217422,0) -- (-0.6949999827818868,0);
\draw[line width=1.5pt,color=wwwwww] (-0.6949999827818868,0) -- (-0.6899999883420314,0);
\draw[line width=1.5pt,color=wwwwww] (-0.6899999883420314,0) -- (-0.6849999939021759,0);
\draw[line width=1.5pt,color=wwwwww] (-0.6849999939021759,0) -- (-0.6799999994623205,0);
\draw[line width=1.5pt,color=wwwwww] (-0.6799999994623205,0) -- (-0.675000005022465,0);
\draw[line width=1.5pt,color=wwwwww] (-0.675000005022465,0) -- (-0.6700000105826096,0);
\draw[line width=1.5pt,color=wwwwww] (-0.6700000105826096,0) -- (-0.6650000161427542,0);
\draw[line width=1.5pt,color=wwwwww] (-0.6650000161427542,0) -- (-0.6600000217028987,0);
\draw[line width=1.5pt,color=wwwwww] (-0.6600000217028987,0) -- (-0.6550000272630433,0);
\draw[line width=1.5pt,color=wwwwww] (-0.6550000272630433,0) -- (-0.6500000328231879,0);
\draw[line width=1.5pt,color=wwwwww] (-0.6500000328231879,0) -- (-0.6450000383833324,0);
\draw[line width=1.5pt,color=wwwwww] (-0.6450000383833324,0) -- (-0.640000043943477,0);
\draw[line width=1.5pt,color=wwwwww] (-0.640000043943477,0) -- (-0.6350000495036215,0);
\draw[line width=1.5pt,color=wwwwww] (-0.6350000495036215,0) -- (-0.6300000550637661,0);
\draw[line width=1.5pt,color=wwwwww] (-0.6300000550637661,0) -- (-0.6250000606239107,0);
\draw[line width=1.5pt,color=wwwwww] (-0.6250000606239107,0) -- (-0.6200000661840552,0);
\draw[line width=1.5pt,color=wwwwww] (-0.6200000661840552,0) -- (-0.6150000717441998,0);
\draw[line width=1.5pt,color=wwwwww] (-0.6150000717441998,0) -- (-0.6100000773043444,0);
\draw[line width=1.5pt,color=wwwwww] (-0.6100000773043444,0) -- (-0.6050000828644889,0);
\draw[line width=1.5pt,color=wwwwww] (-0.6050000828644889,0) -- (-0.6000000884246335,0);
\draw[line width=1.5pt,color=wwwwww] (-0.6000000884246335,0) -- (-0.595000093984778,0);
\draw[line width=1.5pt,color=wwwwww] (-0.595000093984778,0) -- (-0.5900000995449226,0);
\draw[line width=1.5pt,color=wwwwww] (-0.5900000995449226,0) -- (-0.5850001051050672,0);
\draw[line width=1.5pt,color=wwwwww] (-0.5850001051050672,0) -- (-0.5800001106652117,0);
\draw[line width=1.5pt,color=wwwwww] (-0.5800001106652117,0) -- (-0.5750001162253563,0);
\draw[line width=1.5pt,color=wwwwww] (-0.5750001162253563,0) -- (-0.5700001217855009,0);
\draw[line width=1.5pt,color=wwwwww] (-0.5700001217855009,0) -- (-0.5650001273456454,0);
\draw[line width=1.5pt,color=wwwwww] (-0.5650001273456454,0) -- (-0.56000013290579,0);
\draw[line width=1.5pt,color=wwwwww] (-0.56000013290579,0) -- (-0.5550001384659345,0);
\draw[line width=1.5pt,color=wwwwww] (-0.5550001384659345,0) -- (-0.5500001440260791,0);
\draw[line width=1.5pt,color=wwwwww] (-0.5500001440260791,0) -- (-0.5450001495862237,0);
\draw[line width=1.5pt,color=wwwwww] (-0.5450001495862237,0) -- (-0.5400001551463682,0);
\draw[line width=1.5pt,color=wwwwww] (-0.5400001551463682,0) -- (-0.5350001607065128,0);
\draw[line width=1.5pt,color=wwwwww] (-0.5350001607065128,0) -- (-0.5300001662666574,0);
\draw[line width=1.5pt,color=wwwwww] (-0.5300001662666574,0) -- (-0.5250001718268019,0);
\draw[line width=1.5pt,color=wwwwww] (-0.5250001718268019,0) -- (-0.5200001773869465,0);
\draw[line width=1.5pt,color=wwwwww] (-0.5200001773869465,0) -- (-0.515000182947091,0);
\draw[line width=1.5pt,color=wwwwww] (-0.515000182947091,0) -- (-0.5100001885072356,0);
\draw[line width=1.5pt,color=wwwwww] (-0.5100001885072356,0) -- (-0.5050001940673802,0);
\draw[line width=1.5pt,color=wwwwww] (-0.5050001940673802,0) -- (-0.5000001996275247,0);
\draw[line width=1.5pt,color=wwwwww] (-0.5000001996275247,0) -- (-0.4950002051876693,0);
\draw[line width=1.5pt,color=wwwwww] (-0.4950002051876693,0) -- (-0.49000021074781386,0);
\draw[line width=1.5pt,color=wwwwww] (-0.49000021074781386,0) -- (-0.4850002163079584,0);
\draw[line width=1.5pt,color=wwwwww] (-0.4850002163079584,0) -- (-0.480000221868103,0);
\draw[line width=1.5pt,color=wwwwww] (-0.480000221868103,0) -- (-0.47500022742824755,0);
\draw[line width=1.5pt,color=wwwwww] (-0.47500022742824755,0) -- (-0.4700002329883921,0);
\draw[line width=1.5pt,color=wwwwww] (-0.4700002329883921,0) -- (-0.46500023854853667,0);
\draw[line width=1.5pt,color=wwwwww] (-0.46500023854853667,0) -- (-0.46000024410868123,0);
\draw[line width=1.5pt,color=wwwwww] (-0.46000024410868123,0) -- (-0.4550002496688258,0);
\draw[line width=1.5pt,color=wwwwww] (-0.4550002496688258,0) -- (-0.45000025522897036,0);
\draw[line width=1.5pt,color=wwwwww] (-0.45000025522897036,0) -- (-0.4450002607891149,0);
\draw[line width=1.5pt,color=wwwwww] (-0.4450002607891149,0) -- (-0.4400002663492595,0);
\draw[line width=1.5pt,color=wwwwww] (-0.4400002663492595,0) -- (-0.43500027190940405,0);
\draw[line width=1.5pt,color=wwwwww] (-0.43500027190940405,0) -- (-0.4300002774695486,0);
\draw[line width=1.5pt,color=wwwwww] (-0.4300002774695486,0) -- (-0.42500028302969317,0);
\draw[line width=1.5pt,color=wwwwww] (-0.42500028302969317,0) -- (-0.42000028858983773,0);
\draw[line width=1.5pt,color=wwwwww] (-0.42000028858983773,0) -- (-0.4150002941499823,0);
\draw[line width=1.5pt,color=wwwwww] (-0.4150002941499823,0) -- (-0.41000029971012686,0);
\draw[line width=1.5pt,color=wwwwww] (-0.41000029971012686,0) -- (-0.4050003052702714,0);
\draw[line width=1.5pt,color=wwwwww] (-0.4050003052702714,0) -- (-0.400000310830416,0);
\draw[line width=1.5pt,color=wwwwww] (-0.400000310830416,0) -- (-0.39500031639056055,0);
\draw[line width=1.5pt,color=wwwwww] (-0.39500031639056055,0) -- (-0.3900003219507051,0);
\draw[line width=1.5pt,color=wwwwww] (-0.3900003219507051,0) -- (-0.38500032751084967,0);
\draw[line width=1.5pt,color=wwwwww] (-0.38500032751084967,0) -- (-0.38000033307099423,0);
\draw[line width=1.5pt,color=wwwwww] (-0.38000033307099423,0) -- (-0.3750003386311388,0);
\draw[line width=1.5pt,color=wwwwww] (-0.3750003386311388,0) -- (-0.37000034419128336,0);
\draw[line width=1.5pt,color=wwwwww] (-0.37000034419128336,0) -- (-0.3650003497514279,0);
\draw[line width=1.5pt,color=wwwwww] (-0.3650003497514279,0) -- (-0.3600003553115725,0);
\draw[line width=1.5pt,color=wwwwww] (-0.3600003553115725,0) -- (-0.35500036087171705,0);
\draw[line width=1.5pt,color=wwwwww] (-0.35500036087171705,0) -- (-0.3500003664318616,0);
\draw[line width=1.5pt,color=wwwwww] (-0.3500003664318616,0) -- (-0.34500037199200617,0);
\draw[line width=1.5pt,color=wwwwww] (-0.34500037199200617,0) -- (-0.34000037755215073,0);
\draw[line width=1.5pt,color=wwwwww] (-0.34000037755215073,0) -- (-0.3350003831122953,0);
\draw[line width=1.5pt,color=wwwwww] (-0.3350003831122953,0) -- (-0.33000038867243986,0);
\draw[line width=1.5pt,color=wwwwww] (-0.33000038867243986,0) -- (-0.3250003942325844,0);
\draw[line width=1.5pt,color=wwwwww] (-0.3250003942325844,0) -- (-0.320000399792729,0);
\draw[line width=1.5pt,color=wwwwww] (-0.320000399792729,0) -- (-0.31500040535287355,0);
\draw[line width=1.5pt,color=wwwwww] (-0.31500040535287355,0) -- (-0.3100004109130181,0);
\draw[line width=1.5pt,color=wwwwww] (-0.3100004109130181,0) -- (-0.30500041647316267,0);
\draw[line width=1.5pt,color=wwwwww] (-0.30500041647316267,0) -- (-0.30000042203330723,0);
\draw[line width=1.5pt,color=wwwwww] (-0.30000042203330723,0) -- (-0.2950004275934518,0);
\draw[line width=1.5pt,color=wwwwww] (-0.2950004275934518,0) -- (-0.29000043315359636,0);
\draw[line width=1.5pt,color=wwwwww] (-0.29000043315359636,0) -- (-0.2850004387137409,0);
\draw[line width=1.5pt,color=wwwwww] (-0.2850004387137409,0) -- (-0.2800004442738855,0);
\draw[line width=1.5pt,color=wwwwww] (-0.2800004442738855,0) -- (-0.27500044983403005,0);
\draw[line width=1.5pt,color=wwwwww] (-0.27500044983403005,0) -- (-0.2700004553941746,0);
\draw[line width=1.5pt,color=wwwwww] (-0.2700004553941746,0) -- (-0.2650004609543192,0);
\draw[line width=1.5pt,color=wwwwww] (-0.2650004609543192,0) -- (-0.26000046651446374,0.0010158786442253781);
\draw[line width=1.5pt,color=wwwwww] (-0.26000046651446374,0.0010158786442253781) -- (-0.2550004720746083,0.0011829120583370853);
\draw[line width=1.5pt,color=wwwwww] (-0.2550004720746083,0.0011829120583370853) -- (-0.25000047763475286,0.001373085641467603);
\draw[line width=1.5pt,color=wwwwww] (-0.25000047763475286,0.001373085641467603) -- (-0.2450004831948974,0.0015888481415996707);
\draw[line width=1.5pt,color=wwwwww] (-0.2450004831948974,0.0015888481415996707) -- (-0.24000048875504193,0.0018327856924856107);
\draw[line width=1.5pt,color=wwwwww] (-0.24000048875504193,0.0018327856924856107) -- (-0.23500049431518646,0.0021076103421265265);
\draw[line width=1.5pt,color=wwwwww] (-0.23500049431518646,0.0021076103421265265) -- (-0.230000499875331,0.0024161449301925886);
\draw[line width=1.5pt,color=wwwwww] (-0.230000499875331,0.0024161449301925886) -- (-0.22500050543547553,0.002761304063720257);
\draw[line width=1.5pt,color=wwwwww] (-0.22500050543547553,0.002761304063720257) -- (-0.22000051099562007,0.0031460709882955387);
\draw[line width=1.5pt,color=wwwwww] (-0.22000051099562007,0.0031460709882955387) -- (-0.2150005165557646,0.0035734702114050146);
\draw[line width=1.5pt,color=wwwwww] (-0.2150005165557646,0.0035734702114050146) -- (-0.21000052211590914,0.004046535805647372);
\draw[line width=1.5pt,color=wwwwww] (-0.21000052211590914,0.004046535805647372) -- (-0.20500052767605367,0.004568275401693047);
\draw[line width=1.5pt,color=wwwwww] (-0.20500052767605367,0.004568275401693047) -- (-0.2000005332361982,0.005141629973425231);
\draw[line width=1.5pt,color=wwwwww] (-0.2000005332361982,0.005141629973425231) -- (-0.19500053879634274,0.005769429619522134);
\draw[line width=1.5pt,color=wwwwww] (-0.19500053879634274,0.005769429619522134) -- (-0.19000054435648728,0.006454345655065347);
\draw[line width=1.5pt,color=wwwwww] (-0.19000054435648728,0.006454345655065347) -- (-0.1850005499166318,0.007198839441829416);
\draw[line width=1.5pt,color=wwwwww] (-0.1850005499166318,0.007198839441829416) -- (-0.18000055547677635,0.008005108503965918);
\draw[line width=1.5pt,color=wwwwww] (-0.18000055547677635,0.008005108503965918) -- (-0.17500056103692088,0.008875030594420017);
\draw[line width=1.5pt,color=wwwwww] (-0.17500056103692088,0.008875030594420017) -- (-0.17000056659706542,0.009810106493135877);
\draw[line width=1.5pt,color=wwwwww] (-0.17000056659706542,0.009810106493135877) -- (-0.16500057215720995,0.010811402427829606);
\draw[line width=1.5pt,color=wwwwww] (-0.16500057215720995,0.010811402427829606) -- (-0.1600005777173545,0.011879493108150787);
\draw[line width=1.5pt,color=wwwwww] (-0.1600005777173545,0.011879493108150787) -- (-0.15500058327749902,0.013014406450763304);
\draw[line width=1.5pt,color=wwwwww] (-0.15500058327749902,0.013014406450763304) -- (-0.15000058883764356,0.01421557114279303);
\draw[line width=1.5pt,color=wwwwww] (-0.15000058883764356,0.01421557114279303) -- (-0.1450005943977881,0.01548176824061216);
\draw[line width=1.5pt,color=wwwwww] (-0.1450005943977881,0.01548176824061216) -- (-0.14000059995793263,0.016811088026951262);
\draw[line width=1.5pt,color=wwwwww] (-0.14000059995793263,0.016811088026951262) -- (-0.13500060551807716,0.01820089334909349);
\draw[line width=1.5pt,color=wwwwww] (-0.13500060551807716,0.01820089334909349) -- (-0.1300006110782217,0.019647790632034225);
\draw[line width=1.5pt,color=wwwwww] (-0.1300006110782217,0.019647790632034225) -- (-0.12500061663836623,0.021147609701474772);
\draw[line width=1.5pt,color=wwwwww] (-0.12500061663836623,0.021147609701474772) -- (-0.12000062219851076,0.022695393461376057);
\draw[line width=1.5pt,color=wwwwww] (-0.12000062219851076,0.022695393461376057) -- (-0.1150006277586553,0.02428539834947803);
\draw[line width=1.5pt,color=wwwwww] (-0.1150006277586553,0.02428539834947803) -- (-0.11000063331879983,0.025911106342526132);
\draw[line width=1.5pt,color=wwwwww] (-0.11000063331879983,0.025911106342526132) -- (-0.10500063887894437,0.027565249102706815);
\draw[line width=1.5pt,color=wwwwww] (-0.10500063887894437,0.027565249102706815) -- (-0.1000006444390889,0.029239844650526772);
\draw[line width=1.5pt,color=wwwwww] (-0.1000006444390889,0.029239844650526772) -- (-0.09500064999923344,0.030926246720955272);
\draw[line width=1.5pt,color=wwwwww] (-0.09500064999923344,0.030926246720955272) -- (-0.09000065555937797,0.0326152067132076);
\draw[line width=1.5pt,color=wwwwww] (-0.09000065555937797,0.0326152067132076) -- (-0.08500066111952251,0.03429694788564047);
\draw[line width=1.5pt,color=wwwwww] (-0.08500066111952251,0.03429694788564047) -- (-0.08000066667966704,0.035961251181661553);
\draw[line width=1.5pt,color=wwwwww] (-0.08000066667966704,0.035961251181661553) -- (-0.07500067223981158,0.03759755180660738);
\draw[line width=1.5pt,color=wwwwww] (-0.07500067223981158,0.03759755180660738) -- (-0.07000067779995611,0.03919504541633291);
\draw[line width=1.5pt,color=wwwwww] (-0.07000067779995611,0.03919504541633291) -- (-0.06500068336010065,0.04074280253237034);
\draw[line width=1.5pt,color=wwwwww] (-0.06500068336010065,0.04074280253237034) -- (-0.06000068892024518,0.042229889573169165);
\draw[line width=1.5pt,color=wwwwww] (-0.06000068892024518,0.042229889573169165) -- (-0.05500069448038972,0.04364549469222448);
\draw[line width=1.5pt,color=wwwwww] (-0.05500069448038972,0.04364549469222448) -- (-0.05000070004053425,0.04497905644887732);
\draw[line width=1.5pt,color=wwwwww] (-0.05000070004053425,0.04497905644887732) -- (-0.045000705600678786,0.04622039321041269);
\draw[line width=1.5pt,color=wwwwww] (-0.045000705600678786,0.04622039321041269) -- (-0.04000071116082332,0.04735983110074312);
\draw[line width=1.5pt,color=wwwwww] (-0.04000071116082332,0.04735983110074312) -- (-0.035000716720967856,0.04838832827357467);
\draw[line width=1.5pt,color=wwwwww] (-0.035000716720967856,0.04838832827357467) -- (-0.030000722281112394,0.049297593300075396);
\draw[line width=1.5pt,color=wwwwww] (-0.030000722281112394,0.049297593300075396) -- (-0.025000727841256933,0.050080195522853595);
\draw[line width=1.5pt,color=wwwwww] (-0.025000727841256933,0.050080195522853595) -- (-0.02000073340140147,0.050729665339384);
\draw[line width=1.5pt,color=wwwwww] (-0.02000073340140147,0.050729665339384) -- (-0.015000738961546009,0.05124058253748152);
\draw[line width=1.5pt,color=wwwwww] (-0.015000738961546009,0.05124058253748152) -- (-0.010000744521690547,0.05160865100950992);
\draw[line width=1.5pt,color=wwwwww] (-0.010000744521690547,0.05160865100950992) -- (-0.005000750081835085,0.05183075841626306);
\draw[line width=1.5pt,color=wwwwww] (-0.005000750081835085,0.05183075841626306) -- (0,0.051905019651199924);
\draw[line width=1.5pt,color=wwwwww] (0,0.051905019651199924) -- (0.004999994439855463,0.051830780841227574);
\draw[line width=1.5pt,color=wwwwww] (0.004999994439855463,0.051830780841227574) -- (0.009999988879710925,0.05160869566717246);
\draw[line width=1.5pt,color=wwwwww] (0.009999988879710925,0.05160869566717246) -- (0.014999983319566389,0.0512406490493376);
\draw[line width=1.5pt,color=wwwwww] (0.014999983319566389,0.0512406490493376) -- (0.01999997775942185,0.0507297531443143);
\draw[line width=1.5pt,color=wwwwww] (0.01999997775942185,0.0507297531443143) -- (0.024999972199277312,0.050080303884980824);
\draw[line width=1.5pt,color=wwwwww] (0.024999972199277312,0.050080303884980824) -- (0.029999966639132774,0.04929772131889991);
\draw[line width=1.5pt,color=wwwwww] (0.029999966639132774,0.04929772131889991) -- (0.034999961078988236,0.04838847489636353);
\draw[line width=1.5pt,color=wwwwww] (0.034999961078988236,0.04838847489636353) -- (0.0399999555188437,0.04735999513680574);
\draw[line width=1.5pt,color=wwwwww] (0.0399999555188437,0.04735999513680574) -- (0.044999949958699166,0.04622057334709136);
\draw[line width=1.5pt,color=wwwwww] (0.044999949958699166,0.04622057334709136) -- (0.04999994439855463,0.044979251268877946);
\draw[line width=1.5pt,color=wwwwww] (0.04999994439855463,0.044979251268877946) -- (0.054999938838410097,0.04364570269200154);
\draw[line width=1.5pt,color=wwwwww] (0.054999938838410097,0.04364570269200154) -- (0.05999993327826556,0.042230109182034556);
\draw[line width=1.5pt,color=wwwwww] (0.05999993327826556,0.042230109182034556) -- (0.06499992771812102,0.04074303213200654);
\draw[line width=1.5pt,color=wwwwww] (0.06499992771812102,0.04074303213200654) -- (0.06999992215797649,0.03919528336033917);
\draw[line width=1.5pt,color=wwwwww] (0.06999992215797649,0.03919528336033917) -- (0.07499991659783195,0.03759779643981601);
\draw[line width=1.5pt,color=wwwwww] (0.07499991659783195,0.03759779643981601) -- (0.07999991103768742,0.035961500858786404);
\draw[line width=1.5pt,color=wwwwww] (0.07999991103768742,0.035961500858786404) -- (0.08499990547754288,0.0342972009891532);
\draw[line width=1.5pt,color=wwwwww] (0.08499990547754288,0.0342972009891532) -- (0.08999989991739835,0.032615461670027886);
\draw[line width=1.5pt,color=wwwwww] (0.08999989991739835,0.032615461670027886) -- (0.09499989435725381,0.03092650201780733);
\draw[line width=1.5pt,color=wwwwww] (0.09499989435725381,0.03092650201780733) -- (0.09999988879710928,0.029240098847746566);
\draw[line width=1.5pt,color=wwwwww] (0.09999988879710928,0.029240098847746566) -- (0.10499988323696474,0.027565500846359423);
\draw[line width=1.5pt,color=wwwwww] (0.10499988323696474,0.027565500846359423) -- (0.1099998776768202,0.025911354374698294);
\draw[line width=1.5pt,color=wwwwww] (0.1099998776768202,0.025911354374698294) -- (0.11499987211667567,0.024285641516699367);
\draw[line width=1.5pt,color=wwwwww] (0.11499987211667567,0.024285641516699367) -- (0.11999986655653114,0.022695630721116997);
\draw[line width=1.5pt,color=wwwwww] (0.11999986655653114,0.022695630721116997) -- (0.1249998609963866,0.021147840126739283);
\draw[line width=1.5pt,color=wwwwww] (0.1249998609963866,0.021147840126739283) -- (0.12999985543624207,0.019648013414066502);
\draw[line width=1.5pt,color=wwwwww] (0.12999985543624207,0.019648013414066502) -- (0.13499984987609753,0.018201107798322194);
\draw[line width=1.5pt,color=wwwwww] (0.13499984987609753,0.018201107798322194) -- (0.139999844315953,0.016811293572206697);
\draw[line width=1.5pt,color=wwwwww] (0.139999844315953,0.016811293572206697) -- (0.14499983875580846,0.015481964426801153);
\draw[line width=1.5pt,color=wwwwww] (0.14499983875580846,0.015481964426801153) -- (0.14999983319566393,0.014215757627158392);
\draw[line width=1.5pt,color=wwwwww] (0.14999983319566393,0.014215757627158392) -- (0.1549998276355194,0.013014582997893763);
\draw[line width=1.5pt,color=wwwwww] (0.1549998276355194,0.013014582997893763) -- (0.15999982207537486,0.011879659583919521);
\draw[line width=1.5pt,color=wwwwww] (0.15999982207537486,0.011879659583919521) -- (0.16499981651523032,0.010811558792445046);
\draw[line width=1.5pt,color=wwwwww] (0.16499981651523032,0.010811558792445046) -- (0.1699998109550858,0.009810252793489663);
\draw[line width=1.5pt,color=wwwwww] (0.1699998109550858,0.009810252793489663) -- (0.17499980539494125,0.008875166955922787);
\draw[line width=1.5pt,color=wwwwww] (0.17499980539494125,0.008875166955922787) -- (0.17999979983479672,0.008005235122035282);
\draw[line width=1.5pt,color=wwwwww] (0.17999979983479672,0.008005235122035282) -- (0.18499979427465219,0.007198956573206471);
\draw[line width=1.5pt,color=wwwwww] (0.18499979427465219,0.007198956573206471) -- (0.18999978871450765,0.00645445360912101);
\draw[line width=1.5pt,color=wwwwww] (0.18999978871450765,0.00645445360912101) -- (0.19499978315436312,0.0057695287496852005);
\draw[line width=1.5pt,color=wwwwww] (0.19499978315436312,0.0057695287496852005) -- (0.19999977759421858,0.005141720668868383);
\draw[line width=1.5pt,color=wwwwww] (0.19999977759421858,0.005141720668868383) -- (0.20499977203407405,0.004568358079372935);
\draw[line width=1.5pt,color=wwwwww] (0.20499977203407405,0.004568358079372935) -- (0.2099997664739295,0.0040466109028134975);
\draw[line width=1.5pt,color=wwwwww] (0.2099997664739295,0.0040466109028134975) -- (0.21499976091378498,0.003573538178622157);
\draw[line width=1.5pt,color=wwwwww] (0.21499976091378498,0.003573538178622157) -- (0.21999975535364044,0.0031461322830617076);
\draw[line width=1.5pt,color=wwwwww] (0.21999975535364044,0.0031461322830617076) -- (0.2249997497934959,0.0027613591447002697);
\draw[line width=1.5pt,color=wwwwww] (0.2249997497934959,0.0027613591447002697) -- (0.22999974423335137,0.0024161942521022622);
\draw[line width=1.5pt,color=wwwwww] (0.22999974423335137,0.0024161942521022622) -- (0.23499973867320684,0.002107654351269768);
\draw[line width=1.5pt,color=wwwwww] (0.23499973867320684,0.002107654351269768) -- (0.2399997331130623,0.0018328248229393085);
\draw[line width=1.5pt,color=wwwwww] (0.2399997331130623,0.0018328248229393085) -- (0.24499972755291777,0.0015888828120332553);
\draw[line width=1.5pt,color=wwwwww] (0.24499972755291777,0.0015888828120332553) -- (0.24999972199277323,0.00137311625256969);
\draw[line width=1.5pt,color=wwwwww] (0.24999972199277323,0.00137311625256969) -- (0.2549997164326287,0.001182938990811512);
\draw[line width=1.5pt,color=wwwwww] (0.2549997164326287,0.001182938990811512) -- (0.25999971087248414,0.001015902257319819);
\draw[line width=1.5pt,color=wwwwww] (0.25999971087248414,0.001015902257319819) -- (0.2649997053123396,0);
\draw[line width=1.5pt,color=wwwwww] (0.2649997053123396,0) -- (0.269999699752195,0);
\draw[line width=1.5pt,color=wwwwww] (0.269999699752195,0) -- (0.27499969419205045,0);
\draw[line width=1.5pt,color=wwwwww] (0.27499969419205045,0) -- (0.2799996886319059,0);
\draw[line width=1.5pt,color=wwwwww] (0.2799996886319059,0) -- (0.2849996830717613,0);
\draw[line width=1.5pt,color=wwwwww] (0.2849996830717613,0) -- (0.28999967751161676,0);
\draw[line width=1.5pt,color=wwwwww] (0.28999967751161676,0) -- (0.2949996719514722,0);
\draw[line width=1.5pt,color=wwwwww] (0.2949996719514722,0) -- (0.29999966639132764,0);
\draw[line width=1.5pt,color=wwwwww] (0.29999966639132764,0) -- (0.3049996608311831,0);
\draw[line width=1.5pt,color=wwwwww] (0.3049996608311831,0) -- (0.3099996552710385,0);
\draw[line width=1.5pt,color=wwwwww] (0.3099996552710385,0) -- (0.31499964971089395,0);
\draw[line width=1.5pt,color=wwwwww] (0.31499964971089395,0) -- (0.3199996441507494,0);
\draw[line width=1.5pt,color=wwwwww] (0.3199996441507494,0) -- (0.3249996385906048,0);
\draw[line width=1.5pt,color=wwwwww] (0.3249996385906048,0) -- (0.32999963303046026,0);
\draw[line width=1.5pt,color=wwwwww] (0.32999963303046026,0) -- (0.3349996274703157,0);
\draw[line width=1.5pt,color=wwwwww] (0.3349996274703157,0) -- (0.33999962191017113,0);
\draw[line width=1.5pt,color=wwwwww] (0.33999962191017113,0) -- (0.34499961635002657,0);
\draw[line width=1.5pt,color=wwwwww] (0.34499961635002657,0) -- (0.349999610789882,0);
\draw[line width=1.5pt,color=wwwwww] (0.349999610789882,0) -- (0.35499960522973745,0);
\draw[line width=1.5pt,color=wwwwww] (0.35499960522973745,0) -- (0.3599995996695929,0);
\draw[line width=1.5pt,color=wwwwww] (0.3599995996695929,0) -- (0.3649995941094483,0);
\draw[line width=1.5pt,color=wwwwww] (0.3649995941094483,0) -- (0.36999958854930376,0);
\draw[line width=1.5pt,color=wwwwww] (0.36999958854930376,0) -- (0.3749995829891592,0);
\draw[line width=1.5pt,color=wwwwww] (0.3749995829891592,0) -- (0.37999957742901463,0);
\draw[line width=1.5pt,color=wwwwww] (0.37999957742901463,0) -- (0.38499957186887007,0);
\draw[line width=1.5pt,color=wwwwww] (0.38499957186887007,0) -- (0.3899995663087255,0);
\draw[line width=1.5pt,color=wwwwww] (0.3899995663087255,0) -- (0.39499956074858095,0);
\draw[line width=1.5pt,color=wwwwww] (0.39499956074858095,0) -- (0.3999995551884364,0);
\draw[line width=1.5pt,color=wwwwww] (0.3999995551884364,0) -- (0.4049995496282918,0);
\draw[line width=1.5pt,color=wwwwww] (0.4049995496282918,0) -- (0.40999954406814726,0);
\draw[line width=1.5pt,color=wwwwww] (0.40999954406814726,0) -- (0.4149995385080027,0);
\draw[line width=1.5pt,color=wwwwww] (0.4149995385080027,0) -- (0.41999953294785813,0);
\draw[line width=1.5pt,color=wwwwww] (0.41999953294785813,0) -- (0.42499952738771357,0);
\draw[line width=1.5pt,color=wwwwww] (0.42499952738771357,0) -- (0.429999521827569,0);
\draw[line width=1.5pt,color=wwwwww] (0.429999521827569,0) -- (0.43499951626742445,0);
\draw[line width=1.5pt,color=wwwwww] (0.43499951626742445,0) -- (0.4399995107072799,0);
\draw[line width=1.5pt,color=wwwwww] (0.4399995107072799,0) -- (0.4449995051471353,0);
\draw[line width=1.5pt,color=wwwwww] (0.4449995051471353,0) -- (0.44999949958699076,0);
\draw[line width=1.5pt,color=wwwwww] (0.44999949958699076,0) -- (0.4549994940268462,0);
\draw[line width=1.5pt,color=wwwwww] (0.4549994940268462,0) -- (0.45999948846670163,0);
\draw[line width=1.5pt,color=wwwwww] (0.45999948846670163,0) -- (0.46499948290655707,0);
\draw[line width=1.5pt,color=wwwwww] (0.46499948290655707,0) -- (0.4699994773464125,0);
\draw[line width=1.5pt,color=wwwwww] (0.4699994773464125,0) -- (0.47499947178626795,0);
\draw[line width=1.5pt,color=wwwwww] (0.47499947178626795,0) -- (0.4799994662261234,0);
\draw[line width=1.5pt,color=wwwwww] (0.4799994662261234,0) -- (0.4849994606659788,0);
\draw[line width=1.5pt,color=wwwwww] (0.4849994606659788,0) -- (0.48999945510583426,0);
\draw[line width=1.5pt,color=wwwwww] (0.48999945510583426,0) -- (0.4949994495456897,0);
\draw[line width=1.5pt,color=wwwwww] (0.4949994495456897,0) -- (0.49999944398554513,0);
\draw[line width=1.5pt,color=wwwwww] (0.49999944398554513,0) -- (0.5049994384254006,0);
\draw[line width=1.5pt,color=wwwwww] (0.5049994384254006,0) -- (0.5099994328652561,0);
\draw[line width=1.5pt,color=wwwwww] (0.5099994328652561,0) -- (0.5149994273051115,0);
\draw[line width=1.5pt,color=wwwwww] (0.5149994273051115,0) -- (0.5199994217449669,0);
\draw[line width=1.5pt,color=wwwwww] (0.5199994217449669,0) -- (0.5249994161848224,0);
\draw[line width=1.5pt,color=wwwwww] (0.5249994161848224,0) -- (0.5299994106246778,0);
\draw[line width=1.5pt,color=wwwwww] (0.5299994106246778,0) -- (0.5349994050645333,0);
\draw[line width=1.5pt,color=wwwwww] (0.5349994050645333,0) -- (0.5399993995043887,0);
\draw[line width=1.5pt,color=wwwwww] (0.5399993995043887,0) -- (0.5449993939442441,0);
\draw[line width=1.5pt,color=wwwwww] (0.5449993939442441,0) -- (0.5499993883840996,0);
\draw[line width=1.5pt,color=wwwwww] (0.5499993883840996,0) -- (0.554999382823955,0);
\draw[line width=1.5pt,color=wwwwww] (0.554999382823955,0) -- (0.5599993772638104,0);
\draw[line width=1.5pt,color=wwwwww] (0.5599993772638104,0) -- (0.5649993717036659,0);
\draw[line width=1.5pt,color=wwwwww] (0.5649993717036659,0) -- (0.5699993661435213,0);
\draw[line width=1.5pt,color=wwwwww] (0.5699993661435213,0) -- (0.5749993605833768,0);
\draw[line width=1.5pt,color=wwwwww] (0.5749993605833768,0) -- (0.5799993550232322,0);
\draw[line width=1.5pt,color=wwwwww] (0.5799993550232322,0) -- (0.5849993494630876,0);
\draw[line width=1.5pt,color=wwwwww] (0.5849993494630876,0) -- (0.5899993439029431,0);
\draw[line width=1.5pt,color=wwwwww] (0.5899993439029431,0) -- (0.5949993383427985,0);
\draw[line width=1.5pt,color=wwwwww] (0.5949993383427985,0) -- (0.5999993327826539,0);
\draw[line width=1.5pt,color=wwwwww] (0.5999993327826539,0) -- (0.6049993272225094,0);
\draw[line width=1.5pt,color=wwwwww] (0.6049993272225094,0) -- (0.6099993216623648,0);
\draw[line width=1.5pt,color=wwwwww] (0.6099993216623648,0) -- (0.6149993161022203,0);
\draw[line width=1.5pt,color=wwwwww] (0.6149993161022203,0) -- (0.6199993105420757,0);
\draw[line width=1.5pt,color=wwwwww] (0.6199993105420757,0) -- (0.6249993049819311,0);
\draw[line width=1.5pt,color=wwwwww] (0.6249993049819311,0) -- (0.6299992994217866,0);
\draw[line width=1.5pt,color=wwwwww] (0.6299992994217866,0) -- (0.634999293861642,0);
\draw[line width=1.5pt,color=wwwwww] (0.634999293861642,0) -- (0.6399992883014974,0);
\draw[line width=1.5pt,color=wwwwww] (0.6399992883014974,0) -- (0.6449992827413529,0);
\draw[line width=1.5pt,color=wwwwww] (0.6449992827413529,0) -- (0.6499992771812083,0);
\draw[line width=1.5pt,color=wwwwww] (0.6499992771812083,0) -- (0.6549992716210638,0);
\draw[line width=1.5pt,color=wwwwww] (0.6549992716210638,0) -- (0.6599992660609192,0);
\draw[line width=1.5pt,color=wwwwww] (0.6599992660609192,0) -- (0.6649992605007746,0);
\draw[line width=1.5pt,color=wwwwww] (0.6649992605007746,0) -- (0.6699992549406301,0);
\draw[line width=1.5pt,color=wwwwww] (0.6699992549406301,0) -- (0.6749992493804855,0);
\draw[line width=1.5pt,color=wwwwww] (0.6749992493804855,0) -- (0.6799992438203409,0);
\draw[line width=1.5pt,color=wwwwww] (0.6799992438203409,0) -- (0.6849992382601964,0);
\draw[line width=1.5pt,color=wwwwww] (0.6849992382601964,0) -- (0.6899992327000518,0);
\draw[line width=1.5pt,color=wwwwww] (0.6899992327000518,0) -- (0.6949992271399072,0);
\draw[line width=1.5pt,color=wwwwww] (0.6949992271399072,0) -- (0.6999992215797627,0);
\draw[line width=1.5pt,color=wwwwww] (0.6999992215797627,0) -- (0.7049992160196181,0);
\draw[line width=1.5pt,color=wwwwww] (0.7049992160196181,0) -- (0.7099992104594736,0);
\draw[line width=1.5pt,color=wwwwww] (0.7099992104594736,0) -- (0.714999204899329,0);
\draw[line width=1.5pt,color=wwwwww] (0.714999204899329,0) -- (0.7199991993391844,0);
\draw[line width=1.5pt,color=wwwwww] (0.7199991993391844,0) -- (0.7249991937790399,0);
\draw[line width=1.5pt,color=wwwwww] (0.7249991937790399,0) -- (0.7299991882188953,0);
\draw[line width=1.5pt,color=wwwwww] (0.7299991882188953,0) -- (0.7349991826587507,0);
\draw[line width=1.5pt,color=wwwwww] (0.7349991826587507,0) -- (0.7399991770986062,0);
\draw[line width=1.5pt,color=wwwwww] (0.7399991770986062,0) -- (0.7449991715384616,0);
\draw[line width=1.5pt,color=wwwwww] (0.7449991715384616,0) -- (0.7499991659783171,0);
\draw[line width=1.5pt,color=wwwwww] (0.7499991659783171,0) -- (0.7549991604181725,0);
\draw[line width=1.5pt,color=wwwwww] (0.7549991604181725,0) -- (0.7599991548580279,0);
\draw[line width=1.5pt,color=wwwwww] (0.7599991548580279,0) -- (0.7649991492978834,0);
\draw[line width=1.5pt,color=wwwwww] (0.7649991492978834,0) -- (0.7699991437377388,0);
\draw[line width=1.5pt,color=wwwwww] (0.7699991437377388,0) -- (0.7749991381775942,0);
\draw[line width=1.5pt,color=wwwwww] (0.7749991381775942,0) -- (0.7799991326174497,0);
\draw[line width=1.5pt,color=wwwwww] (0.7799991326174497,0) -- (0.7849991270573051,0);
\draw[line width=1.5pt,color=wwwwww] (0.7849991270573051,0) -- (0.7899991214971606,0);
\draw[line width=1.5pt,color=wwwwww] (0.7899991214971606,0) -- (0.794999115937016,0);
\draw[line width=1.5pt,color=wwwwww] (0.794999115937016,0) -- (0.7999991103768714,0);
\draw[line width=1.5pt,color=wwwwww] (0.7999991103768714,0) -- (0.8049991048167269,0);
\draw[line width=1.5pt,color=wwwwww] (0.8049991048167269,0) -- (0.8099990992565823,0);
\draw[line width=1.5pt,color=wwwwww] (0.8099990992565823,0) -- (0.8149990936964377,0);
\draw[line width=1.5pt,color=wwwwww] (0.8149990936964377,0) -- (0.8199990881362932,0);
\draw[line width=1.5pt,color=wwwwww] (0.8199990881362932,0) -- (0.8249990825761486,0);
\draw[line width=1.5pt,color=wwwwww] (0.8249990825761486,0) -- (0.8299990770160041,0);
\draw[line width=1.5pt,color=wwwwww] (0.8299990770160041,0) -- (0.8349990714558595,0);
\draw[line width=1.5pt,color=wwwwww] (0.8349990714558595,0) -- (0.8399990658957149,0);
\draw[line width=1.5pt,color=wwwwww] (0.8399990658957149,0) -- (0.8449990603355704,0);
\draw[line width=1.5pt,color=wwwwww] (0.8449990603355704,0) -- (0.8499990547754258,0);
\draw[line width=1.5pt,color=wwwwww] (0.8499990547754258,0) -- (0.8549990492152812,0);
\draw[line width=1.5pt,color=wwwwww] (0.8549990492152812,0) -- (0.8599990436551367,0);
\draw[line width=1.5pt,color=wwwwww] (0.8599990436551367,0) -- (0.8649990380949921,0);
\draw[line width=1.5pt,color=wwwwww] (0.8649990380949921,0) -- (0.8699990325348476,0);
\draw[line width=1.5pt,color=wwwwww] (0.8699990325348476,0) -- (0.874999026974703,0);
\draw[line width=1.5pt,color=wwwwww] (0.874999026974703,0) -- (0.8799990214145584,0);
\draw[line width=1.5pt,color=wwwwww] (0.8799990214145584,0) -- (0.8849990158544139,0);
\draw[line width=1.5pt,color=wwwwww] (0.8849990158544139,0) -- (0.8899990102942693,0);
\draw[line width=1.5pt,color=wwwwww] (0.8899990102942693,0) -- (0.8949990047341247,0);
\draw[line width=1.5pt,color=wwwwww] (0.8949990047341247,0) -- (0.8999989991739802,0);
\draw[line width=1.5pt,color=wwwwww] (0.8999989991739802,0) -- (0.9049989936138356,0);
\draw[line width=1.5pt,color=wwwwww] (0.9049989936138356,0) -- (0.9099989880536911,0);
\draw[line width=1.5pt,color=wwwwww] (0.9099989880536911,0) -- (0.9149989824935465,0);
\draw[line width=1.5pt,color=wwwwww] (0.9149989824935465,0) -- (0.9199989769334019,0);
\draw[line width=1.5pt,color=wwwwww] (0.9199989769334019,0) -- (0.9249989713732574,0);
\draw[line width=1.5pt,color=wwwwww] (0.9249989713732574,0) -- (0.9299989658131128,0);
\draw[line width=1.5pt,color=wwwwww] (0.9299989658131128,0) -- (0.9349989602529682,0);
\draw[line width=1.5pt,color=wwwwww] (0.9349989602529682,0) -- (0.9399989546928237,0);
\draw[line width=1.5pt,color=wwwwww] (0.9399989546928237,0) -- (0.9449989491326791,0);
\draw[line width=1.5pt,color=wwwwww] (0.9449989491326791,0) -- (0.9499989435725346,0);
\draw[line width=1.5pt,color=wwwwww] (0.9499989435725346,0) -- (0.95499893801239,0);
\draw[line width=1.5pt,color=wwwwww] (0.95499893801239,0) -- (0.9599989324522454,0);
\draw[line width=1.5pt,color=wwwwww] (0.9599989324522454,0) -- (0.9649989268921009,0);
\draw[line width=1.5pt,color=wwwwww] (0.9649989268921009,0) -- (0.9699989213319563,0);
\draw[line width=1.5pt,color=wwwwww] (0.9699989213319563,0) -- (0.9749989157718117,0);
\draw[line width=1.5pt,color=wwwwww] (0.9749989157718117,0) -- (0.9799989102116672,0);
\draw[line width=1.5pt,color=wwwwww] (0.9799989102116672,0) -- (0.9849989046515226,0);
\draw[line width=1.5pt,color=wwwwww] (0.9849989046515226,0) -- (0.9899988990913781,0);
\draw[line width=1.5pt,color=wwwwww] (0.9899988990913781,0) -- (0.9949988935312335,0);
\end{axis}
\end{tikzpicture}} \\
    \subfloat[$ h=0, \beta>1 $: si ha magnetizzazione spontanea.]{ 
%<<<<<<<WARNING>>>>>>>
% PGF/Tikz doesn't support the following mathematical functions:
% cosh, acosh, sinh, asinh, tanh, atanh,
% x^r with r not integer

% Plotting will be done using GNUPLOT
% GNUPLOT must be installed and you must allow Latex to call external
% programs by adding the following option to your compiler
% shell-escape    OR    enable-write18 
% Example: pdflatex --shell-escape file.tex 

\definecolor{wwwwww}{rgb}{0.4,0.4,0.4}
\begin{tikzpicture}[line cap=round,line join=round,>=triangle 45,x=1cm,y=1cm]
\begin{axis}[
x=3.5cm,y=4cm,
axis lines=middle,
xmin=-1.2,
xmax=1.2,
ymin=-0.027640247367904026,
ymax=0.1,
xtick={-1,-0.8,...,1},
ytick={-0.02,0,...,0.1},
y post scale = 12,
ylabel = $f(x)$,
xlabel = $x$]
\clip(-1.3753518986999413,-0.02671256188785095) rectangle (1.4564267138000453,0.1350097836780805);
\draw[line width=1.5pt,color=wwwwww] (-0.9999996436130685,0) -- (-0.9999996436130685,0);
\draw[line width=1.5pt,color=wwwwww] (-0.9999996436130685,0) -- (-0.994999649173213,0);
\draw[line width=1.5pt,color=wwwwww] (-0.994999649173213,0) -- (-0.9899996547333576,0);
\draw[line width=1.5pt,color=wwwwww] (-0.9899996547333576,0) -- (-0.9849996602935022,0.0017843285199680616);
\draw[line width=1.5pt,color=wwwwww] (-0.9849996602935022,0.0017843285199680616) -- (-0.9799996658536467,0.003028862103820273);
\draw[line width=1.5pt,color=wwwwww] (-0.9799996658536467,0.003028862103820273) -- (-0.9749996714137913,0.004780947453532711);
\draw[line width=1.5pt,color=wwwwww] (-0.9749996714137913,0.004780947453532711) -- (-0.9699996769739359,0.007104867706036516);
\draw[line width=1.5pt,color=wwwwww] (-0.9699996769739359,0.007104867706036516) -- (-0.9649996825340804,0.010030464122631554);
\draw[line width=1.5pt,color=wwwwww] (-0.9649996825340804,0.010030464122631554) -- (-0.959999688094225,0.013545191902715716);
\draw[line width=1.5pt,color=wwwwww] (-0.959999688094225,0.013545191902715716) -- (-0.9549996936543695,0.017590813460223998);
\draw[line width=1.5pt,color=wwwwww] (-0.9549996936543695,0.017590813460223998) -- (-0.9499996992145141,0.022065171923097354);
\draw[line width=1.5pt,color=wwwwww] (-0.9499996992145141,0.022065171923097354) -- (-0.9449997047746587,0.026828710841736922);
\draw[line width=1.5pt,color=wwwwww] (-0.9449997047746587,0.026828710841736922) -- (-0.9399997103348032,0.03171476357735713);
\draw[line width=1.5pt,color=wwwwww] (-0.9399997103348032,0.03171476357735713) -- (-0.9349997158949478,0.036542216894462486);
\draw[line width=1.5pt,color=wwwwww] (-0.9349997158949478,0.036542216894462486) -- (-0.9299997214550924,0.041128990722845565);
\draw[line width=1.5pt,color=wwwwww] (-0.9299997214550924,0.041128990722845565) -- (-0.9249997270152369,0.045304852355203304);
\draw[line width=1.5pt,color=wwwwww] (-0.9249997270152369,0.045304852355203304) -- (-0.9199997325753815,0.04892234511645694);
\draw[line width=1.5pt,color=wwwwww] (-0.9199997325753815,0.04892234511645694) -- (-0.914999738135526,0.0518649871015366);
\draw[line width=1.5pt,color=wwwwww] (-0.914999738135526,0.0518649871015366) -- (-0.9099997436956706,0.05405231131256859);
\draw[line width=1.5pt,color=wwwwww] (-0.9099997436956706,0.05405231131256859) -- (-0.9049997492558152,0.05544171123889713);
\draw[line width=1.5pt,color=wwwwww] (-0.9049997492558152,0.05544171123889713) -- (-0.8999997548159597,0.056027380107384894);
\draw[line width=1.5pt,color=wwwwww] (-0.8999997548159597,0.056027380107384894) -- (-0.8949997603761043,0.05583686223735475);
\draw[line width=1.5pt,color=wwwwww] (-0.8949997603761043,0.05583686223735475) -- (-0.8899997659362489,0.054925864284924124);
\draw[line width=1.5pt,color=wwwwww] (-0.8899997659362489,0.054925864284924124) -- (-0.8849997714963934,0.05337201042529576);
\draw[line width=1.5pt,color=wwwwww] (-0.8849997714963934,0.05337201042529576) -- (-0.879999777056538,0.05126818612241565);
\draw[line width=1.5pt,color=wwwwww] (-0.879999777056538,0.05126818612241565) -- (-0.8749997826166825,0.0487160221997759);
\draw[line width=1.5pt,color=wwwwww] (-0.8749997826166825,0.0487160221997759) -- (-0.8699997881768271,0.04581994713337477);
\draw[line width=1.5pt,color=wwwwww] (-0.8699997881768271,0.04581994713337477) -- (-0.8649997937369717,0.04268210106525689);
\draw[line width=1.5pt,color=wwwwww] (-0.8649997937369717,0.04268210106525689) -- (-0.8599997992971162,0.039398275990160776);
\draw[line width=1.5pt,color=wwwwww] (-0.8599997992971162,0.039398275990160776) -- (-0.8549998048572608,0.0360549339943863);
\draw[line width=1.5pt,color=wwwwww] (-0.8549998048572608,0.0360549339943863) -- (-0.8499998104174054,0.032727265656073486);
\draw[line width=1.5pt,color=wwwwww] (-0.8499998104174054,0.032727265656073486) -- (-0.8449998159775499,0.029478185998130825);
\draw[line width=1.5pt,color=wwwwww] (-0.8449998159775499,0.029478185998130825) -- (-0.8399998215376945,0.02635812484849368);
\draw[line width=1.5pt,color=wwwwww] (-0.8399998215376945,0.02635812484849368) -- (-0.834999827097839,0.02340544914333471);
\draw[line width=1.5pt,color=wwwwww] (-0.834999827097839,0.02340544914333471) -- (-0.8299998326579836,0.020647352487432564);
\draw[line width=1.5pt,color=wwwwww] (-0.8299998326579836,0.020647352487432564) -- (-0.8249998382181282,0.018101057659327603);
\draw[line width=1.5pt,color=wwwwww] (-0.8249998382181282,0.018101057659327603) -- (-0.8199998437782727,0.01577519639469831);
\draw[line width=1.5pt,color=wwwwww] (-0.8199998437782727,0.01577519639469831) -- (-0.8149998493384173,0.013671253920860894);
\draw[line width=1.5pt,color=wwwwww] (-0.8149998493384173,0.013671253920860894) -- (-0.8099998548985619,0.011784990304802045);
\draw[line width=1.5pt,color=wwwwww] (-0.8099998548985619,0.011784990304802045) -- (-0.8049998604587064,0.010107774459143043);
\draw[line width=1.5pt,color=wwwwww] (-0.8049998604587064,0.010107774459143043) -- (-0.799999866018851,0.008627788111554938);
\draw[line width=1.5pt,color=wwwwww] (-0.799999866018851,0.008627788111554938) -- (-0.7949998715789955,0.007331075314999194);
\draw[line width=1.5pt,color=wwwwww] (-0.7949998715789955,0.007331075314999194) -- (-0.7899998771391401,0.006202427802731299);
\draw[line width=1.5pt,color=wwwwww] (-0.7899998771391401,0.006202427802731299) -- (-0.7849998826992847,0.0052261077002934645);
\draw[line width=1.5pt,color=wwwwww] (-0.7849998826992847,0.0052261077002934645) -- (-0.7799998882594292,0.004386417078320341);
\draw[line width=1.5pt,color=wwwwww] (-0.7799998882594292,0.004386417078320341) -- (-0.7749998938195738,0.003668128998582687);
\draw[line width=1.5pt,color=wwwwww] (-0.7749998938195738,0.003668128998582687) -- (-0.7699998993797184,0.003056797572102079);
\draw[line width=1.5pt,color=wwwwww] (-0.7699998993797184,0.003056797572102079) -- (-0.7649999049398629,0.0025389656196645265);
\draw[line width=1.5pt,color=wwwwww] (-0.7649999049398629,0.0025389656196645265) -- (-0.7599999105000075,0.002102288274580746);
\draw[line width=1.5pt,color=wwwwww] (-0.7599999105000075,0.002102288274580746) -- (-0.754999916060152,0.0017355897089754541);
\draw[line width=1.5pt,color=wwwwww] (-0.754999916060152,0.0017355897089754541) -- (-0.7499999216202966,0.0014288684421673284);
\draw[line width=1.5pt,color=wwwwww] (-0.7499999216202966,0.0014288684421673284) -- (-0.7449999271804412,0.0011732646747829591);
\draw[line width=1.5pt,color=wwwwww] (-0.7449999271804412,0.0011732646747829591) -- (-0.7399999327405857,0);
\draw[line width=1.5pt,color=wwwwww] (-0.7399999327405857,0) -- (-0.7349999383007303,0);
\draw[line width=1.5pt,color=wwwwww] (-0.7349999383007303,0) -- (-0.7299999438608749,0);
\draw[line width=1.5pt,color=wwwwww] (-0.7299999438608749,0) -- (-0.7249999494210194,0);
\draw[line width=1.5pt,color=wwwwww] (-0.7249999494210194,0) -- (-0.719999954981164,0);
\draw[line width=1.5pt,color=wwwwww] (-0.719999954981164,0) -- (-0.7149999605413085,0);
\draw[line width=1.5pt,color=wwwwww] (-0.7149999605413085,0) -- (-0.7099999661014531,0);
\draw[line width=1.5pt,color=wwwwww] (-0.7099999661014531,0) -- (-0.7049999716615977,0);
\draw[line width=1.5pt,color=wwwwww] (-0.7049999716615977,0) -- (-0.6999999772217422,0);
\draw[line width=1.5pt,color=wwwwww] (-0.6999999772217422,0) -- (-0.6949999827818868,0);
\draw[line width=1.5pt,color=wwwwww] (-0.6949999827818868,0) -- (-0.6899999883420314,0);
\draw[line width=1.5pt,color=wwwwww] (-0.6899999883420314,0) -- (-0.6849999939021759,0);
\draw[line width=1.5pt,color=wwwwww] (-0.6849999939021759,0) -- (-0.6799999994623205,0);
\draw[line width=1.5pt,color=wwwwww] (-0.6799999994623205,0) -- (-0.675000005022465,0);
\draw[line width=1.5pt,color=wwwwww] (-0.675000005022465,0) -- (-0.6700000105826096,0);
\draw[line width=1.5pt,color=wwwwww] (-0.6700000105826096,0) -- (-0.6650000161427542,0);
\draw[line width=1.5pt,color=wwwwww] (-0.6650000161427542,0) -- (-0.6600000217028987,0);
\draw[line width=1.5pt,color=wwwwww] (-0.6600000217028987,0) -- (-0.6550000272630433,0);
\draw[line width=1.5pt,color=wwwwww] (-0.6550000272630433,0) -- (-0.6500000328231879,0);
\draw[line width=1.5pt,color=wwwwww] (-0.6500000328231879,0) -- (-0.6450000383833324,0);
\draw[line width=1.5pt,color=wwwwww] (-0.6450000383833324,0) -- (-0.640000043943477,0);
\draw[line width=1.5pt,color=wwwwww] (-0.640000043943477,0) -- (-0.6350000495036215,0);
\draw[line width=1.5pt,color=wwwwww] (-0.6350000495036215,0) -- (-0.6300000550637661,0);
\draw[line width=1.5pt,color=wwwwww] (-0.6300000550637661,0) -- (-0.6250000606239107,0);
\draw[line width=1.5pt,color=wwwwww] (-0.6250000606239107,0) -- (-0.6200000661840552,0);
\draw[line width=1.5pt,color=wwwwww] (-0.6200000661840552,0) -- (-0.6150000717441998,0);
\draw[line width=1.5pt,color=wwwwww] (-0.6150000717441998,0) -- (-0.6100000773043444,0);
\draw[line width=1.5pt,color=wwwwww] (-0.6100000773043444,0) -- (-0.6050000828644889,0);
\draw[line width=1.5pt,color=wwwwww] (-0.6050000828644889,0) -- (-0.6000000884246335,0);
\draw[line width=1.5pt,color=wwwwww] (-0.6000000884246335,0) -- (-0.595000093984778,0);
\draw[line width=1.5pt,color=wwwwww] (-0.595000093984778,0) -- (-0.5900000995449226,0);
\draw[line width=1.5pt,color=wwwwww] (-0.5900000995449226,0) -- (-0.5850001051050672,0);
\draw[line width=1.5pt,color=wwwwww] (-0.5850001051050672,0) -- (-0.5800001106652117,0);
\draw[line width=1.5pt,color=wwwwww] (-0.5800001106652117,0) -- (-0.5750001162253563,0);
\draw[line width=1.5pt,color=wwwwww] (-0.5750001162253563,0) -- (-0.5700001217855009,0);
\draw[line width=1.5pt,color=wwwwww] (-0.5700001217855009,0) -- (-0.5650001273456454,0);
\draw[line width=1.5pt,color=wwwwww] (-0.5650001273456454,0) -- (-0.56000013290579,0);
\draw[line width=1.5pt,color=wwwwww] (-0.56000013290579,0) -- (-0.5550001384659345,0);
\draw[line width=1.5pt,color=wwwwww] (-0.5550001384659345,0) -- (-0.5500001440260791,0);
\draw[line width=1.5pt,color=wwwwww] (-0.5500001440260791,0) -- (-0.5450001495862237,0);
\draw[line width=1.5pt,color=wwwwww] (-0.5450001495862237,0) -- (-0.5400001551463682,0);
\draw[line width=1.5pt,color=wwwwww] (-0.5400001551463682,0) -- (-0.5350001607065128,0);
\draw[line width=1.5pt,color=wwwwww] (-0.5350001607065128,0) -- (-0.5300001662666574,0);
\draw[line width=1.5pt,color=wwwwww] (-0.5300001662666574,0) -- (-0.5250001718268019,0);
\draw[line width=1.5pt,color=wwwwww] (-0.5250001718268019,0) -- (-0.5200001773869465,0);
\draw[line width=1.5pt,color=wwwwww] (-0.5200001773869465,0) -- (-0.515000182947091,0);
\draw[line width=1.5pt,color=wwwwww] (-0.515000182947091,0) -- (-0.5100001885072356,0);
\draw[line width=1.5pt,color=wwwwww] (-0.5100001885072356,0) -- (-0.5050001940673802,0);
\draw[line width=1.5pt,color=wwwwww] (-0.5050001940673802,0) -- (-0.5000001996275247,0);
\draw[line width=1.5pt,color=wwwwww] (-0.5000001996275247,0) -- (-0.4950002051876693,0);
\draw[line width=1.5pt,color=wwwwww] (-0.4950002051876693,0) -- (-0.49000021074781386,0);
\draw[line width=1.5pt,color=wwwwww] (-0.49000021074781386,0) -- (-0.4850002163079584,0);
\draw[line width=1.5pt,color=wwwwww] (-0.4850002163079584,0) -- (-0.480000221868103,0);
\draw[line width=1.5pt,color=wwwwww] (-0.480000221868103,0) -- (-0.47500022742824755,0);
\draw[line width=1.5pt,color=wwwwww] (-0.47500022742824755,0) -- (-0.4700002329883921,0);
\draw[line width=1.5pt,color=wwwwww] (-0.4700002329883921,0) -- (-0.46500023854853667,0);
\draw[line width=1.5pt,color=wwwwww] (-0.46500023854853667,0) -- (-0.46000024410868123,0);
\draw[line width=1.5pt,color=wwwwww] (-0.46000024410868123,0) -- (-0.4550002496688258,0);
\draw[line width=1.5pt,color=wwwwww] (-0.4550002496688258,0) -- (-0.45000025522897036,0);
\draw[line width=1.5pt,color=wwwwww] (-0.45000025522897036,0) -- (-0.4450002607891149,0);
\draw[line width=1.5pt,color=wwwwww] (-0.4450002607891149,0) -- (-0.4400002663492595,0);
\draw[line width=1.5pt,color=wwwwww] (-0.4400002663492595,0) -- (-0.43500027190940405,0);
\draw[line width=1.5pt,color=wwwwww] (-0.43500027190940405,0) -- (-0.4300002774695486,0);
\draw[line width=1.5pt,color=wwwwww] (-0.4300002774695486,0) -- (-0.42500028302969317,0);
\draw[line width=1.5pt,color=wwwwww] (-0.42500028302969317,0) -- (-0.42000028858983773,0);
\draw[line width=1.5pt,color=wwwwww] (-0.42000028858983773,0) -- (-0.4150002941499823,0);
\draw[line width=1.5pt,color=wwwwww] (-0.4150002941499823,0) -- (-0.41000029971012686,0);
\draw[line width=1.5pt,color=wwwwww] (-0.41000029971012686,0) -- (-0.4050003052702714,0);
\draw[line width=1.5pt,color=wwwwww] (-0.4050003052702714,0) -- (-0.400000310830416,0);
\draw[line width=1.5pt,color=wwwwww] (-0.400000310830416,0) -- (-0.39500031639056055,0);
\draw[line width=1.5pt,color=wwwwww] (-0.39500031639056055,0) -- (-0.3900003219507051,0);
\draw[line width=1.5pt,color=wwwwww] (-0.3900003219507051,0) -- (-0.38500032751084967,0);
\draw[line width=1.5pt,color=wwwwww] (-0.38500032751084967,0) -- (-0.38000033307099423,0);
\draw[line width=1.5pt,color=wwwwww] (-0.38000033307099423,0) -- (-0.3750003386311388,0);
\draw[line width=1.5pt,color=wwwwww] (-0.3750003386311388,0) -- (-0.37000034419128336,0);
\draw[line width=1.5pt,color=wwwwww] (-0.37000034419128336,0) -- (-0.3650003497514279,0);
\draw[line width=1.5pt,color=wwwwww] (-0.3650003497514279,0) -- (-0.3600003553115725,0);
\draw[line width=1.5pt,color=wwwwww] (-0.3600003553115725,0) -- (-0.35500036087171705,0);
\draw[line width=1.5pt,color=wwwwww] (-0.35500036087171705,0) -- (-0.3500003664318616,0);
\draw[line width=1.5pt,color=wwwwww] (-0.3500003664318616,0) -- (-0.34500037199200617,0);
\draw[line width=1.5pt,color=wwwwww] (-0.34500037199200617,0) -- (-0.34000037755215073,0);
\draw[line width=1.5pt,color=wwwwww] (-0.34000037755215073,0) -- (-0.3350003831122953,0);
\draw[line width=1.5pt,color=wwwwww] (-0.3350003831122953,0) -- (-0.33000038867243986,0);
\draw[line width=1.5pt,color=wwwwww] (-0.33000038867243986,0) -- (-0.3250003942325844,0);
\draw[line width=1.5pt,color=wwwwww] (-0.3250003942325844,0) -- (-0.320000399792729,0);
\draw[line width=1.5pt,color=wwwwww] (-0.320000399792729,0) -- (-0.31500040535287355,0);
\draw[line width=1.5pt,color=wwwwww] (-0.31500040535287355,0) -- (-0.3100004109130181,0);
\draw[line width=1.5pt,color=wwwwww] (-0.3100004109130181,0) -- (-0.30500041647316267,0);
\draw[line width=1.5pt,color=wwwwww] (-0.30500041647316267,0) -- (-0.30000042203330723,0);
\draw[line width=1.5pt,color=wwwwww] (-0.30000042203330723,0) -- (-0.2950004275934518,0);
\draw[line width=1.5pt,color=wwwwww] (-0.2950004275934518,0) -- (-0.29000043315359636,0);
\draw[line width=1.5pt,color=wwwwww] (-0.29000043315359636,0) -- (-0.2850004387137409,0);
\draw[line width=1.5pt,color=wwwwww] (-0.2850004387137409,0) -- (-0.2800004442738855,0);
\draw[line width=1.5pt,color=wwwwww] (-0.2800004442738855,0) -- (-0.27500044983403005,0);
\draw[line width=1.5pt,color=wwwwww] (-0.27500044983403005,0) -- (-0.2700004553941746,0);
\draw[line width=1.5pt,color=wwwwww] (-0.2700004553941746,0) -- (-0.2650004609543192,0);
\draw[line width=1.5pt,color=wwwwww] (-0.2650004609543192,0) -- (-0.26000046651446374,0);
\draw[line width=1.5pt,color=wwwwww] (-0.26000046651446374,0) -- (-0.2550004720746083,0);
\draw[line width=1.5pt,color=wwwwww] (-0.2550004720746083,0) -- (-0.25000047763475286,0);
\draw[line width=1.5pt,color=wwwwww] (-0.25000047763475286,0) -- (-0.2450004831948974,0);
\draw[line width=1.5pt,color=wwwwww] (-0.2450004831948974,0) -- (-0.24000048875504193,0);
\draw[line width=1.5pt,color=wwwwww] (-0.24000048875504193,0) -- (-0.23500049431518646,0);
\draw[line width=1.5pt,color=wwwwww] (-0.23500049431518646,0) -- (-0.230000499875331,0);
\draw[line width=1.5pt,color=wwwwww] (-0.230000499875331,0) -- (-0.22500050543547553,0);
\draw[line width=1.5pt,color=wwwwww] (-0.22500050543547553,0) -- (-0.22000051099562007,0);
\draw[line width=1.5pt,color=wwwwww] (-0.22000051099562007,0) -- (-0.2150005165557646,0);
\draw[line width=1.5pt,color=wwwwww] (-0.2150005165557646,0) -- (-0.21000052211590914,0);
\draw[line width=1.5pt,color=wwwwww] (-0.21000052211590914,0) -- (-0.20500052767605367,0);
\draw[line width=1.5pt,color=wwwwww] (-0.20500052767605367,0) -- (-0.2000005332361982,0);
\draw[line width=1.5pt,color=wwwwww] (-0.2000005332361982,0) -- (-0.19500053879634274,0);
\draw[line width=1.5pt,color=wwwwww] (-0.19500053879634274,0) -- (-0.19000054435648728,0);
\draw[line width=1.5pt,color=wwwwww] (-0.19000054435648728,0) -- (-0.1850005499166318,0);
\draw[line width=1.5pt,color=wwwwww] (-0.1850005499166318,0) -- (-0.18000055547677635,0);
\draw[line width=1.5pt,color=wwwwww] (-0.18000055547677635,0) -- (-0.17500056103692088,0);
\draw[line width=1.5pt,color=wwwwww] (-0.17500056103692088,0) -- (-0.17000056659706542,0);
\draw[line width=1.5pt,color=wwwwww] (-0.17000056659706542,0) -- (-0.16500057215720995,0);
\draw[line width=1.5pt,color=wwwwww] (-0.16500057215720995,0) -- (-0.1600005777173545,0);
\draw[line width=1.5pt,color=wwwwww] (-0.1600005777173545,0) -- (-0.15500058327749902,0);
\draw[line width=1.5pt,color=wwwwww] (-0.15500058327749902,0) -- (-0.15000058883764356,0);
\draw[line width=1.5pt,color=wwwwww] (-0.15000058883764356,0) -- (-0.1450005943977881,0);
\draw[line width=1.5pt,color=wwwwww] (-0.1450005943977881,0) -- (-0.14000059995793263,0);
\draw[line width=1.5pt,color=wwwwww] (-0.14000059995793263,0) -- (-0.13500060551807716,0);
\draw[line width=1.5pt,color=wwwwww] (-0.13500060551807716,0) -- (-0.1300006110782217,0);
\draw[line width=1.5pt,color=wwwwww] (-0.1300006110782217,0) -- (-0.12500061663836623,0);
\draw[line width=1.5pt,color=wwwwww] (-0.12500061663836623,0) -- (-0.12000062219851076,0);
\draw[line width=1.5pt,color=wwwwww] (-0.12000062219851076,0) -- (-0.1150006277586553,0);
\draw[line width=1.5pt,color=wwwwww] (-0.1150006277586553,0) -- (-0.11000063331879983,0);
\draw[line width=1.5pt,color=wwwwww] (-0.11000063331879983,0) -- (-0.10500063887894437,0);
\draw[line width=1.5pt,color=wwwwww] (-0.10500063887894437,0) -- (-0.1000006444390889,0);
\draw[line width=1.5pt,color=wwwwww] (-0.1000006444390889,0) -- (-0.09500064999923344,0);
\draw[line width=1.5pt,color=wwwwww] (-0.09500064999923344,0) -- (-0.09000065555937797,0);
\draw[line width=1.5pt,color=wwwwww] (-0.09000065555937797,0) -- (-0.08500066111952251,0);
\draw[line width=1.5pt,color=wwwwww] (-0.08500066111952251,0) -- (-0.08000066667966704,0);
\draw[line width=1.5pt,color=wwwwww] (-0.08000066667966704,0) -- (-0.07500067223981158,0);
\draw[line width=1.5pt,color=wwwwww] (-0.07500067223981158,0) -- (-0.07000067779995611,0);
\draw[line width=1.5pt,color=wwwwww] (-0.07000067779995611,0) -- (-0.06500068336010065,0);
\draw[line width=1.5pt,color=wwwwww] (-0.06500068336010065,0) -- (-0.06000068892024518,0);
\draw[line width=1.5pt,color=wwwwww] (-0.06000068892024518,0) -- (-0.05500069448038972,0);
\draw[line width=1.5pt,color=wwwwww] (-0.05500069448038972,0) -- (-0.05000070004053425,0);
\draw[line width=1.5pt,color=wwwwww] (-0.05000070004053425,0) -- (-0.045000705600678786,0);
\draw[line width=1.5pt,color=wwwwww] (-0.045000705600678786,0) -- (-0.04000071116082332,0);
\draw[line width=1.5pt,color=wwwwww] (-0.04000071116082332,0) -- (-0.035000716720967856,0);
\draw[line width=1.5pt,color=wwwwww] (-0.035000716720967856,0) -- (-0.030000722281112394,0);
\draw[line width=1.5pt,color=wwwwww] (-0.030000722281112394,0) -- (-0.025000727841256933,0);
\draw[line width=1.5pt,color=wwwwww] (-0.025000727841256933,0) -- (-0.02000073340140147,0);
\draw[line width=1.5pt,color=wwwwww] (-0.02000073340140147,0) -- (-0.015000738961546009,0);
\draw[line width=1.5pt,color=wwwwww] (-0.015000738961546009,0) -- (-0.010000744521690547,0);
\draw[line width=1.5pt,color=wwwwww] (-0.010000744521690547,0) -- (-0.005000750081835085,0);
\draw[line width=1.5pt,color=wwwwww] (-0.005000750081835085,0) -- (0,0);
\draw[line width=1.5pt,color=wwwwww] (0,0) -- (0.004999994439855463,0);
\draw[line width=1.5pt,color=wwwwww] (0.004999994439855463,0) -- (0.009999988879710925,0);
\draw[line width=1.5pt,color=wwwwww] (0.009999988879710925,0) -- (0.014999983319566389,0);
\draw[line width=1.5pt,color=wwwwww] (0.014999983319566389,0) -- (0.01999997775942185,0);
\draw[line width=1.5pt,color=wwwwww] (0.01999997775942185,0) -- (0.024999972199277312,0);
\draw[line width=1.5pt,color=wwwwww] (0.024999972199277312,0) -- (0.029999966639132774,0);
\draw[line width=1.5pt,color=wwwwww] (0.029999966639132774,0) -- (0.034999961078988236,0);
\draw[line width=1.5pt,color=wwwwww] (0.034999961078988236,0) -- (0.0399999555188437,0);
\draw[line width=1.5pt,color=wwwwww] (0.0399999555188437,0) -- (0.044999949958699166,0);
\draw[line width=1.5pt,color=wwwwww] (0.044999949958699166,0) -- (0.04999994439855463,0);
\draw[line width=1.5pt,color=wwwwww] (0.04999994439855463,0) -- (0.054999938838410097,0);
\draw[line width=1.5pt,color=wwwwww] (0.054999938838410097,0) -- (0.05999993327826556,0);
\draw[line width=1.5pt,color=wwwwww] (0.05999993327826556,0) -- (0.06499992771812102,0);
\draw[line width=1.5pt,color=wwwwww] (0.06499992771812102,0) -- (0.06999992215797649,0);
\draw[line width=1.5pt,color=wwwwww] (0.06999992215797649,0) -- (0.07499991659783195,0);
\draw[line width=1.5pt,color=wwwwww] (0.07499991659783195,0) -- (0.07999991103768742,0);
\draw[line width=1.5pt,color=wwwwww] (0.07999991103768742,0) -- (0.08499990547754288,0);
\draw[line width=1.5pt,color=wwwwww] (0.08499990547754288,0) -- (0.08999989991739835,0);
\draw[line width=1.5pt,color=wwwwww] (0.08999989991739835,0) -- (0.09499989435725381,0);
\draw[line width=1.5pt,color=wwwwww] (0.09499989435725381,0) -- (0.09999988879710928,0);
\draw[line width=1.5pt,color=wwwwww] (0.09999988879710928,0) -- (0.10499988323696474,0);
\draw[line width=1.5pt,color=wwwwww] (0.10499988323696474,0) -- (0.1099998776768202,0);
\draw[line width=1.5pt,color=wwwwww] (0.1099998776768202,0) -- (0.11499987211667567,0);
\draw[line width=1.5pt,color=wwwwww] (0.11499987211667567,0) -- (0.11999986655653114,0);
\draw[line width=1.5pt,color=wwwwww] (0.11999986655653114,0) -- (0.1249998609963866,0);
\draw[line width=1.5pt,color=wwwwww] (0.1249998609963866,0) -- (0.12999985543624207,0);
\draw[line width=1.5pt,color=wwwwww] (0.12999985543624207,0) -- (0.13499984987609753,0);
\draw[line width=1.5pt,color=wwwwww] (0.13499984987609753,0) -- (0.139999844315953,0);
\draw[line width=1.5pt,color=wwwwww] (0.139999844315953,0) -- (0.14499983875580846,0);
\draw[line width=1.5pt,color=wwwwww] (0.14499983875580846,0) -- (0.14999983319566393,0);
\draw[line width=1.5pt,color=wwwwww] (0.14999983319566393,0) -- (0.1549998276355194,0);
\draw[line width=1.5pt,color=wwwwww] (0.1549998276355194,0) -- (0.15999982207537486,0);
\draw[line width=1.5pt,color=wwwwww] (0.15999982207537486,0) -- (0.16499981651523032,0);
\draw[line width=1.5pt,color=wwwwww] (0.16499981651523032,0) -- (0.1699998109550858,0);
\draw[line width=1.5pt,color=wwwwww] (0.1699998109550858,0) -- (0.17499980539494125,0);
\draw[line width=1.5pt,color=wwwwww] (0.17499980539494125,0) -- (0.17999979983479672,0);
\draw[line width=1.5pt,color=wwwwww] (0.17999979983479672,0) -- (0.18499979427465219,0);
\draw[line width=1.5pt,color=wwwwww] (0.18499979427465219,0) -- (0.18999978871450765,0);
\draw[line width=1.5pt,color=wwwwww] (0.18999978871450765,0) -- (0.19499978315436312,0);
\draw[line width=1.5pt,color=wwwwww] (0.19499978315436312,0) -- (0.19999977759421858,0);
\draw[line width=1.5pt,color=wwwwww] (0.19999977759421858,0) -- (0.20499977203407405,0);
\draw[line width=1.5pt,color=wwwwww] (0.20499977203407405,0) -- (0.2099997664739295,0);
\draw[line width=1.5pt,color=wwwwww] (0.2099997664739295,0) -- (0.21499976091378498,0);
\draw[line width=1.5pt,color=wwwwww] (0.21499976091378498,0) -- (0.21999975535364044,0);
\draw[line width=1.5pt,color=wwwwww] (0.21999975535364044,0) -- (0.2249997497934959,0);
\draw[line width=1.5pt,color=wwwwww] (0.2249997497934959,0) -- (0.22999974423335137,0);
\draw[line width=1.5pt,color=wwwwww] (0.22999974423335137,0) -- (0.23499973867320684,0);
\draw[line width=1.5pt,color=wwwwww] (0.23499973867320684,0) -- (0.2399997331130623,0);
\draw[line width=1.5pt,color=wwwwww] (0.2399997331130623,0) -- (0.24499972755291777,0);
\draw[line width=1.5pt,color=wwwwww] (0.24499972755291777,0) -- (0.24999972199277323,0);
\draw[line width=1.5pt,color=wwwwww] (0.24999972199277323,0) -- (0.2549997164326287,0);
\draw[line width=1.5pt,color=wwwwww] (0.2549997164326287,0) -- (0.25999971087248414,0);
\draw[line width=1.5pt,color=wwwwww] (0.25999971087248414,0) -- (0.2649997053123396,0);
\draw[line width=1.5pt,color=wwwwww] (0.2649997053123396,0) -- (0.269999699752195,0);
\draw[line width=1.5pt,color=wwwwww] (0.269999699752195,0) -- (0.27499969419205045,0);
\draw[line width=1.5pt,color=wwwwww] (0.27499969419205045,0) -- (0.2799996886319059,0);
\draw[line width=1.5pt,color=wwwwww] (0.2799996886319059,0) -- (0.2849996830717613,0);
\draw[line width=1.5pt,color=wwwwww] (0.2849996830717613,0) -- (0.28999967751161676,0);
\draw[line width=1.5pt,color=wwwwww] (0.28999967751161676,0) -- (0.2949996719514722,0);
\draw[line width=1.5pt,color=wwwwww] (0.2949996719514722,0) -- (0.29999966639132764,0);
\draw[line width=1.5pt,color=wwwwww] (0.29999966639132764,0) -- (0.3049996608311831,0);
\draw[line width=1.5pt,color=wwwwww] (0.3049996608311831,0) -- (0.3099996552710385,0);
\draw[line width=1.5pt,color=wwwwww] (0.3099996552710385,0) -- (0.31499964971089395,0);
\draw[line width=1.5pt,color=wwwwww] (0.31499964971089395,0) -- (0.3199996441507494,0);
\draw[line width=1.5pt,color=wwwwww] (0.3199996441507494,0) -- (0.3249996385906048,0);
\draw[line width=1.5pt,color=wwwwww] (0.3249996385906048,0) -- (0.32999963303046026,0);
\draw[line width=1.5pt,color=wwwwww] (0.32999963303046026,0) -- (0.3349996274703157,0);
\draw[line width=1.5pt,color=wwwwww] (0.3349996274703157,0) -- (0.33999962191017113,0);
\draw[line width=1.5pt,color=wwwwww] (0.33999962191017113,0) -- (0.34499961635002657,0);
\draw[line width=1.5pt,color=wwwwww] (0.34499961635002657,0) -- (0.349999610789882,0);
\draw[line width=1.5pt,color=wwwwww] (0.349999610789882,0) -- (0.35499960522973745,0);
\draw[line width=1.5pt,color=wwwwww] (0.35499960522973745,0) -- (0.3599995996695929,0);
\draw[line width=1.5pt,color=wwwwww] (0.3599995996695929,0) -- (0.3649995941094483,0);
\draw[line width=1.5pt,color=wwwwww] (0.3649995941094483,0) -- (0.36999958854930376,0);
\draw[line width=1.5pt,color=wwwwww] (0.36999958854930376,0) -- (0.3749995829891592,0);
\draw[line width=1.5pt,color=wwwwww] (0.3749995829891592,0) -- (0.37999957742901463,0);
\draw[line width=1.5pt,color=wwwwww] (0.37999957742901463,0) -- (0.38499957186887007,0);
\draw[line width=1.5pt,color=wwwwww] (0.38499957186887007,0) -- (0.3899995663087255,0);
\draw[line width=1.5pt,color=wwwwww] (0.3899995663087255,0) -- (0.39499956074858095,0);
\draw[line width=1.5pt,color=wwwwww] (0.39499956074858095,0) -- (0.3999995551884364,0);
\draw[line width=1.5pt,color=wwwwww] (0.3999995551884364,0) -- (0.4049995496282918,0);
\draw[line width=1.5pt,color=wwwwww] (0.4049995496282918,0) -- (0.40999954406814726,0);
\draw[line width=1.5pt,color=wwwwww] (0.40999954406814726,0) -- (0.4149995385080027,0);
\draw[line width=1.5pt,color=wwwwww] (0.4149995385080027,0) -- (0.41999953294785813,0);
\draw[line width=1.5pt,color=wwwwww] (0.41999953294785813,0) -- (0.42499952738771357,0);
\draw[line width=1.5pt,color=wwwwww] (0.42499952738771357,0) -- (0.429999521827569,0);
\draw[line width=1.5pt,color=wwwwww] (0.429999521827569,0) -- (0.43499951626742445,0);
\draw[line width=1.5pt,color=wwwwww] (0.43499951626742445,0) -- (0.4399995107072799,0);
\draw[line width=1.5pt,color=wwwwww] (0.4399995107072799,0) -- (0.4449995051471353,0);
\draw[line width=1.5pt,color=wwwwww] (0.4449995051471353,0) -- (0.44999949958699076,0);
\draw[line width=1.5pt,color=wwwwww] (0.44999949958699076,0) -- (0.4549994940268462,0);
\draw[line width=1.5pt,color=wwwwww] (0.4549994940268462,0) -- (0.45999948846670163,0);
\draw[line width=1.5pt,color=wwwwww] (0.45999948846670163,0) -- (0.46499948290655707,0);
\draw[line width=1.5pt,color=wwwwww] (0.46499948290655707,0) -- (0.4699994773464125,0);
\draw[line width=1.5pt,color=wwwwww] (0.4699994773464125,0) -- (0.47499947178626795,0);
\draw[line width=1.5pt,color=wwwwww] (0.47499947178626795,0) -- (0.4799994662261234,0);
\draw[line width=1.5pt,color=wwwwww] (0.4799994662261234,0) -- (0.4849994606659788,0);
\draw[line width=1.5pt,color=wwwwww] (0.4849994606659788,0) -- (0.48999945510583426,0);
\draw[line width=1.5pt,color=wwwwww] (0.48999945510583426,0) -- (0.4949994495456897,0);
\draw[line width=1.5pt,color=wwwwww] (0.4949994495456897,0) -- (0.49999944398554513,0);
\draw[line width=1.5pt,color=wwwwww] (0.49999944398554513,0) -- (0.5049994384254006,0);
\draw[line width=1.5pt,color=wwwwww] (0.5049994384254006,0) -- (0.5099994328652561,0);
\draw[line width=1.5pt,color=wwwwww] (0.5099994328652561,0) -- (0.5149994273051115,0);
\draw[line width=1.5pt,color=wwwwww] (0.5149994273051115,0) -- (0.5199994217449669,0);
\draw[line width=1.5pt,color=wwwwww] (0.5199994217449669,0) -- (0.5249994161848224,0);
\draw[line width=1.5pt,color=wwwwww] (0.5249994161848224,0) -- (0.5299994106246778,0);
\draw[line width=1.5pt,color=wwwwww] (0.5299994106246778,0) -- (0.5349994050645333,0);
\draw[line width=1.5pt,color=wwwwww] (0.5349994050645333,0) -- (0.5399993995043887,0);
\draw[line width=1.5pt,color=wwwwww] (0.5399993995043887,0) -- (0.5449993939442441,0);
\draw[line width=1.5pt,color=wwwwww] (0.5449993939442441,0) -- (0.5499993883840996,0);
\draw[line width=1.5pt,color=wwwwww] (0.5499993883840996,0) -- (0.554999382823955,0);
\draw[line width=1.5pt,color=wwwwww] (0.554999382823955,0) -- (0.5599993772638104,0);
\draw[line width=1.5pt,color=wwwwww] (0.5599993772638104,0) -- (0.5649993717036659,0);
\draw[line width=1.5pt,color=wwwwww] (0.5649993717036659,0) -- (0.5699993661435213,0);
\draw[line width=1.5pt,color=wwwwww] (0.5699993661435213,0) -- (0.5749993605833768,0);
\draw[line width=1.5pt,color=wwwwww] (0.5749993605833768,0) -- (0.5799993550232322,0);
\draw[line width=1.5pt,color=wwwwww] (0.5799993550232322,0) -- (0.5849993494630876,0);
\draw[line width=1.5pt,color=wwwwww] (0.5849993494630876,0) -- (0.5899993439029431,0);
\draw[line width=1.5pt,color=wwwwww] (0.5899993439029431,0) -- (0.5949993383427985,0);
\draw[line width=1.5pt,color=wwwwww] (0.5949993383427985,0) -- (0.5999993327826539,0);
\draw[line width=1.5pt,color=wwwwww] (0.5999993327826539,0) -- (0.6049993272225094,0);
\draw[line width=1.5pt,color=wwwwww] (0.6049993272225094,0) -- (0.6099993216623648,0);
\draw[line width=1.5pt,color=wwwwww] (0.6099993216623648,0) -- (0.6149993161022203,0);
\draw[line width=1.5pt,color=wwwwww] (0.6149993161022203,0) -- (0.6199993105420757,0);
\draw[line width=1.5pt,color=wwwwww] (0.6199993105420757,0) -- (0.6249993049819311,0);
\draw[line width=1.5pt,color=wwwwww] (0.6249993049819311,0) -- (0.6299992994217866,0);
\draw[line width=1.5pt,color=wwwwww] (0.6299992994217866,0) -- (0.634999293861642,0);
\draw[line width=1.5pt,color=wwwwww] (0.634999293861642,0) -- (0.6399992883014974,0);
\draw[line width=1.5pt,color=wwwwww] (0.6399992883014974,0) -- (0.6449992827413529,0);
\draw[line width=1.5pt,color=wwwwww] (0.6449992827413529,0) -- (0.6499992771812083,0);
\draw[line width=1.5pt,color=wwwwww] (0.6499992771812083,0) -- (0.6549992716210638,0);
\draw[line width=1.5pt,color=wwwwww] (0.6549992716210638,0) -- (0.6599992660609192,0);
\draw[line width=1.5pt,color=wwwwww] (0.6599992660609192,0) -- (0.6649992605007746,0);
\draw[line width=1.5pt,color=wwwwww] (0.6649992605007746,0) -- (0.6699992549406301,0);
\draw[line width=1.5pt,color=wwwwww] (0.6699992549406301,0) -- (0.6749992493804855,0);
\draw[line width=1.5pt,color=wwwwww] (0.6749992493804855,0) -- (0.6799992438203409,0);
\draw[line width=1.5pt,color=wwwwww] (0.6799992438203409,0) -- (0.6849992382601964,0);
\draw[line width=1.5pt,color=wwwwww] (0.6849992382601964,0) -- (0.6899992327000518,0);
\draw[line width=1.5pt,color=wwwwww] (0.6899992327000518,0) -- (0.6949992271399072,0);
\draw[line width=1.5pt,color=wwwwww] (0.6949992271399072,0) -- (0.6999992215797627,0);
\draw[line width=1.5pt,color=wwwwww] (0.6999992215797627,0) -- (0.7049992160196181,0);
\draw[line width=1.5pt,color=wwwwww] (0.7049992160196181,0) -- (0.7099992104594736,0);
\draw[line width=1.5pt,color=wwwwww] (0.7099992104594736,0) -- (0.714999204899329,0);
\draw[line width=1.5pt,color=wwwwww] (0.714999204899329,0) -- (0.7199991993391844,0);
\draw[line width=1.5pt,color=wwwwww] (0.7199991993391844,0) -- (0.7249991937790399,0);
\draw[line width=1.5pt,color=wwwwww] (0.7249991937790399,0) -- (0.7299991882188953,0);
\draw[line width=1.5pt,color=wwwwww] (0.7299991882188953,0) -- (0.7349991826587507,0);
\draw[line width=1.5pt,color=wwwwww] (0.7349991826587507,0) -- (0.7399991770986062,0);
\draw[line width=1.5pt,color=wwwwww] (0.7399991770986062,0) -- (0.7449991715384616,0.0011732295040225186);
\draw[line width=1.5pt,color=wwwwww] (0.7449991715384616,0.0011732295040225186) -- (0.7499991659783171,0.0014288261602412313);
\draw[line width=1.5pt,color=wwwwww] (0.7499991659783171,0.0014288261602412313) -- (0.7549991604181725,0.0017355390604703543);
\draw[line width=1.5pt,color=wwwwww] (0.7549991604181725,0.0017355390604703543) -- (0.7599991548580279,0.0021022278354811403);
\draw[line width=1.5pt,color=wwwwww] (0.7599991548580279,0.0021022278354811403) -- (0.7649991492978834,0.0025388937901419414);
\draw[line width=1.5pt,color=wwwwww] (0.7649991492978834,0.0025388937901419414) -- (0.7699991437377388,0.0030567125741375253);
\draw[line width=1.5pt,color=wwwwww] (0.7699991437377388,0.0030567125741375253) -- (0.7749991381775942,0.0036680288802873563);
\draw[line width=1.5pt,color=wwwwww] (0.7749991381775942,0.0036680288802873563) -- (0.7799991326174497,0.004386299727150289);
\draw[line width=1.5pt,color=wwwwww] (0.7799991326174497,0.004386299727150289) -- (0.7849991270573051,0.005225970867623111);
\draw[line width=1.5pt,color=wwwwww] (0.7849991270573051,0.005225970867623111) -- (0.7899991214971606,0.006202269142511116);
\draw[line width=1.5pt,color=wwwwww] (0.7899991214971606,0.006202269142511116) -- (0.794999115937016,0.007330892439286536);
\draw[line width=1.5pt,color=wwwwww] (0.794999115937016,0.007330892439286536) -- (0.7999991103768714,0.008627578665711766);
\draw[line width=1.5pt,color=wwwwww] (0.7999991103768714,0.008627578665711766) -- (0.8049991048167269,0.01010753621915846);
\draw[line width=1.5pt,color=wwwwww] (0.8049991048167269,0.01010753621915846) -- (0.8099990992565823,0.01178472129865923);
\draw[line width=1.5pt,color=wwwwww] (0.8099990992565823,0.01178472129865923) -- (0.8149990936964377,0.013670952574817324);
\draw[line width=1.5pt,color=wwwwww] (0.8149990936964377,0.013670952574817324) -- (0.8199990881362932,0.015774861703991463);
\draw[line width=1.5pt,color=wwwwww] (0.8199990881362932,0.015774861703991463) -- (0.8249990825761486,0.01810068938075886);
\draw[line width=1.5pt,color=wwwwww] (0.8249990825761486,0.01810068938075886) -- (0.8299990770160041,0.020646951348932054);
\draw[line width=1.5pt,color=wwwwww] (0.8299990770160041,0.020646951348932054) -- (0.8349990714558595,0.02340501706241394);
\draw[line width=1.5pt,color=wwwwww] (0.8349990714558595,0.02340501706241394) -- (0.8399990658957149,0.026357665148170045);
\draw[line width=1.5pt,color=wwwwww] (0.8399990658957149,0.026357665148170045) -- (0.8449990603355704,0.02947770360523067);
\draw[line width=1.5pt,color=wwwwww] (0.8449990603355704,0.02947770360523067) -- (0.8499990547754258,0.032726767262981675);
\draw[line width=1.5pt,color=wwwwww] (0.8499990547754258,0.032726767262981675) -- (0.8549990492152812,0.0360544281622679);
\draw[line width=1.5pt,color=wwwwww] (0.8549990492152812,0.0360544281622679) -- (0.8599990436551367,0.039397773169233735);
\draw[line width=1.5pt,color=wwwwww] (0.8599990436551367,0.039397773169233735) -- (0.8649990380949921,0.042681613506995515);
\draw[line width=1.5pt,color=wwwwww] (0.8649990380949921,0.042681613506995515) -- (0.8699990325348476,0.04581948867119018);
\draw[line width=1.5pt,color=wwwwww] (0.8699990325348476,0.04581948867119018) -- (0.874999026974703,0.04871560787889069);
\draw[line width=1.5pt,color=wwwwww] (0.874999026974703,0.04871560787889069) -- (0.8799990214145584,0.05126783166772997);
\draw[line width=1.5pt,color=wwwwww] (0.8799990214145584,0.05126783166772997) -- (0.8849990158544139,0.05337173154764506);
\draw[line width=1.5pt,color=wwwwww] (0.8849990158544139,0.05337173154764506) -- (0.8899990102942693,0.05492567584169165);
\draw[line width=1.5pt,color=wwwwww] (0.8899990102942693,0.05492567584169165) -- (0.8949990047341247,0.055836777281521525);
\draw[line width=1.5pt,color=wwwwww] (0.8949990047341247,0.055836777281521525) -- (0.8999989991739802,0.05602740887979974);
\draw[line width=1.5pt,color=wwwwww] (0.8999989991739802,0.05602740887979974) -- (0.9049989936138356,0.05544186017810303);
\draw[line width=1.5pt,color=wwwwww] (0.9049989936138356,0.05544186017810303) -- (0.9099989880536911,0.05405258217371688);
\draw[line width=1.5pt,color=wwwwww] (0.9099989880536911,0.05405258217371688) -- (0.9149989824935465,0.05186537628659741);
\draw[line width=1.5pt,color=wwwwww] (0.9149989824935465,0.05186537628659741) -- (0.9199989769334019,0.048922843318169144);
\draw[line width=1.5pt,color=wwwwww] (0.9199989769334019,0.048922843318169144) -- (0.9249989713732574,0.045305444605672186);
\draw[line width=1.5pt,color=wwwwww] (0.9249989713732574,0.045305444605672186) -- (0.9299989658131128,0.041129656910248676);
\draw[line width=1.5pt,color=wwwwww] (0.9299989658131128,0.041129656910248676) -- (0.9349989602529682,0.03654293276899554);
\draw[line width=1.5pt,color=wwwwww] (0.9349989602529682,0.03654293276899554) -- (0.9399989546928237,0.0317155022125931);
\draw[line width=1.5pt,color=wwwwww] (0.9399989546928237,0.0317155022125931) -- (0.9449989491326791,0.026829444455114128);
\draw[line width=1.5pt,color=wwwwww] (0.9449989491326791,0.026829444455114128) -- (0.9499989435725346,0.022065873897565216);
\draw[line width=1.5pt,color=wwwwww] (0.9499989435725346,0.022065873897565216) -- (0.95499893801239,0.017591460359155046);
\draw[line width=1.5pt,color=wwwwww] (0.95499893801239,0.017591460359155046) -- (0.9599989324522454,0.013545765242847316);
\draw[line width=1.5pt,color=wwwwww] (0.9599989324522454,0.013545765242847316) -- (0.9649989268921009,0.010030951675413511);
\draw[line width=1.5pt,color=wwwwww] (0.9649989268921009,0.010030951675413511) -- (0.9699989213319563,0.007105264141820101);
\draw[line width=1.5pt,color=wwwwww] (0.9699989213319563,0.007105264141820101) -- (0.9749989157718117,0.0047812542247871);
\draw[line width=1.5pt,color=wwwwww] (0.9749989157718117,0.0047812542247871) -- (0.9799989102116672,0.0030290865737074286);
\draw[line width=1.5pt,color=wwwwww] (0.9799989102116672,0.0030290865737074286) -- (0.9849989046515226,0.0017844824637001553);
\draw[line width=1.5pt,color=wwwwww] (0.9849989046515226,0.0017844824637001553) -- (0.9899988990913781,0);
\draw[line width=1.5pt,color=wwwwww] (0.9899988990913781,0) -- (0.9949988935312335,0);
\end{axis}
\end{tikzpicture}}
    \subfloat[$ h>0 $: indipendentemente da $ \beta $ il picco è spostato nel verso del campo esterno.]{
%<<<<<<<WARNING>>>>>>>
% PGF/Tikz doesn't support the following mathematical functions:
% cosh, acosh, sinh, asinh, tanh, atanh,
% x^r with r not integer

% Plotting will be done using GNUPLOT
% GNUPLOT must be installed and you must allow Latex to call external
% programs by adding the following option to your compiler
% shell-escape    OR    enable-write18 
% Example: pdflatex --shell-escape file.tex 

\definecolor{wwwwww}{rgb}{0.4,0.4,0.4}
\begin{tikzpicture}[line cap=round,line join=round,>=triangle 45,x=1cm,y=1cm]
\begin{axis}[
x=3.5cm,y=4cm,
axis lines=middle,
xmin=-1.2,
xmax=1.2,
ymin=-0.027640247367904026,
ymax=0.1,
xtick={-1,-0.8,...,1},
ytick={-0.02,0,...,0.1},
y post scale = 12,
ylabel = $f(x)$,
xlabel = $x$]
\clip(-1.3493722784017763,-0.026488569719477113) rectangle (1.4824063340982103,0.13523377584645435);
\draw[line width=1.5pt,color=wwwwww] (-0.9999987603059788,0) -- (-0.9999987603059788,0);
\draw[line width=1.5pt,color=wwwwww] (-0.9999987603059788,0) -- (-0.99499876823097,0);
\draw[line width=1.5pt,color=wwwwww] (-0.99499876823097,0) -- (-0.9899987761559612,0);
\draw[line width=1.5pt,color=wwwwww] (-0.9899987761559612,0) -- (-0.9849987840809524,0);
\draw[line width=1.5pt,color=wwwwww] (-0.9849987840809524,0) -- (-0.9799987920059436,0);
\draw[line width=1.5pt,color=wwwwww] (-0.9799987920059436,0) -- (-0.9749987999309349,0);
\draw[line width=1.5pt,color=wwwwww] (-0.9749987999309349,0) -- (-0.9699988078559261,0);
\draw[line width=1.5pt,color=wwwwww] (-0.9699988078559261,0) -- (-0.9649988157809173,0);
\draw[line width=1.5pt,color=wwwwww] (-0.9649988157809173,0) -- (-0.9599988237059085,0);
\draw[line width=1.5pt,color=wwwwww] (-0.9599988237059085,0) -- (-0.9549988316308997,0);
\draw[line width=1.5pt,color=wwwwww] (-0.9549988316308997,0) -- (-0.9499988395558909,0);
\draw[line width=1.5pt,color=wwwwww] (-0.9499988395558909,0) -- (-0.9449988474808821,0);
\draw[line width=1.5pt,color=wwwwww] (-0.9449988474808821,0) -- (-0.9399988554058734,0);
\draw[line width=1.5pt,color=wwwwww] (-0.9399988554058734,0) -- (-0.9349988633308646,0);
\draw[line width=1.5pt,color=wwwwww] (-0.9349988633308646,0) -- (-0.9299988712558558,0);
\draw[line width=1.5pt,color=wwwwww] (-0.9299988712558558,0) -- (-0.924998879180847,0);
\draw[line width=1.5pt,color=wwwwww] (-0.924998879180847,0) -- (-0.9199988871058382,0);
\draw[line width=1.5pt,color=wwwwww] (-0.9199988871058382,0) -- (-0.9149988950308294,0);
\draw[line width=1.5pt,color=wwwwww] (-0.9149988950308294,0) -- (-0.9099989029558206,0);
\draw[line width=1.5pt,color=wwwwww] (-0.9099989029558206,0) -- (-0.9049989108808119,0);
\draw[line width=1.5pt,color=wwwwww] (-0.9049989108808119,0) -- (-0.8999989188058031,0);
\draw[line width=1.5pt,color=wwwwww] (-0.8999989188058031,0) -- (-0.8949989267307943,0);
\draw[line width=1.5pt,color=wwwwww] (-0.8949989267307943,0) -- (-0.8899989346557855,0);
\draw[line width=1.5pt,color=wwwwww] (-0.8899989346557855,0) -- (-0.8849989425807767,0);
\draw[line width=1.5pt,color=wwwwww] (-0.8849989425807767,0) -- (-0.8799989505057679,0);
\draw[line width=1.5pt,color=wwwwww] (-0.8799989505057679,0) -- (-0.8749989584307591,0);
\draw[line width=1.5pt,color=wwwwww] (-0.8749989584307591,0) -- (-0.8699989663557504,0);
\draw[line width=1.5pt,color=wwwwww] (-0.8699989663557504,0) -- (-0.8649989742807416,0);
\draw[line width=1.5pt,color=wwwwww] (-0.8649989742807416,0) -- (-0.8599989822057328,0);
\draw[line width=1.5pt,color=wwwwww] (-0.8599989822057328,0) -- (-0.854998990130724,0);
\draw[line width=1.5pt,color=wwwwww] (-0.854998990130724,0) -- (-0.8499989980557152,0);
\draw[line width=1.5pt,color=wwwwww] (-0.8499989980557152,0) -- (-0.8449990059807064,0);
\draw[line width=1.5pt,color=wwwwww] (-0.8449990059807064,0) -- (-0.8399990139056976,0);
\draw[line width=1.5pt,color=wwwwww] (-0.8399990139056976,0) -- (-0.8349990218306889,0);
\draw[line width=1.5pt,color=wwwwww] (-0.8349990218306889,0) -- (-0.8299990297556801,0);
\draw[line width=1.5pt,color=wwwwww] (-0.8299990297556801,0) -- (-0.8249990376806713,0);
\draw[line width=1.5pt,color=wwwwww] (-0.8249990376806713,0) -- (-0.8199990456056625,0);
\draw[line width=1.5pt,color=wwwwww] (-0.8199990456056625,0) -- (-0.8149990535306537,0);
\draw[line width=1.5pt,color=wwwwww] (-0.8149990535306537,0) -- (-0.8099990614556449,0);
\draw[line width=1.5pt,color=wwwwww] (-0.8099990614556449,0) -- (-0.8049990693806361,0);
\draw[line width=1.5pt,color=wwwwww] (-0.8049990693806361,0) -- (-0.7999990773056274,0);
\draw[line width=1.5pt,color=wwwwww] (-0.7999990773056274,0) -- (-0.7949990852306186,0);
\draw[line width=1.5pt,color=wwwwww] (-0.7949990852306186,0) -- (-0.7899990931556098,0);
\draw[line width=1.5pt,color=wwwwww] (-0.7899990931556098,0) -- (-0.784999101080601,0);
\draw[line width=1.5pt,color=wwwwww] (-0.784999101080601,0) -- (-0.7799991090055922,0);
\draw[line width=1.5pt,color=wwwwww] (-0.7799991090055922,0) -- (-0.7749991169305834,0);
\draw[line width=1.5pt,color=wwwwww] (-0.7749991169305834,0) -- (-0.7699991248555746,0);
\draw[line width=1.5pt,color=wwwwww] (-0.7699991248555746,0) -- (-0.7649991327805659,0);
\draw[line width=1.5pt,color=wwwwww] (-0.7649991327805659,0) -- (-0.7599991407055571,0);
\draw[line width=1.5pt,color=wwwwww] (-0.7599991407055571,0) -- (-0.7549991486305483,0);
\draw[line width=1.5pt,color=wwwwww] (-0.7549991486305483,0) -- (-0.7499991565555395,0);
\draw[line width=1.5pt,color=wwwwww] (-0.7499991565555395,0) -- (-0.7449991644805307,0);
\draw[line width=1.5pt,color=wwwwww] (-0.7449991644805307,0) -- (-0.7399991724055219,0);
\draw[line width=1.5pt,color=wwwwww] (-0.7399991724055219,0) -- (-0.7349991803305131,0);
\draw[line width=1.5pt,color=wwwwww] (-0.7349991803305131,0) -- (-0.7299991882555044,0);
\draw[line width=1.5pt,color=wwwwww] (-0.7299991882555044,0) -- (-0.7249991961804956,0);
\draw[line width=1.5pt,color=wwwwww] (-0.7249991961804956,0) -- (-0.7199992041054868,0);
\draw[line width=1.5pt,color=wwwwww] (-0.7199992041054868,0) -- (-0.714999212030478,0);
\draw[line width=1.5pt,color=wwwwww] (-0.714999212030478,0) -- (-0.7099992199554692,0);
\draw[line width=1.5pt,color=wwwwww] (-0.7099992199554692,0) -- (-0.7049992278804604,0);
\draw[line width=1.5pt,color=wwwwww] (-0.7049992278804604,0) -- (-0.6999992358054516,0);
\draw[line width=1.5pt,color=wwwwww] (-0.6999992358054516,0) -- (-0.6949992437304429,0);
\draw[line width=1.5pt,color=wwwwww] (-0.6949992437304429,0) -- (-0.6899992516554341,0);
\draw[line width=1.5pt,color=wwwwww] (-0.6899992516554341,0) -- (-0.6849992595804253,0);
\draw[line width=1.5pt,color=wwwwww] (-0.6849992595804253,0) -- (-0.6799992675054165,0);
\draw[line width=1.5pt,color=wwwwww] (-0.6799992675054165,0) -- (-0.6749992754304077,0);
\draw[line width=1.5pt,color=wwwwww] (-0.6749992754304077,0) -- (-0.6699992833553989,0);
\draw[line width=1.5pt,color=wwwwww] (-0.6699992833553989,0) -- (-0.6649992912803901,0);
\draw[line width=1.5pt,color=wwwwww] (-0.6649992912803901,0) -- (-0.6599992992053814,0);
\draw[line width=1.5pt,color=wwwwww] (-0.6599992992053814,0) -- (-0.6549993071303726,0);
\draw[line width=1.5pt,color=wwwwww] (-0.6549993071303726,0) -- (-0.6499993150553638,0);
\draw[line width=1.5pt,color=wwwwww] (-0.6499993150553638,0) -- (-0.644999322980355,0);
\draw[line width=1.5pt,color=wwwwww] (-0.644999322980355,0) -- (-0.6399993309053462,0);
\draw[line width=1.5pt,color=wwwwww] (-0.6399993309053462,0) -- (-0.6349993388303374,0);
\draw[line width=1.5pt,color=wwwwww] (-0.6349993388303374,0) -- (-0.6299993467553286,0);
\draw[line width=1.5pt,color=wwwwww] (-0.6299993467553286,0) -- (-0.6249993546803199,0);
\draw[line width=1.5pt,color=wwwwww] (-0.6249993546803199,0) -- (-0.6199993626053111,0);
\draw[line width=1.5pt,color=wwwwww] (-0.6199993626053111,0) -- (-0.6149993705303023,0);
\draw[line width=1.5pt,color=wwwwww] (-0.6149993705303023,0) -- (-0.6099993784552935,0);
\draw[line width=1.5pt,color=wwwwww] (-0.6099993784552935,0) -- (-0.6049993863802847,0);
\draw[line width=1.5pt,color=wwwwww] (-0.6049993863802847,0) -- (-0.5999993943052759,0);
\draw[line width=1.5pt,color=wwwwww] (-0.5999993943052759,0) -- (-0.5949994022302671,0);
\draw[line width=1.5pt,color=wwwwww] (-0.5949994022302671,0) -- (-0.5899994101552584,0);
\draw[line width=1.5pt,color=wwwwww] (-0.5899994101552584,0) -- (-0.5849994180802496,0);
\draw[line width=1.5pt,color=wwwwww] (-0.5849994180802496,0) -- (-0.5799994260052408,0);
\draw[line width=1.5pt,color=wwwwww] (-0.5799994260052408,0) -- (-0.574999433930232,0);
\draw[line width=1.5pt,color=wwwwww] (-0.574999433930232,0) -- (-0.5699994418552232,0);
\draw[line width=1.5pt,color=wwwwww] (-0.5699994418552232,0) -- (-0.5649994497802144,0);
\draw[line width=1.5pt,color=wwwwww] (-0.5649994497802144,0) -- (-0.5599994577052056,0);
\draw[line width=1.5pt,color=wwwwww] (-0.5599994577052056,0) -- (-0.5549994656301969,0);
\draw[line width=1.5pt,color=wwwwww] (-0.5549994656301969,0) -- (-0.5499994735551881,0);
\draw[line width=1.5pt,color=wwwwww] (-0.5499994735551881,0) -- (-0.5449994814801793,0);
\draw[line width=1.5pt,color=wwwwww] (-0.5449994814801793,0) -- (-0.5399994894051705,0);
\draw[line width=1.5pt,color=wwwwww] (-0.5399994894051705,0) -- (-0.5349994973301617,0);
\draw[line width=1.5pt,color=wwwwww] (-0.5349994973301617,0) -- (-0.5299995052551529,0);
\draw[line width=1.5pt,color=wwwwww] (-0.5299995052551529,0) -- (-0.5249995131801442,0);
\draw[line width=1.5pt,color=wwwwww] (-0.5249995131801442,0) -- (-0.5199995211051354,0);
\draw[line width=1.5pt,color=wwwwww] (-0.5199995211051354,0) -- (-0.5149995290301266,0);
\draw[line width=1.5pt,color=wwwwww] (-0.5149995290301266,0) -- (-0.5099995369551178,0);
\draw[line width=1.5pt,color=wwwwww] (-0.5099995369551178,0) -- (-0.504999544880109,0);
\draw[line width=1.5pt,color=wwwwww] (-0.504999544880109,0) -- (-0.49999955280510017,0);
\draw[line width=1.5pt,color=wwwwww] (-0.49999955280510017,0) -- (-0.4949995607300913,0);
\draw[line width=1.5pt,color=wwwwww] (-0.4949995607300913,0) -- (-0.4899995686550825,0);
\draw[line width=1.5pt,color=wwwwww] (-0.4899995686550825,0) -- (-0.48499957658007364,0);
\draw[line width=1.5pt,color=wwwwww] (-0.48499957658007364,0) -- (-0.4799995845050648,0);
\draw[line width=1.5pt,color=wwwwww] (-0.4799995845050648,0) -- (-0.47499959243005596,0);
\draw[line width=1.5pt,color=wwwwww] (-0.47499959243005596,0) -- (-0.4699996003550471,0);
\draw[line width=1.5pt,color=wwwwww] (-0.4699996003550471,0) -- (-0.4649996082800383,0);
\draw[line width=1.5pt,color=wwwwww] (-0.4649996082800383,0) -- (-0.45999961620502944,0);
\draw[line width=1.5pt,color=wwwwww] (-0.45999961620502944,0) -- (-0.4549996241300206,0);
\draw[line width=1.5pt,color=wwwwww] (-0.4549996241300206,0) -- (-0.44999963205501176,0);
\draw[line width=1.5pt,color=wwwwww] (-0.44999963205501176,0) -- (-0.4449996399800029,0);
\draw[line width=1.5pt,color=wwwwww] (-0.4449996399800029,0) -- (-0.4399996479049941,0);
\draw[line width=1.5pt,color=wwwwww] (-0.4399996479049941,0) -- (-0.43499965582998523,0);
\draw[line width=1.5pt,color=wwwwww] (-0.43499965582998523,0) -- (-0.4299996637549764,0);
\draw[line width=1.5pt,color=wwwwww] (-0.4299996637549764,0) -- (-0.42499967167996755,0);
\draw[line width=1.5pt,color=wwwwww] (-0.42499967167996755,0) -- (-0.4199996796049587,0);
\draw[line width=1.5pt,color=wwwwww] (-0.4199996796049587,0) -- (-0.41499968752994987,0);
\draw[line width=1.5pt,color=wwwwww] (-0.41499968752994987,0) -- (-0.409999695454941,0);
\draw[line width=1.5pt,color=wwwwww] (-0.409999695454941,0) -- (-0.4049997033799322,0);
\draw[line width=1.5pt,color=wwwwww] (-0.4049997033799322,0) -- (-0.39999971130492334,0);
\draw[line width=1.5pt,color=wwwwww] (-0.39999971130492334,0) -- (-0.3949997192299145,0);
\draw[line width=1.5pt,color=wwwwww] (-0.3949997192299145,0) -- (-0.38999972715490566,0);
\draw[line width=1.5pt,color=wwwwww] (-0.38999972715490566,0) -- (-0.3849997350798968,0);
\draw[line width=1.5pt,color=wwwwww] (-0.3849997350798968,0) -- (-0.379999743004888,0);
\draw[line width=1.5pt,color=wwwwww] (-0.379999743004888,0) -- (-0.37499975092987914,0);
\draw[line width=1.5pt,color=wwwwww] (-0.37499975092987914,0) -- (-0.3699997588548703,0);
\draw[line width=1.5pt,color=wwwwww] (-0.3699997588548703,0) -- (-0.36499976677986146,0);
\draw[line width=1.5pt,color=wwwwww] (-0.36499976677986146,0) -- (-0.3599997747048526,0);
\draw[line width=1.5pt,color=wwwwww] (-0.3599997747048526,0) -- (-0.3549997826298438,0);
\draw[line width=1.5pt,color=wwwwww] (-0.3549997826298438,0) -- (-0.34999979055483493,0);
\draw[line width=1.5pt,color=wwwwww] (-0.34999979055483493,0) -- (-0.3449997984798261,0);
\draw[line width=1.5pt,color=wwwwww] (-0.3449997984798261,0) -- (-0.33999980640481725,0);
\draw[line width=1.5pt,color=wwwwww] (-0.33999980640481725,0) -- (-0.3349998143298084,0);
\draw[line width=1.5pt,color=wwwwww] (-0.3349998143298084,0) -- (-0.32999982225479957,0);
\draw[line width=1.5pt,color=wwwwww] (-0.32999982225479957,0) -- (-0.3249998301797907,0);
\draw[line width=1.5pt,color=wwwwww] (-0.3249998301797907,0) -- (-0.3199998381047819,0);
\draw[line width=1.5pt,color=wwwwww] (-0.3199998381047819,0) -- (-0.31499984602977305,0);
\draw[line width=1.5pt,color=wwwwww] (-0.31499984602977305,0) -- (-0.3099998539547642,0);
\draw[line width=1.5pt,color=wwwwww] (-0.3099998539547642,0) -- (-0.30499986187975536,0);
\draw[line width=1.5pt,color=wwwwww] (-0.30499986187975536,0) -- (-0.2999998698047465,0);
\draw[line width=1.5pt,color=wwwwww] (-0.2999998698047465,0) -- (-0.2949998777297377,0);
\draw[line width=1.5pt,color=wwwwww] (-0.2949998777297377,0) -- (-0.28999988565472884,0);
\draw[line width=1.5pt,color=wwwwww] (-0.28999988565472884,0) -- (-0.28499989357972,0);
\draw[line width=1.5pt,color=wwwwww] (-0.28499989357972,0) -- (-0.27999990150471116,0);
\draw[line width=1.5pt,color=wwwwww] (-0.27999990150471116,0) -- (-0.2749999094297023,0);
\draw[line width=1.5pt,color=wwwwww] (-0.2749999094297023,0) -- (-0.2699999173546935,0);
\draw[line width=1.5pt,color=wwwwww] (-0.2699999173546935,0) -- (-0.26499992527968463,0);
\draw[line width=1.5pt,color=wwwwww] (-0.26499992527968463,0) -- (-0.2599999332046758,0);
\draw[line width=1.5pt,color=wwwwww] (-0.2599999332046758,0) -- (-0.25499994112966695,0);
\draw[line width=1.5pt,color=wwwwww] (-0.25499994112966695,0) -- (-0.24999994905465814,0);
\draw[line width=1.5pt,color=wwwwww] (-0.24999994905465814,0) -- (-0.24499995697964932,0);
\draw[line width=1.5pt,color=wwwwww] (-0.24499995697964932,0) -- (-0.2399999649046405,0);
\draw[line width=1.5pt,color=wwwwww] (-0.2399999649046405,0) -- (-0.2349999728296317,0);
\draw[line width=1.5pt,color=wwwwww] (-0.2349999728296317,0) -- (-0.22999998075462288,0);
\draw[line width=1.5pt,color=wwwwww] (-0.22999998075462288,0) -- (-0.22499998867961407,0);
\draw[line width=1.5pt,color=wwwwww] (-0.22499998867961407,0) -- (-0.21999999660460526,0);
\draw[line width=1.5pt,color=wwwwww] (-0.21999999660460526,0) -- (-0.21500000452959644,0);
\draw[line width=1.5pt,color=wwwwww] (-0.21500000452959644,0) -- (-0.21000001245458763,0);
\draw[line width=1.5pt,color=wwwwww] (-0.21000001245458763,0) -- (-0.20500002037957882,0);
\draw[line width=1.5pt,color=wwwwww] (-0.20500002037957882,0) -- (-0.20000002830457,0);
\draw[line width=1.5pt,color=wwwwww] (-0.20000002830457,0) -- (-0.1950000362295612,0);
\draw[line width=1.5pt,color=wwwwww] (-0.1950000362295612,0) -- (-0.19000004415455238,0);
\draw[line width=1.5pt,color=wwwwww] (-0.19000004415455238,0) -- (-0.18500005207954356,0);
\draw[line width=1.5pt,color=wwwwww] (-0.18500005207954356,0) -- (-0.18000006000453475,0);
\draw[line width=1.5pt,color=wwwwww] (-0.18000006000453475,0) -- (-0.17500006792952594,0);
\draw[line width=1.5pt,color=wwwwww] (-0.17500006792952594,0) -- (-0.17000007585451712,0);
\draw[line width=1.5pt,color=wwwwww] (-0.17000007585451712,0) -- (-0.1650000837795083,0);
\draw[line width=1.5pt,color=wwwwww] (-0.1650000837795083,0) -- (-0.1600000917044995,0);
\draw[line width=1.5pt,color=wwwwww] (-0.1600000917044995,0) -- (-0.15500009962949068,0);
\draw[line width=1.5pt,color=wwwwww] (-0.15500009962949068,0) -- (-0.15000010755448187,0);
\draw[line width=1.5pt,color=wwwwww] (-0.15000010755448187,0) -- (-0.14500011547947306,0);
\draw[line width=1.5pt,color=wwwwww] (-0.14500011547947306,0) -- (-0.14000012340446424,0);
\draw[line width=1.5pt,color=wwwwww] (-0.14000012340446424,0) -- (-0.13500013132945543,0);
\draw[line width=1.5pt,color=wwwwww] (-0.13500013132945543,0) -- (-0.13000013925444662,0);
\draw[line width=1.5pt,color=wwwwww] (-0.13000013925444662,0) -- (-0.1250001471794378,0);
\draw[line width=1.5pt,color=wwwwww] (-0.1250001471794378,0) -- (-0.12000015510442899,0);
\draw[line width=1.5pt,color=wwwwww] (-0.12000015510442899,0) -- (-0.11500016302942018,0);
\draw[line width=1.5pt,color=wwwwww] (-0.11500016302942018,0) -- (-0.11000017095441136,0);
\draw[line width=1.5pt,color=wwwwww] (-0.11000017095441136,0) -- (-0.10500017887940255,0);
\draw[line width=1.5pt,color=wwwwww] (-0.10500017887940255,0) -- (-0.10000018680439374,0);
\draw[line width=1.5pt,color=wwwwww] (-0.10000018680439374,0) -- (-0.09500019472938492,0);
\draw[line width=1.5pt,color=wwwwww] (-0.09500019472938492,0) -- (-0.09000020265437611,0);
\draw[line width=1.5pt,color=wwwwww] (-0.09000020265437611,0) -- (-0.0850002105793673,0);
\draw[line width=1.5pt,color=wwwwww] (-0.0850002105793673,0) -- (-0.08000021850435848,0);
\draw[line width=1.5pt,color=wwwwww] (-0.08000021850435848,0) -- (-0.07500022642934967,0);
\draw[line width=1.5pt,color=wwwwww] (-0.07500022642934967,0) -- (-0.07000023435434086,0);
\draw[line width=1.5pt,color=wwwwww] (-0.07000023435434086,0) -- (-0.06500024227933204,0);
\draw[line width=1.5pt,color=wwwwww] (-0.06500024227933204,0) -- (-0.06000025020432323,0);
\draw[line width=1.5pt,color=wwwwww] (-0.06000025020432323,0) -- (-0.05500025812931442,0);
\draw[line width=1.5pt,color=wwwwww] (-0.05500025812931442,0) -- (-0.0500002660543056,0);
\draw[line width=1.5pt,color=wwwwww] (-0.0500002660543056,0) -- (-0.04500027397929679,0);
\draw[line width=1.5pt,color=wwwwww] (-0.04500027397929679,0) -- (-0.040000281904287976,0);
\draw[line width=1.5pt,color=wwwwww] (-0.040000281904287976,0) -- (-0.03500028982927916,0);
\draw[line width=1.5pt,color=wwwwww] (-0.03500028982927916,0) -- (-0.030000297754270346,0);
\draw[line width=1.5pt,color=wwwwww] (-0.030000297754270346,0) -- (-0.02500030567926153,0);
\draw[line width=1.5pt,color=wwwwww] (-0.02500030567926153,0) -- (-0.020000313604252713,0);
\draw[line width=1.5pt,color=wwwwww] (-0.020000313604252713,0) -- (-0.015000321529243896,0);
\draw[line width=1.5pt,color=wwwwww] (-0.015000321529243896,0) -- (-0.010000329454235079,0);
\draw[line width=1.5pt,color=wwwwww] (-0.010000329454235079,0) -- (-0.005000337379226262,0);
\draw[line width=1.5pt,color=wwwwww] (-0.005000337379226262,0) -- (0,0);
\draw[line width=1.5pt,color=wwwwww] (0,0) -- (0.004999992075008817,0);
\draw[line width=1.5pt,color=wwwwww] (0.004999992075008817,0) -- (0.009999984150017634,0);
\draw[line width=1.5pt,color=wwwwww] (0.009999984150017634,0) -- (0.01499997622502645,0);
\draw[line width=1.5pt,color=wwwwww] (0.01499997622502645,0) -- (0.019999968300035267,0);
\draw[line width=1.5pt,color=wwwwww] (0.019999968300035267,0) -- (0.024999960375044084,0);
\draw[line width=1.5pt,color=wwwwww] (0.024999960375044084,0) -- (0.0299999524500529,0);
\draw[line width=1.5pt,color=wwwwww] (0.0299999524500529,0) -- (0.03499994452506172,0);
\draw[line width=1.5pt,color=wwwwww] (0.03499994452506172,0) -- (0.039999936600070535,0);
\draw[line width=1.5pt,color=wwwwww] (0.039999936600070535,0) -- (0.04499992867507935,0);
\draw[line width=1.5pt,color=wwwwww] (0.04499992867507935,0) -- (0.04999992075008816,0);
\draw[line width=1.5pt,color=wwwwww] (0.04999992075008816,0) -- (0.054999912825096975,0);
\draw[line width=1.5pt,color=wwwwww] (0.054999912825096975,0) -- (0.05999990490010579,0);
\draw[line width=1.5pt,color=wwwwww] (0.05999990490010579,0) -- (0.0649998969751146,0);
\draw[line width=1.5pt,color=wwwwww] (0.0649998969751146,0) -- (0.06999988905012342,0);
\draw[line width=1.5pt,color=wwwwww] (0.06999988905012342,0) -- (0.07499988112513223,0);
\draw[line width=1.5pt,color=wwwwww] (0.07499988112513223,0) -- (0.07999987320014104,0);
\draw[line width=1.5pt,color=wwwwww] (0.07999987320014104,0) -- (0.08499986527514986,0);
\draw[line width=1.5pt,color=wwwwww] (0.08499986527514986,0) -- (0.08999985735015867,0);
\draw[line width=1.5pt,color=wwwwww] (0.08999985735015867,0) -- (0.09499984942516748,0);
\draw[line width=1.5pt,color=wwwwww] (0.09499984942516748,0) -- (0.0999998415001763,0);
\draw[line width=1.5pt,color=wwwwww] (0.0999998415001763,0) -- (0.10499983357518511,0);
\draw[line width=1.5pt,color=wwwwww] (0.10499983357518511,0) -- (0.10999982565019392,0);
\draw[line width=1.5pt,color=wwwwww] (0.10999982565019392,0) -- (0.11499981772520274,0);
\draw[line width=1.5pt,color=wwwwww] (0.11499981772520274,0) -- (0.11999980980021155,0);
\draw[line width=1.5pt,color=wwwwww] (0.11999980980021155,0) -- (0.12499980187522036,0);
\draw[line width=1.5pt,color=wwwwww] (0.12499980187522036,0) -- (0.12999979395022918,0);
\draw[line width=1.5pt,color=wwwwww] (0.12999979395022918,0) -- (0.134999786025238,0);
\draw[line width=1.5pt,color=wwwwww] (0.134999786025238,0) -- (0.1399997781002468,0);
\draw[line width=1.5pt,color=wwwwww] (0.1399997781002468,0) -- (0.14499977017525562,0);
\draw[line width=1.5pt,color=wwwwww] (0.14499977017525562,0) -- (0.14999976225026443,0);
\draw[line width=1.5pt,color=wwwwww] (0.14999976225026443,0) -- (0.15499975432527324,0);
\draw[line width=1.5pt,color=wwwwww] (0.15499975432527324,0) -- (0.15999974640028206,0);
\draw[line width=1.5pt,color=wwwwww] (0.15999974640028206,0) -- (0.16499973847529087,0);
\draw[line width=1.5pt,color=wwwwww] (0.16499973847529087,0) -- (0.16999973055029968,0);
\draw[line width=1.5pt,color=wwwwww] (0.16999973055029968,0) -- (0.1749997226253085,0);
\draw[line width=1.5pt,color=wwwwww] (0.1749997226253085,0) -- (0.1799997147003173,0);
\draw[line width=1.5pt,color=wwwwww] (0.1799997147003173,0) -- (0.18499970677532612,0);
\draw[line width=1.5pt,color=wwwwww] (0.18499970677532612,0) -- (0.18999969885033494,0);
\draw[line width=1.5pt,color=wwwwww] (0.18999969885033494,0) -- (0.19499969092534375,0);
\draw[line width=1.5pt,color=wwwwww] (0.19499969092534375,0) -- (0.19999968300035256,0);
\draw[line width=1.5pt,color=wwwwww] (0.19999968300035256,0) -- (0.20499967507536138,0);
\draw[line width=1.5pt,color=wwwwww] (0.20499967507536138,0) -- (0.2099996671503702,0);
\draw[line width=1.5pt,color=wwwwww] (0.2099996671503702,0) -- (0.214999659225379,0);
\draw[line width=1.5pt,color=wwwwww] (0.214999659225379,0) -- (0.21999965130038782,0);
\draw[line width=1.5pt,color=wwwwww] (0.21999965130038782,0) -- (0.22499964337539663,0);
\draw[line width=1.5pt,color=wwwwww] (0.22499964337539663,0) -- (0.22999963545040544,0);
\draw[line width=1.5pt,color=wwwwww] (0.22999963545040544,0) -- (0.23499962752541426,0);
\draw[line width=1.5pt,color=wwwwww] (0.23499962752541426,0) -- (0.23999961960042307,0);
\draw[line width=1.5pt,color=wwwwww] (0.23999961960042307,0) -- (0.24499961167543188,0);
\draw[line width=1.5pt,color=wwwwww] (0.24499961167543188,0) -- (0.2499996037504407,0);
\draw[line width=1.5pt,color=wwwwww] (0.2499996037504407,0) -- (0.2549995958254495,0);
\draw[line width=1.5pt,color=wwwwww] (0.2549995958254495,0) -- (0.25999958790045835,0);
\draw[line width=1.5pt,color=wwwwww] (0.25999958790045835,0) -- (0.2649995799754672,0);
\draw[line width=1.5pt,color=wwwwww] (0.2649995799754672,0) -- (0.26999957205047603,0);
\draw[line width=1.5pt,color=wwwwww] (0.26999957205047603,0) -- (0.2749995641254849,0);
\draw[line width=1.5pt,color=wwwwww] (0.2749995641254849,0) -- (0.2799995562004937,0);
\draw[line width=1.5pt,color=wwwwww] (0.2799995562004937,0) -- (0.28499954827550256,0);
\draw[line width=1.5pt,color=wwwwww] (0.28499954827550256,0) -- (0.2899995403505114,0);
\draw[line width=1.5pt,color=wwwwww] (0.2899995403505114,0) -- (0.29499953242552024,0);
\draw[line width=1.5pt,color=wwwwww] (0.29499953242552024,0) -- (0.2999995245005291,0);
\draw[line width=1.5pt,color=wwwwww] (0.2999995245005291,0) -- (0.3049995165755379,0);
\draw[line width=1.5pt,color=wwwwww] (0.3049995165755379,0) -- (0.30999950865054676,0);
\draw[line width=1.5pt,color=wwwwww] (0.30999950865054676,0) -- (0.3149995007255556,0);
\draw[line width=1.5pt,color=wwwwww] (0.3149995007255556,0) -- (0.31999949280056444,0);
\draw[line width=1.5pt,color=wwwwww] (0.31999949280056444,0) -- (0.3249994848755733,0);
\draw[line width=1.5pt,color=wwwwww] (0.3249994848755733,0) -- (0.3299994769505821,0);
\draw[line width=1.5pt,color=wwwwww] (0.3299994769505821,0) -- (0.33499946902559097,0);
\draw[line width=1.5pt,color=wwwwww] (0.33499946902559097,0) -- (0.3399994611005998,0);
\draw[line width=1.5pt,color=wwwwww] (0.3399994611005998,0) -- (0.34499945317560865,0);
\draw[line width=1.5pt,color=wwwwww] (0.34499945317560865,0) -- (0.3499994452506175,0);
\draw[line width=1.5pt,color=wwwwww] (0.3499994452506175,0) -- (0.35499943732562633,0);
\draw[line width=1.5pt,color=wwwwww] (0.35499943732562633,0) -- (0.3599994294006352,0);
\draw[line width=1.5pt,color=wwwwww] (0.3599994294006352,0) -- (0.364999421475644,0);
\draw[line width=1.5pt,color=wwwwww] (0.364999421475644,0) -- (0.36999941355065286,0);
\draw[line width=1.5pt,color=wwwwww] (0.36999941355065286,0) -- (0.3749994056256617,0.0010565550205939394);
\draw[line width=1.5pt,color=wwwwww] (0.3749994056256617,0.0010565550205939394) -- (0.37999939770067054,0.0011557430029352958);
\draw[line width=1.5pt,color=wwwwww] (0.37999939770067054,0.0011557430029352958) -- (0.3849993897756794,0.001263933093987126);
\draw[line width=1.5pt,color=wwwwww] (0.3849993897756794,0.001263933093987126) -- (0.3899993818506882,0.0013818832746321063);
\draw[line width=1.5pt,color=wwwwww] (0.3899993818506882,0.0013818832746321063) -- (0.39499937392569706,0.0015104059919194934);
\draw[line width=1.5pt,color=wwwwww] (0.39499937392569706,0.0015104059919194934) -- (0.3999993660007059,0.001650370670162193);
\draw[line width=1.5pt,color=wwwwww] (0.3999993660007059,0.001650370670162193) -- (0.40499935807571474,0.0018027061175485302);
\draw[line width=1.5pt,color=wwwwww] (0.40499935807571474,0.0018027061175485302) -- (0.4099993501507236,0.0019684027829292034);
\draw[line width=1.5pt,color=wwwwww] (0.4099993501507236,0.0019684027829292034) -- (0.4149993422257324,0.002148514810915702);
\draw[line width=1.5pt,color=wwwwww] (0.4149993422257324,0.002148514810915702) -- (0.41999933430074127,0.002344161836399814);
\draw[line width=1.5pt,color=wwwwww] (0.41999933430074127,0.002344161836399814) -- (0.4249993263757501,0.0025565304521283122);
\draw[line width=1.5pt,color=wwwwww] (0.4249993263757501,0.0025565304521283122) -- (0.42999931845075895,0.002786875275061721);
\draw[line width=1.5pt,color=wwwwww] (0.42999931845075895,0.002786875275061721) -- (0.4349993105257678,0.0030365195289739776);
\draw[line width=1.5pt,color=wwwwww] (0.4349993105257678,0.0030365195289739776) -- (0.43999930260077663,0.003306855052220548);
\draw[line width=1.5pt,color=wwwwww] (0.43999930260077663,0.003306855052220548) -- (0.4449992946757855,0.003599341630860562);
\draw[line width=1.5pt,color=wwwwww] (0.4449992946757855,0.003599341630860562) -- (0.4499992867507943,0.003915505548563207);
\draw[line width=1.5pt,color=wwwwww] (0.4499992867507943,0.003915505548563207) -- (0.45499927882580316,0.004256937236104108);
\draw[line width=1.5pt,color=wwwwww] (0.45499927882580316,0.004256937236104108) -- (0.459999270900812,0.004625287894904148);
\draw[line width=1.5pt,color=wwwwww] (0.459999270900812,0.004625287894904148) -- (0.46499926297582084,0.0050222649613362);
\draw[line width=1.5pt,color=wwwwww] (0.46499926297582084,0.0050222649613362) -- (0.4699992550508297,0.005449626271588772);
\draw[line width=1.5pt,color=wwwwww] (0.4699992550508297,0.005449626271588772) -- (0.4749992471258385,0.00590917278108634);
\draw[line width=1.5pt,color=wwwwww] (0.4749992471258385,0.00590917278108634) -- (0.47999923920084736,0.006402739688268006);
\draw[line width=1.5pt,color=wwwwww] (0.47999923920084736,0.006402739688268006) -- (0.4849992312758562,0.006932185810145329);
\draw[line width=1.5pt,color=wwwwww] (0.4849992312758562,0.006932185810145329) -- (0.48999922335086504,0.007499381057178103);
\draw[line width=1.5pt,color=wwwwww] (0.48999922335086504,0.007499381057178103) -- (0.4949992154258739,0.00810619185794204);
\draw[line width=1.5pt,color=wwwwww] (0.4949992154258739,0.00810619185794204) -- (0.4999992075008827,0.008754464390419431);
\draw[line width=1.5pt,color=wwwwww] (0.4999992075008827,0.008754464390419431) -- (0.5049991995758916,0.009446005487158462);
\draw[line width=1.5pt,color=wwwwww] (0.5049991995758916,0.009446005487158462) -- (0.5099991916509004,0.010182561096480374);
\draw[line width=1.5pt,color=wwwwww] (0.5099991916509004,0.010182561096480374) -- (0.5149991837259091,0.010965792202107274);
\draw[line width=1.5pt,color=wwwwww] (0.5149991837259091,0.010965792202107274) -- (0.5199991758009179,0.011797248129604075);
\draw[line width=1.5pt,color=wwwwww] (0.5199991758009179,0.011797248129604075) -- (0.5249991678759267,0.01267833720044384);
\draw[line width=1.5pt,color=wwwwww] (0.5249991678759267,0.01267833720044384) -- (0.5299991599509355,0.013610294733978491);
\draw[line width=1.5pt,color=wwwwww] (0.5299991599509355,0.013610294733978491) -- (0.5349991520259443,0.014594148444444732);
\draw[line width=1.5pt,color=wwwwww] (0.5349991520259443,0.014594148444444732) -- (0.5399991441009531,0.015630681335060962);
\draw[line width=1.5pt,color=wwwwww] (0.5399991441009531,0.015630681335060962) -- (0.5449991361759619,0.016720392254155062);
\draw[line width=1.5pt,color=wwwwww] (0.5449991361759619,0.016720392254155062) -- (0.5499991282509706,0.01786345434965023);
\draw[line width=1.5pt,color=wwwwww] (0.5499991282509706,0.01786345434965023) -- (0.5549991203259794,0.019059671737681656);
\draw[line width=1.5pt,color=wwwwww] (0.5549991203259794,0.019059671737681656) -- (0.5599991124009882,0.020308434788478345);
\draw[line width=1.5pt,color=wwwwww] (0.5599991124009882,0.020308434788478345) -- (0.564999104475997,0.02160867452723477);
\draw[line width=1.5pt,color=wwwwww] (0.564999104475997,0.02160867452723477) -- (0.5699990965510058,0.02295881674857448);
\draw[line width=1.5pt,color=wwwwww] (0.5699990965510058,0.02295881674857448) -- (0.5749990886260146,0.024356736548848616);
\draw[line width=1.5pt,color=wwwwww] (0.5749990886260146,0.024356736548848616) -- (0.5799990807010234,0.025799714089300426);
\draw[line width=1.5pt,color=wwwwww] (0.5799990807010234,0.025799714089300426) -- (0.5849990727760321,0.02728439251287536);
\draw[line width=1.5pt,color=wwwwww] (0.5849990727760321,0.02728439251287536) -- (0.5899990648510409,0.028806739044860717);
\draw[line width=1.5pt,color=wwwwww] (0.5899990648510409,0.028806739044860717) -- (0.5949990569260497,0.03036201041018717);
\draw[line width=1.5pt,color=wwwwww] (0.5949990569260497,0.03036201041018717) -- (0.5999990490010585,0.03194472379345559);
\draw[line width=1.5pt,color=wwwwww] (0.5999990490010585,0.03194472379345559) -- (0.6049990410760673,0.03354863464793392);
\draw[line width=1.5pt,color=wwwwww] (0.6049990410760673,0.03354863464793392) -- (0.6099990331510761,0.03516672272170806);
\draw[line width=1.5pt,color=wwwwww] (0.6099990331510761,0.03516672272170806) -- (0.6149990252260849,0.03679118770814491);
\draw[line width=1.5pt,color=wwwwww] (0.6149990252260849,0.03679118770814491) -- (0.6199990173010936,0.03841345593803556);
\draw[line width=1.5pt,color=wwwwww] (0.6199990173010936,0.03841345593803556) -- (0.6249990093761024,0.04002419950684424);
\draw[line width=1.5pt,color=wwwwww] (0.6249990093761024,0.04002419950684424) -- (0.6299990014511112,0.0416133691673563);
\draw[line width=1.5pt,color=wwwwww] (0.6299990014511112,0.0416133691673563) -- (0.63499899352612,0.04317024220922947);
\draw[line width=1.5pt,color=wwwwww] (0.63499899352612,0.04317024220922947) -- (0.6399989856011288,0.04468348638892832);
\draw[line width=1.5pt,color=wwwwww] (0.6399989856011288,0.04468348638892832) -- (0.6449989776761376,0.046141240761938665);
\draw[line width=1.5pt,color=wwwwww] (0.6449989776761376,0.046141240761938665) -- (0.6499989697511463,0.0475312140002036);
\draw[line width=1.5pt,color=wwwwww] (0.6499989697511463,0.0475312140002036) -- (0.6549989618261551,0.0488408004512859);
\draw[line width=1.5pt,color=wwwwww] (0.6549989618261551,0.0488408004512859) -- (0.6599989539011639,0.050057213810839926);
\draw[line width=1.5pt,color=wwwwww] (0.6599989539011639,0.050057213810839926) -- (0.6649989459761727,0.05116763784021485);
\draw[line width=1.5pt,color=wwwwww] (0.6649989459761727,0.05116763784021485) -- (0.6699989380511815,0.052159393069259555);
\draw[line width=1.5pt,color=wwwwww] (0.6699989380511815,0.052159393069259555) -- (0.6749989301261903,0.05302011789111775);
\draw[line width=1.5pt,color=wwwwww] (0.6749989301261903,0.05302011789111775) -- (0.6799989222011991,0.05373796188839514);
\draw[line width=1.5pt,color=wwwwww] (0.6799989222011991,0.05373796188839514) -- (0.6849989142762078,0.0543017886445293);
\draw[line width=1.5pt,color=wwwwww] (0.6849989142762078,0.0543017886445293) -- (0.6899989063512166,0.054701384705982695);
\draw[line width=1.5pt,color=wwwwww] (0.6899989063512166,0.054701384705982695) -- (0.6949988984262254,0.05492767079005253);
\draw[line width=1.5pt,color=wwwwww] (0.6949988984262254,0.05492767079005253) -- (0.6999988905012342,0.054972910802518815);
\draw[line width=1.5pt,color=wwwwww] (0.6999988905012342,0.054972910802518815) -- (0.704998882576243,0.05483091376243221);
\draw[line width=1.5pt,color=wwwwww] (0.704998882576243,0.05483091376243221) -- (0.7099988746512518,0.05449722335602175);
\draw[line width=1.5pt,color=wwwwww] (0.7099988746512518,0.05449722335602175) -- (0.7149988667262606,0.053969289583133954);
\draw[line width=1.5pt,color=wwwwww] (0.7149988667262606,0.053969289583133954) -- (0.7199988588012693,0.053246616845087705);
\draw[line width=1.5pt,color=wwwwww] (0.7199988588012693,0.053246616845087705) -- (0.7249988508762781,0.052330882876785056);
\draw[line width=1.5pt,color=wwwwww] (0.7249988508762781,0.052330882876785056) -- (0.7299988429512869,0.05122602316881701);
\draw[line width=1.5pt,color=wwwwww] (0.7299988429512869,0.05122602316881701) -- (0.7349988350262957,0.04993827597598734);
\draw[line width=1.5pt,color=wwwwww] (0.7349988350262957,0.04993827597598734) -- (0.7399988271013045,0.04847618367367263);
\draw[line width=1.5pt,color=wwwwww] (0.7399988271013045,0.04847618367367263) -- (0.7449988191763133,0.04685054710899291);
\draw[line width=1.5pt,color=wwwwww] (0.7449988191763133,0.04685054710899291) -- (0.7499988112513221,0.045074330690186334);
\draw[line width=1.5pt,color=wwwwww] (0.7499988112513221,0.045074330690186334) -- (0.7549988033263308,0.04316251724802531);
\draw[line width=1.5pt,color=wwwwww] (0.7549988033263308,0.04316251724802531) -- (0.7599987954013396,0.04113191315932845);
\draw[line width=1.5pt,color=wwwwww] (0.7599987954013396,0.04113191315932845) -- (0.7649987874763484,0.039000905804538716);
\draw[line width=1.5pt,color=wwwwww] (0.7649987874763484,0.039000905804538716) -- (0.7699987795513572,0.03678917708641498);
\draw[line width=1.5pt,color=wwwwww] (0.7699987795513572,0.03678917708641498) -- (0.774998771626366,0.034517378406007396);
\draw[line width=1.5pt,color=wwwwww] (0.774998771626366,0.034517378406007396) -- (0.7799987637013748,0.03220677410326315);
\draw[line width=1.5pt,color=wwwwww] (0.7799987637013748,0.03220677410326315) -- (0.7849987557763836,0.029878861850176405);
\draw[line width=1.5pt,color=wwwwww] (0.7849987557763836,0.029878861850176405) -- (0.7899987478513923,0.027554979754433197);
\draw[line width=1.5pt,color=wwwwww] (0.7899987478513923,0.027554979754433197) -- (0.7949987399264011,0.02525591091721059);
\draw[line width=1.5pt,color=wwwwww] (0.7949987399264011,0.02525591091721059) -- (0.7999987320014099,0.023001496818782923);
\draw[line width=1.5pt,color=wwwwww] (0.7999987320014099,0.023001496818782923) -- (0.8049987240764187,0.02081027112212076);
\draw[line width=1.5pt,color=wwwwww] (0.8049987240764187,0.02081027112212076) -- (0.8099987161514275,0.018699125243113356);
\draw[line width=1.5pt,color=wwwwww] (0.8099987161514275,0.018699125243113356) -- (0.8149987082264363,0.016683016313850548);
\draw[line width=1.5pt,color=wwwwww] (0.8149987082264363,0.016683016313850548) -- (0.8199987003014451,0.014774726961333582);
\draw[line width=1.5pt,color=wwwwww] (0.8199987003014451,0.014774726961333582) -- (0.8249986923764538,0.012984684664537241);
\draw[line width=1.5pt,color=wwwwww] (0.8249986923764538,0.012984684664537241) -- (0.8299986844514626,0.011320846390341567);
\draw[line width=1.5pt,color=wwwwww] (0.8299986844514626,0.011320846390341567) -- (0.8349986765264714,0.00978865182295903);
\draw[line width=1.5pt,color=wwwwww] (0.8349986765264714,0.00978865182295903) -- (0.8399986686014802,0.00839104589562747);
\draw[line width=1.5pt,color=wwwwww] (0.8399986686014802,0.00839104589562747) -- (0.844998660676489,0.0071285686301413055);
\draw[line width=1.5pt,color=wwwwww] (0.844998660676489,0.0071285686301413055) -- (0.8499986527514978,0.00599950762589206);
\draw[line width=1.5pt,color=wwwwww] (0.8499986527514978,0.00599950762589206) -- (0.8549986448265066,0.00500010605692987);
\draw[line width=1.5pt,color=wwwwww] (0.8549986448265066,0.00500010605692987) -- (0.8599986369015153,0.004124816871329981);
\draw[line width=1.5pt,color=wwwwww] (0.8599986369015153,0.004124816871329981) -- (0.8649986289765241,0.0033665921670367827);
\draw[line width=1.5pt,color=wwwwww] (0.8649986289765241,0.0033665921670367827) -- (0.8699986210515329,0.0027171955445978286);
\draw[line width=1.5pt,color=wwwwww] (0.8699986210515329,0.0027171955445978286) -- (0.8749986131265417,0.002167524681701134);
\draw[line width=1.5pt,color=wwwwww] (0.8749986131265417,0.002167524681701134) -- (0.8799986052015505,0.001707931471223592);
\draw[line width=1.5pt,color=wwwwww] (0.8799986052015505,0.001707931471223592) -- (0.8849985972765593,0.0013285278061760466);
\draw[line width=1.5pt,color=wwwwww] (0.8849985972765593,0.0013285278061760466) -- (0.8899985893515681,0.0010194664315544243);
\draw[line width=1.5pt,color=wwwwww] (0.8899985893515681,0.0010194664315544243) -- (0.8949985814265768,0);
\draw[line width=1.5pt,color=wwwwww] (0.8949985814265768,0) -- (0.8999985735015856,0);
\draw[line width=1.5pt,color=wwwwww] (0.8999985735015856,0) -- (0.9049985655765944,0);
\draw[line width=1.5pt,color=wwwwww] (0.9049985655765944,0) -- (0.9099985576516032,0);
\draw[line width=1.5pt,color=wwwwww] (0.9099985576516032,0) -- (0.914998549726612,0);
\draw[line width=1.5pt,color=wwwwww] (0.914998549726612,0) -- (0.9199985418016208,0);
\draw[line width=1.5pt,color=wwwwww] (0.9199985418016208,0) -- (0.9249985338766296,0);
\draw[line width=1.5pt,color=wwwwww] (0.9249985338766296,0) -- (0.9299985259516383,0);
\draw[line width=1.5pt,color=wwwwww] (0.9299985259516383,0) -- (0.9349985180266471,0);
\draw[line width=1.5pt,color=wwwwww] (0.9349985180266471,0) -- (0.9399985101016559,0);
\draw[line width=1.5pt,color=wwwwww] (0.9399985101016559,0) -- (0.9449985021766647,0);
\draw[line width=1.5pt,color=wwwwww] (0.9449985021766647,0) -- (0.9499984942516735,0);
\draw[line width=1.5pt,color=wwwwww] (0.9499984942516735,0) -- (0.9549984863266823,0);
\draw[line width=1.5pt,color=wwwwww] (0.9549984863266823,0) -- (0.9599984784016911,0);
\draw[line width=1.5pt,color=wwwwww] (0.9599984784016911,0) -- (0.9649984704766998,0);
\draw[line width=1.5pt,color=wwwwww] (0.9649984704766998,0) -- (0.9699984625517086,0);
\draw[line width=1.5pt,color=wwwwww] (0.9699984625517086,0) -- (0.9749984546267174,0);
\draw[line width=1.5pt,color=wwwwww] (0.9749984546267174,0) -- (0.9799984467017262,0);
\draw[line width=1.5pt,color=wwwwww] (0.9799984467017262,0) -- (0.984998438776735,0);
\draw[line width=1.5pt,color=wwwwww] (0.984998438776735,0) -- (0.9899984308517438,0);
\draw[line width=1.5pt,color=wwwwww] (0.9899984308517438,0) -- (0.9949984229267526,0);
\end{axis}
\end{tikzpicture}
}
    \caption{distribuzione di $ m_n $ data dalla \eqref{eq:distrMagn} per $ n = 165 $ al variare di $ h $ e $ \beta $.}
    \label{fig:distrMagn}
\end{figure}
\fi

\subsection{Valore atteso di un'osservabile generica}
Poiché lo spazio degli stati è finito, una generica osservabile $ \mathcal{O}\colon \Sigma\to\R $ è completamente caratterizzata da $ \card{\Sigma} = 2^n $ valori. Possiamo quindi vedere $ \mathcal{O} $ come una funzione di $ n $ variabili appartenenti $ \{+1,-1\} $. Estendiamo ora tale osservabile ad una funzione $ \tilde{\mathcal{O}}\colon\R^n\to\R $, ad esempio facendo passare un polinomio in $ n $ variabili per i $ 2^n $ punti assegnati. Vogliamo cioè trovare una funzione del tipo
\[ \tilde{\mathcal{O}} (x_1,\ldots,x_n) = \sum_{\abs{i}=0}^{N(n)} \alpha_{i} x^i \]
dove $ i $ è un multi indice e $ x = (x_1, \ldots, x_n) $, in modo tale che
\[ \tilde{\mathcal{O}}(\sigma_1, \ldots, \sigma_n) = \mathcal{O}((\sigma_1,\ldots,\sigma_n)). \]
Volendo studiare $ \lim_{n \to +\infty} \mean{\mathcal{O}}_{P_\beta} $, possiamo quindi ridurci allo studio di generici prodotti della forma $ \sigma_1 \cdots \sigma_k $.

\textcolor{red}{
    Non è chiaro il senso di tutto ciò. Innanzi tutto non è ben definito cosa voglia dire considerare la stessa osservabile in sistemi a diversa taglia, poiché non c'è sempre un modo ovvio (come invece nel caso di $ m_n $) per esprimere l'osservabile come funzione di $ n $.
    Inoltre, sia il grado $ N $ che i coefficienti del polinomio interpolante dipendono, in generale, da $ n $, per cui non è sufficiente conoscere $ \mean{x}_{P_\beta} $ per calcolare $ \lim_{n \to +\infty} \mean{\mathcal{O}}_{P_\beta} $.
}

\begin{thm}
    Se $ \beta < 1 $ oppure $ \beta \geq 1 \wedge h \neq 0 $ vale
    \[ \lim_{n \to +\infty} \mean{\sigma_1, \ldots, \sigma_k}_{P_\beta} = \overline{M}^k(\beta, h). \]
\end{thm}
\begin{proof}
    Si procede per induzione.
\begin{pbase}
    \[ \mean{m_n}_{P_\beta} = \frac{1}{n} \sum_{i=1}^{n}\mean{\sigma_i}_{P_\beta} = \mean{\sigma_1}_{P_\beta} \]
    ma $ \mean{m_n}_{P_\beta} \to_n \overline{M}(\beta,h) $, da cui la tesi.
\end{pbase}
\begin{pind}

    \begin{lemma}\label{lemma:cusumano}
        Siano $ f_n\colon \Sigma_n \to \R $ equilimitate e sia $ \mean{f} \coloneqq \lim_{n \to +\infty}\mean{f_n}_{P_{\beta, n}} $. Allora
        \[ \mean{f_n m_n}_{P_{\beta,n}} \to \overline{M}(\beta,h) \mean{f}. \]
    \end{lemma}
    \begin{proof}
        Per la diseguaglianza di Cauchy-Schwartz si ha
        \begin{align*}
            \abs{ \mean{f_n\cdot \left(m_n-\overline{M}(\beta,h) \right) }_{P_{\beta,n}} }^2  & \leq \mean{f_n^2}_{P_{\beta,n}} \cdot \mean{\left (m_n-\overline{M}(\beta,h)\right )^2}_{P_{\beta,n}} \\
            & \leq K \cdot \left( \mean{m_n^2}_{P_{\beta,n}} + \overline{M}^2(\beta,h) - 2\overline{M}(\beta,h)\mean{m_n}_{P_{\beta,n}} \right) \\
            & \to K \left( \overline{M}^2(\beta,h) + \overline{M}^2(\beta,h) - 2\overline{M}^2(\beta,h) \right) = 0
        \end{align*}
        e quindi
        \begin{align*}
            \lim_{n \to +\infty} \mean{f_n m_n}_{P_{\beta,n}} & = \lim_{n \to +\infty} \mean{f_n\cdot \left(m_n-\overline{M}(\beta,h) \right) }_{P_{\beta,n}} + \lim_{n \to +\infty}\mean{f_n}_{P_{\beta,n}} \overline{M}(\beta,h) \\
            & = \overline{M}(\beta,h)\mean{f}. \qedhere
        \end{align*}
        \begin{align*}
            \mean{\sigma_1 \cdots \sigma_k m_n}_{P_{\beta,n}} & = \frac{1}{n} \sum_{i=1}^{n} \mean{\sigma_1 \cdots \sigma_k \sigma_i}_{P_{\beta,n}} = \frac{1}{n} \sum_{i=1}^k\mean{\sigma_1\cdots\sigma_k\sigma_i}_{P_{\beta,n}} + \frac{1}{n} \sum_{i=k+1}^n \mean{\sigma_1\cdots\sigma_k\sigma_i}_{P_{\beta,n}} \\
            & = \frac{1}{n} \sum_{i=1}^k {\overbrace{\mean{\sigma_1 \cdots \sigma_k}}^{k-1}}_{P_{\beta,n}} + \frac{1}{n} \sum_{i=k+1}^n \mean{\sigma_1 \cdots \sigma_k \sigma_{k+1}}_{P_{\beta,n}} \\
            & = \frac{k}{n} \mean{\sigma_1 \cdots \sigma_k}_{P_{\beta,n}} + \frac{n-k}{n} \mean{\sigma_1 \cdots \sigma_k \sigma_{k+1}}_{P_{\beta,n}}.
        \end{align*}
        Passando al limite, usando il lemma \ref{lemma:cusumano} con $ f(\sigma_1, \ldots, \sigma_n) = \sigma_1 \cdots \sigma_k $ e l'ipotesi induttiva si ottiene la tesi.
    \end{proof}
\end{pind}
\end{proof}

\subsection{Altri metodi per il calcolo dell'energia libera}
\textcolor{red}{Mancante}

\section{Lezione del 10/10/2018}
\subsection{Introduzione}
La Teoria della Misura nasce a inizio '900 per formalizzare la probabilità e per cercare di fondare una teoria dell'integrazione che risulti più efficace di quella di Riemann. Per alcuni sottoinsiemi di $\R$, in particolare per gli intervalli limitati, abbiamo un concetto intuitivo di "misura", ovvero la lunghezza dell'intervallo:
\[\lambda\left([a,b]\right) = b-a.\]
L'obiettivo della teoria della misura è estendere questa nozione ad altri sottoinsiemi di $\R$ in modo coerente, ovvero in modo da rispettare, ad esempio, la proprietà di additività:
\[A \cap B = \emptyset \implies \lambda(A\cup B) = \lambda(A) + \lambda(B)\]
\vspace{0.3cm}

Sia $X$ un insieme, $\mathscr{P}(X)$ l'insieme delle parti di X.

{\bf Definizione: } $\mathcal{A} \subseteq \mathscr{P}(X)$ è detto {\it anello} se è chiuso per intersezione, unione e differenza: $\forall A, B \in \mathcal A \; \; A\cup B, A\cap B, A\backslash B \in \mathcal{A}$.

{\bf Definizione: } $\mathcal{A}$ è un'{\it algebra} se $X\in \mathcal{A}$.

{\bf Definizione: } $\mathcal{F} $ è una {\it $\sigma$-algebra} se è un'algebra ed è chiusa per unione numerabile:
$\forall (A_n)_{n\in\mathbb{N}}\subseteq \mathcal{F}, \underset{n\in \mathbb{N}}{\cup} A_n \in \mathcal{F}$

{\bf Esercizio: } Dimostrare che intersezioni di anelli, algebre, $\sigma$-algebre sono ancora rispettivamente anelli, algebre e $\sigma$-algebre
\vspace{0.3cm}

{\bf Definizione: } Dato $C\subseteq \mathscr{P}(X)$, la $\sigma$-algebra generata da $C$ è:
$$\sigma(C)=\cap \mathcal{F}: \mathcal{F} \text{ è una } \sigma \text{-algebra} \wedge C\subseteq \mathcal{F} $$ 
{\color{red}NOTA: rifare la notazione meglio}

{\bf Definizione: } Sia $\mathcal{A}$ anello e sia data una funzione $\mu:\mathcal{A} \rightarrow[0,+\infty]$. Allora:

1) $\mu$ è {\it additiva} se $\forall A,B \in \mathcal{A}, A\cap B=\varnothing, \mu( A\cup B)=\mu(A)+\mu(B)$.
 
2)$\mu$ è {\it $\sigma$-additiva} se $\forall (A_n)_{n\in\mathbb{N}}\subseteq \mathcal{A}, \forall i\neq j \in \mathbb{N} A_i\cap A_j =\varnothing$, allora chiamato $A=\underset{n\in\mathbb{N}}{\cup}A_n$ si ha $\mu(A)=\sum\limits_{n\in\mathbb{N}} \mu(A_n)$
 
\vspace{0.3cm}

\Oss Se $\mu$ è additiva, $\mu(\varnothing)=\mu(\varnothing)+\mu(\varnothing)$ da cui $\mu(\varnothing)=0$.
\vspace{0.3cm}


{\bf Definizione: } Si chiama {\it spazio di misura} una tripla $(X,\mathcal{F}, \mu)$ dove $X$ è un insieme, $\mathcal{F}\subseteq \mathscr{P}(X)$ una $\sigma$-algebra e $\mu$ è $\sigma$-additiva. $\mu$ viene detta {\it misura} e gli insiemi di $\mathcal{F}$ vengono detti {\it misurabili}.

Se $\mu(X)<+\infty$, allora $\mu$ viene detta {\it misura finita}. Questo implica che $\forall A \in \mathcal{F}, \mu(A) < +\infty$.

Se $\mu(X)=1$ allora $\mu$ viene detta {\it probabilità}.

Se $\exists (A_n)_{n\in\mathbb{N}}\subseteq\mathcal{F}, \underset{n\in\mathbb{N}}{\cup}A_n=X$ e $\forall n \; \mu(A_n) <+\infty$ allora $\mu$ si dice $\sigma$-finita.
\vspace{0.3cm}

{\bf Notazione da probabilisti: } $A_n \uparrow A$ se $\forall n \; A_n \subseteq A_{n+1}$ e $\underset{n\in \mathbb{N}}{\bigcup}A_n =A$.

$A_n \downarrow A$ se $\forall n \; A_n \supseteq A_{n+1}$ e $\underset{n\in \mathbb{N}}{\bigcap}A_n =A$.

Dato $(A_n)_{n\in\mathbb{N}}$, si definiscono $\underset{n\rightarrow +\infty}{\text{liminf}}A_n=\underset{n\in\mathbb{N}}{\bigcap}\underset{k\geq n}{\bigcup}A_k$ l'insieme di tutti gli elementi che appaiono frequentemente tra gli $A_n$, e $\underset{n\rightarrow +\infty}{\text{limsup}}A_n=\underset{n\in\mathbb{N}}{\bigcup}\underset{k\geq n}{\bigcap}A_k$ l'insieme di tutti gli elementi che appaiono definitivamente tra gli $A_n$.
\vspace{0.3cm}


{\bf Esercizio: } Data $\mu$ additiva su $\mathcal{A}$ anello, $\mu$ è $\sigma$-additiva se e soltanto se $\forall A_n \uparrow A$ con $\forall n\; A_n,A \in \mathcal{A}$ $\mu(A)=\lim_{n\rightarrow +\infty} \mu(A_n)$.
\vspace{0.3cm}

{\bf Esercizio: } Data $\mu$ $\sigma$-additiva su $\mathcal{A}$ anello, allora $\forall A_n \downarrow A$ con $\forall n\; A_n,A \in \mathcal{A}$ e $\mu(A_n) < + \infty$ definitivamente, vale $\mu(A)=\lim_{n\rightarrow +\infty} \mu(A_n)$.
\vspace{0.3cm}

{\bf Esercizio: } Dato $(X,\mathcal{F}, \mu)$ spazio di misura, $\forall (A_n)_{n\in\mathbb{N}}\subseteq\mathcal{F}$  si ha $\mu(\underset{n\rightarrow +\infty}{\text{liminf}} A_n) \leq \underset{n\rightarrow +\infty}{\text{liminf}} \mu (A_n)$. Inoltre se $\mu(\underset{n\in\mathbb{N}}{\bigcup} A_n) <+\infty)$, allora $\mu(\underset{n\rightarrow +\infty}{\text{limsup}} A_n) \leq \underset{n\rightarrow +\infty}{\text{limsup}} \mu (A_n)$
 \vspace{0.3cm}

{\bf Esercizio: } {\it (Borel-Cantelli)} Se $\sum\limits_{n\in\mathbb{N}} \mu(A_n) < +\infty$, allora $\mu(\underset{n\rightarrow +\infty}{\text{limsup}} A_n)=0$.

\vspace{0.5cm}

{\bf Esempio: } Dato $X$ insieme, $\mathcal{F}\subseteq \mathscr{P}(X)$ una $\sigma$-algebra, sono esempi di misure:

1) {\it Misura finita} Se $A$ è finito allora $\mu(A)=|A|$, altrimenti $\mu(A)=+\infty$

2) {\it Misura di Dirac} Dato $x\in X$, se $x\in A$ allora $\delta_x(A)=1$, altrimenti $\delta_x(A)=0$.

3) {\it $\sigma$-algebra dei singoletti} È quella generata dall'insieme di tutti i singoletti di $X$,  $\mathcal{F}=\sigma(\{ \{x\} | x\in X\}) =\{A\subseteq X | A \text{ o } X\backslash A \text{ è finito o numerabile} \}$

\vspace{0.4cm}
Prendendo $X=\mathbb{N}, \mathcal{F} =\mathscr{P}(\mathbb{N})$, si possono definire le distribuzioni di probabilità discrete e le relative misure. Data una funzione di probabilità $p: \mathbb{N} \rightarrow [0,+\infty)$ che soddisfa $\sum\limits_{n\in\mathbb{N}} p(n)=1$, questa induce una misura $\mu$ definita come $\mu(A)=\sum\limits_{n\in A} p(n)$. Esempi sono:
 \begin{itemize}
  \item Distribuzione di Bernoulli di parametro $0<q<1$: $p(0)=1-q$, $p(1)=q$, $p(n)=0$ per $n\geq 2$.
 
\item Distribuzione Binomiale di parametro $0<q<1$: $p(k)=\displaystyle\binom{n}{k} q^k (1-q)^{n-k}$.
 
 \item Distribuzione di Poisson di parametro $\lambda\in\mathbb{R}$: $p(n)=\dfrac{\lambda^ne^{-\lambda}}{n!}$.
\end{itemize}

 \vspace{0.3cm}
 
 {\bf Esempio: } misura $\mu$ non $\sigma$-finita. Prendo $X$ un insieme più che numerabile e la sua $\sigma$-algebra dei singoletti, e una misura $\mu$ per cui se $A$ è finito o numerabile $\mu(A)=0$ e se $A$ è più che numerabile $\mu(A)=+\infty$. Allora per ogni sequenza $(A_n)_{n\in\mathbb{N}}\subseteq\mathcal{F}, \underset{n\in\mathbb{N}}{\cup}A_n=X$, almeno uno degli $A_n$ deve essere più che numerabile, quindi la sua misura è $+\infty$.
 
 \vspace{0.3cm}
 
 \Prop Sia $(X,\mathcal{F}, \mu)$ uno spazio di misura e sia $\mathcal{E} \subseteq \mathcal{F}$ una sotto-$\sigma$-algebra, allora $\mu_{|\mathcal{E}}$ è una misura di $(X,\mathcal{E})$.
 
 \Dim: bohboh
 
{\bf Definizione: } Se $Y\subseteq X$, la {\it sotto $sigma$-algebra di sottoinsiemi} di $Y$ è $\mathcal{E}_Y=\{A\in \mathcal{F} | A \subseteq Y\}$.

{\bf Esercizio: } $\mu_{|Y}(A)=\mu_{|\mathcal{E}_Y}(A)$. Inoltre se $Y$ è misurabile (cioé $Y\in \mathcal{F}$), allora questo è uguale a $\mu(A\cap Y)$.

{\bf Definizione: } Dato $(X,\mathcal{F}, \mu)$ uno spazio di misura, se per $A\in \mathcal{F}$ vale $\mu(A)=0$ allora $A$ si dice {\it trascurabile}. L'insieme $R=\bigcup \{A\in \mathcal{F} | \mu(A)=0\}$ è l'insieme di tutti i punti trascurabili. Se una certa proprietà vale $\forall x \in X\backslash R$ si dice che vale {\it quasi ovunque}. 

\vspace{0.3cm}
\Teo{Coincidenza delle misure}Sia $X$ insieme, $\mathcal{F}$ una $\sigma$-algebra e $\mu, \nu$ due misure su $(X,\mathcal{F})$. Sia $K\subseteq \mathcal{F}, K\neq \varnothing$ con le seguenti proprietà:

1) $\forall A \in K, \mu(A)=\nu(A)$, cioé $\mu$ e $\nu$ coincidono su tutto $K$.

2) $K$ è chiuso per intersezioni

3) $\sigma(K)=\mathcal{F}$

4) $\exists (X_n)_{n\in \mathbb{N}}\subseteq K, \; X_n \uparrow X$ e $\forall n\; \mu(X_n)=\nu(X_n)<+\infty$

Allora $\mu$ e $\nu$ coincidono su tutto $\mathcal{F}$

\Dim VOLEEEEVI

{\bf Esercizio: } Trovare un controesempio al teorema se non vale il punto 4), ovvero $X, \mathcal{F} \supseteq K$ e due misure $\mu,\nu$ che coincidano su $K$ ma non su tutto $\mathcal{F}$.

\vspace{0.4cm}

\Teo{Estensione di Caratheodory}Sia $X$ insieme e $\mathcal{A}\subseteq \mathcal{P}(X)$ un anello. Se $\mu$ è $\sigma$-additiva su $\mathcal{A}$, esiste un estensione di $\mu$ ad una misura $\mu_1$ su $\sigma(\mathcal{A})$. Inoltre se l'estensione $\mu_1$ è $\sigma$-finita, allora è unica.

\Dim Banale

\vspace{0.3cm}

{\bf Esempio: } $X=\mathbb{R}$, $\mathcal{A}$ è l'anello generato dagli intervalli $[a.b)$, $\mu=$roba

Per il teorema di Caratheodory, esiste una msiura $\lambda$ su $\sigma(\mathcal{A})$=Boreliani che estende $\mu$.

 
 
 {\bf Definizione: } $(X,\tau)$ spazio topologico, $\sigma(\tau)$ è la $\sigma$-algebra dei Boreliani $\mathcal{B}$.
 
 Nel caso $(X,\tau)=(\mathbb{R},\tau_{\text{eu}})$, $| \mathcal{B} | = | 2^{\mathbb{N}} |$
 
 {\bf Esercizio: } Dato $(\mathbb{R}^d, \mathcal{B}, \mu)$ spazio di misura sulla $\sigma$-algebra dei Boreliani, con $\mu$ misura finita sui compatti e invariante per traslazione (ovvero $\forall x \in \mathbb{R}^d \; \forall A \in \mathcal{B} \; \; \mu(x+A)=\mu(A)$), allora $\exists c>0 \mu=c\lambda$ dove $\lambda$ è la misura di Lebesgue.
 
{\bf Esercizio: } È possibile non solo traslare lo spazio, ma anche trasformarlo tramite applicazioni lineari. Sia $\mathbb{T}\in GL(d,\mathbb{R}), \phi:\mathbb{R}^d \rightarrow \mathbb{R}^d\phi(x)=\mathbb{T}x$. Allora $\forall A \in \mathcal{B}$, $\lambda(\phi(A))=|\det \mathbb{T} | \lambda(A)$

{\it Hint:} 1) $\mu([0,1]^d)=c\Rightarrow \mu([0, \frac{1}{k}]^d)=\frac{c}{k^d}\Rightarrow \mu=c\lambda$ su $\mathcal{A}$, usare il teorema sopra {\color{red}[NOTA: quale dei due?]}

2) Se $\mathbb{T} \in Gl(n,\mathbb{R})$, allora esistono $P_1, P_2$ ortogonali e $D$ diagonale tali che $\mathbb{T}=P_1DP_2$.

\Teo{Completamento di Misura}

{\bf Definizione: } \(\mathcal{A}\subseteq \mathscr{P}(X)\) è un $\sigma$-ideale di \(\mathscr{P}(X)\) se $\forall B \in \mathcal{A}, C \subseteq B$ implica $C\in\mathcal{A}$.

{\bf Definizione: } \((X,\mathcal{F},\mu)\), la misura \(\mu\) è {\it completa} se \textcolor{red}{Nstort [Fstorta??]} è un $\sigma$-ideale di $\mathscr{P}(X)$.

Dunque la misura sui Boreliani $(\mathbb{R}^d, \mathcal{B}, \lambda)$ si estende a $(\mathbb{R}^d, \mathcal{M}, \lambda_1)$ con $\lambda_1$ completa su \(\mathcal{M}\) insieme dei Lebesgue-misurabili.

{\bf Esercizio: } $|\mathcal{M}|=|2^{\mathbb{R}}|=|\mathscr{\mathbb{R}^d}|$. {\it Hint:} Prendere i sottoinsiemi dell'insieme di Cantor.

{\bf Esercizio: } * Controesempio di Vitali: Esiste $V\subseteq \mathbb{R}$, $V\not\in \mathcal{M}$, ovvero non è misurabile secondo Lebesgue.

{\bf Esercizio: } ** Dimostrare che $| \mathcal{B} | = | 2^{\mathbb{N}} |$.

{\bf Esercizio: } *** Esiste un estensione propria di $(\mathbb{R}^d, \mathcal{M}, \lambda)$ invariante per traslazione
 
{\bf Esercizio: } ? In ZF senza l'assioma della scelta, esiste un modello in cui $\mathcal{M}=\mathscr{P}(\mathbb{R}^d)$, ovvero tutti gli insiemi sono Lebesgue-misurabili.
\vspace{0.4cm}

{\large\color{red} FUNZIONI MISURABILI}

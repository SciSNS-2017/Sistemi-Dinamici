\section{Lezione del 10/10/2018 [Grotto]}
\subsection{Introduzione}
La Teoria della Misura nasce a inizio '900 per formalizzare la probabilità e per cercare di fondare una teoria dell'integrazione che risulti più efficace di quella di Riemann. Per alcuni sottoinsiemi di $\R$, in particolare per gli intervalli limitati, abbiamo un concetto intuitivo di ``misura'', ovvero la lunghezza dell'intervallo:
\[ \lambda\left([a,b]\right) = b-a. \]
L'obiettivo della teoria della misura è estendere questa nozione ad altri sottoinsiemi di $\R$ in modo coerente, ovvero in modo da rispettare, ad esempio, la proprietà di additività:
\[ A \cap B = \emptyset \Rightarrow \lambda(A\cup B) = \lambda(A) + \lambda(B) \]

Sia $X$ un insieme, $\mathscr{P}(X)$ l'insieme delle parti di X.

\begin{definition}[anello]
	$\mathcal{A} \subseteq \mathscr{P}(X)$ è detto anello se è chiuso per intersezione, unione e differenza:
	\[ \forall A, B \in \mathcal A \quad A\cup B,\ A\cap B,\ A\setminus B \in \mathcal{A} \]
\end{definition}

\begin{definition}[algebra]
	$\mathcal{A}$ è un'algebra se è un anello e $X\in \mathcal{A}$.
\end{definition}

\begin{definition}[$ \sigma $-algebra]
	$\mathcal{F} $ è una $\sigma$-algebra se è un'algebra ed è chiusa per unione numerabile:
	\[ \forall (A_n)_{n\in\N}\subseteq \mathcal{F},\ \bigcup_{n\in \N} A_n \in \mathcal{F} \]
\end{definition}

\begin{exercise}
	Dimostrare che intersezioni di anelli, algebre, $\sigma$-algebre sono ancora rispettivamente anelli, algebre e $\sigma$-algebre.
\end{exercise}

\begin{definition}[$ \sigma $-algebra generata]
	Dato $C\subseteq \mathscr{P}(X)$, la $\sigma$-algebra generata da $C$ è:
	\[ \sigma(C) \coloneqq \left\{\bigcap \mathcal{F}: \mathcal{F} \text{ è una $\sigma$-algebra} \wedge C\subseteq \mathcal{F} \right\} \]
\end{definition}

\begin{definition}
	Sia $\mathcal{A}$ anello e sia data una funzione $\mu:\mathcal{A} \rightarrow[0,+\infty]$. Allora:
	\begin{itemize}
		\item $\mu$ è \emph{additiva} se $\forall A,B \in \mathcal{A}, A\cap B=\varnothing,\ \mu( A\cup B)=\mu(A)+\mu(B)$;
		\item $\mu$ è \emph{$\sigma$-additiva} se $\forall (A_n)_{n\in\N}\subseteq \mathcal{A}, \forall i\neq j \in \N \ A_i\cap A_j =\varnothing$, allora chiamato $ A=\bigcup\limits_{n\in\N}A_n$ si ha $\mu(A)=\sum\limits_{n\in\N} \mu(A_n)$.
	\end{itemize}
\end{definition}

\begin{oss}
	Se $\mu$ è additiva, $\mu(\varnothing)=\mu(\varnothing)+\mu(\varnothing)$ da cui $\mu(\varnothing)=0$\footnote{Potrebbe essere $ \mu(\emptyset) = +\infty $ ma allora, per additività, ogni insieme avrebbe misura infinita}.
\end{oss}

\subsection{Spazi di misura}
\begin{definition}[spazio di misura]
	Si chiama spazio di misura una terna $(X,\mathcal{F}, \mu)$ dove $X$ è un insieme, ${\mathcal{F}\subseteq \mathscr{P}(X)}$ una $\sigma$-algebra e $\mu\colon\mathcal{F} \to [0,+\infty]$ è $\sigma$-additiva. $\mu$ viene detta \emph{misura} e gli insiemi di $\mathcal{F}$ vengono detti \emph{misurabili}.
	Inoltre:
	\begin{itemize}
		\item Se $\mu(X)<+\infty$, allora $\mu$ viene detta \emph{misura finita}. Questo implica che $\forall A \in \mathcal{F}, {\mu(A) < +\infty} $;
		\item Se $\mu(X)=1$ allora $\mu$ viene detta \emph{probabilità};
		\item Se $\exists (A_n)_{n\in\N}\subseteq\mathcal{F} : \bigcup\limits_{n\in\N}A_n=X$ e $\forall n \in\N \ \mu(A_n) <+\infty$, allora $\mu$ si dice \emph{$\sigma$-finita}.
	\end{itemize}
\end{definition}
Introduciamo le seguenti notazioni:
\[ A_n \uparrow   A \text{ se } \forall n \in\N \ A_n \subseteq A_{n+1} \text{ e } \bigcup\limits_{n\in \N}A_n = A \]
\[ A_n \downarrow A \text{ se } \forall n \in\N \ A_n \supseteq A_{n+1} \text{ e } \bigcap\limits_{n\in \N}A_n = A \]
Dato $(A_n)_{n\in\N}$, si definiscono $\underset{n\to+\infty}{\liminf}A_n \coloneqq\underset{n\in\N}{\bigcap}\underset{k\geq n}{\bigcup}A_k$ l'insieme di tutti gli elementi che appaiono frequentemente tra gli $A_n$, e $\underset{n\to +\infty}{\limsup}A_n\coloneqq\underset{n\in\N}{\bigcup}\underset{k\geq n}{\bigcap}A_k$ l'insieme di tutti gli elementi che appaiono definitivamente tra gli $A_n$.

\begin{exercise}
	Data $\mu$ additiva su $\mathcal{A}$ anello, $\mu$ è $\sigma$-additiva se e soltanto se $\forall A_n \uparrow A$ con $\forall n\; A_n,A \in \mathcal{A}$ $\mu(A)=\lim_{n\to +\infty} \mu(A_n)$.
\end{exercise}
\begin{exercise}
	Data $\mu$ $\sigma$-additiva su $\mathcal{A}$ anello, allora $\forall A_n \downarrow A$ con $\forall n\; A_n,A \in \mathcal{A}$ e $\mu(A_n) < + \infty$ definitivamente, vale $\mu(A)=\lim_{n\rightarrow +\infty} \mu(A_n)$.
\end{exercise}
\begin{exercise}
	Dato $(X,\mathcal{F}, \mu)$ spazio di misura, $\forall (A_n)_{n\in\N}\subseteq\mathcal{F}$  si ha $\mu\left(\underset{n\rightarrow +\infty}{\liminf} A_n\right) \leq \underset{n\rightarrow +\infty}{\liminf} \mu (A_n)$. Inoltre se $\mu\left(\underset{n\in\N}{\bigcup} A_n \right) < +\infty$, allora $\mu\left(\underset{n\rightarrow +\infty}{\limsup} A_n\right) \leq \underset{n\rightarrow +\infty}{\limsup} \mu (A_n)$.
\end{exercise}
\begin{exercise}[Borel-Cantelli]
	Sia $(X,\mathcal{F}, \mu)$ spazio di misura, e $ (A_n)_{n\in\N}\subseteq\mathcal{F}$. Allora 
    \[
        \sum_{n\in\N} \mu(A_n) < +\infty \quad \Rightarrow \quad   \mu\left({\limsup_{n\rightarrow +\infty}} A_n\right)=0.
    \]
\end{exercise}
\begin{example}
	Dato $X$ insieme, $\mathcal{F}\subseteq \mathscr{P}(X)$ una $\sigma$-algebra, sono esempi di misure:
	\begin{itemize}
		\item \emph{Misura finita}: se $A \subseteq X$ è finito allora $\mu(A)=\card{A}$, altrimenti $\mu(A)=+\infty$;
		\item \emph{Misura di Dirac}: Fissato $x\in X$, poniamo $ \delta_x(A) = 1 $ se $x\in A$ e $\delta_x(A)=0$ altrimenti;
		\item Definita la \emph{$\sigma$-algebra dei singoletti} di $ X $ come:
		\[ \mathcal{F} \coloneqq \sigma\left( \left\{ \{x\} : x\in X \right\} \right) = \left\{ A\subseteq X : A \text{ o } X\setminus A \text{ è al più numerabile} \right\} \]
		Supponiamo $ X $ sia più che numerabile. Poniamo: $ \mu(A) = 0 $ se $ A $ è al massimo numerabile, $ \mu(A) = +\infty $ altrimenti. 
		Allora per ogni sequenza $(A_n)_{n\in\N}\subseteq\mathcal{F},\ \bigcup_{n\in\N} A_n=X$, almeno uno degli $A_n$ deve essere più che numerabile, quindi la sua misura è $+\infty$ e pertanto $ \mu $ è una misura non $ \sigma $-finita.
	\end{itemize}
\end{example}

\begin{example}[misura di probabilità discreta]
	Prendendo $X=\N,\ \mathcal{F} =\mathscr{P}(\N)$, si possono definire le distribuzioni di probabilità discrete e le relative misure. Data una funzione di probabilità $p: \N \rightarrow [0,+\infty)$ che soddisfi $\sum_{n\in\N} p(n)=1$, questa induce una misura $\mu$ definita come $\mu(A)\coloneqq\sum_{n\in A} p(n)$. Esempi sono:
	\begin{itemize}
	\item Distribuzione di Bernoulli di parametro $0<q<1$: $p(0)=1-q$, $p(1)=q$, $p(n)=0$ per $n\geq 2$;
	\item Distribuzione Binomiale di parametro $0<q<1$: $\displaystyle p(k)=\binom{n}{k} q^k (1-q)^{n-k}$;
	\item Distribuzione di Poisson di parametro $\lambda\in\R$: $\displaystyle p(n)=\frac{\lambda^ne^{-\lambda}}{n!}$.
	\end{itemize}
\end{example}

\begin{proposition}[restrizione di una misura]
	Sia $(X,\mathcal{F}, \mu)$ uno spazio di misura e sia $\mathcal{E} \subseteq \mathcal{F}$ una sotto-$\sigma$-algebra, allora $\mu\lvert_{\mathcal{E}}$ è una misura di $(X,\mathcal{E})$.
\end{proposition}

\begin{definition}
	Se $Y\subseteq X$, la \emph{sotto $\sigma$-algebra di sottoinsiemi} di $Y$ è $\mathcal{E}_Y=\{A\in \mathcal{F} : A \subseteq Y\}$.
\end{definition}

\begin{exercise}
	Mostrare che $\mu\lvert_{Y}(A)=\mu\lvert_{\mathcal{E}_Y}(A)$. Mostrare inoltre che se $Y$ è misurabile (cioè $Y\in \mathcal{F}$), allora questo è uguale a $\mu(A\cap Y)$.
\end{exercise}

\begin{definition}[insieme trascurabile]
	Dato $(X,\mathcal{F}, \mu)$ uno spazio di misura, se per $A\in \mathcal{F}$ vale $\mu(A)=0$ allora $A$ si dice \emph{trascurabile}. L'insieme $\mathscr{N}\coloneqq\left\{A\in \mathcal{F} : \mu(A)=0 \right\}$ è l'insieme di tutti i punti trascurabili. \\\
    Se una certa proprietà vale $\forall x \in X \setminus \mathscr{N}$ si dice che vale \emph{quasi ovunque} (o \emph{quasi certamente} se $ \mu $ è una probabilità).
\end{definition}

\begin{thm}[coincidenza delle misure] \label{thm:coincidenza}
	Sia $X$ insieme, $\mathcal{F}$ una $\sigma$-algebra e $\mu, \nu$ due misure su $(X,\mathcal{F})$. Sia $K\subseteq \mathcal{F}, K\neq \varnothing$ con le seguenti proprietà:
	\begin{enumerate}[label=(\roman*)]
	\item $\forall A \in K, \ \mu(A)=\nu(A)$, cioè $\mu$ e $\nu$ coincidono su tutto $K$;
	\item $K$ è chiuso per intersezioni;
	\item $\sigma(K)=\mathcal{F}$;
	\item\label{th:coin:cusumano} $\exists (X_n)_{n\in \N}\subseteq K $ tale che $ X_n \uparrow X$ e $\forall n \in \N, \ \mu(X_n)=\nu(X_n)<+\infty$.
	\end{enumerate}
Allora $\mu$ e $\nu$ coincidono su tutto $\mathcal{F}$.
\end{thm}

\begin{exercise}
	Trovare un controesempio al teorema se non vale il punto \ref{th:coin:cusumano}, ovvero due spazi di misura $(X, \mathcal{F}, \mu)$ e $ (X, \mathcal{F}, \nu) $ e un $ K \subseteq \mathcal{F} $ tale che le misure $\mu,\nu$ che coincidano su $K$ ma non su tutto $\mathcal{F}$.
\end{exercise}

\begin{thm}[estensione di Carathéodory]
	Sia $X$ insieme e $\mathcal{A}\subseteq \mathscr{P}(X)$ un anello. Se ${\mu\colon\mathcal{A}\to[0,+\infty]}$ è $\sigma$-additiva, esiste un'estensione di $\mu$ ad una misura $\tilde\mu$ su $\sigma(\mathcal{A})$. Inoltre se l'estensione $\tilde\mu$ è $\sigma$-finita, allora essa è unica.
\end{thm}

\begin{example}
	Siano $ X = \R $ con l'usuale topologia euclidea $ \tau_{\mathrm{eu}} $ e $ \mathcal{A} $ l'anello generato dagli intervalli della forma $[a,b)$. Sia:
	\[ \mu\left( \bigcup_{i=1}^N [a_i,b_i) \right) \coloneqq \sum_{i=1}^N (b_i-a_i) \]
	Per il teorema di Carathéodory, esiste una misura $\lambda$ su $\sigma(\mathcal{A})=\sigma(\tau_{\mathrm{eu}})$ che estende $\mu$.
\end{example}

\begin{definition}[boreliani]
    Dato uno spazio topologico $ (X, \tau) $, la $ \sigma $-algebra dei sottoinsiemi boreliani di $ X $ è la $ \sigma $-algebra generata dalla topologia, $ \mathcal{B} \coloneqq \sigma(\tau) $.
\end{definition}

\begin{exercise}
	Preso $ (\R,\tau_{\text{eu}}) $, $\card{\mathcal{B}} = |2^{\N}| $.
	\textcolor{red}{Hint: facile per induzione transfinita.}
\end{exercise} 
\begin{exercise}
	Dato $(\R^d, \mathcal{B}, \mu)$ spazio di misura sulla $\sigma$-algebra dei Boreliani, con $\mu$ misura finita sui compatti e invariante per traslazione (ovvero $\forall x \in \R^d, \ \forall A \in \mathcal{B}, \ \mu(x+A)=\mu(A)$), allora $\exists c > 0 : \mu=c\lambda$ dove $\lambda$ è la \textcolor{red}{misura di Lebesgue}. \\
    Hint: $\mu ([0,1]^d)=c \Rightarrow \mu ([0, 1/k]^d) = c/k^d \Rightarrow \mu = c\lambda$ su $\mathcal{A}$; concludere con il Teorema \ref{thm:coincidenza}.
\end{exercise}
\begin{exercise}
	È possibile non solo traslare lo spazio, ma anche trasformarlo tramite applicazioni lineari. Sia $T\in\mathrm{GL}(d,\R),\ \phi\colon\R^d \to \R^d,\ \phi(x)=Tx$. Allora $\forall A \in \mathcal{B}$ vale
    \[
        \lambda(\phi(A))=\lambda(A)\abs{\det T}. 
    \]
	Hint: Se $\mathbb{T} \in \mathrm{GL}(n,\R)$, allora esistono $P_1, P_2$ ortogonali e $D$ diagonale tali che $\mathbb{T}=P_1DP_2$.
\end{exercise}

\begin{definition}[$ \sigma $-ideale]
	Sia $ (X,\mathcal{F},\mu) $ uno spazio di misura. Diciamo che $ I\subseteq\mathcal{F} $ è un \emph{$ \sigma $-ideale} di $ \mathcal{F} $ se sono soddisfatte le seguenti proprietà:
	\begin{enumerate}[label=(\roman*)]
		\item $ \emptyset\in I $;
		\item $ \forall N\in I,\ A\subseteq N \Rightarrow A \in I $;
		\item $ \left\{A_n \right\}_{n\in\mathbb{N}} \subseteq I \Rightarrow \bigcup\limits_{n\in\mathbb{N}} A_n\in I $.
	\end{enumerate}
\end{definition}

\begin{definition}[misura completa]
	Una misura è completa se l'insieme degli insiemi trascurabili $ \mathscr{N} $ è $ \sigma $-ideale di $ \mathscr{P}(X) $.
\end{definition}

\begin{thm}[completamento di misura]
	Sia $ (X,\mathcal{F},\mu) $ uno spazio di misura. Allora $ \mu $ si estende a una misura completa massimale su:
	\[ \overline{\mathcal{F}} \coloneqq \left\{ A = E \cup N : E\in \mathcal{F},\ N\subseteq N' \in \mathscr{N} \right\} \]
	\textcolor{red}{Siamo sicuri?}
\end{thm}
Dunque la misura sui Boreliani $\big(\R^d, \mathcal{B}, \lambda\big)$ si estende a $\big(\R^d, \mathcal{M}, \tilde\lambda \big)$ con $\tilde\lambda$ completa detta \emph{misura di Lebesgue} su \(\mathcal{M}\) insieme dei Lebesgue-misurabili.

\begin{exercise}
	$\card{\mathcal{M}}=|2^{\R}|$.
	Hint: Prendere i sottoinsiemi dell'insieme di Cantor. \textcolor{red}{Difficile per induzione transfinita.}
\end{exercise}

\begin{exercise}[insieme di Vitali]
	Esiste $V\subseteq \R$, $V\not\in \mathcal{M}$, ovvero non misurabile secondo Lebesgue.
\end{exercise}

\begin{exercise}
	Esiste un'estensione propria di $(\R^d, \mathcal{M}, \lambda)$ invariante per traslazione.
\end{exercise}

\begin{exercise}
	In ZF senza l'assioma della scelta, esiste un modello in cui $\mathcal{M}=\mathscr{P}(\R^d)$, ovvero tutti gli insiemi sono Lebesgue-misurabili.
\end{exercise}

\subsection{Funzioni misurabili}
\begin{definition}[funzione misurabile]
	Siano $ (X,\mathcal{E},\mu) $ e $ (Y,\mathcal{F},\nu) $ spazi di misura e sia \linebreak$ f\colon X\to Y $. Diciamo che $ f $ è $ \mathcal{E} $-misurabile se le controimmagini di misurabili sono misurabili:
	\[ 
        \forall F \in \mathcal{F}, \ f^{-1}(F) \in \mathcal{E}.
    \]
\end{definition}

\begin{oss}
    Per verificare che una funzione è misurabile è sufficiente verificarlo un una classe $ K $ di insiemi la cui sigma algebra generata è $ \mathcal{F} $.
\end{oss}

\begin{proposition}
    Siano $ (X,\mathcal{E},\mu) $, $ (Y,\mathcal{F},\nu) $ e $ (Z, \mathcal{G}, \xi) $ spazi di misura. Se $ f \colon X \to Y $ e $ g \colon Y \to Z $ sono misurabili allora $ g \circ f \colon X \to Z $ è misurabile.
\end{proposition}

\begin{exercise}
	La funzione indicatrice $ \chi_A \colon X \to \{0,1\} $ è misurabile se \textcolor{red}{e solo se} $ A $ è misurabile.
\end{exercise}

\begin{oss}
	La funzione vuota è misurabile.
\end{oss}

\begin{definition}[step function]
	Sia $ \phi\colon (X,\mathcal{E}) \to (\R \cup \{\pm\infty\}, \mathcal{B}) $. Diciamo che $ \phi $ è una step function o funzione semplice se si può scrivere come combinazione lineare di indicatrici di insiemi misurabili:
	\[ 
        \phi(x) = \sum_{i=0}^N a_i \chi_{A_i}(x) 
    \]
	con $ a_i \in (-\infty, +\infty] $ e $ A_i \in \mathcal{E} $. \textcolor{red}{Non sono sicuro sugli infiniti.}
\end{definition}

\begin{oss}
	Le step functions sono misurabili.
\end{oss}

\begin{proposition}
	Sia $ f\colon (X,\mathcal{E}) \to [0, +\infty] $ misurabile. Allora
	\[ \exists (\phi_n)_{n\in\N},\ \phi_n \text{ semplici} : \phi_n \nearrow f \]
	cioè ogni funzione misurabile $ f $ si può approssimare con una successione di funzioni semplici $ \phi_n $ crescenti in $ n $ e che tendono puntualmente a $ f $.
	Inoltre, se $ f\colon (X,\mathcal{E}) \to [0,+\infty) $, si può avere convergenza uniforme $ \phi_n \touf f $.
\end{proposition}
\begin{proof}
	Poniamo:
	\[ \phi_n (x) \coloneqq
    \begin{cases}
			\frac{k}{2^n} & \text{se } x \in f^{-1}\left( \left[\frac{k}{2^n}, \frac{k+1}{2^n} \right)  \right) \quad\text{per } k=0,\cdots,n2^n-1\\
			n             & \text{se } x \in f^{-1}\left( [n, +\infty] \right)
		\end{cases}
	 \]
	Questa è una funzione semplice ben definita in quanto è costante su insiemi misurabili (le controimmagini tramite $ f $), $ \phi_n $ è non decrescente ed ovviamente vale $ \phi_n(x) \leq f(x)\ \forall x\in X $.
	Fissiamo $ x\in X $. Se $ f(x)<+\infty $ fissiamo $ \epsilon > 0 $ e prendiamo $ \bar n > \max\{ f(x), -\log_2(\epsilon) \} $. Allora $ \exists k : x\in f^{-1}\left( \left[\dfrac{k}{2^{\bar n} }, \dfrac{k+1}{2^{\bar n} } \right) \right) $ e quindi $ \forall n\ge \bar{n}, \ \abs{\phi_n(x) - f(x)} \leq \abs{\phi_{\bar{n}}(x) - f(x)} \leq \dfrac{1}{2^{\bar n}} < \epsilon $. \\
	Se invece $ f(x) = +\infty $ allora $ \forall n \in \N, \ x\in f^{-1}\left( [n, +\infty] \right) $ e quindi $ \phi_n(x) = n \to +\infty = f(x) $. Questo dimostra la convergenza puntuale.

	\textcolor{red}{Convergenza uniforme?}
\end{proof}
\begin{proposition}[criterio di Doob]
	Data $ f \colon (X, \mathcal{E}) \to (Y, \mathcal{F}) $ misurabile e $ g \colon (X, \mathcal{E}) \to (\R, \mathcal{B}) $ $ f $-misurabile, esiste $ h \colon (Y, \mathcal{F}) \to (\R, \mathcal{B}) $ misurabile tale che $ g=h\circ f $. 
	\begin{center}
		\begin{tikzcd}
			(X,\mathcal{E}) \arrow[r, "g \text{ mis.}"] \arrow[d, "f \text{ mis.}" left] & (\R,\mathcal{B}) \\
			(Y,\mathcal{F}) \arrow[ru, dashed, "\exists h \text{ mis.}" below right]
		\end{tikzcd}
	\end{center}
\end{proposition}
\begin{proof}
	\textcolor{red}{content}
\end{proof}

\begin{definition}[misura immagine]
	Sia $ f\colon (X,\mathcal{E}) \to (Y,\mathcal{F}) $ e sia $ \mu $ una misura su $ (X,\mathcal{E}) $. Chiamiamo misura immagine di $ \mu $ tramite $ f $ la misura su $ (Y, \mathcal{F}) $ così definita:
	\begin{align*}
		f_\sharp \mu\colon \mathcal{F} & \to [0,+\infty]                      \\
		A                              & \mapsto \mu \left( f^{-1}(A) \right)
	\end{align*}
	Se $ (X,\mathcal{E}) = (Y,\mathcal{F}) $ e $ f_\sharp\mu = \mu $ diciamo che $ \mu $ è una misura \emph{$ f $-invariante}.
\end{definition}

\begin{example}
	Sia $ X = [0,1] $ con la misura di Lebesgue $ \lambda $ (che in questo caso è una probabilità). Sia $ f\colon X\to X $.
	\begin{itemize}
		\item Se $ f(x) \equiv 0 $, allora $ f_\sharp \lambda = \delta_0 $.
		\item \emph{Lancio della moneta}: se $ f(x) = \chi_{[0,\frac{1}{2}]}(x) $, allora $ f_\sharp \lambda (A) =
		\begin{cases}
			1	& \text{se } 0\in A \wedge 1 \in A \\
			1/2 & \text{se } 0\in A \veebar 1\in A \\
			0			& \text{altrimenti}
		\end{cases} $
	\end{itemize}
\end{example}

\begin{example}
	Sia $ \tau_x\colon \R^d\to\R^d $ la traslazione di $ x\in\R^d $: $ \tau_x(y) \coloneqq x+y $. Allora $ (\tau_x)_\sharp\lambda = \lambda $, cioè la misura di Lebesgue è invariante per traslazioni. \\
    Consideriamo invece $ \phi_A\colon\R^d\to\R^d $ definita come $ \phi(x) \coloneqq Ax $ con $ A \in\mathrm{GL}(d,\R) $. Allora 
    \[
        (\phi_A)_\sharp\lambda = \abs{\det A}^{-1} \cdot \lambda.
    \] 
\end{example}

\begin{definition}[misura invariante]
    Sia $ (X, \mathcal{F}, \mu) $ uno spazio di misura e $ f \colon X \to X $ misurabile. La misura $ \mu $ si dice $ f $-invariante se la misura immagine coincide con la misura, cioè $ \forall F \in \mathcal{F}, f_\sharp \mu(F) = \mu(f^{-1}(F)) = \mu(F) $. 
\end{definition}
\section{Lezione del 10/10/2018}
\subsection{Introduzione}
La Teoria della Misura nasce a inizio '900 per formalizzare la probabilità e per cercare di fondare una teoria dell'integrazione che risulti più efficace di quella di Riemann. Per alcuni sottoinsiemi di $\R$, in particolare per gli intervalli limitati, abbiamo un concetto intuitivo di "misura", ovvero la lunghezza dell'intervallo:
\[\lambda\left([a,b]\right) = b-a.\]
L'obiettivo della teoria della misura è estendere questa nozione ad altri sottoinsiemi di $\R$ in modo coerente, ovvero in modo da rispettare, ad esempio, la proprietà di additività:
\[A \cap B = \emptyset \implies \lambda(A\cup B) = \lambda(A) + \lambda(B)\]
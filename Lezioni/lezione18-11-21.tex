\section{Lezione del 21/11/2018 [Marmi]}

\begin{thm}[di ricorrenza di Poincaré]
    Sia $ (X, \mathcal{A}, \mu, f) $ un sistema dinamico misurabile. Allora $ \forall A \in \mathcal{A} $, per $ \mu $-q.o. $ x \in A $, $ x $ è ricorrente in $ A $.
\end{thm}
\begin{proof}
    Sia $ A_r \coloneqq \{x \in A : x \text{ è ricorrente in } A\} \subseteq A $. Osserviamo che possiamo scrivere
    \[
        A_r = A \setminus \bigcup_{n \geq 1} B_n
    \]
    dove $ B_n $ è l'insieme degli $ x \in A $ che non visitano più $ A $ dopo il tempo $ n $ o più formalmente
    \[
        B_n \coloneqq A \setminus \bigcup_{j \geq n} f^{-j}(A).
    \]
    Essendo $ A_r $ differenza e unione numerabile di insiemi misurabili, anche $ A_r $ misurabile. Osservando che $ A \subseteq \bigcup_{j \geq 0} f^{-j}(A) $ essendo $ f^{0}(A) = A $ otteniamo che
    \[
        B_n \subseteq \bigcup_{j \geq 0} f^{-j}(A) \setminus  \bigcup_{j \geq n} f^{-j}(A).
    \]
    Definendo allora $ \overline{A} \coloneqq \bigcup_{j \geq 0} f^{-j}(A) $ abbiamo che
    \[
        \mu(B_n) \leq \mu(\overline{A}) - \mu\left(\textstyle{\bigcup_{j \geq n}} f^{-j}(A)\right) = \mu(\overline{A}) - \mu\left(f^{-n}\left(\textstyle{\bigcup_{j \geq 0}} f^{-j}(A)\right)\right) = \mu(\overline{A}) - \mu(f^{-n}(\overline{A})).
    \]
    Ma $ \mu $ è $ f $-invariate quindi
    \[
        \mu(B_n) \leq \mu(\overline{A}) - f_{\sharp}\mu(\overline{A}) = \mu(\overline{A}) - \mu(\overline{A}) = 0.
    \]
    da cui $ \mu\left(\bigcup_{n \geq 1} B_n\right) \leq \sum_{n \geq 1} \mu(B_n) = 0 $. Così concludiamo che
    \[
        \mu(A_r) = \mu(A) - \mu\left(\textstyle{\bigcup_{n \geq 1}} B_n\right) = \mu(A)
     \]
    cioè $ \mu $-q.o. $ x \in A $ è ricorrente in $ A $.
\end{proof}

\begin{thm}[Birkhoff] \label{thm:Birkoff}
    Sia $ (X, \mathcal{A}, \mu, f) $ un sistema dinamico misurabile. Allora $ \forall A \in \mathcal{A} $, per $ \mu $-q.o. $ x \in X $, $ \overline{\nu}(x, A) = \underline{\nu}(x, A) $ cioè
    \[
        \exists \, \lim_{n \to +\infty} \frac{1}{n} \sum_{j = 0}^{n-1} \chi_A(f^{j}(x)) \eqqcolon \nu(x, A).
    \]
    Inoltre $ \forall \varphi \in L^1(X, \mathcal{A}, \mu; \R) $ e per $ \mu $-q.o. $ x \in X $
    \[
        \exists \, \lim_{n \to +\infty} \frac{1}{n} \sum_{j=0}^{n-1} (\varphi \circ f^j)(x) \eqqcolon \tilde{\varphi}(x).
    \]
\end{thm}
\begin{proof}
    Trattiamo solo la parte con le funzioni caratteristiche. Sappiamo che esistono
    \[
        \underline{\nu}(x, A) \coloneqq \liminf_{n \to +\infty} \nu(x, A, n) \qquad \overline{\nu}(x, A) \coloneqq \limsup_{n \to +\infty} \nu(x, A, n)
    \]
    dove $ \nu(x, A, n) \coloneqq \frac{1}{n} \sum_{j = 0}^{n-1} \chi_A(f^{j}(x)) = \frac{1}{n} T(x, A, n) $. Sicuramente vale $ \underline{\nu}(x, A) \leq \overline{\nu}(x, A) $ quindi è sufficiente mostrare la disuguaglianza opposta. \\
    Fissato $ \epsilon > 0 $ sia
    \[
        \overline{\tau}(x, A, \epsilon) \coloneqq \min{\{n \in \N : \nu(x, A, n) \geq \overline{\nu}(x, A) - \epsilon\}}.
    \]
    Supponiamo per ora che tale quantità sia uniformemente limitata cioè $ \exists M_\epsilon > 0 : \forall x \in X, \ \overline{\tau}(x, A, \epsilon) \leq M_\epsilon $. L'idea è di dividere l'orbita da 0 a $ n $ (abbiamo in mente $ n \gg M_\epsilon $ in quanto manderemo poi $ n \to +\infty $) in sotto-segmenti di lunghezza $ \overline{\tau} $ su cui abbiamo un controllo essendo $ \overline{\tau} $ limitato. Dato $ n > M_\epsilon $ consideriamo quindi il segmento di orbita $ (f^{j}(x))_{j = 0}^{n-1} $ di lunghezza $ n $. Definiamo la successione ricorsiva
    \[
        \begin{cases}
        x_0 = x & \tau_0 = \overline{\tau}(x_0, A, \epsilon) \\
        x_{k+1} = f^{\overline{\tau}(x, A, \epsilon)}(x_k) = f^{\tau_k}(x_0) & \tau_k = \sum_{j = 0}^{k} \overline{\tau}(x_j, A, \epsilon)
        \end{cases}
    \]
    definita da $ k = 0 $ fino a $ k = K $ tale che $ \tau_{K} < n $ ma $ \tau_{K+1} \geq n $. \\
    Ora per definizione $ \nu(x, A, \overline{\tau}(x, A , \epsilon)) \geq \overline{\nu}(x, A) - \epsilon $ mentre per $ f $-invarianza $ \overline{\nu}(f(x), A) = \overline{\nu}(x, A) $ quindi per ogni $ k \in \{0, \ldots, K\} $ il numero di visite in ogni sotto-segmento si stima come
    \begin{align*}
        T(x_k, A, \overline{\tau}(x_k, A, \epsilon)) & = \overline{\tau}(x_k, A, \epsilon) \cdot \nu(x_k, A, \overline{\tau}(x, A , \epsilon)) \\
        & \geq \overline{\tau}(x_k, A, \epsilon) \cdot (\overline{\nu}(x_k, A) - \epsilon) \\
        & = \overline{\tau}(x_k, A, \epsilon) \cdot (\overline{\nu}(x, A) - \epsilon).
    \end{align*}
    Il numero totale di visite lungo il segmento di orbita di lunghezza $ n $ si stima considerando i sotto-segmenti come
    \begin{align*}
        T(x, A, n) & = \left[\sum_{k = 0}^{K} T(x_k, A, \overline{\tau}(x, A, \epsilon))\right] + T(x_{K}, A, n - \tau_{K}) \\
        & \geq \left[\sum_{k = 0}^{K}  \overline{\tau}(x_k, A, \epsilon)\right] (\overline{\nu}(x, A) - \epsilon) = \tau_{K} \cdot (\overline{\nu}(x, A) - \epsilon) \\
        & \geq (n - M_\epsilon)(\overline{\nu}(x, A) - \epsilon)
    \end{align*}
    dove abbiamo maggiorato il termine $ T(x_{K}, A, n - \tau_{K}) $ con $ 0 $ in quanto è il pezzo di orbita su cui non abbiamo controllo e abbiamo osservato che per costruzione la somma dei tempi $ \tau_K \geq n - M_{\epsilon} $. Ora per $ f $-invarianza della misura
    \[
        \int_{X} T(x, A, n) \dif{\mu} = \sum_{j = 0}^{n-1} \int_{X} \chi_A(f^{j}(x)) \dif{\mu}(x) = \sum_{j = 0}^{n-1} \int_{X} \chi_A(x) \dif{\mu}(x) = n \cdot \mu(A)
    \]
    così per monotonia
    \begin{gather*}
        n \cdot \mu(A) = \int_{X} T(x, A, n) \geq (n - M_\epsilon) \left(\int_{X}(\overline{\nu}(x, A) - \epsilon) \dif{\mu}(x)\right) \\
        \mu(A) \geq \frac{n - M_\epsilon}{n} \left(\int_{X}\overline{\nu}(x, A) \dif{\mu}(x) - \epsilon\right)
    \end{gather*}
    Passando prima al limite in $ n $ e poi all'estremo superiore su $ \epsilon $ otteniamo infine
    \[
        \mu(A) \geq \int_{X}\overline{\nu}(x, A) \dif{\mu}(x).
    \]
    Ripetendo lo stesso ragionamento con $ \underline{\nu} $, cioè supponendo che la quantità
    \[
        \underline{\tau}(x, A, \epsilon) \coloneqq \min{\{n \in \N : \nu(x, A, n) \leq \overline{\nu}(x, A) + \epsilon\}}
    \]
    sia uniformemente limitata e dividendo il segmento di orbita da $ 0 $ a $ (n-1) $ in sotto-segmenti giungiamo alla disuguaglianza opposta
    \[
        \mu(A) \leq \int_{X}\underline{\nu}(x, A) \dif{\mu}(x).
    \]
    Combinando le due disuguaglianze abbiamo quindi che
    \[
        \int_{X}\underline{\nu}(x, A) \dif{\mu}(x) \geq \int_{X}\overline{\nu}(x, A) \dif{\mu}(x).
    \]
    Ma d'altra parte essendo $ \underline{\nu}(x, A) \leq \overline{\nu}(x, A) $ per ogni $ x \in X $ anche
    \[
        \int_{X}\underline{\nu}(x, A) \dif{\mu}(x) \leq \int_{X}\overline{\nu}(x, A) \dif{\mu}(x).
    \]
    Così
    \[
        \int_{X} \overline{\nu} \dif{\mu} = \int_{X} \underline{\nu} \dif{\mu} \quad \Rightarrow \quad \int_{X} \left(\overline{\nu} - \underline{\nu}\right) \dif{\mu} = 0.
    \]
    Ma dato che l'integrando è positivo concludiamo che $ \mu $-q.o. si ha $ \overline{\nu}(x, A) = \underline{\nu}(x, A) $. \\
    \textcolor{red}{Dobbiamo ora rimuovere l'ipotesi della limitatezza.}
\end{proof}

\begin{proposition}
    Sia $ (X, \mathcal{A}, \mu_1, f) $ un sistema ergodico e $ \mu_2 $ una misura di probabilità su $ (X, \mathcal{A}) $, $ f $-invariante. Allora i seguenti fatti sono equivalenti:
    \begin{enumerate}[label=(\roman*)]
        \item $ \mu_1 \neq \mu_2 $;
        \item $ \mu_2 $ non è assolutamente continua rispetto a $ \mu_1 $, cioè $ \exists A \in \mathcal{A} $ tale che $ \mu_1(A) = 0 $ ma $ \mu_2(A) > 0 $;
        \item $ \exists A \in \mathcal{A} $ $ f $-invariante tale che $ \mu_1(A) = 0 $ e $ \mu_2(A) > 0 $.
    \end{enumerate}
\end{proposition}

Tale teorema stabilisce che un sistema dinamico misurabile può essere ergodico rispetto a due misure ``che non si parlano''. Ci sono tuttavia dei sistemi dinamici che ammettono una sola misura invariante. In tale caso si dà la seguente
\begin{definition}[sistema unicamente ergodico]
    $ (X; \mathcal{A}, \mu, f) $ si dice unicamente ergodico se esiste un'unica misura di probabilità su $ (X; \mathcal{A}) $ che sia $ f $-invariante.
\end{definition}

Tale definizione è ben posta perché un sistema unicamente ergodico è anche ergodico. Se infatti per assurdo $ (X, \mathcal{A}, \mu, f) $ non fosse ergodico esisterebbe $ A \in \mathcal{A} $ $ f $-invariante tale che $ \mu(A) > 0 $ e $ \mu(X \setminus A) > 0 $. Possiamo allora definire le misure
\[
     \nu_1(E) \coloneqq \frac{\mu(A \cap E)}{\mu(A)}
     \qquad
     \nu_2(E) \coloneqq \frac{\mu((X \setminus A) \cap E)}{\mu(X \setminus A)}
\]
che sono misure di probabilità diverse e $ f $-invarianti contro l'ipotesi. Per l'invarianza basta osservare che per l'invarianza di $ \mu $ si ha
\[
    \nu_1(f^{-1}(E)) = \frac{\mu(A \cap f^{-1}(E))}{\mu(A)} \textcolor{red}{=} \frac{\mu(f^{-1}(A) \cap f^{-1}(E))}{\mu(A)} = \frac{\mu(f^{-1}(A \cap E))}{\mu(A)} = \frac{\mu(A \cap E)}{\mu(A)} = \nu_1(E).
\]

\begin{thm}
    Se $ (X, \mathcal{B}, \mu, f) $, con $ X $ spazio topologico e $ \mathcal{B} $ la $ \sigma $-algebra dei boreliani, è un sistema unicamente ergodico e $ \varphi \colon X \to \R $ è continua allora si ha convergenza uniforme della media temporale dell'osservabile:
    \[
        \frac{1}{n} \sum_{j=0}^{n-1} (\varphi \circ f^j)(x) \, \touf \, \int_X \varphi \dif{\mu}.
    \]
\end{thm}

\subsection{Iterated function systems}

\section{Lezione del 14/11/2018 [Marmi]}

\subsection{Dinamica misurabile e teoria ergodica}

\begin{definition}[sistema dinamico misurabile]
    Un sistema dinamico misurabile è una quaterna $ (X, \mathcal{A}, \mu, f) $ dove
    \begin{enumerate}[label=(\roman*)]
        \item $ (X, \mathcal{A}, \mu) $ è uno spazio di probabilità, cioè un insieme $ X $ (spazio delle fasi) con una $ \sigma $-algebra $ \mathcal{A} $ e una misura $ \mu $ tale che $ \mu(X) = 1 $;
        \item $ f \colon X \to X $ è una funzione misurabile e tale che $ \mu $ sia $ f $-invariante, cioè tale che $ \forall A \in \mathcal{A}, \ f_{\sharp}\mu (A) = \mu(f^{-1}(A)) = \mu(A) $;
        \item la dinamica è data dall'iterazione di $ f $.
    \end{enumerate}
\end{definition}

\textcolor{red}{Ho riscritto un po' la definizione seguente per renderla più simmetrica, controllare che sia equivalente a quella data a lezione}

\begin{definition}[isomorfismo di sistemi dinamici misurabili]
    Due sistemi dinamici misurabili $ (X, \mathcal{A}, \mu, f) $ e $ (Y, \mathcal{B}, \nu, g) $ si dicono isomorfi se esistono una funzione $ h \colon X \to Y $ e una funzione $ k \colon Y \to X $ che siano una l'inversa dell'altra a meno di un insieme di misura nulla e facciano commutare il seguente diagramma:
    \begin{center}
        \begin{tikzcd}
        X \arrow[r, "f"] \arrow[d, shift right, "h" left] & X \arrow[d, shift left, "h" right]\\
        Y \arrow[r, "g" below] \arrow[u, shift right, "k" right] & Y \arrow[u, shift left, "k" left]
        \end{tikzcd}
    \end{center}
    Più precisamente chiediamo che $ \exists X' \subseteq X : \mu(X \setminus X') = 0 $, $ \exists Y' \subseteq Y : \nu(Y \setminus Y') = 0 $ e $ \exists h \colon X' \to Y' $ e $ k \colon Y' \to X' $ tali che
    \begin{enumerate}[label=(\roman*)]
        \item $ h \circ k = \Id_{Y'} $;
        \item $ k \circ h = \Id_{X'} $;
        \item $ h $ faccia commutare il diagramma di sopra, cioè $ g \circ h = h \circ f $ $ \mu $-q.o. oppure $ k $ faccia commutare il diagramma, cioè $ f \circ k = k \circ g $ $ \nu $-q.o.;
        \item $ h $ sia misurabile, cioè $ \forall B \in \mathcal{B}, \ h^{-1}(B) \in \mathcal{A} $;
        \item $ \nu $ sia la misura immagine di $ \mu $ secondo $ h $, cioè $ {\forall B \in \mathcal{B}, \ h_\sharp \mu(B) = \mu(h^{-1}(B)) = \nu(B)} $;
        \item $ k $ sia misurabile, cioè $ \forall A \in \mathcal{A}, \ h^{-1}(A) \in \mathcal{B} $;
        \item $ \mu $ sia la misura immagine di $ \nu $ secondo $ k $, cioè $ \forall A \in \mathcal{A}, \ k_\sharp \nu(A) = \nu(k^{-1}(A)) = \mu(A) $.
    \end{enumerate}
\end{definition}

\begin{oss}
    Le nozioni di dinamica misurabile e teoria ergodica che andremo ad enunciare sono invarianti per isomorfismo di sistemi dinamici.
\end{oss}

In Tabella \ref{tab:ergodica-vs-probabilia}, si riporta un confronto tra il linguaggio probabilistico e quello usato in teoria ergodica.

\begin{table}[h!]
    \centering
    \begin{tabularx}{\textwidth}{cXX}
        & \textsc{Teoria ergodica} & \textsc{Probabilità} \\ \toprule
        $ X $ & spazio delle fasi & spazio dei campioni \\
        $ \mathcal{A} $ & $ \sigma $-algebra dei misurabili & collezione degli eventi \\
        $ \mu $ & misura $ \mu(A) $ & probabilità $ \PP{(x \in A)} $ \\
        $ \varphi \colon X \to \R $ & osservabile & variabile aleatoria \\
        $ \varphi_n \coloneqq (\varphi \circ f^{n})_{n \in \N} $ & valore di un'osservabile lungo un'orbita & processo stocastico con distribuzione  $ \PP{(\varphi_1 \in A_1, \ldots, \varphi_k \in A_k)} =  \mu\left(\bigcap_{j = 1}^{k} \{x \in X : \varphi(f^{j}(x)) \in A_j\}\right) $ \\
        & $ f $-invarianza di $ \mu $ & processo stazionario \\ \bottomrule
    \end{tabularx}
    \caption{Teoria ergodica e probabilità}
    \label{tab:ergodica-vs-probabilia}
\end{table}

\begin{exercise}
    Sia $ f \colon [0, 1] \to [0, 1] $ monotona e $ C^1 $ a tratti (anche numerabili). Su $ [0, 1] $ è posta una misura con densità $ \rho(x) $, cioè tale che $ \mu(A) \coloneqq \int_A \rho(x) \dif{x} $ dove $ \dif{x} $ è la misura di Lebesgue. Mostrare che $ \mu $ è $ f $-invariante se e solo se
    \[
        \sum_{x \in f^{-1}(\{y\})} \frac{\rho(x)}{\abs{f'(x)}} = \rho(y).
    \]
\end{exercise}

\begin{example}
    I sistemi dinamici nell'esempio \ref{ex:Ulam_tenda} sono isomorfi tramite la mappa $ h $ ivi definita.
\end{example}

\begin{exercise}[mappa di Gauss]
    Sia $ G\colon [0,1]\to [0,1] $ definita come $ G(x) \coloneqq \left\{ \frac{1}{x} \right\} $. Verificare che $ G $ conserva la misura con densità:
    \[ \dif{\mu}(x) = \frac{\dif x}{(\log 2)(1+x)} \, . \]
\end{exercise}

\begin{exercise}[mappa di Farey]
    Sia $ F\colon (0,1)\to (0,1) $ la mappa
    \[
        F(x) \coloneqq
        \begin{cases}
            \frac{x}{1-x}   & \text{se } 0 < x \leq \frac{1}{2} \\
            \frac{1}{x} - 1 & \text{se } \frac{1}{2} < x < 1
        \end{cases}
    \]
    Essa conserva la misura infinita $ \dif{\mu}(x) = \frac{\dif{x}}{x} $.
\end{exercise}

\begin{exercise}
    Sia $ f\colon \R\to\R $ definita come $ f(x) \coloneqq \frac{1}{2} \left( x - \frac{1}{x} \right) $. Questa conserva la misura $ \dif{\mu}(x) = \frac{\dif{x}}{\pi(1+x^2)} $.
\end{exercise}

\begin{definition}[frequenze di visita]
    Sia $ (X, \mathcal{A}, \mu, f) $ un sistema dinamico misurabile. Dato $ A \in \mathcal{A} $, $ x \in A $ e $ n \in \N $ definiamo la frequenza media delle visite ad $ A $ dell'orbita di $ x $ da $ 0 $ a $ n $ come
    \[
        \nu(x, A, n) \coloneqq \frac{1}{n} \sum_{j = 0}^{n-1} \chi_A(f^{j}(x)).
    \]
    Definiamo inoltre
    \begin{align*}
        \overline{\nu}(x, A) & \coloneqq \limsup_{n \to +\infty} \nu(x, A, n) \\
        \underline{\nu}(x, A) & \coloneqq \liminf_{n \to +\infty} \nu(x, A, n)
    \end{align*}
    Se $ \overline{\nu} = \underline{\nu} $ allora definiamo la frequenza media delle visite ad $ A $ dell'orbita di $ x $:
    \[
        \nu(x, A) \coloneqq \lim_{n \to +i\infty} \frac{1}{n} \sum_{j = 0}^{n-1} \chi_A(f^{j}(x)).
    \]
\end{definition}
Quest'ultima è una buona definizione in virtù del seguente
\begin{thm}[Birkhoff]
    Sia $ (X,\mathcal{A},\mu,f) $ un sistema dinamico misurabile. Allora $ \forall A\in\mathcal{A} $ e per $ \mu $-q.o. $ x\in X $
    \[ \exists \lim_{n \to +\infty} \nu(x,A,n) = \nu(x,A) \, . \]
    Inoltre, $ \forall \varphi \in L^1(X,\mathcal{A},\mu) $ e per $ \mu $-q.o. $ x\in X $
    \[ \exists \lim_{n \to +\infty} \frac{1}{n} \sum_{j=0}^{n-1} \varphi\circ f^j(x) \eqqcolon \tilde{\varphi}(x) \, . \]
    La funzione $ \tilde{\varphi} $ viene detta \emph{media temporale} dell'osservabile $ \varphi $.
\end{thm}

\begin{definition}[sistema ergodico]
    Un sistema dinamico misurabile $ (X, \mathcal{A}, \mu, f) $ si dice ergodico se $ \forall A \in \mathcal{A} $ e per $ \mu $-q.o. $ x \in A $ esiste la frequenza di visita di $ x $ ad $ A $ e vale
    \[
        \nu(x, A) = \mu(A)
    \]
    cioè se la frequenza statistica delle visite coincide con la probabilità a priori. 
\end{definition}

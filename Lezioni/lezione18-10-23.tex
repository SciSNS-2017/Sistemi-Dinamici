\section{Lezione del 23/10/18 [Bindini]}

\subsection{Spazi di Hilbert}
\emph{Setting}: $ V $ è uno spazio vettoriale su $ \K = \R \text{ o } \C $. 

\begin{definition}[prodotto scalare]
    Una funzione $ {\langle \; , \, \rangle} \colon V \times V \to \K $ è un prodotto scalare se 
    \begin{enumerate}[label = (\roman*)]
        \item $ \forall v \in V, \ {\langle v, v \rangle} \geq 0 $ e $ {\langle v, v \rangle} \iff v = 0 $;
        \item $ \forall a_1, a_2 \in \K, \forall v_1, v_2, w \in V, \ {\langle a_1v_1 + a_2v_2, w \rangle} = a_1{\langle v_1, w\rangle} + a_2{\langle v_2, w\rangle} $;
        \item $ \forall b_1, b_2 \in \K, \forall w_1, w_2, v \in V, \ {\langle v, b_1w_1 + b_2w_2 \rangle} = \bar{b}_1{\langle v, w_1\rangle} + \bar{b}_2{\langle v, w_2\rangle} $
    \end{enumerate}
\end{definition}

Un prodotto scalare su $ V $ induce una funzione $ \norm{\;} \colon V \to \K  $ detta \emph{norma} definita come $ {\norm{v} \coloneqq \sqrt{{\langle v, v \rangle}}} $. Una norma a sua volta indice una distanza $ d \colon V \times V \to \K $ data da \linebreak $ d(v, w) \coloneqq \norm{v - w} $ che definisce una struttura di spazio metrico e di topologia.

\begin{proposition}[Cauchy-Schwarz]
    Per ogni $ v, w \in V $ vale $ \abs{{\langle v, w \rangle}} \leq \norm{v} \cdot \norm{w} $. 
\end{proposition}

\begin{definition}[sottospazio ortogonale]
    Se $ W \subseteq V $ è un sottospazio definiamo l'ortogonale di $ W $ come 
    \[
        W^{\perp} \coloneqq \{v \in V : \forall w \in W, \ {\langle v, w \rangle} = 0\}.
    \]
\end{definition}

\begin{lemma}
    Il prodotto scalare è una funzione continua rispetto alla topologia indotta dalla norma (e a quella euclidea in arrivo).
\end{lemma}
\begin{proof}
    Fissati $ (v_0, w_0) \in V \times V $, usando la triangolare e Cauchy-Schwarz
    \[
        \abs{{\langle v, w \rangle} - {\langle v_0, w_0 \rangle}} = \abs{{\langle v - v_0, w\rangle} - {\langle v_0, w - w_0 \rangle}} \leq \norm{v - v_0} \, \norm{w} + \norm{v_0} \, \norm{w - w_0}.
    \]
    Quindi se $ v \in B_\delta(v_0) $ e $ w \in B_\delta(w_0) $ si ha
    \[
        \abs{{\langle v, w \rangle} - {\langle v_0, w_0 \rangle}} \leq \delta(\norm{w} + \norm{w_0}) \leq \delta(\norm{w_0} + \norm{v_0} + \delta). \qedhere
    \]
\end{proof}

\begin{proposition}
    $ W^\perp $ è un sottospazio chiuso di $ V $.
\end{proposition}
\begin{proof}
    Il fatto che sia un sottospazio è ovvio per la linearità del prodotto scalare nella prima componente. Fissiamo ora $ w \in W $. Allora $ \{w\}^\perp = \{v \in V : {\langle v, w \rangle} = 0\} $ è un chiuso (controimmagine continua di un chiuso) da cui $ W^\perp = \bigcap_{w \in W} \{w\}^\perp $ è chiuso essendo intersezione di chiusi. 
\end{proof}

\begin{proposition}
    Vale che $ (W^\perp)^\perp = \overline{W} $.
\end{proposition}
\begin{proof}
    Se $ w \in W $ allora $ {\langle v, w \rangle} = 0 $ per ogni $ v \in W^\perp $ da cui $ W \subseteq (W^\perp)^\perp \Rightarrow \overline{W} \subseteq (W^\perp)^\perp $ essendo l'ortogonale di $ W^\perp $ un chiuso. \textcolor{red}{Viceversa}
\end{proof}

\begin{definition}[base ortonormale]
    Una base ortonormale di $ V $ è un insieme $ \{e_j\}_{j \in J} $ con $ J $ al più numerabile tale che
    \begin{enumerate}[label=(\roman*)]
        \item $ \forall j, k \in J, \ {\langle e_j, e_k \rangle} = \delta_{jk} $;
        \item\label{def:coin:base} $ \forall v \in V, \ \exists \{v_j\}_{j \in J} \subseteq \K : v = \displaystyle{\sum_{j \in J} v_j e_j} $. 
    \end{enumerate}
\end{definition} 

Nel caso in cui $ J $ sia numerabile la condizione \ref{def:coin:base} è da intendersi come convergenza rispetto alla norma, cioè
\[
    \norm{v - \sum_{j = 1}^{N} v_j e_j} \to 0 \quad \text{ per } N \to +\infty
\]

\begin{definition}[spazio separabile]
   Se $ V $ ammette una base al più numerabile allora si dice separabile.
\end{definition}

\begin{definition}[spazio di Hilbert]
    $ V $ si dice spazio di Hilbert se è separabile e completo rispetto alla norma indotta dal prodotto scalare.
\end{definition}

\begin{thm}[criterio di completezza]
    Sia $ (V, \norm{\;}) $ uno spazio normato. Allora le seguenti proprietà sono equivalenti.
    \begin{enumerate}[label=(\roman*)]
        \item $ V $ è completo.
        \item $ \forall (v_k)_{k \in \N} \subseteq V, \ \displaystyle{\sum_{k \in \N} \norm{v_k} < +\infty \Rightarrow \sum_{k \in \N} v_k} \text{ converge} $. 
    \end{enumerate}
\end{thm}

\begin{example}[$ \R^n, \C^n $]
    
\end{example}

\begin{example}[$ L^2 $]
    content...
\end{example}

\begin{example}[$ l^2 $]
    content...
\end{example}

\begin{proposition}
    Lo spazio $ l^2(\Z) $ è completo. 
\end{proposition}
\begin{proof}
    content...
\end{proof}

\begin{exercise}
    Dimostrare che lo spazio $ L^2([0, 1]) $ è completo. 
\end{exercise}

\section{Lezione del 20/11/2018 [Marmi]}

\emph{Setting}: $ (X, \mathcal{A}, \mu, f) $ è un sistema dinamico misurabile.

\begin{thm}
    Le seguenti proprietà sono equivalenti.
    \begin{enumerate}[label=(\roman*)]
        \item $ (X, \mathcal{A}, \mu, f) $ è ergodico.
        \item $ (X, \mathcal{A}, \mu, f) $ è \emph{metricamente indecomponibile} cioè $ \forall A \in \mathcal{A} : f(A) \subseteq A, \ \mu(A) \mu(X \setminus A) = 0 $, ovvero lo spazio delle fasi non può essere separato in due insiemi disgiunti $ f $-invarianti entrambi di misura non nulla.
        \item $ (X, \mathcal{A}, \mu, f) $ ha solo integrali primi del moto banali cioè $ \forall \varphi \in \textcolor{red}{L^1}(X, \mathcal{A}, \mu; \R) : \varphi \circ f = \varphi $ $ \mu $-q.o. si ha che $ \varphi $ è costante $ \mu $-q.o. in $ X $.
        \item $ \forall \varphi \in \textcolor{red}{L^1}(X, \mathcal{A}, \mu; \R) $ la media temporale di $ \varphi $ (che esiste per il Teorema \ref{thm:Birkhoff}) è $ \mu $-q.o. uguale alla media spaziale
        \[
            \lim_{n \to +\infty} \frac{1}{n} \sum_{j=0}^{n-1} (\varphi \circ f^j)(x) = \int_{X} \varphi \dif{\mu} \qquad \text{per } \mu\text{-q.o. } x \in X.
        \]
        \item $ \forall A, B \in \mathcal{A}, \ \displaystyle{\lim_{n \to +\infty} \frac{1}{n} \sum_{j=0}^{n-1} \mu(f^{-j}(A) \cap B) = \mu(A) \mu(B)} $ per $ \mu $-q.o. $ x \in X $, cioè il sistema dinamico è \emph{mescolante in media}.
    \end{enumerate}
\end{thm}
\begin{proof}
    Mostriamo le varie implicazioni.
    \begin{description}
        \item[$ (i) \Rightarrow (ii) $] Sia $ A\in\mathcal{A} : f(A) \subseteq A, \mu(A) > 0 $. Poiché $ A $ è $ f $-invariante, $ \forall x\in A\ \chi_A(f^j(x)) = 1 $ e dunque $ \nu(x,A) = 1 $ da cui, per l'ergodicità, $ \mu(A) = 1 $.
        \item[$ (ii) \Rightarrow (iii) $]
        \item[$ (iii) \Rightarrow (iv) $]
        \item[$ (iv) \Rightarrow (i) $]
        \item[$ (iv) \Rightarrow (v) $]
        \item[$ (v) \Rightarrow (ii) $]
    \end{description}
\end{proof}

\begin{proposition}\label{prop:rotazioni_erg}
    La rotazione $ R_\alpha \colon \T^1 \to \T^1 $ è ergodica se e solo se $ \alpha \in \R \setminus \Q $. 
\end{proposition}
\begin{proof}
    Usiamo la caratterizzazione con gli integrali primi. Prendo $ \varphi \colon \T^1 \to \R $ in \textcolor{red}{$ L^2 $} e lo sviluppo in serie di Fourier:
    \[
        \varphi(x) = \sum_{n \in \Z} \hat{\varphi}(n) e^{2\pi i n x}
    \]
    Componendola con la rotazione
    \[
        (\varphi \circ R_\alpha)(x) = \varphi(x + \alpha) = \sum_{n \in \Z} \hat{\varphi}(n) e^{2\pi i n \alpha} e^{2\pi i n x}.
    \]
    Affinché $ \varphi $ sia un integrale primo deve valere $ \hat{\varphi}(n) \left(e^{2\pi i n \alpha} - 1 \right) = 0. $ per ogni $ n \in \Z $.
    Ora se $ \alpha \in \R \setminus \Q, \ e^{2\pi i n \alpha} \neq 1 $ per ogni $ n \neq 0 $ così $ \varphi(x) = \hat{\varphi}(0) $ cioè $ \varphi $ è costante q.o. Per mostrare l'implicazione inversa supponiamo per assurdo che $ \alpha \in \Q $ e della forma $ p/q $; allora $ e^{2\pi i n \alpha} - 1 = 0 $ per ogni $ n $ della forma $ kq $ con $ k \in \Z $ da cui possiamo trovare un integrale primo non costante contro l'ipotesi che $ R_\alpha $ fosse ergodico.
\end{proof}

\begin{exercise}\label{ex:potenze_di_due_cancro}
    Sia $ x_j = \text{cifra più significativa di } 2^j $. Calcolare la frequenza di ciascuna cifra nella successione $ (x_j)_{j\in\N} $.
\end{exercise}
\begin{solution}
    Chiamiamo $ c_n $ la cifra più significativa di $ 2^n $, cioè il numero in $ \{1, \cdots, 9\} $ tale che
    \[ c_n \cdot 10^s \leq 2^n < (c_n+1) \cdot 10^s \]
    dove $ s = \lfloor \log_{10} 2^n \rfloor $. Prendendo ora il logaritmo si ha $ \log_{10}c_n + s \leq n\log_{10}2 < \log_{10}(c_n+1) + s $ e quindi
    \[ \log_{10}c_n \leq \{n\log_{10}2\} < \log_{10}(c_n+1) \]
    Considerando ora le rotazioni $ R_{\log_{10}(2)} \colon \T^1\to\T^1 $ possiamo riscrivere $ \{ n\log_{10}2 \} = R^n_{\log_{10}(2)}(0) $;
    se suddividiamo il toro come $ \T^1 = \sqcup_{k=1}^9 I_k $ con $ I_k = \left[\log_{10}k,\log_{10}(k+1)\right) $ la condizione che la cifra più significativa di $ 2^n $ sia $ c_n $ si traduce in
    \[ R^n_{\log_{10}(2)}(0) \in I_c \, . \]
    La sequenza delle potenze di 2 cercata è dunque la dinamica simbolica dell'orbita di 0 tramite la rotazione di $ \log_{10}2 $ secondo la suddetta partizione.

    Il sistema dinamico $ (\T^1, \mathcal{M}, \lambda, R_{\log_{10}(2)}) $ è ergodico per la proposizione \ref{prop:rotazioni_erg} in quanto $ \log_{10}2 $ è irrazionale; dunque la frequenza di visita dell'orbita di 0 agli intervalli $ I_k $ è uguale alla misura di Lebesgue degli intervalli stessi:
    \[ \nu(0,I_c) = \log_{10}\left(1+\frac{1}{c}\right) \, . \]
\end{solution}

\begin{example}[successione di Kolakoski]
    \textcolor{red}{mancante}
\end{example}

\begin{exercise}
    Mostrare che le dilatazioni sul toro sono ergodiche.
\end{exercise}
\begin{solution}
    Sia $ E_m\colon\T^1\to\T^1 $, $ E_m(x) = mx\pmod{1} $ per $ m\in\Z,\ \abs{m} \geq 2 $. Prendiamo $ \varphi\colon\T^1\to\R $ tale che $ \varphi\circ f = \varphi $; espandiamo $ \varphi $ in serie di Fourier e imponiamo che sia un integrale del moto
    \[ \varphi(x) = \sum_{n\in\Z} \hat\varphi(x) e^{2\pi i n x} = \sum_{n\in\Z} \hat\varphi(x) e^{2\pi i n m x} = \varphi(E_m(x)) \quad \forall x\in\T^1 \]
    Per l'ortogonalità della base di Fourier le due somme devono essere eguali termine a termine. Deve dunque valere che $ nx(1-m) = k $ per ogni $ k\in\Z $. L'equazione è banalmente verificata per $ n = 0 $. Se $ n\neq 0 $ basta prendere $ x\in\R\setminus\Q $ affinché l'equazione non sia verificata per nessun $ k\in\Z $.
\end{solution}

\begin{exercise}
    La trasformazione del panettiere è ergodica. \textcolor{red}{Ha scritto una mappa un po' diversa.}
\end{exercise}


\section{Lezione del 20/11/2018 [Marmi]}

\emph{Setting}: $ (X, \mathcal{A}, \mu, f) $ è un sistema dinamico misurabile.

\begin{thm}[Birkoff] \label{thm:Birkoff}
    Sia $ (X, \mathcal{A}, \mu, f) $ un sistema dinamico misurabile. Allora $ \forall A \in \mathcal{A} $, per $ \mu $-q.o. $ x \in X $, $ \overline{\nu}(x, A) = \underline{\nu}(x, A) $ cioè
    \[
    \exists \, \nu(x, A) \coloneqq \lim_{n \to +i\infty} \frac{1}{n} \sum_{j = 0}^{n-1} \chi_A(f^{j}(x)).
    \]
    Inoltre $ \forall \varphi \in L^1(X, \mathcal{A}, \mu; \R) $ e per $ \mu $-q.o. $ x \in X $
    \[
    \exists \, \tilde{\varphi}(x) = \lim_{n \to +\infty} \frac{1}{n} \sum_{j=0}^{n-1} (\varphi \circ f^j)(x). 
    \]
\end{thm}

\begin{thm}
    Le seguenti proprietà sono equivalenti.
    \begin{enumerate}[label=(\roman*)]
        \item $ (X, \mathcal{A}, \mu, f) $ è ergodico.
        \item $ (X, \mathcal{A}, \mu, f) $ è \emph{metricamente indecomponibile} cioè $ \forall A, \in \mathcal{A} : f(A) \subseteq A, \ \mu(A) \nu(X \setminus A) = 0 $ ovvero lo spazio delle fasi non può essere separato in due insiemi disgiunti $ f $-invarianti entrambi di misura non nulla. 
        \item $ (X, \mathcal{A}, \mu, f) $ ha solo integrali primi del moto banali cioè $ \forall \varphi \in \textcolor{red}{L^1}(X, \mathcal{A}, \mu; \R) $ tale che $ \varphi \circ f = \varphi $ $ \mu $-q.o. si ha $ \varphi $ è costante $ \mu $-q.o.
        \item $ \forall \varphi \in \textcolor{red}{L^1}(X, \mathcal{A}, \mu; \R) $ la media temporale di $ \varphi $ (che esiste per il Teorema \ref{thm:Birkoff})è $ \mu $-q.o. uguale alla media spaziale 
        \[
            \lim_{n \to +\infty} \frac{1}{n} \sum_{j=0}^{n-1} (\varphi \circ f^j)(x) = \int_{X} \varphi \dif{\mu} \qquad \text{per } \mu\text{-q.o. } x \in X.
        \]
        \item $ \forall A, B \in \mathcal{A}, \ \displaystyle{\lim_{n \to +\infty} \frac{1}{n} \sum_{j=0}^{n-1} \mu(f^{-j}(A) \cap B) = \mu(A) \mu(B)} $ per $ \mu $-q.o. $ x \in X $. 
    \end{enumerate}
\end{thm}
\begin{proof}
    Mostriamo le varie implicazioni.
    \begin{description}
        \item[$ (i) \Rightarrow (ii) $] 
        \item[$ (ii) \Rightarrow (iii) $] 
        \item[$ (iii) \Rightarrow (iv) $] 
        \item[$ (iv) \Rightarrow (i) $] 
        \item[$ (iv) \Rightarrow (v) $] 
        \item[$ (v) \Rightarrow (ii) $] 
    \end{description}
\end{proof}

\begin{proposition}
    La rotazione $ R_\alpha \colon \T^1 \to \T^1 $ è ergodica se e solo se $ \alpha \in \R \setminus \Q $. 
\end{proposition}
\begin{proof}
    Usiamo la caratterizzazione con gli integrali primi. Prendo $ \varphi \colon \T^1 \to \R $ in $ L^2 $ e lo sviluppo in serie di Fourier:
    \[
        \varphi(x) = \sum_{n \in \Z} \hat{\varphi}(n) e^{2\pi i n x}
    \]
    Componendola con la rotazione
    \[
        (\varphi \circ R_\alpha)(x) = \varphi(x + \alpha) = \sum_{n \in \Z} \hat{\varphi}(n) e^{2\pi i n \alpha} e^{2\pi i n x}.
    \]
    Affinché $ \varphi $ sia un integrale primo deve valere per ogni $ n \in \Z $
    \[
        \hat{\varphi}(n) \left(e^{2\pi i n \alpha} - 1 \right) = 0.
    \]
    Ora se $ \alpha \in \R \setminus \Q, \ e^{2\pi i n \alpha} \neq 1 $ per ogni $ n \neq 0 $ così $ \varphi(x) = \hat{\varphi}(0) $ cioè $ \varphi $ è costante q.o. Per mostrare la freccia inversa supponiamo per assurdo che $ \alpha \in \Q $ e della forma $ p/q $; allora $ e^{2\pi i n \alpha} - 1 = 0 $ per ogni $ n $ della forma $ kq $ con $ k \in \Z $ da cui possiamo trovare un integrale primo non costante contro l'ipotesi che $ R_\alpha $ fosse ergodico. 
\end{proof}

\begin{exercise}
    Mostrare che le dilatazioni sul toro sono ergodiche.
\end{exercise}

\begin{exercise}
    La trasformazione del panettiere è ergodica. 
\end{exercise}


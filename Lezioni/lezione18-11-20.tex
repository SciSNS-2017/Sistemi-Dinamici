\section{Lezione del 20/11/2018 [Marmi]}

\emph{Setting}: $ (X, \mathcal{A}, \mu, f) $ è un sistema dinamico misurabile.

\begin{lemma} \label{lem:insieme-quasi-invariante}
    Sia $ A \in \mathcal{A} $ un insieme quasi invariante, cioè tale che $ f(A) \subseteq A \cup N $ con $ \mu(N) = 0 $. Allora esiste $ A' \in \mathcal{A} $ $ f $-invariante quasi uguale a $ A $, cioè tale che $ \mu(A \Delta A') = 0.
\end{lemma}
\begin{proof}
    Basta prendere $ N' \coloneqq \bigcup_{j \geq 0} f^{-j}(N) $ e $ A' \coloneqq A \setminus N' $. Infatti se $ x \in A' $ allora $ x \in A $ ma $ x \notin N' $ cioè $ \forall j \geq 0, \ x \notin f^{-j}(N) $; quindi $ f(x) \in A \cup N $ e $ \forall j \geq 0, \ f(x) \notin f^{-j}(N) $ da cui in particolare $ f(x) \notin f^0(N) = N $ così $ f(x) \in A \setminus N' $ ovvero $ A' $ è $ f $-invariante. Inoltre per $ f $-invarianza della misura $ \mu(N') \leq \sum_{j \geq 0} \mu(f^{-j}(N)) = 0 $ così $ \mu(A\Delta A') = \mu(A\Delta(A\setminus N')) = \mu(A \cap N') \leq \mu(N') = 0 $.
\end{proof}

\begin{thm}
    Le seguenti proprietà sono equivalenti.
    \begin{enumerate}[label=(\roman*)]
        \item $ (X, \mathcal{A}, \mu, f) $ è ergodico.
        \item $ (X, \mathcal{A}, \mu, f) $ è \emph{metricamente indecomponibile} cioè $ \forall A \in \mathcal{A} : f(A) \subseteq A, \ \mu(A) > 0 \Rightarrow \mu(X \setminus A) = 0 $, ovvero lo spazio delle fasi non può essere separato in due insiemi disgiunti $ f $-invarianti entrambi di misura non nulla.
        \item $ (X, \mathcal{A}, \mu, f) $ ha solo integrali primi del moto banali cioè $ \forall \varphi \in \textcolor{red}{L^1}(X, \mathcal{A}, \mu; \R) : \varphi \circ f = \varphi $ $ \mu $-q.o. si ha che $ \varphi $ è costante $ \mu $-q.o. in $ X $.
        \item $ \forall \varphi \in \textcolor{red}{L^1}(X, \mathcal{A}, \mu; \R) $ la media temporale di $ \varphi $ (che esiste per il Teorema \ref{thm:Birkhoff}) è $ \mu $-q.o. uguale alla media spaziale
        \[
            \lim_{n \to +\infty} \frac{1}{n} \sum_{j=0}^{n-1} (\varphi \circ f^j)(x) = \int_{X} \varphi \dif{\mu} \qquad \text{per } \mu\text{-q.o. } x \in X.
        \]
        \item $ \forall A, B \in \mathcal{A}, \ \displaystyle{\lim_{n \to +\infty} \frac{1}{n} \sum_{j=0}^{n-1} \mu(f^{-j}(A) \cap B) = \mu(A) \mu(B)} $ per $ \mu $-q.o. $ x \in X $, cioè il sistema dinamico è \emph{mescolante in media}.
    \end{enumerate}
\end{thm}
\begin{proof}
    Mostriamo le varie implicazioni.
    \begin{description}
        \item[$ (i) \Rightarrow (ii) $] Sia $ A\in\mathcal{A} : f(A) \subseteq A $ con $\mu(A) > 0 $. Poiché $ A $ è $ f $-invariante, $ \forall x\in A\ \chi_A(f^j(x)) = 1 $ e dunque $ \nu(x,A) = 1 $ da cui, per l'ergodicità, $ \mu(A) = 1 $. Ma allora $ \mu(X \setminus A) = 1 - \mu(A) = 0 $.
        \item[$ (ii) \Rightarrow (iii) $] L'idea è che gli insiemi di livello di un integrale primo del moto separano lo spazio delle fasi: se l'integrale primo è costante allora un'insieme di livello e tutto lo spazio e gli altri sono vuoti, se invece non fosse costante potrei decomporre lo spazio. \\
        Sia $ \varphi $ un integrale primo in $ L^1(X, \mathcal{A}, \mu; \R) $. Supponiamo ora che valga l'ipotesi più forte $ \varphi = \varphi \circ f $ ovunque. Fissato $ \gamma \in \R $ sia $ A_\gamma \coloneqq \{x \in X : \varphi(x) \leq \gamma\} $ il sottolivello che è per ipotesi misurabile. Essendo $ \varphi $ integrale primo, $ f^{-1}(A_\gamma) = \{x \in X : (\varphi \circ f)(x) \leq \gamma\} = \{x \in X : \varphi(x) \leq \gamma\} = A_\gamma $. Ma allora $ f(A_\gamma) = f(f^{-1}(A_\gamma)) \subseteq A_\gamma $, cioè $ A_\gamma $ è $ f $-invariante. Perciò l'ipotesi \emph{(ii)} ci dice che $ \mu(A_\gamma) \in \{0, 1\} $. Consideriamo allora la funzione
        \begin{align*}
            \psi \colon \R & \to [0, +\infty) \\
            \gamma & \mapsto \mu(A_\gamma)
        \end{align*}
        che è monotona non decrescente ($ \gamma_1 < \gamma_2 \Rightarrow A_{\gamma_1} \subseteq A_{\gamma_2} \Rightarrow \mu(A_{\gamma_1}) \leq \mu(A_{\gamma_2}) $). Osserviamo ora che
        \[
            \lim_{\gamma \to -\infty} \mu(A_\gamma) = 0 \qquad \lim_{\gamma \to +\infty} \mu(A_\gamma) = 1
        \]
        da cui deduciamo che esiste $ \overline{\gamma} \coloneqq \inf{\{\gamma \in \R : \mu(A_\gamma) = 1\}} \in \R $. Ma essendo gli $ A_\gamma $ inscatolati
        \[
            A_{\overline{\gamma}} = \bigcap_{n \geq 1} A_{\overline{\gamma} + \frac{1}{n}} \quad \Rightarrow \quad A_{\overline{\gamma} + \frac{1}{n}} \downarrow A_{\overline{\gamma}}
        \]
        da cui
        \[
            \mu(A_{\overline{\gamma}}) = \lim_{n \to +\infty} \mu\left(A_{\overline{\gamma} + \frac{1}{n}}\right) = 1.
        \]
        \textcolor{red}{Si conclude in un qualche modo oscuro.}\\
        Supponiamo ora che $ \varphi = \varphi \circ f $ $ \mu $-q.o. Detto allora $ B_\gamma \coloneqq \{x \in X : (\varphi \circ f) \leq \gamma\} $ si ha $ \mu(A_\gamma \Delta B_\gamma) = 0 $ e pertanto $ \exists N \in \mathcal{A} : \mu(N) = 0 $ tale che $ A_\gamma = B_\gamma \Delta N $. Così $ B_\gamma = f^{-1}(A_\gamma) = f^{-1}(B_\gamma) \Delta f^{-1}(N) $. Quindi posto $ N' \coloneqq f^{-1}(N) $ per invarianza della misura si ha $ \mu(N') = 0 $ e inoltre
        \[
            f(B_\gamma \Delta N') = f(f^{-1}(B_\gamma)) \subseteq B_\gamma = (B_\gamma \Delta N') \Delta N'
        \]
        cioè $ B_\gamma \Delta N' $ è invariante a meno di un insieme di misura nulla. Per il Lemma \ref{lem:insieme-quasi-invariante} sappiamo che esiste $ B'_\gamma \in \mathcal{A} $ tale che $ f(B_\gamma) \subseteq B_\gamma $ e $ \mu(B_\gamma \Delta B'_\gamma) = 0 $. Per l'ipotesi \emph{(ii)} allora $ \mu(B'_\gamma) \in \{0, 1\} $ così\footnote{Se $ F, M \in \mathcal{A} $ e $ \mu(M) = 0 $ allora \[\mu(F \Delta M) = \mu((F \setminus M) \cup (M \setminus F)) = \mu(F \setminus M) + \mu(M \setminus F) = \mu(F) - \mu(F \cap M) + \mu(M \setminus F) = \mu(F)\] in quanto $ F \cap M $ e $ M \setminus F $ sono sottoinsiemi misurabili di un insieme di misura nulla.}
        \[
            \mu(A_\gamma) = \mu(B_\gamma \Delta N') = \mu(B_\gamma) = \mu((B_\gamma \Delta B'_\gamma) \Delta B'_\gamma) = \mu(B'_\gamma) \in \{0, 1\}.
        \]
        A questo punto si conclude come in precedenza.
        \item[$ (iii) \Rightarrow (iv) $] Osserviamo che la media temporale di un'osservabile è un integrale primo del moto in quanto
        \begin{align*}
            (\tilde{\varphi} \circ f)(x) & = \lim_{n \to +\infty} \frac{1}{n} \sum_{j = 0}^{n-1} (\varphi \circ f^{j+1})(x) = \lim_{n \to +\infty} \frac{1}{n} \sum_{j = 1}^{n} (\varphi \circ f^{j})(x) \\
            & = \lim_{n \to +\infty} \left(\frac{1}{n} \sum_{j = 0}^{n-1} (\varphi \circ f^{j})(x) - \frac{\varphi(x)}{n} + \frac{\varphi(f^{n}(x))}{n}\right) \\
            & \textcolor{red}{=} \lim_{n \to +\infty} \frac{1}{n} \sum_{j = 0}^{n-1} (\varphi \circ f^{j})(x) = \tilde{\varphi}(x).
        \end{align*}
        e quindi per ipotesi $ \exists c \in \R : \tilde{\varphi} = c \ \mu\text{-q.o.} $. \\
        Supponiamo per ora $ \varphi \geq 0 $. Essendo $ \frac{1}{n} \sum_{j = 0}^{n-1} (\varphi \circ f^{j}) \in L^1 $ e quindi misurabili possiamo applicare il Lemma di Fatou e usando l'$ f $-invarianza della misura otteniamo
        \begin{align*}
            \tilde{\varphi}(x) & = \int_{X} \tilde{\varphi} \dif{\mu} = \int_{X} \lim_{n \to +\infty} \frac{1}{n} \sum_{j = 0}^{n-1} (\varphi \circ f^{j}) \dif{\mu} = \int_{X} \liminf_{n \to +\infty} \frac{1}{n} \sum_{j = 0}^{n-1} (\varphi \circ f^{j}) \dif{\mu} \\
            & \leq \liminf_{n \to +\infty} \int_{X} \frac{1}{n} \sum_{j = 0}^{n-1} (\varphi \circ f^{j}) \dif{\mu} = \liminf_{n \to +\infty} \frac{1}{n} \sum_{j = 0}^{n-1} \int_{X} (\varphi \circ f^{j}) \dif{\mu} \\
            & = \liminf_{n \to +\infty} \frac{1}{n} \sum_{j = 0}^{n-1} \int_{X} \varphi \dif{\mu} \\
            & = \int_{X} \varphi \dif{\mu}.
        \end{align*}
        \item[$ (iv) \Rightarrow (i) $]
        \item[$ (iv) \Rightarrow (v) $]
        \item[$ (v) \Rightarrow (ii) $]
    \end{description}
\end{proof}

\begin{proposition}\label{prop:rotazioni_erg}
    La rotazione $ R_\alpha \colon \T^1 \to \T^1 $ è ergodica se e solo se $ \alpha \in \R \setminus \Q $. 
\end{proposition}
\begin{proof}
    Usiamo la caratterizzazione con gli integrali primi. Prendo $ \varphi \colon \T^1 \to \R $ in \textcolor{red}{$ L^2 $} e lo sviluppo in serie di Fourier:
    \[
        \varphi(x) = \sum_{n \in \Z} \hat{\varphi}(n) e^{2\pi i n x}
    \]
    Componendola con la rotazione
    \[
        (\varphi \circ R_\alpha)(x) = \varphi(x + \alpha) = \sum_{n \in \Z} \hat{\varphi}(n) e^{2\pi i n \alpha} e^{2\pi i n x}.
    \]
    Affinché $ \varphi $ sia un integrale primo deve valere $ \hat{\varphi}(n) \left(e^{2\pi i n \alpha} - 1 \right) = 0. $ per ogni $ n \in \Z $.
    Ora se $ \alpha \in \R \setminus \Q, \ e^{2\pi i n \alpha} \neq 1 $ per ogni $ n \neq 0 $ così $ \varphi(x) = \hat{\varphi}(0) $ cioè $ \varphi $ è costante q.o. Per mostrare l'implicazione inversa supponiamo per assurdo che $ \alpha \in \Q $ e della forma $ p/q $; allora $ e^{2\pi i n \alpha} - 1 = 0 $ per ogni $ n $ della forma $ kq $ con $ k \in \Z $ da cui possiamo trovare un integrale primo non costante contro l'ipotesi che $ R_\alpha $ fosse ergodico.
\end{proof}

\begin{exercise}\label{ex:potenze_di_due_cancro}
    Sia $ x_j = \text{cifra più significativa di } 2^j $. Calcolare la frequenza di ciascuna cifra nella successione $ (x_j)_{j\in\N} $.
\end{exercise}
\begin{solution}
    Chiamiamo $ c_n $ la cifra più significativa di $ 2^n $, cioè il numero in $ \{1, \cdots, 9\} $ tale che
    \[ c_n \cdot 10^s \leq 2^n < (c_n+1) \cdot 10^s \]
    dove $ s = \lfloor \log_{10} 2^n \rfloor $. Prendendo ora il logaritmo si ha $ \log_{10}c_n + s \leq n\log_{10}2 < \log_{10}(c_n+1) + s $ e quindi
    \[ \log_{10}c_n \leq \{n\log_{10}2\} < \log_{10}(c_n+1) \]
    Considerando ora le rotazioni $ R_{\log_{10}(2)} \colon \T^1\to\T^1 $ possiamo riscrivere $ \{ n\log_{10}2 \} = R^n_{\log_{10}(2)}(0) $;
    se suddividiamo il toro come $ \T^1 = \sqcup_{k=1}^9 I_k $ con $ I_k = \left[\log_{10}k,\log_{10}(k+1)\right) $ la condizione che la cifra più significativa di $ 2^n $ sia $ c_n $ si traduce in
    \[ R^n_{\log_{10}(2)}(0) \in I_c \, . \]
    La sequenza delle potenze di 2 cercata è dunque la dinamica simbolica dell'orbita di 0 tramite la rotazione di $ \log_{10}2 $ secondo la suddetta partizione.

    Il sistema dinamico $ (\T^1, \mathcal{M}, \lambda, R_{\log_{10}(2)}) $ è ergodico per la proposizione \ref{prop:rotazioni_erg} in quanto $ \log_{10}2 $ è irrazionale; dunque la frequenza di visita dell'orbita di 0 agli intervalli $ I_k $ è uguale alla misura di Lebesgue degli intervalli stessi:
    \[ \nu(0,I_c) = \log_{10}\left(1+\frac{1}{c}\right) \, . \]
\end{solution}

\begin{example}[successione di Kolakoski]
    \textcolor{red}{mancante}
\end{example}

\begin{exercise}
    Mostrare che le dilatazioni sul toro sono ergodiche.
\end{exercise}
\begin{solution}
    Sia $ E_m\colon\T^1\to\T^1 $, $ E_m(x) = mx\pmod{1} $ per $ m\in\Z,\ \abs{m} \geq 2 $. Prendiamo $ \varphi\colon\T^1\to\R $ tale che $ \varphi\circ f = \varphi $; espandiamo $ \varphi $ in serie di Fourier e imponiamo che sia un integrale del moto
    \[ \varphi(x) = \sum_{n\in\Z} \hat\varphi(x) e^{2\pi i n x} = \sum_{n\in\Z} \hat\varphi(x) e^{2\pi i n m x} = \varphi(E_m(x)) \quad \forall x\in\T^1 \]
    Per l'ortogonalità della base di Fourier le due somme devono essere eguali termine a termine. Deve dunque valere che $ nx(1-m) = k $ per ogni $ k\in\Z $. L'equazione è banalmente verificata per $ n = 0 $. Se $ n\neq 0 $ basta prendere $ x\in\R\setminus\Q $ affinché l'equazione non sia verificata per nessun $ k\in\Z $.
\end{solution}

\begin{exercise}
    La trasformazione $ T_\alpha \colon \T^2 \to \T^2 $ data da $ T_\alpha(x, y) \coloneqq (x+\alpha, x+y) $ è ergodica se e solo se $ \alpha \in \R \setminus \Q $.
\end{exercise}


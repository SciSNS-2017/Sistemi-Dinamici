\section{Lezione del 19/12/2018 [Marmi]}

\subsection{Entropia di Shannon}
Claude Shannon: \emph{A Mathematical Theory of Communication}, 1948. \\

Sia $ (X, \mathcal{A}, \mu) $ uno spazio di probabilità e consideriamo una partizione finita dello spazio \linebreak $ X = A_1 \sqcup \ldots \sqcup A_n \pmod{0} $ cioè tale che gli insiemi $ A_i $ sono a due a due disgiunti a meno di insiemi di misura nulla e la loro unione è $ X $ a meno di un insieme di misura nulla. A tale partizione possiamo associare un vettore di probabilità $ p \coloneqq (p_1, \ldots, p_n) \in \R^n_+ $ dove $ p_i \coloneqq \mu(A_i) $. Indichiamo con $ \Delta^{(n)} \coloneqq \{(p_1, \ldots, p_n) \in \R^n_+ : \sum_{i=0}^{n} p_i=1\} $ lo spazio dei vettori di probabilità in $ \R^n $. \\

L'idea di Shannon è quella di associare a ogni elemento $ x \in X $ una certa quantità di \emph{informazione} $ I(x) $ partendo dal fatto che gli eventi più rari contengono informazione maggiore. La seconda intuizione di Shannon è di utilizzare la scala logaritmica per quantificare tale informazione: se $ x \in A_i $ della partizione allora $ I(x) = -\log p_i $. In tale modo se $ x $ un evento è ``raro'', ovvero $ p_i=\mu(A_i) $ è ``piccola'', allora $ I(x) $ sarà ``grande''; viceversa se $ x $ è un evento frequente, ovvero $ p_i $ è ``grande'', allora $ I(x) $ sarà ``piccola''. Più formalmente definiamo la funzione che indica l'indice della partizione a cui $ x $ appartiene\footnote{Siccome la partizione è definita $ (\mathrm{mod} 0) $ chiaramente $ i $ sarà una funzione definita quasi ovunque.}
\begin{align*}
    i \colon X & \to \{1, \ldots, n\} \\
    x & \mapsto i(x) = j : x \in A_{j}
\end{align*}
da cui la funzione \emph{informazione} è
\begin{align*}
    I \colon X & \to [0, +\infty) \\
    x & \mapsto -\log p_{i(x)}
\end{align*}
L'\emph{entropia di Shannon} della partizione è una quantità che ci dice qual è l'informazione media contenuta nella partizione, ovvero sarà il valore di aspettazione della funzione informazione
\begin{equation} \label{eqn:entropia-shannon}
    H \coloneqq \mathbb{E}[I] = \int_{X} I(x) \dif{\mu(x)} = -\sum_{i=1}^{n} p_i \log{p_i}
\end{equation}

Il seguente teorema formalizza le nozioni di informazione media sopra esposte e afferma che la definizione di entropia dell'equazione \eqref{eqn:entropia-shannon} è l'unica compatibile (a meno di riscalamenti) con una serie di assiomi che corrispondo a nozioni intuitive legate all'idea di informazione media data a Shannon.

\begin{thm}[entropia di Shannon] \label{thm:Shannon}
    Sia $ H^{(n)} \colon \Delta^{(n)} \to [0, +\infty) $ una funzione reale sullo spazio dei vettori di probabilità che chiamiamo entropia. Supponiamo che:    \begin{enumerate}[label=(\roman*)]
        \item\label{pt:continua} $ H^{(n)} $ sia continua e non identicamente costante;
        \item $ H^{(n)} $ sia simmetrica nei suoi argomenti;
        \item $ H^{(n)}(1, 0, \ldots, 0) = 0 $;
        \item\label{pt:concat} le entropie sono ``concatenate'': $ H^{(n)}(0, p_2, \ldots, p_n) = H^{(n-1)}(p_2, \ldots, p_n) $;
        \item\label{pt:max} l'entropia massima è quella ``più casuale'': $ H^{(n)}(p_1, \ldots, p_n) \leq H^{(n)}\left(\dfrac{1}{n}, \ldots, \dfrac{1}{n}\right) $;
        \item\label{pt:formula} l'entropia di una partizione in $ n \cdot l $ insiemi è l'entropia degli $ n $ raggruppamenti della partizione in $ l $ insiemi più la media pesata delle ``entropie relative'' all'interno di ciascun raggruppamento, calcolate con la probabilità condizionata. In formule
        \[
            H^{(nl)}(\pi_{11}, \ldots, \pi_{1l}, \ldots, \pi_{n1}, \ldots, \pi_{nl}) = H^{(n)}(p_1, \ldots, p_n) + \sum_{i=1}^{n} p_i \, H^{(l)} \left(\dfrac{\pi_{i1}}{p_i}, \ldots, \dfrac{\pi_{il}}{p_i}\right)
        \]
        dove $ p_i \coloneqq \sum_{j = 1}^{l} \pi_{ij} $.
    \end{enumerate}
    Allora $ \exists C > 0 $ costante positiva tale che
    \[
        H^{(n)}(p_1, \ldots, p_n) = -C \cdot \sum_{i = 1}^{n} p_i \, \log{p_i}
    \]
    e viene detta entropia di Shannon.
\end{thm}
\begin{proof}
    Per prima cosa mostreremo che la tesi vale per $ K(n) \coloneqq H^{(n)}\left(\frac{1}{n}, \ldots, \frac{1}{n}\right) = C \cdot \log{n} $. Useremo l'ultima proprietà per trovare l'espressione di $ H^{(n)} $ sui razionali e concluderemo così per densità. \\
    Dalla \ref{pt:concat} e dalla \ref{pt:max} otteniamo che $ K(n) $ è una funzione non decrescente in $ n $
    \[
        K(n+1) = H^{(n+1)}\left(\frac{1}{n+1}, \ldots, \frac{1}{n+1}\right) \geq H^{(n+1)}\left(0, \frac{1}{n}, \ldots, \frac{1}{n}\right) = H^{(n)}\left(\frac{1}{n}, \ldots, \frac{1}{n}\right) = K(n).
    \]
    Mentre dalla \ref{pt:concat} abbiamo che $ K(nl) = K(n) + K(l) $: infatti prendendo $ \pi_{ij} = \frac{1}{nl} $ da cui $ p_i = \sum_{j=1}^{l} \frac{1}{nl} = \frac{1}{n} $ si ottiene
    \begin{align*}
        K(nl) & = H^{(nl)}\left(\frac{1}{nl}, \ldots, \frac{1}{nl}\right) = H^{(n)}\left(\frac{1}{n}, \ldots, \frac{1}{n}\right) + \sum_{i=1}^{n} \frac{1}{n} H^{(l)} \left(\frac{1}{l}, \ldots, \frac{1}{l}\right) \\
        & = K(n) + n\frac{1}{n} K(l)
    \end{align*}
    Ora $ \forall r, l \in \N\setminus\{0\} $ e $ \forall n \in \N $, sia $ m \in \N $ tale che $ l^{m} \leq r^{n} \leq l^{m+1} $. Per quanto detto sulla funzione $ K $ abbiamo che $ K(l^m) \leq K(r^{n}) \leq K(l^{m+1}) $ ovvero
    \[ \frac{m}{n} \leq \frac{K(r)}{K(l)} \leq \frac{m}{n} + \frac{1}{n}. \]
    Se applichiamo il logaritmo alla stessa disuguaglianza otteniamo similmente
    \[ \frac{m}{n} \leq \frac{\log{r}}{\log{l}} \leq \frac{m}{n} + \frac{1}{n} \]
    da cui
    \[ \frac{K(r)}{K(l)} - \frac{1}{n} \leq \frac{\log{r}}{\log{l}} \leq \frac{K(r)}{K(l)} + \frac{1}{n}. \]
    Passando al limite in $ n $ otteniamo che $ \forall r, l \in \N\setminus\{0\} $ vale $ \frac{K(r)}{K(l)} = \frac{\log{r}}{\log{l}} $ da cui esiste una costante $ C > 0 $ tale che $ K(r) = C \cdot \log{r} $. \\
    Ora per la \ref{pt:concat} e la \ref{pt:formula}
    \begin{align*}
        H^{\left(\sum_{i=1}^{n} l_i\right)}(\pi_{11},\ldots,\pi_{1l_1}
        , \ldots, \pi_{n1},\ldots,\pi_{1l_n}) & = H^{(nL)}(\pi_{11}, \ldots, \pi_{1L}, \ldots, \pi_{n1}, \ldots, \pi_{nL}) \\
        & = H^{(n)}(p_1, \ldots, p_n) + \sum_{i=1}^{n} p_i \, H^{(L)} \left(\dfrac{\pi_{i1}}{p_i}, \ldots, \dfrac{\pi_{iL}}{p_i}\right) \\
        & = H^{(n)}(p_1, \ldots, p_n) + \sum_{i=1}^{n} p_i \, H^{(l_i)} \left(\dfrac{\pi_{i1}}{p_i}, \ldots, \dfrac{\pi_{il_i}}{p_i}\right)
    \end{align*}
    dove $ L \coloneqq \max\{l_1, \ldots, l_n\} $ e abbiamo posto a zero le probabilità $ \pi_{i (l_j+1)}, \ldots, \pi_{jL} $. Se ora poniamo $ s\coloneqq\sum_{i=1}^{n}l_i $ e prendiamo $ \pi_{ij} = 1/s $ otteniamo che $ p_i = l_i/s $ così
    \[ K(s) = H^{(s)}\left(\frac{1}{s}. \ldots, \frac{1}{s}\right) = H^{(n)}\left(\frac{l_1}{s}, \ldots, \frac{l_n}{s}\right) + \sum_{i=1}^{n} \frac{l_i}{s} H^{(l_i)}\left(\frac{1}{l_i}, \ldots, \frac{1}{l_i}\right) \]
    ovvero
    \begin{align*}
        H^{(n)}\left(\frac{l_1}{s}, \ldots, \frac{l_n}{s}\right) = K(s) - \sum_{i=1}^{n} \frac{l_i}{s} K(l_i) = C \log{s} \cdot \sum_{i=1}^{n} \frac{l_i}{s} - C \sum_{i=1}^{n} \frac{l_i}{s} \log{\frac{1}{l_i}} = - C \sum_{i=1}^{n}\frac{l_i}{s} \log{\frac{l_i}{s}}.
    \end{align*}
    Tale espressione determina $ H^{(n)} $ su tutti i vettori di probabilità a componenti razionali. Infatti se $ \left(\frac{r_1}{s_1},\ldots,\frac{r_n}{s_n}\right) \in \Delta^{(n)} $ con $ l_i, s_i \in \N $, posto $ s \coloneqq \mathrm{mcm}\{s_1, \ldots, s_n\} $ e $ l_i = r_i \frac{s}{s_i} $ si ha $ \left(\frac{r_1}{s_1},\ldots,\frac{r_n}{s_n}\right) = \left(\frac{l_1}{s}, \ldots, \frac{l_n}{s}\right) $ e inoltre $ \sum_{i=1}^{n} l_i = s $ essendo $ \sum_{i=1}^{n}\frac{r_i}{s_i} $. Così
    \[ H^{(n)}\left(\frac{r_1}{s_1},\ldots,\frac{r_n}{s_n}\right) = H^{(n)}\left(\frac{l_1}{s}, \ldots, \frac{l_n}{s}\right) = - C \sum_{i=1}^{n}\frac{l_i}{s} \log{\frac{l_i}{s}} = - C \sum_{i=1}^{n}\frac{r_i}{s_i} \log{\frac{r_i}{s_i}}. \]
    Dobbiamo ora mostrare che tale formula si estende in modo unico anche ai vettori di probabilità reali. La funzione $ f_n(p_1, \ldots, p_n) = -C\sum_{i=1}^{n}p_i \log{p_i} $ definita su tutto $ \Delta^{(n)} $\footnote{In realtà $ f_n $ è l'estensione continua di $ -C\sum_{i=1}^{n}p_i \log{p_i} $ che è ben definita solo sui vettori di probabilità con componenti tutte non nulle.} è una funzione continua su un compatto e pertanto uniformemente continua. D'altra parte abbiamo visto che $ f_n $ coincide con $ H^{(n)} $ su $ \Delta^{(n)} \cap \Q^{n} $ ed $ H^{(n)} $ la restrizione di $ f_n $ ai vettori di probabilità razionali è uniformemente continua sul suo domino. Così per la \ref{pt:continua}, la funzione $ H^{(n)} $ si estende in modo unico alla chiusura di $ \Delta^{(n)} \cap \Q^{n} $ che è $ \Delta^{(n)} $ e, per unicità dell'estensione, tale estensione deve coincidere con $ f_n $ su tutto $ \Delta^{(n)} $.
\end{proof}

Data una partizione $ \mathcal{P} \coloneqq \{A_1, \ldots, A_n\} $ di $ X \pmod{0} $ scriveremo più brevemente
\[ H(\mathcal{P}) \coloneqq H^{(n)}(\mu(A_1), \ldots, \mu(A_n)). \]

\subsection{Entropia metrica o di Kolmogorov-Sinai}

\begin{definition}[join di due partizioni]
    Se $ \mathcal{P} \coloneqq \{A_1, \ldots, A_n\} $ e $ \mathcal{Q} \coloneqq \{B_1, \ldots, B_m\} $ sono due partizioni di $ X \pmod{0} $ si definisce il join di $ \mathcal{P} $ e $ \mathcal{Q} $ come la partizione
    \[
        \mathcal{P} \vee \mathcal{Q} \coloneqq \{A_i \cap B_j : i = 1, \ldots, n \text{ e } j = 1, \ldots, m\}.
    \]
\end{definition}

Dato un sistema dinamico misurabile $ (X, \mathcal{A}, \mu, f) $ possiamo considerare il join delle iterazioni di $ f $ su $ \mathcal{P} $. Posto $ f^{-k}\mathcal{P} \coloneqq \{f^{-k}(A_i) : A_i \in \mathcal{P}\} $ tale join è
\[
    \mathcal{P}^{(n)} \coloneqq \mathcal{P} \vee (f^{-1}\mathcal{P}) \vee \cdots \vee (f^{-(n-1)}\mathcal{P}) = \bigvee_{i=0}^{n}f^{-i}\mathcal{P}
\]

Tale costruzione ha lo scopo di quantificare l'informazione media che si ottiene facendo la \emph{dinamica simbolica} dell'iterazione di $ f $ usando la partizione $ \mathcal{P} $. Alla partizione $ \mathcal{P} = \{A_1, \ldots, A_n\} $ associamo l'insieme di simboli $ \{a_1, \ldots, a_n\} $. Consideriamo l'orbita di un punto $ x \in X $ sotto l'azione di $ f $: al passo $ n $-esimo dell'iterazione codifichiamo il blocco della partizione in cui sta $ f^{n}(x) $ con il simbolo $ a_{i_n} $ associato alla partizione, e così per ogni $ n \in \N $. In tale modo all'orbita corrisponde una successione $ (a_{i_n}) $ che è la codifica dell'orbita data dalla scelta della partizione.

Ora se $ B \in \mathcal{P}^{(n)} $ vuol dire che $ B = A_{i_0} \cap f^{-1}(A_{i_1}) \cap \cdots \cap f^{-(n-1)}(A_{i_{n-1}}) $ ovvero se $ x \in B $ vuol dire che $ x \in A_{i_0}, f(x) \in A_{i_1}, \ldots, f^{n-1}(x) \in A_{i_{n-1}} $. Pertanto a $ B \in \mathcal{P}^{(n)} $ corrisponde l'insieme degli $ x \in X $ che hanno come dinamica simbolica fino all'$ n $-esima iterazione la stringa $ (a_{i_0}, \ldots, a_{i_{n-1}}) $.

Pertanto $ H(\mathcal{P}^{(n)}) $ quantifica, in un certo senso, l'informazione media che si ottiene sulla dinamica di $ f $ conoscendo i primi $ n $ blocchi in cui è ``passata'' l'orbita di un elemento di $ X $ sotto $ f $. Dividendo tale quantità per $ n $ (la lunghezza della stringa della dinamica simbolica) e prendendo il limite in $ n $ otteniamo quindi l'informazione media che si ottiene sulla dinamica di $ f $ facendo la dinamica simbolica usando come partizione dello spazio $ \mathcal{P} $. Chiaramente tale limite dipende dalla partizione scelta, che poteva per esempio non essere ottimale per la codifica di $ f $: per definire un'entropia associata al sistema dinamico $ (X, \mathcal{A}, \mu, f) $ dovremo quindi prendere l'estremo superiore di tale limite al variare della partizione.

\begin{definition}[entropia di Kolmogorov-Sinai]
    Sia $ (X, \mathcal{A}, \mu, f) $ un sistema dinamico misurabile. L'entropia del sistema è
    \[
        h_\mu(f) \coloneqq \sup_\mathcal{P} {\lim_{n \to +\infty} \frac{1}{n}H(\mathcal{P}^{(n)})}
    \]
    dove $ H $ è l'entropia di Shannon della partizione e l'estremo superiore è fatto al variare di $ \mathcal{P} $ partizione finita (o numerabile) di $ X \pmod{0} $.
\end{definition}
Per mostrare che questa è una buona definizione enunciamo il seguente
\begin{lemma}[Fekete] \label{lem:fekete}
    Se $ a \colon \N\setminus\{0\} \to \R $ è una successione subadditiva (cioè tale che $ a_{n+m} \leq a_n + a_m $) allora esiste
    \[
        \lim_{n\to+\infty} \frac{a_n}{n} = \inf_{n\geq1}\frac{a_n}{n}.
    \]
\end{lemma}
\begin{proof}
    Sia $ \alpha = \inf_{n\geq1} (a_n/n) $. La disuguaglianza $ \liminf_{n\to\infty} (a_n/n) \geq \alpha $ è ovvia. Ci basta dimostrare che $ \limsup_{n\to\infty} (a_n/n) \leq \alpha $. Sia dunque $ \epsilon > 0 $; allora esiste $ \bar{n} $ tale che $ a_{\bar{n}}/\bar{n} \leq \alpha + \epsilon $. Riscriviamo ora $ n $ come $ n = k \bar{n} + r $ per opportuni $ k\in\N $ e $ r\in\{0,\ldots, \bar{n}-1\} $. Per la subaddività si ha $ a_{k\bar{n}+r} \leq a_{k\bar{n}} + a_r \leq ka_{\bar{n}} + a_r $ e quindi
    \[ \frac{a_n}{n} \leq \frac{k}{n}a_{\bar{n}} + \frac{1}{n}a_r \leq \frac{k}{n}a_{\bar{n}} + \frac{1}{n} a_{\bar{r}} \]
    dove $ \bar{r} = \argmax_{r\in\{ 0,\ldots, \bar{n}-1 \}} a_r $. Prendendo il $ \limsup $ si ha
    \[ \limsup_{n\to\infty} \frac{a_n}{n} \leq \frac{a_{\bar{n}}}{\bar{n}} \leq \alpha + \epsilon \]
    da cui, per l'arbitrarietà di $ \epsilon $, si conclude.
\end{proof}
Grazie a tale lemma ci basta allora dimostrare che la successione $ H(\mathcal{P}^{(n)}) $ è subadditiva in $ n $. Ora
\[ H(\mathcal{P}^{(n+m)}) = H\left(\bigvee_{i=0}^{n+m}f^{-i}\mathcal{P}\right) = H\left(P^{(m)} \vee \bigvee_{i=m+1}^{n+m}f^{-i}\mathcal{P}\right) \]
e d'altra parte
\[ H\left(\bigvee_{i=n+1}^{n+m}f^{-i}\mathcal{P}\right) = H\left(f^{-m}\bigvee_{i=1}^{n}f^{-i}\mathcal{P}\right) = H\left(\bigvee_{i=1}^{n}f^{-i}\mathcal{P}\right) = H(\mathcal{P}^{(n)}) \]
in quanto per $ f $-invarianza della misura, se $ \bigvee_{i=1}^{n}f^{-i}\mathcal{P} = \{B_1, \ldots, B_k\} $ allora $ f^{-m}\bigvee_{i=1}^{n}f^{-i}\mathcal{P} = \{f^{-m}(B_1), \ldots, f^{-m}(B_k)\} $ così $ \mu(f^{-m}(B_i)) = \mu(B_i) $ è l'entropia di Shannon delle due partizioni è la stessa. Per concludere ci basta basta quindi mostrare la seguente

\begin{proposition}
    Date due partizioni $ \mathcal{P} $ e $ \mathcal{Q} $ vale $ H(\mathcal{P} \vee \mathcal{Q}) \leq H(\mathcal{P}) + H(\mathcal{Q}) $.
\end{proposition}
\begin{proof}
    Siano $ \mathcal{P} = \{A_1, \ldots, A_n\} $ e $ \mathcal{Q} = \{B_1, \ldots, B_m\} $. Allora per il teorema \ref{thm:Shannon}, punto \ref{pt:formula} abbiamo che
    \[ H(\mathcal{P} \vee \mathcal{Q}) = H(\mathcal{P}) + \sum_{i, j} p_i f(t_{ij}) \]
    dove $ p_i = \mu(A_i) $ e $ t_{ij} = \frac{\mu(A_i \cap B_j)}{\mu(A_i)} $ e $ f(x) = - x \log{x} $. Essendo $ f $ convessa e $ \sum_{i} p_i = 1 $ otteniamo per la disuguaglianza di Jensen
    \[ \sum_{i} p_i f(t_{ij}) \leq f\left(\sum_{i} p_i t_{ij}\right) = f(\mu(B_j)) = - \mu(B_j) \log{\mu(B_j)} \]
    così
    \[ \sum_{ij} p_j f(t_{ij}) = \sum_{j} \sum_{i} p_j f(t_{ij}) \leq - \sum_{j} \mu(B_j) \log{\mu(B_j)} = H(\mathcal{Q}). \qedhere \]
\end{proof}

\begin{exercise}
    \textcolor{red}{Spostare nella lezione dopo quando parla di generatori}.
    Calcolare l'entropia metrica degli \emph{schemi di Bernoulli} (usando i cilindri come partizione).
\end{exercise}
\begin{solution}
    \textcolor{red}{Mancante}
\end{solution}
